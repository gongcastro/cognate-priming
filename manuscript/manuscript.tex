% Options for packages loaded elsewhere
\PassOptionsToPackage{unicode}{hyperref}
\PassOptionsToPackage{hyphens}{url}
\PassOptionsToPackage{dvipsnames,svgnames,x11names}{xcolor}
%
\documentclass[
  letterpaper,
  DIV=11,
  numbers=noendperiod]{scrartcl}

\usepackage{amsmath,amssymb}
\usepackage{iftex}
\ifPDFTeX
  \usepackage[T1]{fontenc}
  \usepackage[utf8]{inputenc}
  \usepackage{textcomp} % provide euro and other symbols
\else % if luatex or xetex
  \usepackage{unicode-math}
  \defaultfontfeatures{Scale=MatchLowercase}
  \defaultfontfeatures[\rmfamily]{Ligatures=TeX,Scale=1}
\fi
\usepackage{lmodern}
\ifPDFTeX\else  
    % xetex/luatex font selection
\fi
% Use upquote if available, for straight quotes in verbatim environments
\IfFileExists{upquote.sty}{\usepackage{upquote}}{}
\IfFileExists{microtype.sty}{% use microtype if available
  \usepackage[]{microtype}
  \UseMicrotypeSet[protrusion]{basicmath} % disable protrusion for tt fonts
}{}
\makeatletter
\@ifundefined{KOMAClassName}{% if non-KOMA class
  \IfFileExists{parskip.sty}{%
    \usepackage{parskip}
  }{% else
    \setlength{\parindent}{0pt}
    \setlength{\parskip}{6pt plus 2pt minus 1pt}}
}{% if KOMA class
  \KOMAoptions{parskip=half}}
\makeatother
\usepackage{xcolor}
\setlength{\emergencystretch}{3em} % prevent overfull lines
\setcounter{secnumdepth}{-\maxdimen} % remove section numbering
% Make \paragraph and \subparagraph free-standing
\ifx\paragraph\undefined\else
  \let\oldparagraph\paragraph
  \renewcommand{\paragraph}[1]{\oldparagraph{#1}\mbox{}}
\fi
\ifx\subparagraph\undefined\else
  \let\oldsubparagraph\subparagraph
  \renewcommand{\subparagraph}[1]{\oldsubparagraph{#1}\mbox{}}
\fi

\usepackage{color}
\usepackage{fancyvrb}
\newcommand{\VerbBar}{|}
\newcommand{\VERB}{\Verb[commandchars=\\\{\}]}
\DefineVerbatimEnvironment{Highlighting}{Verbatim}{commandchars=\\\{\}}
% Add ',fontsize=\small' for more characters per line
\usepackage{framed}
\definecolor{shadecolor}{RGB}{241,243,245}
\newenvironment{Shaded}{\begin{snugshade}}{\end{snugshade}}
\newcommand{\AlertTok}[1]{\textcolor[rgb]{0.68,0.00,0.00}{#1}}
\newcommand{\AnnotationTok}[1]{\textcolor[rgb]{0.37,0.37,0.37}{#1}}
\newcommand{\AttributeTok}[1]{\textcolor[rgb]{0.40,0.45,0.13}{#1}}
\newcommand{\BaseNTok}[1]{\textcolor[rgb]{0.68,0.00,0.00}{#1}}
\newcommand{\BuiltInTok}[1]{\textcolor[rgb]{0.00,0.23,0.31}{#1}}
\newcommand{\CharTok}[1]{\textcolor[rgb]{0.13,0.47,0.30}{#1}}
\newcommand{\CommentTok}[1]{\textcolor[rgb]{0.37,0.37,0.37}{#1}}
\newcommand{\CommentVarTok}[1]{\textcolor[rgb]{0.37,0.37,0.37}{\textit{#1}}}
\newcommand{\ConstantTok}[1]{\textcolor[rgb]{0.56,0.35,0.01}{#1}}
\newcommand{\ControlFlowTok}[1]{\textcolor[rgb]{0.00,0.23,0.31}{#1}}
\newcommand{\DataTypeTok}[1]{\textcolor[rgb]{0.68,0.00,0.00}{#1}}
\newcommand{\DecValTok}[1]{\textcolor[rgb]{0.68,0.00,0.00}{#1}}
\newcommand{\DocumentationTok}[1]{\textcolor[rgb]{0.37,0.37,0.37}{\textit{#1}}}
\newcommand{\ErrorTok}[1]{\textcolor[rgb]{0.68,0.00,0.00}{#1}}
\newcommand{\ExtensionTok}[1]{\textcolor[rgb]{0.00,0.23,0.31}{#1}}
\newcommand{\FloatTok}[1]{\textcolor[rgb]{0.68,0.00,0.00}{#1}}
\newcommand{\FunctionTok}[1]{\textcolor[rgb]{0.28,0.35,0.67}{#1}}
\newcommand{\ImportTok}[1]{\textcolor[rgb]{0.00,0.46,0.62}{#1}}
\newcommand{\InformationTok}[1]{\textcolor[rgb]{0.37,0.37,0.37}{#1}}
\newcommand{\KeywordTok}[1]{\textcolor[rgb]{0.00,0.23,0.31}{#1}}
\newcommand{\NormalTok}[1]{\textcolor[rgb]{0.00,0.23,0.31}{#1}}
\newcommand{\OperatorTok}[1]{\textcolor[rgb]{0.37,0.37,0.37}{#1}}
\newcommand{\OtherTok}[1]{\textcolor[rgb]{0.00,0.23,0.31}{#1}}
\newcommand{\PreprocessorTok}[1]{\textcolor[rgb]{0.68,0.00,0.00}{#1}}
\newcommand{\RegionMarkerTok}[1]{\textcolor[rgb]{0.00,0.23,0.31}{#1}}
\newcommand{\SpecialCharTok}[1]{\textcolor[rgb]{0.37,0.37,0.37}{#1}}
\newcommand{\SpecialStringTok}[1]{\textcolor[rgb]{0.13,0.47,0.30}{#1}}
\newcommand{\StringTok}[1]{\textcolor[rgb]{0.13,0.47,0.30}{#1}}
\newcommand{\VariableTok}[1]{\textcolor[rgb]{0.07,0.07,0.07}{#1}}
\newcommand{\VerbatimStringTok}[1]{\textcolor[rgb]{0.13,0.47,0.30}{#1}}
\newcommand{\WarningTok}[1]{\textcolor[rgb]{0.37,0.37,0.37}{\textit{#1}}}

\providecommand{\tightlist}{%
  \setlength{\itemsep}{0pt}\setlength{\parskip}{0pt}}\usepackage{longtable,booktabs,array}
\usepackage{calc} % for calculating minipage widths
% Correct order of tables after \paragraph or \subparagraph
\usepackage{etoolbox}
\makeatletter
\patchcmd\longtable{\par}{\if@noskipsec\mbox{}\fi\par}{}{}
\makeatother
% Allow footnotes in longtable head/foot
\IfFileExists{footnotehyper.sty}{\usepackage{footnotehyper}}{\usepackage{footnote}}
\makesavenoteenv{longtable}
\usepackage{graphicx}
\makeatletter
\def\maxwidth{\ifdim\Gin@nat@width>\linewidth\linewidth\else\Gin@nat@width\fi}
\def\maxheight{\ifdim\Gin@nat@height>\textheight\textheight\else\Gin@nat@height\fi}
\makeatother
% Scale images if necessary, so that they will not overflow the page
% margins by default, and it is still possible to overwrite the defaults
% using explicit options in \includegraphics[width, height, ...]{}
\setkeys{Gin}{width=\maxwidth,height=\maxheight,keepaspectratio}
% Set default figure placement to htbp
\makeatletter
\def\fps@figure{htbp}
\makeatother
\newlength{\cslhangindent}
\setlength{\cslhangindent}{1.5em}
\newlength{\csllabelwidth}
\setlength{\csllabelwidth}{3em}
\newlength{\cslentryspacingunit} % times entry-spacing
\setlength{\cslentryspacingunit}{\parskip}
\newenvironment{CSLReferences}[2] % #1 hanging-ident, #2 entry spacing
 {% don't indent paragraphs
  \setlength{\parindent}{0pt}
  % turn on hanging indent if param 1 is 1
  \ifodd #1
  \let\oldpar\par
  \def\par{\hangindent=\cslhangindent\oldpar}
  \fi
  % set entry spacing
  \setlength{\parskip}{#2\cslentryspacingunit}
 }%
 {}
\usepackage{calc}
\newcommand{\CSLBlock}[1]{#1\hfill\break}
\newcommand{\CSLLeftMargin}[1]{\parbox[t]{\csllabelwidth}{#1}}
\newcommand{\CSLRightInline}[1]{\parbox[t]{\linewidth - \csllabelwidth}{#1}\break}
\newcommand{\CSLIndent}[1]{\hspace{\cslhangindent}#1}

\usepackage{booktabs}
\usepackage{longtable}
\usepackage{array}
\usepackage{multirow}
\usepackage{wrapfig}
\usepackage{float}
\usepackage{colortbl}
\usepackage{pdflscape}
\usepackage{tabu}
\usepackage{threeparttable}
\usepackage{threeparttablex}
\usepackage[normalem]{ulem}
\usepackage{makecell}
\usepackage{xcolor}
\usepackage{caption}
\usepackage{tipa}
\usepackage{booktabs}
\hyphenpenalty=100000
\exhyphenpenalty=100000
\KOMAoption{captions}{tableheading}
\makeatletter
\makeatother
\makeatletter
\makeatother
\makeatletter
\@ifpackageloaded{caption}{}{\usepackage{caption}}
\AtBeginDocument{%
\ifdefined\contentsname
  \renewcommand*\contentsname{Table of contents}
\else
  \newcommand\contentsname{Table of contents}
\fi
\ifdefined\listfigurename
  \renewcommand*\listfigurename{List of Figures}
\else
  \newcommand\listfigurename{List of Figures}
\fi
\ifdefined\listtablename
  \renewcommand*\listtablename{List of Tables}
\else
  \newcommand\listtablename{List of Tables}
\fi
\ifdefined\figurename
  \renewcommand*\figurename{Figure}
\else
  \newcommand\figurename{Figure}
\fi
\ifdefined\tablename
  \renewcommand*\tablename{Table}
\else
  \newcommand\tablename{Table}
\fi
}
\@ifpackageloaded{float}{}{\usepackage{float}}
\floatstyle{ruled}
\@ifundefined{c@chapter}{\newfloat{codelisting}{h}{lop}}{\newfloat{codelisting}{h}{lop}[chapter]}
\floatname{codelisting}{Listing}
\newcommand*\listoflistings{\listof{codelisting}{List of Listings}}
\makeatother
\makeatletter
\@ifpackageloaded{caption}{}{\usepackage{caption}}
\@ifpackageloaded{subcaption}{}{\usepackage{subcaption}}
\makeatother
\makeatletter
\@ifpackageloaded{tcolorbox}{}{\usepackage[skins,breakable]{tcolorbox}}
\makeatother
\makeatletter
\@ifundefined{shadecolor}{\definecolor{shadecolor}{rgb}{.97, .97, .97}}
\makeatother
\makeatletter
\makeatother
\makeatletter
\makeatother
\ifLuaTeX
\usepackage[bidi=basic]{babel}
\else
\usepackage[bidi=default]{babel}
\fi
\babelprovide[main,import]{english}
% get rid of language-specific shorthands (see #6817):
\let\LanguageShortHands\languageshorthands
\def\languageshorthands#1{}
\ifLuaTeX
  \usepackage{selnolig}  % disable illegal ligatures
\fi
\IfFileExists{bookmark.sty}{\usepackage{bookmark}}{\usepackage{hyperref}}
\IfFileExists{xurl.sty}{\usepackage{xurl}}{} % add URL line breaks if available
\urlstyle{same} % disable monospaced font for URLs
\hypersetup{
  pdftitle={Developmental trajectories of bilingual word recognition},
  pdfauthor={Gonzalo Garcia-Castro; Serene Siow; Nuria Sebastian-Galles; Kim Plunkett},
  pdflang={en},
  pdfkeywords={cognate, word recognition, lexicon, language
acquisition, vocabulary, bilingualism, general additive mixed
models, bayesian},
  colorlinks=true,
  linkcolor={blue},
  filecolor={Maroon},
  citecolor={Blue},
  urlcolor={Blue},
  pdfcreator={LaTeX via pandoc}}

\title{Developmental trajectories of bilingual word recognition}
\author{Gonzalo Garcia-Castro \and Serene Siow \and Nuria
Sebastian-Galles \and Kim Plunkett}
\date{}

\begin{document}
\maketitle
\ifdefined\Shaded\renewenvironment{Shaded}{\begin{tcolorbox}[boxrule=0pt, borderline west={3pt}{0pt}{shadecolor}, frame hidden, sharp corners, enhanced, interior hidden, breakable]}{\end{tcolorbox}}\fi

\begin{center}\rule{0.5\linewidth}{0.5pt}\end{center}

\begin{center}\rule{0.5\linewidth}{0.5pt}\end{center}

\hypertarget{introduction}{%
\section{Introduction}\label{introduction}}

The formation of a mental lexicon is a critical developmental
achievement for infants\_ learning words allows infants to retrieve
socially relevant concepts from fairly arbitrary linguistic forms
embedded in speech.

Learning words allows infants to access the rich world made of concepts
through the recognition of fairly arbitrary linguistic forms, i.e.,
words. This remarkable developmental achievement is

The bilingual lexicon is language-non selective.

The role of cognates in lexical processing.

How does language co-activation shape lexical development?

Implicit naming as a paradigm to study cross-language activation. (Mani
\& Plunkett, 2010, 2011a).

\hypertarget{methods}{%
\section{Methods}\label{methods}}

All materials, data, and reproducible code can be found at the OSF
(\href{https://osf.io/ckydb/}{https://osf.io/hy984/}) and GitHub
(\url{https://github.com/gongcastro/cognate-priming}) repositories. This
study was conducted according to guidelines laid down in the Declaration
of Helsinki, and was approved by the Drug Research Ethical Committee
(CEIm) of the IMIM Parc de Salut Mar, reference 2020/9080/I. Before
every testing session, caretakers were asked to read and sign an
informed consent form, and were given a small gift at the end of it

\hypertarget{participants}{%
\subsection{Participants}\label{participants}}

\begin{Shaded}
\begin{Highlighting}[]
\NormalTok{n\_participants\_total }\OtherTok{\textless{}{-}} \FunctionTok{length}\NormalTok{(}\FunctionTok{unique}\NormalTok{(participants}\SpecialCharTok{$}\NormalTok{id))}

\NormalTok{n\_participants\_sessions }\OtherTok{\textless{}{-}} \FunctionTok{count}\NormalTok{(participants, id, }\AttributeTok{name =} \StringTok{"n\_sessions"}\NormalTok{) }\SpecialCharTok{|\textgreater{}} 
    \FunctionTok{count}\NormalTok{(n\_sessions) }\SpecialCharTok{|\textgreater{}} 
    \FunctionTok{group\_split}\NormalTok{(n\_sessions) }\SpecialCharTok{|\textgreater{}} 
    \FunctionTok{set\_names}\NormalTok{(}\FunctionTok{paste0}\NormalTok{(}\StringTok{"session\_"}\NormalTok{, }\DecValTok{1}\SpecialCharTok{:}\DecValTok{3}\NormalTok{))}

\NormalTok{n\_sessions\_total }\OtherTok{\textless{}{-}} \FunctionTok{count}\NormalTok{(participants)}

\NormalTok{n\_sessions\_age\_group }\OtherTok{\textless{}{-}}\NormalTok{ participants }\SpecialCharTok{|\textgreater{}} 
    \FunctionTok{group\_by}\NormalTok{(age\_group) }\SpecialCharTok{|\textgreater{}} 
    \FunctionTok{summarise}\NormalTok{(}\FunctionTok{across}\NormalTok{(age, }\FunctionTok{lst}\NormalTok{(mean, sd, min, max)),}
              \AttributeTok{n =} \FunctionTok{n}\NormalTok{(),}
              \AttributeTok{.groups =} \StringTok{"drop"}\NormalTok{) }\SpecialCharTok{|\textgreater{}} 
    \FunctionTok{mutate}\NormalTok{(}\FunctionTok{across}\NormalTok{(age\_mean}\SpecialCharTok{:}\NormalTok{age\_max, round, }\DecValTok{2}\NormalTok{)) }\SpecialCharTok{|\textgreater{}} 
    \FunctionTok{group\_split}\NormalTok{(age\_group) }\SpecialCharTok{|\textgreater{}} 
    \FunctionTok{set\_names}\NormalTok{(}\FunctionTok{c}\NormalTok{(}\StringTok{"age\_21"}\NormalTok{, }\StringTok{"age\_25"}\NormalTok{, }\StringTok{"age\_30"}\NormalTok{))}
\end{Highlighting}
\end{Shaded}

\begin{verbatim}
Warning: There was 1 warning in `mutate()`.
i In argument: `across(age_mean:age_max, round, 2)`.
Caused by warning:
! The `...` argument of `across()` is deprecated as of dplyr 1.1.0.
Supply arguments directly to `.fns` through an anonymous function instead.

  # Previously
  across(a:b, mean, na.rm = TRUE)

  # Now
  across(a:b, \(x) mean(x, na.rm = TRUE))
\end{verbatim}

\begin{Shaded}
\begin{Highlighting}[]
\NormalTok{n\_sessions\_dominance }\OtherTok{\textless{}{-}} \FunctionTok{count}\NormalTok{(participants, test\_language) }\SpecialCharTok{|\textgreater{}} 
    \FunctionTok{group\_split}\NormalTok{(test\_language) }\SpecialCharTok{|\textgreater{}} 
    \FunctionTok{set\_names}\NormalTok{(}\FunctionTok{c}\NormalTok{(}\StringTok{"catalan"}\NormalTok{, }\StringTok{"spanish"}\NormalTok{))}

\NormalTok{n\_sessions\_dominance\_age\_group }\OtherTok{\textless{}{-}} \FunctionTok{count}\NormalTok{(participants, age\_group, test\_language) }\SpecialCharTok{|\textgreater{}} 
    \FunctionTok{group\_split}\NormalTok{(test\_language) }\SpecialCharTok{|\textgreater{}} 
    \FunctionTok{set\_names}\NormalTok{(}\FunctionTok{c}\NormalTok{(}\StringTok{"catalan"}\NormalTok{, }\StringTok{"spanish"}\NormalTok{)) }\SpecialCharTok{|\textgreater{}} 
    \FunctionTok{map}\NormalTok{(\textbackslash{}(x) }\FunctionTok{group\_split}\NormalTok{(x, age\_group) }\SpecialCharTok{|\textgreater{}} 
            \FunctionTok{set\_names}\NormalTok{(}\FunctionTok{c}\NormalTok{(}\StringTok{"age\_21"}\NormalTok{, }\StringTok{"age\_25"}\NormalTok{, }\StringTok{"age\_30"}\NormalTok{)))}

\NormalTok{n\_sessions\_lp }\OtherTok{\textless{}{-}}\NormalTok{ participants }\SpecialCharTok{|\textgreater{}} 
    \FunctionTok{count}\NormalTok{(lp) }\SpecialCharTok{|\textgreater{}} 
    \FunctionTok{group\_split}\NormalTok{(lp) }\SpecialCharTok{|\textgreater{}} 
    \FunctionTok{set\_names}\NormalTok{(}\FunctionTok{c}\NormalTok{(}\StringTok{"monolingual"}\NormalTok{, }\StringTok{"bilingual"}\NormalTok{))}

\NormalTok{n\_sessions\_lp\_age\_group }\OtherTok{\textless{}{-}}\NormalTok{ participants }\SpecialCharTok{|\textgreater{}} 
    \FunctionTok{count}\NormalTok{(lp, age\_group) }\SpecialCharTok{|\textgreater{}} 
    \FunctionTok{group\_split}\NormalTok{(lp) }\SpecialCharTok{|\textgreater{}} 
    \FunctionTok{set\_names}\NormalTok{(}\FunctionTok{c}\NormalTok{(}\StringTok{"monolingual"}\NormalTok{, }\StringTok{"bilingual"}\NormalTok{)) }\SpecialCharTok{|\textgreater{}} 
    \FunctionTok{map}\NormalTok{(\textbackslash{}(x) }\FunctionTok{group\_split}\NormalTok{(x, age\_group) }\SpecialCharTok{|\textgreater{}} 
            \FunctionTok{set\_names}\NormalTok{(}\FunctionTok{c}\NormalTok{(}\StringTok{"age\_21"}\NormalTok{, }\StringTok{"age\_25"}\NormalTok{, }\StringTok{"age\_30"}\NormalTok{)))}
\end{Highlighting}
\end{Shaded}

We collected data from 180 monolingual and bilingual participants living
in the Metropolitan Area of Barcelona (Spain), who were exposed to at
least Catalan and/or Spanish from birth. Families were recruited from
maternity room in private hospitals in Barcelona, and contacted via
phone when the child's age spanned between our age intervals of
interest. Families were invited to participate at three age points: 21,
25, and 30 months. 94 participants were tested at one age point, 56 at
two age points, and 30 at the three age points. In total, we gathered
data from 296 testing sessions: 94 at 21 months (\emph{Mean} = 20.98,
\emph{SD} = 0.97, \emph{Range} = 20--26.35), 96 at 25 months
(\emph{Mean} = 25.15, \emph{SD} = 0.99, \emph{Range} = 23.35--30.77),
and 106 at 30 months (\emph{Mean} = 29.42, \emph{SD} = 0.99,
\emph{Range} = 19.37--36.07).

\hypertarget{sec-lp}{%
\subsubsection{Language profile}\label{sec-lp}}

We assessed participants' language profile using the Language Exposure
Questionnaire (LEQ, \protect\hyperlink{ref-bosch2001evidence}{Bosch \&
Sebastián-Gallés, 2001}). Before each experimental session, the
experimenter asked the caretakers to estimate the amount of hours per
day they and other people in the infant's social circle have spent
speaking to the infant in any language since birth. The output of this
interview is an estimated degree of exposure (DoE) to each language,
indicated by the proportion of time the infant was reported to have
listened to each language. According to this estimate, we classified
participants as Catalan- or Spanish-dominant if the language with
highest DoE was Catalan or Spanish, respectively, and tested the
participant in the stimuli set that contained words in their native
language. We collected data from 178 Catalan-dominant participants in
Catalan (52 at 21 months, 62 at 25 months, and 64 at 30 months). We
further classified participants as monolinguals if the DoE to their
dominant language exceeded 80\% of the total DoE to Catalan and Spanish,
and as bilinguals otherwise. Participants with DoE to language other
than Catalan or Spanish were excluded from analyses. This divided the
sample into 162 monolinguals ( at 21 months, at 25 months, and at 30
months), and 134 bilinguals ( at 21 months, 2, 2, 44 at 25 months, and
2, 3, 46 at 30 months). Table~\ref{tbl-participants-lp} shows a detailed
description of the linguistic profile of our sample.

\begin{Shaded}
\begin{Highlighting}[]
\NormalTok{participants }\SpecialCharTok{|\textgreater{}} 
    \FunctionTok{filter}\NormalTok{(is\_valid\_participant) }\SpecialCharTok{|\textgreater{}} 
    \FunctionTok{select}\NormalTok{(id, age\_group, age, lp,}
\NormalTok{           doe\_catalan, doe\_spanish, test\_language) }\SpecialCharTok{|\textgreater{}} 
    \FunctionTok{mutate}\NormalTok{(}\AttributeTok{id =} \FunctionTok{paste0}\NormalTok{(id, }\StringTok{" ("}\NormalTok{, age\_group, }\StringTok{")"}\NormalTok{)) }\SpecialCharTok{|\textgreater{}}
    \FunctionTok{add\_count}\NormalTok{(lp, }
              \AttributeTok{name =} \StringTok{"n\_lp"}\NormalTok{) }\SpecialCharTok{|\textgreater{}} 
    \FunctionTok{add\_count}\NormalTok{(age\_group, }
              \AttributeTok{name =} \StringTok{"n\_age\_group"}\NormalTok{) }\SpecialCharTok{|\textgreater{}} 
    \FunctionTok{pivot\_longer}\NormalTok{(}\FunctionTok{starts\_with}\NormalTok{(}\StringTok{"doe\_"}\NormalTok{),}
                 \AttributeTok{names\_to =} \StringTok{"language"}\NormalTok{,}
                 \AttributeTok{values\_to =} \StringTok{"doe"}\NormalTok{) }\SpecialCharTok{|\textgreater{}}
    \FunctionTok{add\_count}\NormalTok{(age\_group,}
\NormalTok{              test\_language,}
              \AttributeTok{name =} \StringTok{"n\_age\_test"}\NormalTok{) }\SpecialCharTok{|\textgreater{}} 
    \FunctionTok{mutate}\NormalTok{(}\AttributeTok{language =} \FunctionTok{str\_to\_sentence}\NormalTok{(}\FunctionTok{str\_remove\_all}\NormalTok{(language, }\StringTok{"doe\_"}\NormalTok{)),}
           \AttributeTok{age\_group =} \FunctionTok{paste0}\NormalTok{(age\_group, }\StringTok{" (N = "}\NormalTok{, n\_age\_group, }\StringTok{")"}\NormalTok{),}
           \AttributeTok{test\_language =} \FunctionTok{paste0}\NormalTok{(}\StringTok{"Tested in "}\NormalTok{, test\_language, }
                               \StringTok{" (N = "}\NormalTok{, n\_age\_test, }\StringTok{")"}\NormalTok{),}
           \AttributeTok{lp =} \FunctionTok{factor}\NormalTok{(lp, }\AttributeTok{levels =} \FunctionTok{rev}\NormalTok{(}\FunctionTok{unique}\NormalTok{(lp))))  }\SpecialCharTok{|\textgreater{}} 
    \FunctionTok{summarise}\NormalTok{(}\FunctionTok{across}\NormalTok{(}\FunctionTok{c}\NormalTok{(doe, age), }\FunctionTok{lst}\NormalTok{(mean, sd)),}
              \AttributeTok{.by =} \FunctionTok{c}\NormalTok{(age\_group, lp, test\_language, language)) }\SpecialCharTok{|\textgreater{}} 
    \FunctionTok{pivot\_wider}\NormalTok{(}\AttributeTok{id\_cols =} \FunctionTok{c}\NormalTok{(age\_group, test\_language),}
                \AttributeTok{names\_from =} \FunctionTok{c}\NormalTok{(language, lp),}
                \AttributeTok{values\_from =} \FunctionTok{c}\NormalTok{(}\FunctionTok{matches}\NormalTok{(}\StringTok{"doe"}\NormalTok{), age\_mean, age\_sd),}
                \AttributeTok{names\_repair =}\NormalTok{ janitor}\SpecialCharTok{::}\NormalTok{make\_clean\_names) }\SpecialCharTok{|\textgreater{}} 
    \FunctionTok{relocate}\NormalTok{(age\_group, test\_language,}
             \FunctionTok{matches}\NormalTok{(}\StringTok{"monolingual"}\NormalTok{),}
             \FunctionTok{matches}\NormalTok{(}\StringTok{"bilingual"}\NormalTok{)) }\SpecialCharTok{|\textgreater{}} 
    \FunctionTok{select}\NormalTok{(}\SpecialCharTok{{-}}\FunctionTok{c}\NormalTok{(age\_mean\_spanish\_monolingual,}
\NormalTok{              age\_mean\_spanish\_bilingual,}
\NormalTok{              age\_sd\_spanish\_monolingual,}
\NormalTok{              age\_sd\_spanish\_bilingual)) }\SpecialCharTok{|\textgreater{}} 
    \FunctionTok{arrange}\NormalTok{(age\_group, test\_language) }\SpecialCharTok{|\textgreater{}} 
    \FunctionTok{gt}\NormalTok{(}\AttributeTok{rowname\_col =} \StringTok{"test\_language"}\NormalTok{, }
       \AttributeTok{groupname\_col =} \StringTok{"age\_group"}\NormalTok{, }
       \AttributeTok{row\_group.sep =} \StringTok{": "}\NormalTok{) }\SpecialCharTok{|\textgreater{}} 
    \FunctionTok{tab\_spanner}\NormalTok{(}\FunctionTok{md}\NormalTok{(}\StringTok{"Monolingual (*N* = 162)"}\NormalTok{), }\FunctionTok{matches}\NormalTok{(}\StringTok{"monolingual"}\NormalTok{)) }\SpecialCharTok{|\textgreater{}} 
    \FunctionTok{tab\_spanner}\NormalTok{(}\FunctionTok{md}\NormalTok{(}\StringTok{"Bilingual (*N* = 133)"}\NormalTok{), }\FunctionTok{matches}\NormalTok{(}\StringTok{"bilingual"}\NormalTok{)) }\SpecialCharTok{|\textgreater{}}
    \FunctionTok{fmt\_number}\NormalTok{(}\FunctionTok{matches}\NormalTok{(}\StringTok{"doe"}\NormalTok{), }\AttributeTok{decimals =} \DecValTok{1}\NormalTok{, }\AttributeTok{scale\_by =} \DecValTok{100}\NormalTok{) }\SpecialCharTok{|\textgreater{}} 
    \FunctionTok{fmt\_number}\NormalTok{(}\FunctionTok{matches}\NormalTok{(}\StringTok{"age"}\NormalTok{), }\AttributeTok{decimals =} \DecValTok{1}\NormalTok{) }\SpecialCharTok{|\textgreater{}} 
    \FunctionTok{cols\_merge\_uncert}\NormalTok{(}\AttributeTok{col\_val =}\NormalTok{ age\_mean\_catalan\_monolingual, }
                      \AttributeTok{col\_uncert =}\NormalTok{ age\_sd\_catalan\_monolingual) }\SpecialCharTok{|\textgreater{}} 
    \FunctionTok{cols\_merge\_uncert}\NormalTok{(}\AttributeTok{col\_val =}\NormalTok{ age\_mean\_catalan\_bilingual, }
                      \AttributeTok{col\_uncert =}\NormalTok{ age\_sd\_catalan\_bilingual) }\SpecialCharTok{|\textgreater{}} 
    \FunctionTok{cols\_merge\_uncert}\NormalTok{(}\AttributeTok{col\_val =}\NormalTok{ doe\_mean\_catalan\_monolingual,}
                      \AttributeTok{col\_uncert =}\NormalTok{ doe\_sd\_catalan\_monolingual) }\SpecialCharTok{|\textgreater{}} 
    \FunctionTok{cols\_merge\_uncert}\NormalTok{(}\AttributeTok{col\_val =}\NormalTok{ doe\_mean\_catalan\_bilingual,}
                      \AttributeTok{col\_uncert =}\NormalTok{ doe\_sd\_catalan\_bilingual) }\SpecialCharTok{|\textgreater{}} 
    \FunctionTok{cols\_merge\_uncert}\NormalTok{(}\AttributeTok{col\_val =}\NormalTok{ doe\_mean\_spanish\_monolingual, }
                      \AttributeTok{col\_uncert =}\NormalTok{ doe\_sd\_spanish\_monolingual) }\SpecialCharTok{|\textgreater{}} 
    \FunctionTok{cols\_merge\_uncert}\NormalTok{(}\AttributeTok{col\_val =}\NormalTok{ doe\_mean\_spanish\_bilingual, }
                      \AttributeTok{col\_uncert =}\NormalTok{ doe\_sd\_spanish\_bilingual) }\SpecialCharTok{|\textgreater{}} 
    \FunctionTok{cols\_label}\NormalTok{(}\AttributeTok{age\_mean\_catalan\_monolingual =} \StringTok{"Age (months)"}\NormalTok{,}
               \AttributeTok{age\_mean\_catalan\_bilingual =} \StringTok{"Age (months)"}\NormalTok{,}
               \AttributeTok{doe\_mean\_catalan\_monolingual =} \StringTok{"Catalan (\%)"}\NormalTok{,}
               \AttributeTok{doe\_mean\_catalan\_bilingual =} \StringTok{"Catalan (\%)"}\NormalTok{,}
               \AttributeTok{doe\_mean\_spanish\_monolingual =} \StringTok{"Spanish (\%)"}\NormalTok{,}
               \AttributeTok{doe\_mean\_spanish\_bilingual =} \StringTok{"Spanish (\%)"}\NormalTok{) }\SpecialCharTok{|\textgreater{}} 
    \FunctionTok{tab\_style}\NormalTok{(}\FunctionTok{cell\_text}\NormalTok{(}\AttributeTok{weight =} \StringTok{"bold"}\NormalTok{),}
              \FunctionTok{list}\NormalTok{(}\FunctionTok{cells\_column\_spanners}\NormalTok{())) }\SpecialCharTok{|\textgreater{}} 
    \FunctionTok{tab\_style}\NormalTok{(}\FunctionTok{cell\_text}\NormalTok{(}\AttributeTok{size =} \StringTok{"medium"}\NormalTok{),}
              \FunctionTok{list}\NormalTok{(}\FunctionTok{cells\_body}\NormalTok{(),}
                 \FunctionTok{cells\_stub}\NormalTok{()))}
\end{Highlighting}
\end{Shaded}

\hypertarget{tbl-participants-lp}{}
\begin{longtable}{l|rrrrrr}
\caption{\label{tbl-participants-lp}Description of language profile of test participants. Data are
summarised for each age group, and for monolinguals and bilinguals
separately. }\tabularnewline

\toprule
\multicolumn{1}{l}{} & \multicolumn{3}{c}{Monolingual (\emph{N} = 162)} & \multicolumn{3}{c}{Bilingual (\emph{N} = 133)} \\ 
\cmidrule(lr){2-4} \cmidrule(lr){5-7}
\multicolumn{1}{l}{} & Catalan (\%) & Spanish (\%) & Age (months) & Catalan (\%) & Spanish (\%) & Age (months) \\ 
\midrule
\multicolumn{7}{l}{21 months (N = 29)} \\ 
\midrule
Tested in Catalan (N = 38) & $98.8$ ± $1.8$ & $1.2$ ± $1.8$ & $20.8$ ± $0.4$ & $61.6$ ± $7.3$ & $38.4$ ± $7.3$ & $21.1$ ± $0.4$ \\ 
Tested in Spanish (N = 20) & $10.1$ ± $6.2$ & $89.9$ ± $6.2$ & $20.8$ ± $0.5$ & $56.0$ ± $17.0$ & $44.0$ ± $17.0$ & $20.4$ ± $0.3$ \\ 
\midrule
\multicolumn{7}{l}{25 months (N = 45)} \\ 
\midrule
Tested in Catalan (N = 68) & $85.5$ ± $25.1$ & $14.5$ ± $25.1$ & $25.1$ ± $0.5$ & $58.9$ ± $12.7$ & $40.3$ ± $13.1$ & $24.7$ ± $0.6$ \\ 
Tested in Spanish (N = 22) & $45.9$ ± $44.3$ & $53.7$ ± $43.9$ & $25.5$ ± $0.4$ & $43.8$ ± $6.2$ & $55.2$ ± $6.2$ & $25.3$ ± $0.6$ \\ 
\midrule
\multicolumn{7}{l}{30 months (N = 39)} \\ 
\midrule
Tested in Catalan (N = 52) & $94.7$ ± $5.9$ & $5.3$ ± $5.8$ & $30.2$ ± $1.1$ & $59.7$ ± $8.8$ & $40.0$ ± $8.7$ & $29.6$ ± $1.3$ \\ 
Tested in Spanish (N = 26) & $9.2$ ± $7.8$ & $89.4$ ± $6.6$ & $29.7$ ± $1.0$ & $42.5$ ± $12.2$ & $55.9$ ± $13.8$ & $29.9$ ± $1.1$ \\ 
\bottomrule
\end{longtable}

\hypertarget{vocabulary-size}{%
\subsection{Vocabulary size}\label{vocabulary-size}}

We collected vocabulary data using parental responses to the Barcelona
Vocabulary Inventory (BVQ,
\protect\hyperlink{ref-garcia-castro2023bvq}{\textbf{garcia-castro2023bvq?}}),
an online vocabulary checklist inspired in several adaptations of the
the Communicative Developmental Inventory (CDI,
\protect\hyperlink{ref-fenson2004variability}{\textbf{fenson2004variability?}})
developed to assess the vocabulary size of Catalan-Spanish bilingual
toddlers. Families received a link to the BVQ immediately after each
experimental session, and were given two weeks to fill it. Two weeks
after being sent the link, 141 (48\% families failed to provide a
complete response to the BVQ. We imputed their vocabulary size scores
using multiple imputation
(\protect\hyperlink{ref-vanbuuren2018flexible}{\textbf{vanbuuren2018flexible?}}).
We used predictive mean matching to impute missing scores using
participants' information about their age group and language profile. We
obtained several receptive and expressive vocabulary size measures from
each participant's vocabulary: Catalan vocabulary size (proportion of
the words in the Catalan checklist reported as acquired), Spanish
vocabulary size (proportion of the words in the Catalan checklist
reported as acquired), and total vocabulary size (proportion of the
words in both checklists reported as acquired).

\hypertarget{stimuli}{%
\subsubsection{Stimuli}\label{stimuli}}

\begin{Shaded}
\begin{Highlighting}[]
\NormalTok{n\_words }\OtherTok{\textless{}{-}} \FunctionTok{unique}\NormalTok{(}\FunctionTok{with}\NormalTok{(stimuli, }\FunctionTok{unlist}\NormalTok{(prime, target, distractor)))}
\end{Highlighting}
\end{Shaded}

We used arbre, ànec, camió, finestra, vaca, barret, osset, claus, cotxe,
cadira, vaixell, ampolla, flor, formatge, botó, ulleres, gat, gos, mico,
pa, moto, mitjó, avió, granota, ovella, ocell, peix, plàtan, pilota,
poma, guitarra, maduixa, botella, globo, mono, coche, cama, zumo, pato,
cebra, cerdo, oreja, mesa, galleta, gafas, tomate, zapato, manzana,
vaso, queso, oveja, ojo, pelota, tele, pez, perro, barco, hoja, tren,
tenedor, casa, gato distinct words included in the BVQ to create the
stimuli lists. We created six stimuli lists: three in Catalan, and three
in Spanish. Each list contained 32 trials, each involving a
prime-target-distractor group. Each word played a role as either prime,
\emph{or} as target and distractor across the three lists in their
corresponding language. For instance, the Catalan word \emph{cadira}
appeared as \emph{prime} in the three lists, but never as \emph{target}
or \emph{distractor}; the Catalan word \emph{bici} appeared as
\emph{target} and \emph{distractor} across the three lists, but never as
a prime. Target-distractor pairings were held constant across the three
lists in each language. For instance, in all Catalan lists the word
\emph{bici} was paired with the word \emph{porta}. Target-distractor
pairings were also yoked, so that each member of the same
target-distractor pair appeared once as target and once as distractor in
each list. For instance, the \emph{bici}-\emph{porta} paired appeared
twice in each of the three Catalan lists: once with \emph{bici} as
target and \emph{porta} as distractor, and once with \emph{porta} as
target and \emph{bici} as distractor. This counterbalancing avoided
participants encountering looking at the target word guided solely by
that word having being named in a previous trial. Finally, prime words
appeared only once in each list: each target-distractor pair was
associated with a different prime word in both appearances. In each
list, the same prime word was presented alongside a different
target-distractor pair. For instance, the Catalan prime word
\emph{barret} was presented with the \emph{bici}-\emph{porta}
target-distractor pair in one list, with the \emph{bici}-\emph{porta}
pair in another list, and with \emph{berenar}-\emph{amanida} in the
remaining list. The order of the trials was randomised across
experimental session, so that each time a participant was tested, the
order in which the prime-target-distractor was presented was randomised.
Each participant was randomly assigned to one of the three list in the
corresponding language (their dominant language, see
Section~\ref{sec-lp}), and always the same list across their
experimental sessions in the case of a recurrent participant.

In 16 of the 32 trials of the same list (henceforth \emph{related}
trials), the prime and the target words were phonologically related,
sharing phonological onset (at least first phoneme). In the other 16
trials (\emph{unrelated} trials), prime and target did not share
phonological onset. 8 of the 16 \emph{related} trials included a cognate
prime (\emph{cognate} trials), and the other 8 included a non-cognate
trials (\emph{non-cognate} trials). A prime word was considered cognate
if its Catalan and Spanish translation shared phonological onset.
Especial attention was paid to avoiding semantic or taxonomic
relationships between prime and target words, and between prime and
distractor words. Target and distractor word pairs were phonologically
unrelated (did not share phonological onset). Some of them shared
semantic features or a taxonomic relationship. This is the case of words
associated with especially salient referents such as animals or food. To
avoid infants guiding their gaze to these objects based on their
saliency, we paired animals and food items together.

We examined the overall equivalence of the three trial types by
comparing them across three variables relating to the target word:
lexical frequency, word prevalence, animacy Table~\ref{tbl-stimuli}
shows a detailed summary of the stimuli properties, broken down by trial
type and testing language. Lexical frequencies were extracted from the
Catalan and Spanish corpora of the CHILDES database
(\protect\hyperlink{ref-reference}{\textbf{reference?}};
\protect\hyperlink{ref-reference}{\textbf{reference?}}) as counts per
million words, and transformed into Zipf scores for easier
cross-language comparison
(\protect\hyperlink{ref-reference}{\textbf{reference?}};
\protect\hyperlink{ref-reference}{\textbf{reference?}}). We defined word
prevalence as the proportion of same-aged infants who were reported to
understand the word in the BVQ database.

\hypertarget{auditory-stimuli}{%
\paragraph{Auditory stimuli}\label{auditory-stimuli}}

The auditory stimuli were natural exemplars of the selected target
words, spoken by Catalan-Spanish proficient bilingual female speaker who
was instructed to pronounce each word in a toddler-directed manner.
Recordings were made with an Audio-Tecnica 328 microphone (AT2050) at a
sampling rate of 44100 Hz, in a soundproof room at the \emph{Laboratori
de Recerca en Infancia} at University Pompeu Fabra. We used the Audacity
(\protect\hyperlink{ref-reference}{\textbf{reference?}}) and Praat
(\protect\hyperlink{ref-reference}{\textbf{reference?}}) to record and
edit the audio files. The speaker was presented with a list of words in
Catalan. The order of the words was pseudo-randomised, and each word was
produced three times in a row before moving to the next word in the
list. After going through all the words in the list, the speaker went
through the word list again generating three tokens for each word, now
in an inverse order (from bottom of the list to the top). We then
repeated the same procedure for the list of Spanish words. The resulting
audios were manually chunked into individual word-forms. For each of the
six tokens produced for each word, the most adequate was selected for
further processing. The audios were then transformed to stereo by
duplicating them into two channels, denoised, and finally normalised.
The mean duration of the final audios was 1.23 (\emph{SD} = 0.17) and
1.08 (\emph{SD} = 0.14) seconds for the Catalan and Spanish lists.

To make the pronunciation of the words as familiar as possible to each
infant, we generated additional pronunciation variants for some words in
Catalan and Spanish. Catalan words involving the /\textipa{L}/ phoneme
in their Central Catalan variant (e.g., /\textipa{'Lu.n@}) were also
recorded with such phoneme replaced by /j/ (e.g., /\textipa{'ju.n@}), a
phonological process common in the Metropolitan Area of Barcelona
(\protect\hyperlink{ref-reference}{\textbf{reference?}}). Spanish words
involving the /\textipa{T}/ phoneme were also generated replacing such
phoneme with /\textipa{s}/ to better accommodate Latin variants of
Spanish. Before every experimental session, caregivers were asked to
utter three written words involving the /\textipa{L}/ phoneme (in the
case of participants tested in Catalan) or the /\textipa{T}/ phoneme (in
the case of participants tested in Spanish). Each token contained the
critical phoneme at onset, inter-vocalic position, and coda. The
experimenter assigned the participant to the Catalan or Spanish stimuli
list involving the closest variant to that of caregivers'.

\hypertarget{visual-stimuli}{%
\paragraph{Visual stimuli}\label{visual-stimuli}}

For each word, we created a picture with a typical referent.

\begin{Shaded}
\begin{Highlighting}[]
\NormalTok{stimuli }\SpecialCharTok{|\textgreater{}} 
    \FunctionTok{summarise}\NormalTok{(}\FunctionTok{across}\NormalTok{(}\FunctionTok{c}\NormalTok{(}\FunctionTok{matches}\NormalTok{(}\StringTok{"familiarity\_|freq\_|animate\_"}\NormalTok{), duration),}
\NormalTok{                     tibble}\SpecialCharTok{::}\FunctionTok{lst}\NormalTok{(mean, sd, min, max)),}
              \AttributeTok{.by =} \FunctionTok{c}\NormalTok{(test\_language, trial\_type)) }\SpecialCharTok{|\textgreater{}} 
    \FunctionTok{select}\NormalTok{(}\SpecialCharTok{{-}}\FunctionTok{matches}\NormalTok{(}\StringTok{"familiarity\_se"}\NormalTok{), }\SpecialCharTok{{-}}\FunctionTok{matches}\NormalTok{(}\StringTok{"prime"}\NormalTok{),}
           \SpecialCharTok{{-}}\FunctionTok{c}\NormalTok{(is\_animate\_target\_sd,}
\NormalTok{              is\_animate\_target\_min,}
\NormalTok{              is\_animate\_target\_max)) }\SpecialCharTok{|\textgreater{}} 
    \FunctionTok{gt}\NormalTok{(}\AttributeTok{groupname\_col =} \StringTok{"test\_language"}\NormalTok{,}
       \AttributeTok{rowname\_col =} \StringTok{"list"}\NormalTok{) }\SpecialCharTok{|\textgreater{}} 
    \FunctionTok{tab\_spanner}\NormalTok{(}\StringTok{"Prevalence (\%)"}\NormalTok{, }\FunctionTok{matches}\NormalTok{(}\StringTok{"familiarity"}\NormalTok{)) }\SpecialCharTok{|\textgreater{}} 
    \FunctionTok{tab\_spanner}\NormalTok{(}\StringTok{"Frequency (Zipf)"}\NormalTok{, }\FunctionTok{matches}\NormalTok{(}\StringTok{"freq"}\NormalTok{)) }\SpecialCharTok{|\textgreater{}}
    \FunctionTok{tab\_spanner}\NormalTok{(}\StringTok{"Animacy (\%)"}\NormalTok{, }\FunctionTok{matches}\NormalTok{(}\StringTok{"animate"}\NormalTok{)) }\SpecialCharTok{|\textgreater{}} 
    \FunctionTok{tab\_spanner}\NormalTok{(}\StringTok{"Duration (s)"}\NormalTok{, }\FunctionTok{matches}\NormalTok{(}\StringTok{"duration"}\NormalTok{)) }\SpecialCharTok{|\textgreater{}} 
    \FunctionTok{fmt\_number}\NormalTok{(is.numeric) }\SpecialCharTok{|\textgreater{}} 
    \FunctionTok{cols\_merge\_range}\NormalTok{(familiarity\_target\_min, familiarity\_target\_max) }\SpecialCharTok{|\textgreater{}} 
    \FunctionTok{cols\_merge\_range}\NormalTok{(freq\_target\_min, freq\_target\_max) }\SpecialCharTok{|\textgreater{}} 
    \FunctionTok{cols\_merge\_range}\NormalTok{(duration\_min, duration\_max) }\SpecialCharTok{|\textgreater{}}
    \FunctionTok{cols\_merge\_uncert}\NormalTok{(freq\_target\_mean, freq\_target\_sd) }\SpecialCharTok{|\textgreater{}} 
    \FunctionTok{cols\_merge\_uncert}\NormalTok{(familiarity\_target\_mean, familiarity\_target\_sd) }\SpecialCharTok{|\textgreater{}} 
    \FunctionTok{cols\_merge\_uncert}\NormalTok{(duration\_mean, duration\_sd) }\SpecialCharTok{|\textgreater{}} 
    \FunctionTok{cols\_label}\NormalTok{(}\AttributeTok{trial\_type =} \StringTok{""}\NormalTok{,}
               \AttributeTok{familiarity\_target\_mean =} \StringTok{"Mean ± SD"}\NormalTok{,}
               \CommentTok{\# familiarity\_target\_sd = "SD",}
               \AttributeTok{familiarity\_target\_min =} \StringTok{"Range"}\NormalTok{,}
               \AttributeTok{freq\_target\_mean =} \StringTok{"Mean ± SD"}\NormalTok{,}
               \CommentTok{\# freq\_target\_sd = "SD",}
               \AttributeTok{freq\_target\_min =} \StringTok{"Range"}\NormalTok{,}
               \AttributeTok{is\_animate\_target\_mean =} \StringTok{""}\NormalTok{,}
               \AttributeTok{duration\_mean =} \StringTok{"Mean ± SD"}\NormalTok{,}
               \CommentTok{\# duration\_sd = "SD",}
               \AttributeTok{duration\_min =} \StringTok{"Range"}\NormalTok{) }
\end{Highlighting}
\end{Shaded}

\hypertarget{tbl-stimuli}{}
\begin{longtable}{lrrrrrrr}
\caption{\label{tbl-stimuli}Summary of stimuli properties by trial type. }\tabularnewline

\toprule
 & \multicolumn{2}{c}{Prevalence (\%)} & \multicolumn{2}{c}{Frequency (Zipf)} & Animacy (\%) & \multicolumn{2}{c}{Duration (s)} \\ 
\cmidrule(lr){2-3} \cmidrule(lr){4-5} \cmidrule(lr){6-6} \cmidrule(lr){7-8}
 & Mean ± SD & Range & Mean ± SD & Range &  & Mean ± SD & Range \\ 
\midrule
\multicolumn{8}{l}{Catalan} \\ 
\midrule
Cognate & $0.37$ ± $0.18$ & $0.09$–$0.84$ & $3.90$ ± $0.61$ & $3.04$–$5.05$ & $0.21$ & $1.20$ ± $0.17$ & $0.86$–$1.55$ \\ 
Non-cognate & $0.38$ ± $0.16$ & $0.09$–$0.84$ & $3.90$ ± $0.52$ & $3.04$–$5.05$ & $0.25$ & $1.25$ ± $0.17$ & $0.88$–$1.55$ \\ 
Unrelated & $0.36$ ± $0.15$ & $0.09$–$0.84$ & $3.86$ ± $0.58$ & $2.94$–$5.05$ & $0.22$ & $1.23$ ± $0.17$ & $0.86$–$1.55$ \\ 
\midrule
\multicolumn{8}{l}{Spanish} \\ 
\midrule
Cognate & $0.29$ ± $0.13$ & $0.04$–$0.50$ & $4.21$ ± $0.51$ & $2.94$–$5.11$ & $0.10$ & $1.11$ ± $0.14$ & $0.85$–$1.45$ \\ 
Non-cognate & $0.30$ ± $0.14$ & $0.04$–$0.50$ & $4.27$ ± $0.53$ & $2.94$–$5.11$ & $0.15$ & $1.09$ ± $0.14$ & $0.83$–$1.45$ \\ 
Unrelated & $0.28$ ± $0.14$ & $0.04$–$0.50$ & $4.23$ ± $0.46$ & $2.94$–$5.11$ & $0.17$ & $1.07$ ± $0.15$ & $0.83$–$1.54$ \\ 
\bottomrule
\end{longtable}

\hypertarget{procedure}{%
\subsection{Procedure}\label{procedure}}

Testing took place in a sound-proof room. Participants sat on their
caregivers' lap in a dimly lit testing booth while the experimenter
conducted the experiment from outside. Caregivers were instructed to
keep their eyes shut (to avoid recording their gaze, instead of the
participant's), to be still, and to avoid interacting with the
participant verbally or non-verbally. Participants sat at approximately
65 cm from the eye-tracker and a XX-in screen of \(1929\times1080\)
screen resolution. We used a custom Matlab XXXX script using the
PsychToolbox XXX extension
(\protect\hyperlink{ref-brainard}{\textbf{brainard?}}) to present the
stimuli, and the Tobii Analytics SDK 3.0 to interact with the
eye-tracking while the experiment was running. Sampling rate was set at
120 Hz. A 5-point calibration was performed before every experimental
session, in which the picture of a colourful beach ball was presented.
We set a 55\% grey background for the calibration and stimuli
presentation. Auditory stimuli were presented through two loudspeakers
located behind the screen, one to each side. The experimenter monitored
the experimental from outside the room using a centrally located video
camera place above the screen. After a successful calibration the
experimenter triggered the onset of the first trial. Trials were
presented uninterruptedly and without intervention of the experimenter
until the 32 trials were presented, or the experimental session had to
be stopped because of the participant's behaviour.

Each trial started with the presentation of an attention getter for
3,000 milliseconds. Then, the prime picture was presented in silence in
the centre of the screen for 1,500 milliseconds. Fifty milliseconds
after the offset of the prime image, an auditory label was played from
the loudspeakers and, 700 milliseconds after the onset of the auditory
label, the target and distractor pictures were presented side-by-side
during 1,000 milliseconds until the end of the trial. After this, the
attention getter of the next trial was immediately presented. Each
experimental session took approximately 10 minutes.

\hypertarget{data-analysis}{%
\subsection{Data analysis}\label{data-analysis}}

\hypertarget{data-processing}{%
\subsubsection{Data processing}\label{data-processing}}

\begin{Shaded}
\begin{Highlighting}[]
\NormalTok{n\_trials }\OtherTok{\textless{}{-}} \FunctionTok{nrow}\NormalTok{(attrition\_trials)}

\NormalTok{n\_sessions }\OtherTok{\textless{}{-}} \FunctionTok{distinct}\NormalTok{(attrition\_trials, filename) }\SpecialCharTok{|\textgreater{}} 
    \FunctionTok{nrow}\NormalTok{()}

\NormalTok{n\_participants }\OtherTok{\textless{}{-}} \FunctionTok{distinct}\NormalTok{(attrition\_trials, id) }\SpecialCharTok{|\textgreater{}} 
    \FunctionTok{nrow}\NormalTok{()}

\NormalTok{n\_trials\_prime }\OtherTok{\textless{}{-}}\NormalTok{ attrition\_trials }\SpecialCharTok{|\textgreater{}} 
    \FunctionTok{filter}\NormalTok{(is\_valid\_gaze\_prime)}

\NormalTok{n\_trials\_test }\OtherTok{\textless{}{-}}\NormalTok{ n\_trials\_prime }\SpecialCharTok{|\textgreater{}} 
    \FunctionTok{filter}\NormalTok{(is\_valid\_gaze\_test)}

\NormalTok{n\_trials\_each }\OtherTok{\textless{}{-}}\NormalTok{ n\_trials\_test }\SpecialCharTok{|\textgreater{}} 
    \FunctionTok{filter}\NormalTok{(is\_valid\_gaze\_test\_each)}

\NormalTok{n\_trials\_vocab }\OtherTok{\textless{}{-}}\NormalTok{ n\_trials\_each }\SpecialCharTok{|\textgreater{}} 
    \FunctionTok{filter}\NormalTok{(is\_valid\_vocab)}

\NormalTok{n\_trials\_participant }\OtherTok{\textless{}{-}}\NormalTok{ n\_trials\_vocab }\SpecialCharTok{|\textgreater{}} 
    \FunctionTok{inner\_join}\NormalTok{(}\FunctionTok{filter}\NormalTok{(attrition\_participants,}
\NormalTok{                      is\_valid\_participant),}
               \AttributeTok{by =} \FunctionTok{join\_by}\NormalTok{(filename))}

\NormalTok{n\_sessions\_valid }\OtherTok{\textless{}{-}}\NormalTok{ attrition\_participants }\SpecialCharTok{|\textgreater{}} 
    \FunctionTok{filter}\NormalTok{(is\_valid\_participant)}

\NormalTok{n\_participants\_valid }\OtherTok{\textless{}{-}}\NormalTok{ n\_trials\_participant }\SpecialCharTok{|\textgreater{}} 
    \FunctionTok{distinct}\NormalTok{(filename)}

\NormalTok{n\_longitudinal }\OtherTok{\textless{}{-}} \FunctionTok{count}\NormalTok{(n\_sessions\_valid, filename, }\AttributeTok{name =} \StringTok{"sessions"}\NormalTok{) }\SpecialCharTok{|\textgreater{}} 
    \FunctionTok{count}\NormalTok{(sessions)}

\NormalTok{n\_participants\_include }\OtherTok{\textless{}{-}} \FunctionTok{read\_csv}\NormalTok{(}\StringTok{"data{-}raw/participants.csv"}\NormalTok{,}
                                   \AttributeTok{show\_col\_types =} \ConstantTok{FALSE}\NormalTok{)}
\end{Highlighting}
\end{Shaded}

We gathered data for 9,472 trials from 296 experimental sessions,
provided by 180 distinct participants. Missing eye-tracker samples were
interpolated using the last-observation-carried-forward (see
\protect\hyperlink{ref-zettersten2022peekbank}{\textbf{zettersten2022peekbank?}}
for a similar approach). We excluded trials in which the participant did
not fixate the prime at least half of the time of the prime phase
(\$\geq\(750 ms, *n* = 1,810), or the target and distractor during the test phase (\)\geq\$1,000
ms, \emph{n} = 502). We also excluded trials in which the participant
failed to fixate both target \emph{and} distractor pictures for at least
100 ms (i.e., 5\% of the time) (1,304 trials excluded). We then excluded
trials in which the participant was not reported to understand the prime
\emph{and} the target words, according to their caregivers' responses to
the vocabulary checklist. This resulted in the exclusion of 3,641
trials. Finally, we excluded trials from experimental sessions in which
the participant did not provide at least two valid trials in each of the
three experimental conditions (cognate prime, non-cognate prime, and
unrelated prime) which resulted in the exclusion of 34 experimental
sessions and their 219 remaining trials. The final data set comprised 0
from c(``CognatePriming\_019\_2019-11-28.csv'',
``CognatePriming\_021\_2019-11-30.csv'',
``CognatePriming\_026\_2019-12-14.csv'',
``CognatePriming\_027\_2019-12-14\_1.csv'',
``CognatePriming\_049\_2021-02-13.csv'',
``CognatePriming\_050\_2020-11-07.csv'',
``CognatePriming\_050\_2021-03-13.csv'',
``CognatePriming\_050\_2021-10-02.csv'',
``CognatePriming\_051\_2021-03-06\_1.csv'',
``CognatePriming\_051\_2021-10-01.csv'',
``CognatePriming\_053\_2021-04-17.csv'',
``CognatePriming\_054\_2020-11-14.csv'',
``CognatePriming\_055\_2020-11-14.csv'',
``CognatePriming\_055\_2021-02-20\_1.csv'',
``CognatePriming\_056\_2020-11-14.csv'',
``CognatePriming\_058\_2020-11-17.csv'',
``CognatePriming\_059\_2020-11-18.csv'',
``CognatePriming\_060\_2021-04-26.csv'',
``CognatePriming\_064\_2020-11-21.csv'',
``CognatePriming\_065\_2020-11-25.csv'',
``CognatePriming\_066\_2020-11-26.csv'',
``CognatePriming\_067\_2020-11-27.csv'',
``CognatePriming\_068\_2020-11-27.csv'',
``CognatePriming\_069\_2020-11-28.csv'',
``CognatePriming\_069\_2021-05-08.csv'',
``CognatePriming\_070\_2020-12-01.csv'',
``CognatePriming\_070\_2021-09-30\_1.csv'',
``CognatePriming\_071\_2020-12-01.csv'',
``CognatePriming\_072\_2020-12-04.csv'',
``CognatePriming\_073\_2021-10-02.csv'',
``CognatePriming\_074\_2020-12-05.csv'',
``CognatePriming\_075\_2020-12-09.csv'',
``CognatePriming\_076\_2021-05-08.csv'',
``CognatePriming\_076\_2021-10-16.csv'',
``CognatePriming\_078\_2021-04-29.csv'',
``CognatePriming\_079\_2020-12-16.csv'',
``CognatePriming\_079\_2021-10-07\_1.csv'',
``CognatePriming\_080\_2021-04-10.csv'',
``CognatePriming\_081\_2020-12-21.csv'',
``CognatePriming\_081\_2021-05-03.csv'',
``CognatePriming\_081\_2021-10-19.csv'',
``CognatePriming\_084\_2021-10-15.csv'',
``CognatePriming\_089\_2021-02-02.csv'',
``CognatePriming\_089\_2021-07-06.csv'',
``CognatePriming\_090\_2021-02-06\_2.csv'',
``CognatePriming\_090\_2021-06-19.csv'',
``CognatePriming\_091\_2021-06-18.csv'',
``CognatePriming\_092\_2021-06-08.csv'',
``CognatePriming\_094\_2021-02-13.csv'',
``CognatePriming\_095\_2021-02-13\_1.csv'',
``CognatePriming\_098\_2021-11-13.csv'',
``CognatePriming\_100\_2021-02-20\_1.csv'',
``CognatePriming\_100\_2021-06-14.csv'',
``CognatePriming\_101\_2021-02-20\_1.csv'',
``CognatePriming\_101\_2021-11-20.csv'',
``CognatePriming\_102\_2021-02-26.csv'',
``CognatePriming\_102\_2021-07-05.csv'',
``CognatePriming\_102\_2021-11-12.csv'',
``CognatePriming\_105\_2021-02-27\_1.csv'',
``CognatePriming\_105\_2021-07-24\_1.csv'',
``CognatePriming\_105\_2021-11-27.csv'',
``CognatePriming\_107\_2022-02-21.csv'',
``CognatePriming\_108\_2021-03-15.csv'',
``CognatePriming\_108\_2022-01-26\_1.csv'',
``CognatePriming\_110\_2022-01-22\_1.csv'',
``CognatePriming\_112\_2022-02-18.csv'',
``CognatePriming\_113\_2021-05-15.csv'',
``CognatePriming\_114\_2021-05-19.csv'',
``CognatePriming\_115\_2021-05-22.csv'',
``CognatePriming\_115\_2021-10-23.csv'',
``CognatePriming\_115\_2022-01-29.csv'',
``CognatePriming\_117\_2021-11-16.csv'',
``CognatePriming\_117\_2022-03-11\_1.csv'',
``CognatePriming\_119\_2021-06-01.csv'',
``CognatePriming\_119\_2022-01-26.csv'',
``CognatePriming\_121\_2021-06-08.csv'',
``CognatePriming\_121\_2021-10-18.csv'',
``CognatePriming\_123\_2021-10-23.csv'',
``CognatePriming\_124\_2022-02-18.csv'',
``CognatePriming\_125\_2021-06-26.csv'',
``CognatePriming\_127\_2021-06-28.csv'',
``CognatePriming\_127\_2021-10-25.csv'',
``CognatePriming\_128\_2021-11-02.csv'',
``CognatePriming\_131\_2022-02-18.csv'',
``CognatePriming\_133\_2021-07-06.csv'',
``CognatePriming\_135\_2021-12-01.csv'',
``CognatePriming\_136\_2021-12-03.csv'',
``CognatePriming\_140\_2021-12-11\_2.csv'',
``CognatePriming\_142\_2022-03-31.csv'',
``CognatePriming\_144\_2022-01-07.csv'',
``CognatePriming\_146\_2022-01-27.csv'',
``CognatePriming\_146\_2022-06-14.csv'',
``CognatePriming\_147\_2022-01-28.csv'',
``CognatePriming\_147\_2022-06-17.csv'',
``CognatePriming\_152\_2022-02-05.csv'',
``CognatePriming\_154\_2022-02-08\_1.csv'',
``CognatePriming\_154\_2022-05-24\_1.csv'',
``CognatePriming\_157\_2022-02-10.csv'',
``CognatePriming\_157\_2022-10-03.csv'',
``CognatePriming\_160\_2022-02-28.csv'',
``CognatePriming\_171\_2022-03-14.csv'',
``CognatePriming\_172\_2022-03-15.csv'',
``CognatePriming\_176\_2022-03-22.csv'',
``CognatePriming\_178\_2022-03-23\_1.csv'',
``CognatePriming\_183\_2022-10-06.csv'',
``CognatePriming\_190\_2022-06-20.csv'',
``CognatePriming\_192\_2022-06-21.csv'',
``CognatePriming\_196\_2022-06-28\_2.csv'',
``CognatePriming\_198\_2022-10-01.csv'',
``CognatePriming\_199\_2022-10-03.csv'',
``CognatePriming\_200\_2022-10-04.csv'',
``CognatePriming\_201\_2022-10-07.csv'',
``CognatePriming\_202\_2022-10-08.csv''), c(8, 5, 8, 7, 7, 6, 7, 7, 3,
5, 4, 5, 3, 3, 2, 4, 6, 8, 4, 4, 2, 5, 7, 4, 6, 3, 6, 3, 2, 3, 6, 4, 4,
6, 4, 3, 5, 2, 2, 5, 6, 6, 6, 5, 5, 5, 4, 2, 5, 2, 4, 6, 3, 3, 5, 6, 6,
6, 3, 5, 6, 5, 2, 5, 5, 6, 8, 2, 5, 5, 7, 5, 7, 7, 6, 5, 4, 4, 4, 5, 2,
4, 4, 5, 7, 5, 4, 5, 2, 7, 6, 3, 2, 7, 3, 8, 6, 2, 5, 3, 6, 3, 5, 4, 7,
6, 7, 4, 7, 4, 7, 3, 2), c(7, 7, 7, 8, 7, 5, 8, 7, 6, 5, 5, 6, 2, 2, 2,
4, 8, 7, 3, 4, 3, 2, 4, 2, 3, 2, 7, 5, 6, 3, 4, 5, 4, 5, 6, 2, 5, 6, 4,
6, 3, 3, 6, 4, 6, 5, 4, 2, 3, 3, 4, 3, 5, 4, 6, 4, 6, 6, 4, 7, 8, 6, 2,
4, 4, 7, 4, 3, 5, 4, 6, 7, 7, 6, 6, 7, 6, 4, 6, 3, 4, 5, 4, 6, 4, 5, 5,
3, 4, 6, 6, 2, 4, 4, 5, 4, 5, 3, 7, 4, 4, 3, 6, 2, 6, 2, 5, 2, 6, 4, 6,
3, 2), c(9, 8, 9, 10, 10, 6, 11, 14, 8, 9, 9, 9, 3, 4, 5, 10, 15, 12, 3,
5, 5, 6, 6, 6, 6, 9, 14, 5, 10, 2, 10, 10, 7, 13, 9, 5, 8, 10, 2, 5, 7,
10, 11, 10, 12, 10, 7, 5, 11, 4, 7, 8, 6, 4, 11, 9, 12, 14, 10, 13, 13,
11, 3, 5, 6, 14, 11, 2, 6, 11, 12, 8, 14, 7, 10, 6, 6, 11, 8, 7, 3, 5,
6, 6, 7, 10, 8, 7, 5, 12, 8, 14, 8, 8, 7, 7, 6, 8, 10, 9, 7, 6, 10, 7,
10, 4, 9, 7, 12, 7, 13, 11, 3), c(TRUE, TRUE, TRUE, TRUE, TRUE, TRUE,
TRUE, TRUE, TRUE, TRUE, TRUE, TRUE, TRUE, TRUE, TRUE, TRUE, TRUE, TRUE,
TRUE, TRUE, TRUE, TRUE, TRUE, TRUE, TRUE, TRUE, TRUE, TRUE, TRUE, TRUE,
TRUE, TRUE, TRUE, TRUE, TRUE, TRUE, TRUE, TRUE, TRUE, TRUE, TRUE, TRUE,
TRUE, TRUE, TRUE, TRUE, TRUE, TRUE, TRUE, TRUE, TRUE, TRUE, TRUE, TRUE,
TRUE, TRUE, TRUE, TRUE, TRUE, TRUE, TRUE, TRUE, TRUE, TRUE, TRUE, TRUE,
TRUE, TRUE, TRUE, TRUE, TRUE, TRUE, TRUE, TRUE, TRUE, TRUE, TRUE, TRUE,
TRUE, TRUE, TRUE, TRUE, TRUE, TRUE, TRUE, TRUE, TRUE, TRUE, TRUE, TRUE,
TRUE, TRUE, TRUE, TRUE, TRUE, TRUE, TRUE, TRUE, TRUE, TRUE, TRUE, TRUE,
TRUE, TRUE, TRUE, TRUE, TRUE, TRUE, TRUE, TRUE, TRUE, TRUE, TRUE),
c(TRUE, TRUE, TRUE, TRUE, TRUE, TRUE, TRUE, TRUE, TRUE, TRUE, TRUE,
TRUE, TRUE, TRUE, TRUE, TRUE, TRUE, TRUE, TRUE, TRUE, TRUE, TRUE, TRUE,
TRUE, TRUE, TRUE, TRUE, TRUE, TRUE, TRUE, TRUE, TRUE, TRUE, TRUE, TRUE,
TRUE, TRUE, TRUE, TRUE, TRUE, TRUE, TRUE, TRUE, TRUE, TRUE, TRUE, TRUE,
TRUE, TRUE, TRUE, TRUE, TRUE, TRUE, TRUE, TRUE, TRUE, TRUE, TRUE, TRUE,
TRUE, TRUE, TRUE, TRUE, TRUE, TRUE, TRUE, TRUE, TRUE, TRUE, TRUE, TRUE,
TRUE, TRUE, TRUE, TRUE, TRUE, TRUE, TRUE, TRUE, TRUE, TRUE, TRUE, TRUE,
TRUE, TRUE, TRUE, TRUE, TRUE, TRUE, TRUE, TRUE, TRUE, TRUE, TRUE, TRUE,
TRUE, TRUE, TRUE, TRUE, TRUE, TRUE, TRUE, TRUE, TRUE, TRUE, TRUE, TRUE,
TRUE, TRUE, TRUE, TRUE, TRUE, TRUE), c(TRUE, TRUE, TRUE, TRUE, TRUE,
TRUE, TRUE, TRUE, TRUE, TRUE, TRUE, TRUE, TRUE, TRUE, TRUE, TRUE, TRUE,
TRUE, TRUE, TRUE, TRUE, TRUE, TRUE, TRUE, TRUE, TRUE, TRUE, TRUE, TRUE,
TRUE, TRUE, TRUE, TRUE, TRUE, TRUE, TRUE, TRUE, TRUE, TRUE, TRUE, TRUE,
TRUE, TRUE, TRUE, TRUE, TRUE, TRUE, TRUE, TRUE, TRUE, TRUE, TRUE, TRUE,
TRUE, TRUE, TRUE, TRUE, TRUE, TRUE, TRUE, TRUE, TRUE, TRUE, TRUE, TRUE,
TRUE, TRUE, TRUE, TRUE, TRUE, TRUE, TRUE, TRUE, TRUE, TRUE, TRUE, TRUE,
TRUE, TRUE, TRUE, TRUE, TRUE, TRUE, TRUE, TRUE, TRUE, TRUE, TRUE, TRUE,
TRUE, TRUE, TRUE, TRUE, TRUE, TRUE, TRUE, TRUE, TRUE, TRUE, TRUE, TRUE,
TRUE, TRUE, TRUE, TRUE, TRUE, TRUE, TRUE, TRUE, TRUE, TRUE, TRUE, TRUE),
c(TRUE, TRUE, TRUE, TRUE, TRUE, TRUE, TRUE, TRUE, TRUE, TRUE, TRUE,
TRUE, TRUE, TRUE, TRUE, TRUE, TRUE, TRUE, TRUE, TRUE, TRUE, TRUE, TRUE,
TRUE, TRUE, TRUE, TRUE, TRUE, TRUE, TRUE, TRUE, TRUE, TRUE, TRUE, TRUE,
TRUE, TRUE, TRUE, TRUE, TRUE, TRUE, TRUE, TRUE, TRUE, TRUE, TRUE, TRUE,
TRUE, TRUE, TRUE, TRUE, TRUE, TRUE, TRUE, TRUE, TRUE, TRUE, TRUE, TRUE,
TRUE, TRUE, TRUE, TRUE, TRUE, TRUE, TRUE, TRUE, TRUE, TRUE, TRUE, TRUE,
TRUE, TRUE, TRUE, TRUE, TRUE, TRUE, TRUE, TRUE, TRUE, TRUE, TRUE, TRUE,
TRUE, TRUE, TRUE, TRUE, TRUE, TRUE, TRUE, TRUE, TRUE, TRUE, TRUE, TRUE,
TRUE, TRUE, TRUE, TRUE, TRUE, TRUE, TRUE, TRUE, TRUE, TRUE, TRUE, TRUE,
TRUE, TRUE, TRUE, TRUE, TRUE, TRUE), c(19, 21, 26, 27, 49, 50, 50, 50,
51, 51, 53, 54, 55, 55, 56, 58, 59, 60, 64, 65, 66, 67, 68, 69, 69, 70,
70, 71, 72, 73, 74, 75, 76, 76, 78, 79, 79, 80, 81, 81, 81, 84, 89, 89,
90, 90, 91, 92, 94, 95, 98, 100, 100, 101, 101, 102, 102, 102, 105, 105,
105, 107, 108, 108, 110, 112, 113, 114, 115, 115, 115, 117, 117, 119,
119, 121, 121, 123, 124, 125, 127, 127, 128, 131, 133, 135, 136, 140,
142, 144, 146, 146, 147, 147, 152, 154, 154, 157, 157, 160, 171, 172,
176, 178, 183, 190, 192, 196, 198, 199, 200, 201, 202), c(3, 3, 3, 3, 2,
1, 2, 3, 2, 3, 3, 1, 1, 2, 1, 2, 2, 3, 2, 2, 2, 2, 2, 2, 3, 1, 3, 1, 1,
3, 2, 2, 2, 3, 2, 1, 3, 3, 1, 2, 3, 3, 1, 2, 1, 2, 2, 2, 1, 1, 3, 1, 2,
1, 3, 1, 2, 3, 1, 2, 3, 3, 1, 3, 3, 3, 1, 1, 1, 2, 3, 2, 3, 1, 3, 1, 2,
2, 3, 1, 1, 2, 2, 3, 1, 2, 2, 2, 2, 2, 2, 3, 2, 3, 1, 2, 3, 2, 3, 1, 2,
2, 2, 2, 3, 2, 3, 1, 3, 3, 3, 2, 3), c(2, 2, 2, 1, 1, 1, 1, 1, 1, 1, 1,
1, 2, 2, 1, 1, 1, 2, 1, 1, 2, 2, 2, 1, 1, 1, 1, 2, 1, 1, 1, 1, 2, 2, 1,
1, 1, 2, 1, 1, 1, 1, 1, 1, 2, 2, 1, 1, 1, 1, 2, 2, 2, 1, 1, 1, 1, 1, 2,
2, 2, 1, 2, 2, 2, 1, 1, 1, 1, 1, 1, 2, 2, 1, 1, 1, 1, 2, 2, 1, 1, 1, 2,
1, 1, 1, 2, 1, 2, 2, 1, 1, 1, 1, 2, 1, 1, 1, 1, 2, 1, 1, 1, 2, 1, 2, 2,
2, 1, 1, 2, 1, 1) experimental sessions, provided by
c(``CognatePriming\_019\_2019-11-28.csv'',
``CognatePriming\_021\_2019-11-30.csv'',
``CognatePriming\_026\_2019-12-14.csv'',
``CognatePriming\_027\_2019-12-14\_1.csv'',
``CognatePriming\_049\_2021-02-13.csv'',
``CognatePriming\_050\_2020-11-07.csv'',
``CognatePriming\_050\_2021-03-13.csv'',
``CognatePriming\_050\_2021-10-02.csv'',
``CognatePriming\_051\_2021-03-06\_1.csv'',
``CognatePriming\_051\_2021-10-01.csv'',
``CognatePriming\_053\_2021-04-17.csv'',
``CognatePriming\_054\_2020-11-14.csv'',
``CognatePriming\_055\_2020-11-14.csv'',
``CognatePriming\_055\_2021-02-20\_1.csv'',
``CognatePriming\_056\_2020-11-14.csv'',
``CognatePriming\_058\_2020-11-17.csv'',
``CognatePriming\_059\_2020-11-18.csv'',
``CognatePriming\_060\_2021-04-26.csv'',
``CognatePriming\_064\_2020-11-21.csv'',
``CognatePriming\_065\_2020-11-25.csv'',
``CognatePriming\_066\_2020-11-26.csv'',
``CognatePriming\_067\_2020-11-27.csv'',
``CognatePriming\_068\_2020-11-27.csv'',
``CognatePriming\_069\_2020-11-28.csv'',
``CognatePriming\_069\_2021-05-08.csv'',
``CognatePriming\_070\_2020-12-01.csv'',
``CognatePriming\_070\_2021-09-30\_1.csv'',
``CognatePriming\_071\_2020-12-01.csv'',
``CognatePriming\_072\_2020-12-04.csv'',
``CognatePriming\_073\_2021-10-02.csv'',
``CognatePriming\_074\_2020-12-05.csv'',
``CognatePriming\_075\_2020-12-09.csv'',
``CognatePriming\_076\_2021-05-08.csv'',
``CognatePriming\_076\_2021-10-16.csv'',
``CognatePriming\_078\_2021-04-29.csv'',
``CognatePriming\_079\_2020-12-16.csv'',
``CognatePriming\_079\_2021-10-07\_1.csv'',
``CognatePriming\_080\_2021-04-10.csv'',
``CognatePriming\_081\_2020-12-21.csv'',
``CognatePriming\_081\_2021-05-03.csv'',
``CognatePriming\_081\_2021-10-19.csv'',
``CognatePriming\_084\_2021-10-15.csv'',
``CognatePriming\_089\_2021-02-02.csv'',
``CognatePriming\_089\_2021-07-06.csv'',
``CognatePriming\_090\_2021-02-06\_2.csv'',
``CognatePriming\_090\_2021-06-19.csv'',
``CognatePriming\_091\_2021-06-18.csv'',
``CognatePriming\_092\_2021-06-08.csv'',
``CognatePriming\_094\_2021-02-13.csv'',
``CognatePriming\_095\_2021-02-13\_1.csv'',
``CognatePriming\_098\_2021-11-13.csv'',
``CognatePriming\_100\_2021-02-20\_1.csv'',
``CognatePriming\_100\_2021-06-14.csv'',
``CognatePriming\_101\_2021-02-20\_1.csv'',
``CognatePriming\_101\_2021-11-20.csv'',
``CognatePriming\_102\_2021-02-26.csv'',
``CognatePriming\_102\_2021-07-05.csv'',
``CognatePriming\_102\_2021-11-12.csv'',
``CognatePriming\_105\_2021-02-27\_1.csv'',
``CognatePriming\_105\_2021-07-24\_1.csv'',
``CognatePriming\_105\_2021-11-27.csv'',
``CognatePriming\_107\_2022-02-21.csv'',
``CognatePriming\_108\_2021-03-15.csv'',
``CognatePriming\_108\_2022-01-26\_1.csv'',
``CognatePriming\_110\_2022-01-22\_1.csv'',
``CognatePriming\_112\_2022-02-18.csv'',
``CognatePriming\_113\_2021-05-15.csv'',
``CognatePriming\_114\_2021-05-19.csv'',
``CognatePriming\_115\_2021-05-22.csv'',
``CognatePriming\_115\_2021-10-23.csv'',
``CognatePriming\_115\_2022-01-29.csv'',
``CognatePriming\_117\_2021-11-16.csv'',
``CognatePriming\_117\_2022-03-11\_1.csv'',
``CognatePriming\_119\_2021-06-01.csv'',
``CognatePriming\_119\_2022-01-26.csv'',
``CognatePriming\_121\_2021-06-08.csv'',
``CognatePriming\_121\_2021-10-18.csv'',
``CognatePriming\_123\_2021-10-23.csv'',
``CognatePriming\_124\_2022-02-18.csv'',
``CognatePriming\_125\_2021-06-26.csv'',
``CognatePriming\_127\_2021-06-28.csv'',
``CognatePriming\_127\_2021-10-25.csv'',
``CognatePriming\_128\_2021-11-02.csv'',
``CognatePriming\_131\_2022-02-18.csv'',
``CognatePriming\_133\_2021-07-06.csv'',
``CognatePriming\_135\_2021-12-01.csv'',
``CognatePriming\_136\_2021-12-03.csv'',
``CognatePriming\_140\_2021-12-11\_2.csv'',
``CognatePriming\_142\_2022-03-31.csv'',
``CognatePriming\_144\_2022-01-07.csv'',
``CognatePriming\_146\_2022-01-27.csv'',
``CognatePriming\_146\_2022-06-14.csv'',
``CognatePriming\_147\_2022-01-28.csv'',
``CognatePriming\_147\_2022-06-17.csv'',
``CognatePriming\_152\_2022-02-05.csv'',
``CognatePriming\_154\_2022-02-08\_1.csv'',
``CognatePriming\_154\_2022-05-24\_1.csv'',
``CognatePriming\_157\_2022-02-10.csv'',
``CognatePriming\_157\_2022-10-03.csv'',
``CognatePriming\_160\_2022-02-28.csv'',
``CognatePriming\_171\_2022-03-14.csv'',
``CognatePriming\_172\_2022-03-15.csv'',
``CognatePriming\_176\_2022-03-22.csv'',
``CognatePriming\_178\_2022-03-23\_1.csv'',
``CognatePriming\_183\_2022-10-06.csv'',
``CognatePriming\_190\_2022-06-20.csv'',
``CognatePriming\_192\_2022-06-21.csv'',
``CognatePriming\_196\_2022-06-28\_2.csv'',
``CognatePriming\_198\_2022-10-01.csv'',
``CognatePriming\_199\_2022-10-03.csv'',
``CognatePriming\_200\_2022-10-04.csv'',
``CognatePriming\_201\_2022-10-07.csv'',
``CognatePriming\_202\_2022-10-08.csv'') distinct participants. Of those
participants, 1 provided data from one experimental session, 113
provided data from two experimental sessions, and 1 provided data from
three experimental sessions.

We defined our time window of interest from 300 ms after the onset of
the test phase (target and distractor presentation) until the end of the
test phase (2,000 ms). For each trial, we chunked the time domain into
17 time bins of 100 ms of duration. We then calculated, for each
experimental session, time bin, and condition, participant's proportion
of target and distractor fixations. Finally, we computed the empirical
logit of target fixations, which we introduced in the statistical
analyses as our response variable.

\begin{Shaded}
\begin{Highlighting}[]
\NormalTok{attrition\_trials }\SpecialCharTok{|\textgreater{}} 
    \FunctionTok{left\_join}\NormalTok{(}\FunctionTok{select}\NormalTok{(participants, age\_group, lp, id, date\_test),}
              \AttributeTok{by =} \FunctionTok{join\_by}\NormalTok{(id, age\_group)) }\SpecialCharTok{|\textgreater{}} 
    \FunctionTok{left\_join}\NormalTok{(attrition\_participants,}
              \AttributeTok{by =} \FunctionTok{join\_by}\NormalTok{(id, age\_group)) }\SpecialCharTok{|\textgreater{}} 
    \FunctionTok{add\_count}\NormalTok{(age\_group, lp, }\AttributeTok{name =} \StringTok{"n\_trials"}\NormalTok{) }\SpecialCharTok{|\textgreater{}} 
    \FunctionTok{summarise}\NormalTok{(}\AttributeTok{valid\_trial\_count =} \FunctionTok{sum}\NormalTok{(is\_valid\_trial, }\AttributeTok{na.rm =} \ConstantTok{TRUE}\NormalTok{),}
              \AttributeTok{.by =} \FunctionTok{c}\NormalTok{(id, lp, trial\_type, date\_test,}
\NormalTok{                    age\_group, is\_valid\_participant, n\_trials)) }\SpecialCharTok{|\textgreater{}} 
    \FunctionTok{pivot\_wider}\NormalTok{(}\AttributeTok{names\_from =}\NormalTok{ trial\_type, }
                \AttributeTok{values\_from =}\NormalTok{ valid\_trial\_count, }
                \AttributeTok{values\_fill =} \DecValTok{0}\NormalTok{,}
                \AttributeTok{names\_repair =}\NormalTok{ janitor}\SpecialCharTok{::}\NormalTok{make\_clean\_names) }\SpecialCharTok{|\textgreater{}} 
    \FunctionTok{rename}\NormalTok{(}\AttributeTok{noncognate =}\NormalTok{ non\_cognate) }\SpecialCharTok{|\textgreater{}} 
    \FunctionTok{add\_count}\NormalTok{(lp, age\_group) }\SpecialCharTok{|\textgreater{}}
    \FunctionTok{summarise}\NormalTok{(}\FunctionTok{across}\NormalTok{(}\FunctionTok{c}\NormalTok{(cognate, noncognate, unrelated),}
                     \FunctionTok{lst}\NormalTok{(sum, mean, sd)),}
              \AttributeTok{n\_valid =} \FunctionTok{sum}\NormalTok{(is\_valid\_participant),}
              \AttributeTok{.by =} \FunctionTok{c}\NormalTok{(lp, age\_group, n, n\_trials)) }\SpecialCharTok{|\textgreater{}}
    \FunctionTok{relocate}\NormalTok{(n\_valid, }\AttributeTok{.after =}\NormalTok{ n) }\SpecialCharTok{|\textgreater{}} 
    \FunctionTok{mutate}\NormalTok{(}\AttributeTok{n\_prop =}\NormalTok{ n\_valid}\SpecialCharTok{/}\NormalTok{n) }\SpecialCharTok{|\textgreater{}} 
    \FunctionTok{arrange}\NormalTok{(age\_group, lp) }\SpecialCharTok{|\textgreater{}} 
    \FunctionTok{relocate}\NormalTok{(age\_group, lp, n\_valid, n\_prop, }
             \FunctionTok{matches}\NormalTok{(}\StringTok{"cognate"}\NormalTok{),}
             \FunctionTok{matches}\NormalTok{(}\StringTok{"noncognate"}\NormalTok{),}
             \FunctionTok{matches}\NormalTok{(}\StringTok{"unrelated"}\NormalTok{)) }\SpecialCharTok{|\textgreater{}} 
    \FunctionTok{gt}\NormalTok{(}\AttributeTok{groupname\_col =} \StringTok{"age\_group"}\NormalTok{,}
       \AttributeTok{rowname\_col =} \StringTok{"lp"}\NormalTok{)  }\SpecialCharTok{|\textgreater{}}
    \FunctionTok{cols\_hide}\NormalTok{(}\FunctionTok{c}\NormalTok{(n, n\_trials)) }\SpecialCharTok{|\textgreater{}} 
    \FunctionTok{fmt\_number}\NormalTok{(}\FunctionTok{matches}\NormalTok{(}\StringTok{"\_mean|\_sd"}\NormalTok{), }
               \AttributeTok{decimals =} \DecValTok{1}\NormalTok{) }\SpecialCharTok{|\textgreater{}}
    \FunctionTok{fmt\_percent}\NormalTok{(}\FunctionTok{matches}\NormalTok{(}\StringTok{"\_prop"}\NormalTok{),}
                \AttributeTok{decimals =} \DecValTok{1}\NormalTok{) }\SpecialCharTok{|\textgreater{}} 
    \FunctionTok{cols\_merge\_n\_pct}\NormalTok{(}\AttributeTok{col\_n =}\NormalTok{ n\_valid, }
                     \AttributeTok{col\_pct =}\NormalTok{ n\_prop,}
                     \AttributeTok{autohide =} \ConstantTok{TRUE}\NormalTok{)  }\SpecialCharTok{|\textgreater{}} 
    \FunctionTok{tab\_spanner}\NormalTok{(}\StringTok{"Non{-}cognate"}\NormalTok{, }\FunctionTok{matches}\NormalTok{(}\StringTok{"noncognate"}\NormalTok{)) }\SpecialCharTok{|\textgreater{}}
    \FunctionTok{tab\_spanner}\NormalTok{(}\StringTok{"Cognate"}\NormalTok{, }\FunctionTok{starts\_with}\NormalTok{(}\StringTok{"cognate"}\NormalTok{)) }\SpecialCharTok{|\textgreater{}}
    \FunctionTok{tab\_spanner}\NormalTok{(}\StringTok{"Unrelated"}\NormalTok{, }\FunctionTok{matches}\NormalTok{(}\StringTok{"unrelated"}\NormalTok{)) }\SpecialCharTok{|\textgreater{}}
    \FunctionTok{tab\_spanner}\NormalTok{(}\StringTok{"Related"}\NormalTok{, }\FunctionTok{matches}\NormalTok{(}\StringTok{"cognate"}\NormalTok{)) }\SpecialCharTok{|\textgreater{}} 
    \FunctionTok{cols\_merge\_uncert}\NormalTok{(}\AttributeTok{col\_val =}\NormalTok{ cognate\_mean,}
                      \AttributeTok{col\_uncert =}\NormalTok{ cognate\_sd) }\SpecialCharTok{|\textgreater{}} 
    \FunctionTok{cols\_merge\_uncert}\NormalTok{(}\AttributeTok{col\_val =}\NormalTok{ noncognate\_mean,}
                      \AttributeTok{col\_uncert =}\NormalTok{ noncognate\_sd) }\SpecialCharTok{|\textgreater{}} 
    \FunctionTok{cols\_merge\_uncert}\NormalTok{(}\AttributeTok{col\_val =}\NormalTok{ unrelated\_mean, }
                      \AttributeTok{col\_uncert =}\NormalTok{ unrelated\_sd) }\SpecialCharTok{|\textgreater{}} 
    \FunctionTok{cols\_label}\NormalTok{(}\AttributeTok{age\_group =} \StringTok{"Age"}\NormalTok{,}
               \AttributeTok{n\_valid =} \StringTok{"Participants (included)"}\NormalTok{,}
               \AttributeTok{cognate\_sum =} \StringTok{"Trials"}\NormalTok{,}
               \AttributeTok{cognate\_mean =} \StringTok{"Mean ± SD"}\NormalTok{,}
               \AttributeTok{noncognate\_sum =} \StringTok{"Trials"}\NormalTok{,}
               \AttributeTok{noncognate\_mean =} \StringTok{"Mean ± SD"}\NormalTok{,}
               \AttributeTok{unrelated\_sum =} \StringTok{"Trials"}\NormalTok{,}
               \AttributeTok{unrelated\_mean =} \StringTok{"Mean ± SD"}\NormalTok{) }\SpecialCharTok{|\textgreater{}} 
    \FunctionTok{summary\_rows}\NormalTok{(}\AttributeTok{groups =} \ConstantTok{TRUE}\NormalTok{,}
                 \AttributeTok{columns =} \FunctionTok{where}\NormalTok{(is.integer),}
                 \AttributeTok{fns =} \FunctionTok{list}\NormalTok{(}\StringTok{"Sum"} \OtherTok{=} \StringTok{"sum"}\NormalTok{),}
                 \AttributeTok{fmt =}\NormalTok{ fmt\_number,}
                 \AttributeTok{decimals =} \DecValTok{0}\NormalTok{) }\SpecialCharTok{|\textgreater{}}
    \FunctionTok{summary\_rows}\NormalTok{(}\AttributeTok{groups =} \ConstantTok{TRUE}\NormalTok{,}
                 \AttributeTok{columns =} \FunctionTok{c}\NormalTok{(}\FunctionTok{where}\NormalTok{(is.numeric), }\SpecialCharTok{{-}}\NormalTok{n\_valid),}
                 \AttributeTok{fns =} \FunctionTok{list}\NormalTok{(}\StringTok{"Mean"} \OtherTok{=} \StringTok{"mean"}\NormalTok{),}
                 \AttributeTok{fmt =}\NormalTok{ fmt\_number,}
                 \AttributeTok{decimals =} \DecValTok{1}\NormalTok{) }\SpecialCharTok{|\textgreater{}}
    \FunctionTok{grand\_summary\_rows}\NormalTok{(}\AttributeTok{columns =} \FunctionTok{where}\NormalTok{(is.integer),}
                       \AttributeTok{fns =} \FunctionTok{list}\NormalTok{(}\StringTok{"N"} \OtherTok{=} \StringTok{"sum"}\NormalTok{),}
                       \AttributeTok{formatter =}\NormalTok{ fmt\_integer) }
\end{Highlighting}
\end{Shaded}

\begin{verbatim}
Warning: Since gt v0.9.0, the `formatter` argument (and associated `...`) has been
deprecated.
* Please use the `fmt` argument to provide formatting directives.
This warning is displayed once every 8 hours.
\end{verbatim}

\hypertarget{tbl-attrition-group}{}
\begin{longtable}{l|rrrrrrr}
\caption{\label{tbl-attrition-group}Summary of the dataset. }\tabularnewline

\toprule
\multicolumn{1}{l}{} &  & \multicolumn{4}{c}{Related} &  &  \\ 
\cmidrule(lr){3-6}
\multicolumn{1}{l}{} &  & \multicolumn{2}{c}{Cognate} & \multicolumn{2}{c}{Non-cognate} & \multicolumn{2}{c}{Unrelated} \\ 
\cmidrule(lr){3-4} \cmidrule(lr){5-6} \cmidrule(lr){7-8}
\multicolumn{1}{l}{} & Participants (included) & Trials & Mean ± SD & Trials & Mean ± SD & Trials & Mean ± SD \\ 
\midrule
\multicolumn{8}{l}{21 months} \\ 
\midrule
Monolingual & 20 ($40.0\%$) & 103 & $2.1$ ± $2.4$ & 110 & $2.2$ ± $2.2$ & 180 & $3.6$ ± $3.6$ \\ 
Bilingual & 9 ($20.5\%$) & 42 & $1.0$ ± $1.5$ & 52 & $1.2$ ± $1.7$ & 91 & $2.1$ ± $3.2$ \\ 
\midrule
\multicolumn{8}{l}{25 months} \\ 
\midrule
Monolingual & 29 ($55.8\%$) & 142 & $2.7$ ± $2.5$ & 142 & $2.7$ ± $2.5$ & 254 & $4.9$ ± $4.5$ \\ 
Bilingual & 16 ($36.4\%$) & 71 & $1.6$ ± $2.3$ & 67 & $1.5$ ± $2.2$ & 123 & $2.8$ ± $3.9$ \\ 
\midrule
\multicolumn{8}{l}{30 months} \\ 
\midrule
Monolingual & 25 ($41.7\%$) & 140 & $2.3$ ± $2.8$ & 134 & $2.2$ ± $2.7$ & 255 & $4.2$ ± $5.1$ \\ 
Bilingual & 14 ($30.4\%$) & 86 & $1.9$ ± $2.9$ & 83 & $1.8$ ± $2.8$ & 140 & $3.0$ ± $4.8$ \\ 
\midrule 
\midrule 
sum & $113$ & $584$ & — & $588$ & — & $1,043$ & — \\ 
\bottomrule
\end{longtable}

\hypertarget{modelling-approach}{%
\subsubsection{Modelling approach}\label{modelling-approach}}

We conducted two main analyses. First, we estimated the effect of
phonological priming on participants' target looking, comparing
\emph{related} trials with \emph{unrelated} trials. This analysis
included all trials on the data set. Second, we estimated the effect of
cognateness on phonological priming, comparing \emph{cognate} with
\emph{non-cognate} trials, leaving out \emph{unrelated} trials. In both
analyses, we used General Additive Mixed Models (GAMMs) to model the
probability of target fixations across the time course of the trial
using a normal distribution.

In the first analysis. We included \emph{Relatedness} (\texttt{Related}
vs.~\texttt{Unrelated}, sum-coded as \texttt{-0.5} and \texttt{+0.5}),
\emph{Group} (\texttt{Monolingual} vs.~\texttt{Bilingual}, sum-coded as
\texttt{-0.5} and \texttt{+0.5}), and \emph{Age} (participants'
standardised age in months) as fixed, main effects. We also included
cubic regression splines for the main effect of \emph{Time}, and one for
an adjustment of the previous cubic spline by \emph{Group}
(\protect\hyperlink{ref-wood2017generalized}{Wood, 2017}). For both
splines, we specified \(k = 10\) basis functions or \emph{knots}--half
the number of time bins, for computational convenience. Finally, we
added by-participant random intercepts, and random slopes for the main
effect of \emph{Relatedness} and the main effect of \emph{Age}, both
including repeated measures per participant.

To test the contribution of each of the predictors of
interest--\emph{Relatedness}/\emph{Cognateness}, and \emph{Group}--, we
compared each model (\(\mathcal{M_0}\)) against a simplified model
dropping each of the main effects, \emph{Relatedness}/\emph{Cognateness}
(\(\mathcal{M}_1\)) or \emph{Group} (\(\mathcal{M_2}\)). In both
simplified models, the interaction term was dropped. We used
leave-one-out cross-validation (LOO-CV) as a benchmark of model
performance, using Pareto-smoothed importance sampling (PSIS) to
approximate it. We then examined the posterior predictions of the
best-performing model for interpretation.

\[
\begin{aligned}
\textbf{Likelihood:} \\
y_i &\sim \mathcal{N}(\mu_i, \sigma_i) \\ \\
\textbf{Linear model} \\
\text{logit}(\mu_i) &= (\beta_0 + u _{0_{i}}) + (\beta_1 + u _{1_{i}}) \cdot \text{Relatedness} + \beta_{2} \cdot \text{Group} + \\
&\beta_{3} \cdot (\text{Relatedness} \times \text{Group}) + (\beta_4 + u_{3_{i}}) \cdot \text{Age} + \\
&\sum_{j = 1}^k b_{j_{1}}(\beta_{5_{k}} + u_{4_{i}}) \cdot \text{Time} + \\
&\sum_{j = 1}^k b_{j_{1}} (\beta_{6_{k }} + u_{5_{i}}) \cdot (\text{Time} \times \text{Group}) \\
\text{where:} \\
&k \text{ is the number of knots in the spline (10)} \\
\textbf{Prior:} \\
\beta_{0-6} &\sim \mathcal{N}(0, 1) \\
b_{0-1} &\sim MVN(0, 1) \\
\sigma_i &\sim Exp(4) 
\end{aligned}
\]

\hypertarget{results}{%
\section{Results}\label{results}}

\hypertarget{phonological-priming-related-vs.-unrelated}{%
\subsection{Phonological priming: Related
vs.~Unrelated}\label{phonological-priming-related-vs.-unrelated}}

A model including the \emph{Relatedness} \(\times\) \emph{Group}
interaction showed the best of-of-sample predictive performance,
although the model including only \emph{Relatedness} performed
equivalently
(\(\text{ELPD}_{\mathcal{M_0}} - \text{ELPD}_{\mathcal{M_1}}\) = -0.724,
\emph{SE} = 3.487). Both models showed substantially better predictive
performance than the model including only \emph{Group}
(\(\text{ELPD}_{\mathcal{M_0}} - \text{ELPD}_{\mathcal{M_2}}\) =
-48.179, \emph{SE} = 11.672). This indicates that including the
\emph{Relatedness} predictor improved the predictive performance of the
model significantly, that including its interaction with \emph{Group}
slightly increased the performance of the model, and that the main
effect of \emph{Group} by itself barely changed the predictive
performance of the model.

\begin{Shaded}
\begin{Highlighting}[]
\NormalTok{epreds }\OtherTok{\textless{}{-}} \FunctionTok{expand\_grid}\NormalTok{(}\AttributeTok{relatedness =} \FunctionTok{levels}\NormalTok{(data\_time}\SpecialCharTok{$}\NormalTok{relatedness),}
                      \AttributeTok{timebin =} \FunctionTok{seq}\NormalTok{(}\DecValTok{0}\NormalTok{, }\DecValTok{20}\NormalTok{, }\AttributeTok{length.out =} \DecValTok{100}\NormalTok{),}
                      \AttributeTok{age =} \FunctionTok{mean}\NormalTok{(data\_time}\SpecialCharTok{$}\NormalTok{age),}
                      \AttributeTok{lp =} \FunctionTok{levels}\NormalTok{(data\_time}\SpecialCharTok{$}\NormalTok{lp)) }\SpecialCharTok{|\textgreater{}}
    \FunctionTok{add\_epred\_draws}\NormalTok{(model\_fit\_related,}
                    \AttributeTok{ndraws =} \ConstantTok{NULL}\NormalTok{,}
                    \AttributeTok{re\_formula =} \ConstantTok{NA}\NormalTok{) }

\NormalTok{data\_time }\SpecialCharTok{|\textgreater{}} 
    \FunctionTok{summarise}\NormalTok{(}\AttributeTok{logit\_t =} \FunctionTok{mean}\NormalTok{(logit\_t),}
              \AttributeTok{.by =} \FunctionTok{c}\NormalTok{(id, timebin, lp, relatedness, age)) }\SpecialCharTok{|\textgreater{}} 
    \FunctionTok{ggplot}\NormalTok{(}\FunctionTok{aes}\NormalTok{(timebin, logit\_t, }
               \AttributeTok{colour =}\NormalTok{ relatedness,}
               \AttributeTok{fill =}\NormalTok{ relatedness,}
               \AttributeTok{shape =}\NormalTok{ relatedness)) }\SpecialCharTok{+}
    \FunctionTok{facet\_wrap}\NormalTok{(}\SpecialCharTok{\textasciitilde{}}\NormalTok{lp) }\SpecialCharTok{+}
    \CommentTok{\# geom\_line(data = epreds,}
    \CommentTok{\#         aes(y = .epred,}
    \CommentTok{\#           group = interaction(relatedness, .draw)),}
    \CommentTok{\#         linetype = "solid",}
    \CommentTok{\#         alpha = 0.1,}
    \CommentTok{\#         linewidth = 3/4) +}
    \FunctionTok{stat\_summary}\NormalTok{(}\AttributeTok{data =}\NormalTok{ epreds,}
                 \FunctionTok{aes}\NormalTok{(}\AttributeTok{y =}\NormalTok{ .epred),}
                 \AttributeTok{fun.data =}\NormalTok{ \textbackslash{}(x) }\FunctionTok{mean\_qi}\NormalTok{(x, }\AttributeTok{.width =} \FloatTok{0.95}\NormalTok{),}
                 \AttributeTok{geom =} \StringTok{"ribbon"}\NormalTok{,}
                 \AttributeTok{alpha =} \FloatTok{0.5}\NormalTok{,}
                 \AttributeTok{linewidth =} \DecValTok{0}\NormalTok{) }\SpecialCharTok{+}
    \FunctionTok{stat\_summary}\NormalTok{(}\AttributeTok{data =}\NormalTok{ epreds,}
                 \FunctionTok{aes}\NormalTok{(}\AttributeTok{y =}\NormalTok{ .epred,}
                    \AttributeTok{linetype =}\NormalTok{ relatedness),}
                 \AttributeTok{fun =} \StringTok{"mean"}\NormalTok{,}
                 \AttributeTok{geom =} \StringTok{"line"}\NormalTok{,}
                 \AttributeTok{colour =} \StringTok{"black"}\NormalTok{,}
                 \AttributeTok{linewidth =} \DecValTok{3}\SpecialCharTok{/}\DecValTok{4}\NormalTok{) }\SpecialCharTok{+}
    \FunctionTok{stat\_summary}\NormalTok{(}\AttributeTok{fun =}\NormalTok{ mean,}
                 \AttributeTok{geom =} \StringTok{"point"}\NormalTok{,}
                 \AttributeTok{colour =} \StringTok{"black"}\NormalTok{,}
                 \AttributeTok{size =} \FloatTok{2.5}\NormalTok{,}
                 \AttributeTok{stroke =} \DecValTok{3}\SpecialCharTok{/}\DecValTok{4}\NormalTok{) }\SpecialCharTok{+}
    \FunctionTok{labs}\NormalTok{(}\AttributeTok{x =} \StringTok{"Time (ms)"}\NormalTok{,}
         \AttributeTok{y =} \StringTok{"P(Target looking)"}\NormalTok{,}
         \AttributeTok{colour =} \StringTok{"Prime type"}\NormalTok{,}
         \AttributeTok{fill =} \StringTok{"Prime type"}\NormalTok{,}
         \AttributeTok{linetype =} \StringTok{"Prime type"}\NormalTok{,}
         \AttributeTok{shape =} \StringTok{"Prime type"}\NormalTok{) }\SpecialCharTok{+}
    \FunctionTok{scale\_linetype\_manual}\NormalTok{(}\AttributeTok{values =} \FunctionTok{rev}\NormalTok{(}\FunctionTok{c}\NormalTok{(}\StringTok{"solid"}\NormalTok{, }\StringTok{"dashed"}\NormalTok{))) }\SpecialCharTok{+}
    \FunctionTok{scale\_colour\_grey}\NormalTok{(}\AttributeTok{start =} \FloatTok{0.8}\NormalTok{, }\AttributeTok{end =} \DecValTok{0}\NormalTok{) }\SpecialCharTok{+}
    \FunctionTok{scale\_fill\_grey}\NormalTok{(}\AttributeTok{start =} \FloatTok{0.85}\NormalTok{, }\AttributeTok{end =} \FloatTok{0.35}\NormalTok{) }\SpecialCharTok{+}
    \FunctionTok{scale\_shape\_manual}\NormalTok{(}\AttributeTok{values =} \FunctionTok{c}\NormalTok{(}\DecValTok{1}\NormalTok{, }\DecValTok{2}\NormalTok{)) }\SpecialCharTok{+}
    \FunctionTok{scale\_x\_continuous}\NormalTok{(}\AttributeTok{labels =}\NormalTok{ \textbackslash{}(x) }\FunctionTok{format}\NormalTok{(x }\SpecialCharTok{*} \FloatTok{1e2}\NormalTok{, }\AttributeTok{big.mark =} \StringTok{","}\NormalTok{)) }\SpecialCharTok{+}
    \FunctionTok{theme}\NormalTok{(}\AttributeTok{legend.position =} \StringTok{"top"}\NormalTok{,}
          \AttributeTok{legend.title =} \FunctionTok{element\_blank}\NormalTok{())}
\end{Highlighting}
\end{Shaded}

\begin{figure}[H]

{\centering \includegraphics{manuscript_files/figure-pdf/fig-related-1.pdf}

}

\caption{\label{fig-related}\textbf{?(caption)}}

\end{figure}

\hypertarget{cognate-priming-cognate-vs.-non-cognate}{%
\subsection{Cognate priming: Cognate
vs.~Non-cognate}\label{cognate-priming-cognate-vs.-non-cognate}}

A model including the \emph{Cognateness} \(\times\) \emph{Group}
interaction showed the best of-of-sample predictive performance,
although the model including only \emph{Cognateness} performed
equivalently
(\(\text{ELPD}_{\mathcal{M_0}} - \text{ELPD}_{\mathcal{M_1}}\) = -1.835,
\emph{SE} = 2.279). Both models showed substantially better predictive
performance than the model including only \emph{Group}
(\(\text{ELPD}_{\mathcal{M_0}} - \text{ELPD}_{\mathcal{M_1}}\) =
-43.201, \emph{SE} = 10.598). This indicates that including the
\emph{Cognateness} predictor improved the predictive performance of the
model significantly, that including its interaction with \emph{Group}
slightly increased the performance of the model, and that the main
effect of \emph{Group} by itself barely changed the predictive
performance of the model.

\begin{Shaded}
\begin{Highlighting}[]
\NormalTok{epreds }\OtherTok{\textless{}{-}} \FunctionTok{expand\_grid}\NormalTok{(}\AttributeTok{cognateness =} \FunctionTok{levels}\NormalTok{(data\_time\_cognateness\_alt}\SpecialCharTok{$}\NormalTok{cognateness),}
                      \AttributeTok{timebin =} \FunctionTok{seq}\NormalTok{(}\DecValTok{0}\NormalTok{, }\DecValTok{20}\NormalTok{, }\AttributeTok{length.out =} \DecValTok{100}\NormalTok{),}
                      \AttributeTok{age =} \FunctionTok{mean}\NormalTok{(data\_time\_cognateness\_alt}\SpecialCharTok{$}\NormalTok{age),}
                      \AttributeTok{lp =} \FunctionTok{levels}\NormalTok{(data\_time\_cognateness\_alt}\SpecialCharTok{$}\NormalTok{lp)) }\SpecialCharTok{|\textgreater{}}
    \FunctionTok{add\_epred\_draws}\NormalTok{(model\_fit\_cognate\_alt,}
                    \AttributeTok{ndraws =} \ConstantTok{NULL}\NormalTok{,}
                    \AttributeTok{re\_formula =} \ConstantTok{NA}\NormalTok{) }

\NormalTok{data\_time\_cognateness\_alt }\SpecialCharTok{|\textgreater{}} 
    \FunctionTok{summarise}\NormalTok{(}\AttributeTok{logit\_t =} \FunctionTok{mean}\NormalTok{(logit\_t),}
              \AttributeTok{.by =} \FunctionTok{c}\NormalTok{(id, timebin, lp, cognateness, age)) }\SpecialCharTok{|\textgreater{}} 
    \FunctionTok{ggplot}\NormalTok{(}\FunctionTok{aes}\NormalTok{(timebin, logit\_t, }
               \AttributeTok{colour =}\NormalTok{ cognateness,}
               \AttributeTok{fill =}\NormalTok{ cognateness,}
               \AttributeTok{shape =}\NormalTok{ cognateness)) }\SpecialCharTok{+}
    \FunctionTok{facet\_wrap}\NormalTok{(}\SpecialCharTok{\textasciitilde{}}\NormalTok{lp) }\SpecialCharTok{+}
    \CommentTok{\# geom\_line(data = epreds,}
    \CommentTok{\#         aes(y = .epred,}
    \CommentTok{\#           group = interaction(cognateness, .draw)),}
    \CommentTok{\#         linetype = "solid",}
    \CommentTok{\#         alpha = 0.1,}
    \CommentTok{\#         linewidth = 3/4) +}
    \FunctionTok{stat\_summary}\NormalTok{(}\AttributeTok{data =}\NormalTok{ epreds,}
                 \FunctionTok{aes}\NormalTok{(}\AttributeTok{y =}\NormalTok{ .epred),}
                 \AttributeTok{fun.data =}\NormalTok{ \textbackslash{}(x) }\FunctionTok{mean\_qi}\NormalTok{(x, }\AttributeTok{.width =} \FloatTok{0.95}\NormalTok{),}
                 \AttributeTok{geom =} \StringTok{"ribbon"}\NormalTok{,}
                 \AttributeTok{alpha =} \FloatTok{0.5}\NormalTok{,}
                 \AttributeTok{linewidth =} \DecValTok{0}\NormalTok{) }\SpecialCharTok{+}
    \FunctionTok{stat\_summary}\NormalTok{(}\AttributeTok{data =}\NormalTok{ epreds,}
                 \FunctionTok{aes}\NormalTok{(}\AttributeTok{y =}\NormalTok{ .epred,}
                    \AttributeTok{linetype =}\NormalTok{ cognateness),}
                 \AttributeTok{fun =} \StringTok{"mean"}\NormalTok{,}
                 \AttributeTok{geom =} \StringTok{"line"}\NormalTok{,}
                 \AttributeTok{colour =} \StringTok{"black"}\NormalTok{,}
                 \AttributeTok{linewidth =} \DecValTok{3}\SpecialCharTok{/}\DecValTok{4}\NormalTok{) }\SpecialCharTok{+}
    \FunctionTok{stat\_summary}\NormalTok{(}\AttributeTok{fun =}\NormalTok{ mean,}
                 \AttributeTok{geom =} \StringTok{"point"}\NormalTok{,}
                 \AttributeTok{colour =} \StringTok{"black"}\NormalTok{,}
                 \AttributeTok{size =} \FloatTok{2.5}\NormalTok{,}
                 \AttributeTok{stroke =} \DecValTok{3}\SpecialCharTok{/}\DecValTok{4}\NormalTok{) }\SpecialCharTok{+}
    \FunctionTok{labs}\NormalTok{(}\AttributeTok{x =} \StringTok{"Time (ms)"}\NormalTok{,}
         \AttributeTok{y =} \StringTok{"P(Target looking)"}\NormalTok{,}
         \AttributeTok{colour =} \StringTok{"Prime type"}\NormalTok{,}
         \AttributeTok{fill =} \StringTok{"Prime type"}\NormalTok{,}
         \AttributeTok{linetype =} \StringTok{"Prime type"}\NormalTok{,}
         \AttributeTok{shape =} \StringTok{"Prime type"}\NormalTok{) }\SpecialCharTok{+}
    \FunctionTok{scale\_linetype\_manual}\NormalTok{(}\AttributeTok{values =} \FunctionTok{rev}\NormalTok{(}\FunctionTok{c}\NormalTok{(}\StringTok{"solid"}\NormalTok{, }\StringTok{"dashed"}\NormalTok{))) }\SpecialCharTok{+}
    \FunctionTok{scale\_colour\_grey}\NormalTok{(}\AttributeTok{start =} \FloatTok{0.8}\NormalTok{, }\AttributeTok{end =} \DecValTok{0}\NormalTok{) }\SpecialCharTok{+}
    \FunctionTok{scale\_fill\_grey}\NormalTok{(}\AttributeTok{start =} \FloatTok{0.85}\NormalTok{, }\AttributeTok{end =} \FloatTok{0.35}\NormalTok{) }\SpecialCharTok{+}
    \FunctionTok{scale\_shape\_manual}\NormalTok{(}\AttributeTok{values =} \FunctionTok{c}\NormalTok{(}\DecValTok{1}\NormalTok{, }\DecValTok{2}\NormalTok{)) }\SpecialCharTok{+}
    \FunctionTok{scale\_x\_continuous}\NormalTok{(}\AttributeTok{labels =}\NormalTok{ \textbackslash{}(x) }\FunctionTok{format}\NormalTok{(x }\SpecialCharTok{*} \FloatTok{1e2}\NormalTok{, }\AttributeTok{big.mark =} \StringTok{","}\NormalTok{)) }\SpecialCharTok{+}
    \FunctionTok{theme}\NormalTok{(}\AttributeTok{legend.position =} \StringTok{"top"}\NormalTok{,}
          \AttributeTok{legend.title =} \FunctionTok{element\_blank}\NormalTok{())}
\end{Highlighting}
\end{Shaded}

\begin{figure}[H]

{\centering \includegraphics{manuscript_files/figure-pdf/fig-cognate-1.pdf}

}

\caption{\label{fig-cognate}\textbf{?(caption)}}

\end{figure}

\hypertarget{discussion}{%
\section{Discussion}\label{discussion}}

\hypertarget{appendix}{%
\section{Appendix}\label{appendix}}

\hypertarget{relaxing-trial-inclusion-criteria}{%
\subsection{Relaxing trial inclusion
criteria}\label{relaxing-trial-inclusion-criteria}}

\begin{Shaded}
\begin{Highlighting}[]
\NormalTok{epreds }\OtherTok{\textless{}{-}} \FunctionTok{expand\_grid}\NormalTok{(}\AttributeTok{relatedness =} \FunctionTok{levels}\NormalTok{(data\_time\_alt}\SpecialCharTok{$}\NormalTok{relatedness),}
                      \AttributeTok{timebin =} \FunctionTok{seq}\NormalTok{(}\DecValTok{0}\NormalTok{, }\DecValTok{20}\NormalTok{, }\AttributeTok{length.out =} \DecValTok{100}\NormalTok{),}
                      \AttributeTok{age =} \FunctionTok{mean}\NormalTok{(data\_time}\SpecialCharTok{$}\NormalTok{age),}
                      \AttributeTok{lp =} \FunctionTok{levels}\NormalTok{(data\_time\_alt}\SpecialCharTok{$}\NormalTok{lp)) }\SpecialCharTok{|\textgreater{}}
    \FunctionTok{add\_epred\_draws}\NormalTok{(model\_fit\_related\_alt,}
                    \AttributeTok{ndraws =} \ConstantTok{NULL}\NormalTok{,}
                    \AttributeTok{re\_formula =} \ConstantTok{NA}\NormalTok{) }

\NormalTok{data\_time\_alt }\SpecialCharTok{|\textgreater{}} 
    \FunctionTok{summarise}\NormalTok{(}\AttributeTok{logit\_t =} \FunctionTok{mean}\NormalTok{(logit\_t),}
              \AttributeTok{.by =} \FunctionTok{c}\NormalTok{(id, timebin, lp, relatedness, age)) }\SpecialCharTok{|\textgreater{}} 
    \FunctionTok{ggplot}\NormalTok{(}\FunctionTok{aes}\NormalTok{(timebin, logit\_t, }
               \AttributeTok{colour =}\NormalTok{ relatedness,}
               \AttributeTok{fill =}\NormalTok{ relatedness,}
               \AttributeTok{shape =}\NormalTok{ relatedness)) }\SpecialCharTok{+}
    \FunctionTok{facet\_wrap}\NormalTok{(}\SpecialCharTok{\textasciitilde{}}\NormalTok{lp) }\SpecialCharTok{+}
    \CommentTok{\# geom\_line(data = epreds,}
    \CommentTok{\#         aes(y = .epred,}
    \CommentTok{\#           group = interaction(relatedness, .draw)),}
    \CommentTok{\#         linetype = "solid",}
    \CommentTok{\#         alpha = 0.1,}
    \CommentTok{\#         linewidth = 3/4) +}
    \FunctionTok{stat\_summary}\NormalTok{(}\AttributeTok{data =}\NormalTok{ epreds,}
                 \FunctionTok{aes}\NormalTok{(}\AttributeTok{y =}\NormalTok{ .epred),}
                 \AttributeTok{fun.data =}\NormalTok{ \textbackslash{}(x) }\FunctionTok{mean\_qi}\NormalTok{(x, }\AttributeTok{.width =} \FloatTok{0.95}\NormalTok{),}
                 \AttributeTok{geom =} \StringTok{"ribbon"}\NormalTok{,}
                 \AttributeTok{alpha =} \FloatTok{0.5}\NormalTok{,}
                 \AttributeTok{linewidth =} \DecValTok{0}\NormalTok{) }\SpecialCharTok{+}
    \FunctionTok{stat\_summary}\NormalTok{(}\AttributeTok{data =}\NormalTok{ epreds,}
                 \FunctionTok{aes}\NormalTok{(}\AttributeTok{y =}\NormalTok{ .epred,}
                    \AttributeTok{linetype =}\NormalTok{ relatedness),}
                 \AttributeTok{fun =} \StringTok{"mean"}\NormalTok{,}
                 \AttributeTok{geom =} \StringTok{"line"}\NormalTok{,}
                 \AttributeTok{colour =} \StringTok{"black"}\NormalTok{,}
                 \AttributeTok{linewidth =} \DecValTok{3}\SpecialCharTok{/}\DecValTok{4}\NormalTok{) }\SpecialCharTok{+}
    \FunctionTok{stat\_summary}\NormalTok{(}\AttributeTok{fun =}\NormalTok{ mean,}
                 \AttributeTok{geom =} \StringTok{"point"}\NormalTok{,}
                 \AttributeTok{colour =} \StringTok{"black"}\NormalTok{,}
                 \AttributeTok{size =} \FloatTok{2.5}\NormalTok{,}
                 \AttributeTok{stroke =} \DecValTok{3}\SpecialCharTok{/}\DecValTok{4}\NormalTok{) }\SpecialCharTok{+}
    \FunctionTok{labs}\NormalTok{(}\AttributeTok{x =} \StringTok{"Time (ms)"}\NormalTok{,}
         \AttributeTok{y =} \StringTok{"P(Target looking)"}\NormalTok{,}
         \AttributeTok{colour =} \StringTok{"Prime type"}\NormalTok{,}
         \AttributeTok{fill =} \StringTok{"Prime type"}\NormalTok{,}
         \AttributeTok{linetype =} \StringTok{"Prime type"}\NormalTok{,}
         \AttributeTok{shape =} \StringTok{"Prime type"}\NormalTok{) }\SpecialCharTok{+}
    \FunctionTok{scale\_linetype\_manual}\NormalTok{(}\AttributeTok{values =} \FunctionTok{rev}\NormalTok{(}\FunctionTok{c}\NormalTok{(}\StringTok{"solid"}\NormalTok{, }\StringTok{"dashed"}\NormalTok{))) }\SpecialCharTok{+}
    \FunctionTok{scale\_colour\_grey}\NormalTok{(}\AttributeTok{start =} \FloatTok{0.8}\NormalTok{, }\AttributeTok{end =} \DecValTok{0}\NormalTok{) }\SpecialCharTok{+}
    \FunctionTok{scale\_fill\_grey}\NormalTok{(}\AttributeTok{start =} \FloatTok{0.85}\NormalTok{, }\AttributeTok{end =} \FloatTok{0.35}\NormalTok{) }\SpecialCharTok{+}
    \FunctionTok{scale\_shape\_manual}\NormalTok{(}\AttributeTok{values =} \FunctionTok{c}\NormalTok{(}\DecValTok{1}\NormalTok{, }\DecValTok{2}\NormalTok{)) }\SpecialCharTok{+}
    \FunctionTok{scale\_x\_continuous}\NormalTok{(}\AttributeTok{labels =}\NormalTok{ \textbackslash{}(x) }\FunctionTok{format}\NormalTok{(x }\SpecialCharTok{*} \FloatTok{1e2}\NormalTok{, }\AttributeTok{big.mark =} \StringTok{","}\NormalTok{)) }\SpecialCharTok{+}
    \FunctionTok{theme}\NormalTok{(}\AttributeTok{legend.position =} \StringTok{"top"}\NormalTok{,}
          \AttributeTok{legend.title =} \FunctionTok{element\_blank}\NormalTok{())}
\end{Highlighting}
\end{Shaded}

\begin{figure}[H]

{\centering \includegraphics{manuscript_files/figure-pdf/fig-related-alt-1.pdf}

}

\caption{\label{fig-related-alt}\textbf{?(caption)}}

\end{figure}

\begin{Shaded}
\begin{Highlighting}[]
\NormalTok{epreds }\OtherTok{\textless{}{-}} \FunctionTok{expand\_grid}\NormalTok{(}\AttributeTok{cognateness =} \FunctionTok{levels}\NormalTok{(data\_time\_cognateness\_alt}\SpecialCharTok{$}\NormalTok{cognateness),}
                      \AttributeTok{timebin =} \FunctionTok{seq}\NormalTok{(}\DecValTok{0}\NormalTok{, }\DecValTok{20}\NormalTok{, }\AttributeTok{length.out =} \DecValTok{100}\NormalTok{),}
                      \AttributeTok{age =} \FunctionTok{mean}\NormalTok{(data\_time\_cognateness\_alt}\SpecialCharTok{$}\NormalTok{age),}
                      \AttributeTok{lp =} \FunctionTok{levels}\NormalTok{(data\_time\_cognateness\_alt}\SpecialCharTok{$}\NormalTok{lp)) }\SpecialCharTok{|\textgreater{}}
    \FunctionTok{add\_epred\_draws}\NormalTok{(model\_fit\_cognate\_alt,}
                    \AttributeTok{ndraws =} \ConstantTok{NULL}\NormalTok{,}
                    \AttributeTok{re\_formula =} \ConstantTok{NA}\NormalTok{) }

\NormalTok{data\_time\_cognateness\_alt }\SpecialCharTok{|\textgreater{}} 
    \FunctionTok{summarise}\NormalTok{(}\AttributeTok{logit\_t =} \FunctionTok{mean}\NormalTok{(logit\_t),}
              \AttributeTok{.by =} \FunctionTok{c}\NormalTok{(id, timebin, lp, cognateness, age)) }\SpecialCharTok{|\textgreater{}} 
    \FunctionTok{ggplot}\NormalTok{(}\FunctionTok{aes}\NormalTok{(timebin, logit\_t, }
               \AttributeTok{colour =}\NormalTok{ cognateness,}
               \AttributeTok{fill =}\NormalTok{ cognateness,}
               \AttributeTok{shape =}\NormalTok{ cognateness)) }\SpecialCharTok{+}
    \FunctionTok{facet\_wrap}\NormalTok{(}\SpecialCharTok{\textasciitilde{}}\NormalTok{lp) }\SpecialCharTok{+}
    \CommentTok{\# geom\_line(data = epreds,}
    \CommentTok{\#         aes(y = .epred,}
    \CommentTok{\#           group = interaction(cognateness, .draw)),}
    \CommentTok{\#         linetype = "solid",}
    \CommentTok{\#         alpha = 0.1,}
    \CommentTok{\#         linewidth = 3/4) +}
    \FunctionTok{stat\_summary}\NormalTok{(}\AttributeTok{data =}\NormalTok{ epreds,}
                 \FunctionTok{aes}\NormalTok{(}\AttributeTok{y =}\NormalTok{ .epred),}
                 \AttributeTok{fun.data =}\NormalTok{ \textbackslash{}(x) }\FunctionTok{mean\_qi}\NormalTok{(x, }\AttributeTok{.width =} \FloatTok{0.95}\NormalTok{),}
                 \AttributeTok{geom =} \StringTok{"ribbon"}\NormalTok{,}
                 \AttributeTok{alpha =} \FloatTok{0.5}\NormalTok{,}
                 \AttributeTok{linewidth =} \DecValTok{0}\NormalTok{) }\SpecialCharTok{+}
    \FunctionTok{stat\_summary}\NormalTok{(}\AttributeTok{data =}\NormalTok{ epreds,}
                 \FunctionTok{aes}\NormalTok{(}\AttributeTok{y =}\NormalTok{ .epred,}
                    \AttributeTok{linetype =}\NormalTok{ cognateness),}
                 \AttributeTok{fun =} \StringTok{"mean"}\NormalTok{,}
                 \AttributeTok{geom =} \StringTok{"line"}\NormalTok{,}
                 \AttributeTok{colour =} \StringTok{"black"}\NormalTok{,}
                 \AttributeTok{linewidth =} \DecValTok{3}\SpecialCharTok{/}\DecValTok{4}\NormalTok{) }\SpecialCharTok{+}
    \FunctionTok{stat\_summary}\NormalTok{(}\AttributeTok{fun =}\NormalTok{ mean,}
                 \AttributeTok{geom =} \StringTok{"point"}\NormalTok{,}
                 \AttributeTok{colour =} \StringTok{"black"}\NormalTok{,}
                 \AttributeTok{size =} \FloatTok{2.5}\NormalTok{,}
                 \AttributeTok{stroke =} \DecValTok{3}\SpecialCharTok{/}\DecValTok{4}\NormalTok{) }\SpecialCharTok{+}
    \FunctionTok{labs}\NormalTok{(}\AttributeTok{x =} \StringTok{"Time (ms)"}\NormalTok{,}
         \AttributeTok{y =} \StringTok{"P(Target looking)"}\NormalTok{,}
         \AttributeTok{colour =} \StringTok{"Prime type"}\NormalTok{,}
         \AttributeTok{fill =} \StringTok{"Prime type"}\NormalTok{,}
         \AttributeTok{linetype =} \StringTok{"Prime type"}\NormalTok{,}
         \AttributeTok{shape =} \StringTok{"Prime type"}\NormalTok{) }\SpecialCharTok{+}
    \FunctionTok{scale\_linetype\_manual}\NormalTok{(}\AttributeTok{values =} \FunctionTok{rev}\NormalTok{(}\FunctionTok{c}\NormalTok{(}\StringTok{"solid"}\NormalTok{, }\StringTok{"dashed"}\NormalTok{))) }\SpecialCharTok{+}
    \FunctionTok{scale\_colour\_grey}\NormalTok{(}\AttributeTok{start =} \FloatTok{0.8}\NormalTok{, }\AttributeTok{end =} \DecValTok{0}\NormalTok{) }\SpecialCharTok{+}
    \FunctionTok{scale\_fill\_grey}\NormalTok{(}\AttributeTok{start =} \FloatTok{0.85}\NormalTok{, }\AttributeTok{end =} \FloatTok{0.35}\NormalTok{) }\SpecialCharTok{+}
    \FunctionTok{scale\_shape\_manual}\NormalTok{(}\AttributeTok{values =} \FunctionTok{c}\NormalTok{(}\DecValTok{1}\NormalTok{, }\DecValTok{2}\NormalTok{)) }\SpecialCharTok{+}
    \FunctionTok{scale\_x\_continuous}\NormalTok{(}\AttributeTok{labels =}\NormalTok{ \textbackslash{}(x) }\FunctionTok{format}\NormalTok{(x }\SpecialCharTok{*} \FloatTok{1e2}\NormalTok{, }\AttributeTok{big.mark =} \StringTok{","}\NormalTok{)) }\SpecialCharTok{+}
    \FunctionTok{theme}\NormalTok{(}\AttributeTok{legend.position =} \StringTok{"top"}\NormalTok{,}
          \AttributeTok{legend.title =} \FunctionTok{element\_blank}\NormalTok{())}
\end{Highlighting}
\end{Shaded}

\begin{figure}[H]

{\centering \includegraphics{manuscript_files/figure-pdf/fig-cognate-alt-1.pdf}

}

\caption{\label{fig-cognate-alt}\textbf{?(caption)}}

\end{figure}

\hypertarget{refs}{}
\begin{CSLReferences}{1}{0}
\leavevmode\vadjust pre{\hypertarget{ref-bosch2001evidence}{}}%
Bosch, L., \& Sebastián-Gallés, N. (2001). Evidence of early language
discrimination abilities in infants from bilingual environments.
\emph{Infancy}, \emph{2}(1), 29--49.

\leavevmode\vadjust pre{\hypertarget{ref-wood2017generalized}{}}%
Wood, S. N. (2017). \emph{Generalized additive models: An introduction
with r}. CRC press.

\leavevmode\vadjust pre{\hypertarget{ref-bosch2001evidence}{}}%
Bosch, L., \& Sebastián-Gallés, N. (2001). Evidence of early language
discrimination abilities in infants from bilingual environments.
\emph{Infancy}, \emph{2}(1), 29--49.

\leavevmode\vadjust pre{\hypertarget{ref-wood2017generalized}{}}%
Wood, S. N. (2017). \emph{Generalized additive models: An introduction
with r}. CRC press.

\end{CSLReferences}



\end{document}
