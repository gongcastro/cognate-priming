% Options for packages loaded elsewhere
\PassOptionsToPackage{unicode}{hyperref}
\PassOptionsToPackage{hyphens}{url}
\PassOptionsToPackage{dvipsnames,svgnames,x11names}{xcolor}
%
\documentclass[
  letterpaper,
  DIV=11,
  numbers=noendperiod]{scrartcl}

\usepackage{amsmath,amssymb}
\usepackage{iftex}
\ifPDFTeX
  \usepackage[T1]{fontenc}
  \usepackage[utf8]{inputenc}
  \usepackage{textcomp} % provide euro and other symbols
\else % if luatex or xetex
  \usepackage{unicode-math}
  \defaultfontfeatures{Scale=MatchLowercase}
  \defaultfontfeatures[\rmfamily]{Ligatures=TeX,Scale=1}
\fi
\usepackage{lmodern}
\ifPDFTeX\else  
    % xetex/luatex font selection
\fi
% Use upquote if available, for straight quotes in verbatim environments
\IfFileExists{upquote.sty}{\usepackage{upquote}}{}
\IfFileExists{microtype.sty}{% use microtype if available
  \usepackage[]{microtype}
  \UseMicrotypeSet[protrusion]{basicmath} % disable protrusion for tt fonts
}{}
\makeatletter
\@ifundefined{KOMAClassName}{% if non-KOMA class
  \IfFileExists{parskip.sty}{%
    \usepackage{parskip}
  }{% else
    \setlength{\parindent}{0pt}
    \setlength{\parskip}{6pt plus 2pt minus 1pt}}
}{% if KOMA class
  \KOMAoptions{parskip=half}}
\makeatother
\usepackage{xcolor}
\setlength{\emergencystretch}{3em} % prevent overfull lines
\setcounter{secnumdepth}{-\maxdimen} % remove section numbering
% Make \paragraph and \subparagraph free-standing
\ifx\paragraph\undefined\else
  \let\oldparagraph\paragraph
  \renewcommand{\paragraph}[1]{\oldparagraph{#1}\mbox{}}
\fi
\ifx\subparagraph\undefined\else
  \let\oldsubparagraph\subparagraph
  \renewcommand{\subparagraph}[1]{\oldsubparagraph{#1}\mbox{}}
\fi

\usepackage{color}
\usepackage{fancyvrb}
\newcommand{\VerbBar}{|}
\newcommand{\VERB}{\Verb[commandchars=\\\{\}]}
\DefineVerbatimEnvironment{Highlighting}{Verbatim}{commandchars=\\\{\}}
% Add ',fontsize=\small' for more characters per line
\usepackage{framed}
\definecolor{shadecolor}{RGB}{241,243,245}
\newenvironment{Shaded}{\begin{snugshade}}{\end{snugshade}}
\newcommand{\AlertTok}[1]{\textcolor[rgb]{0.68,0.00,0.00}{#1}}
\newcommand{\AnnotationTok}[1]{\textcolor[rgb]{0.37,0.37,0.37}{#1}}
\newcommand{\AttributeTok}[1]{\textcolor[rgb]{0.40,0.45,0.13}{#1}}
\newcommand{\BaseNTok}[1]{\textcolor[rgb]{0.68,0.00,0.00}{#1}}
\newcommand{\BuiltInTok}[1]{\textcolor[rgb]{0.00,0.23,0.31}{#1}}
\newcommand{\CharTok}[1]{\textcolor[rgb]{0.13,0.47,0.30}{#1}}
\newcommand{\CommentTok}[1]{\textcolor[rgb]{0.37,0.37,0.37}{#1}}
\newcommand{\CommentVarTok}[1]{\textcolor[rgb]{0.37,0.37,0.37}{\textit{#1}}}
\newcommand{\ConstantTok}[1]{\textcolor[rgb]{0.56,0.35,0.01}{#1}}
\newcommand{\ControlFlowTok}[1]{\textcolor[rgb]{0.00,0.23,0.31}{#1}}
\newcommand{\DataTypeTok}[1]{\textcolor[rgb]{0.68,0.00,0.00}{#1}}
\newcommand{\DecValTok}[1]{\textcolor[rgb]{0.68,0.00,0.00}{#1}}
\newcommand{\DocumentationTok}[1]{\textcolor[rgb]{0.37,0.37,0.37}{\textit{#1}}}
\newcommand{\ErrorTok}[1]{\textcolor[rgb]{0.68,0.00,0.00}{#1}}
\newcommand{\ExtensionTok}[1]{\textcolor[rgb]{0.00,0.23,0.31}{#1}}
\newcommand{\FloatTok}[1]{\textcolor[rgb]{0.68,0.00,0.00}{#1}}
\newcommand{\FunctionTok}[1]{\textcolor[rgb]{0.28,0.35,0.67}{#1}}
\newcommand{\ImportTok}[1]{\textcolor[rgb]{0.00,0.46,0.62}{#1}}
\newcommand{\InformationTok}[1]{\textcolor[rgb]{0.37,0.37,0.37}{#1}}
\newcommand{\KeywordTok}[1]{\textcolor[rgb]{0.00,0.23,0.31}{#1}}
\newcommand{\NormalTok}[1]{\textcolor[rgb]{0.00,0.23,0.31}{#1}}
\newcommand{\OperatorTok}[1]{\textcolor[rgb]{0.37,0.37,0.37}{#1}}
\newcommand{\OtherTok}[1]{\textcolor[rgb]{0.00,0.23,0.31}{#1}}
\newcommand{\PreprocessorTok}[1]{\textcolor[rgb]{0.68,0.00,0.00}{#1}}
\newcommand{\RegionMarkerTok}[1]{\textcolor[rgb]{0.00,0.23,0.31}{#1}}
\newcommand{\SpecialCharTok}[1]{\textcolor[rgb]{0.37,0.37,0.37}{#1}}
\newcommand{\SpecialStringTok}[1]{\textcolor[rgb]{0.13,0.47,0.30}{#1}}
\newcommand{\StringTok}[1]{\textcolor[rgb]{0.13,0.47,0.30}{#1}}
\newcommand{\VariableTok}[1]{\textcolor[rgb]{0.07,0.07,0.07}{#1}}
\newcommand{\VerbatimStringTok}[1]{\textcolor[rgb]{0.13,0.47,0.30}{#1}}
\newcommand{\WarningTok}[1]{\textcolor[rgb]{0.37,0.37,0.37}{\textit{#1}}}

\providecommand{\tightlist}{%
  \setlength{\itemsep}{0pt}\setlength{\parskip}{0pt}}\usepackage{longtable,booktabs,array}
\usepackage{calc} % for calculating minipage widths
% Correct order of tables after \paragraph or \subparagraph
\usepackage{etoolbox}
\makeatletter
\patchcmd\longtable{\par}{\if@noskipsec\mbox{}\fi\par}{}{}
\makeatother
% Allow footnotes in longtable head/foot
\IfFileExists{footnotehyper.sty}{\usepackage{footnotehyper}}{\usepackage{footnote}}
\makesavenoteenv{longtable}
\usepackage{graphicx}
\makeatletter
\def\maxwidth{\ifdim\Gin@nat@width>\linewidth\linewidth\else\Gin@nat@width\fi}
\def\maxheight{\ifdim\Gin@nat@height>\textheight\textheight\else\Gin@nat@height\fi}
\makeatother
% Scale images if necessary, so that they will not overflow the page
% margins by default, and it is still possible to overwrite the defaults
% using explicit options in \includegraphics[width, height, ...]{}
\setkeys{Gin}{width=\maxwidth,height=\maxheight,keepaspectratio}
% Set default figure placement to htbp
\makeatletter
\def\fps@figure{htbp}
\makeatother
\newlength{\cslhangindent}
\setlength{\cslhangindent}{1.5em}
\newlength{\csllabelwidth}
\setlength{\csllabelwidth}{3em}
\newlength{\cslentryspacingunit} % times entry-spacing
\setlength{\cslentryspacingunit}{\parskip}
\newenvironment{CSLReferences}[2] % #1 hanging-ident, #2 entry spacing
 {% don't indent paragraphs
  \setlength{\parindent}{0pt}
  % turn on hanging indent if param 1 is 1
  \ifodd #1
  \let\oldpar\par
  \def\par{\hangindent=\cslhangindent\oldpar}
  \fi
  % set entry spacing
  \setlength{\parskip}{#2\cslentryspacingunit}
 }%
 {}
\usepackage{calc}
\newcommand{\CSLBlock}[1]{#1\hfill\break}
\newcommand{\CSLLeftMargin}[1]{\parbox[t]{\csllabelwidth}{#1}}
\newcommand{\CSLRightInline}[1]{\parbox[t]{\linewidth - \csllabelwidth}{#1}\break}
\newcommand{\CSLIndent}[1]{\hspace{\cslhangindent}#1}

\usepackage{booktabs}
\usepackage{longtable}
\usepackage{array}
\usepackage{multirow}
\usepackage{wrapfig}
\usepackage{float}
\usepackage{colortbl}
\usepackage{pdflscape}
\usepackage{tabu}
\usepackage{threeparttable}
\usepackage{threeparttablex}
\usepackage[normalem]{ulem}
\usepackage{makecell}
\usepackage{xcolor}
\usepackage{caption}
\usepackage{tipa}
\usepackage{booktabs}
\hyphenpenalty=100000
\exhyphenpenalty=100000
\KOMAoption{captions}{tableheading}
\makeatletter
\makeatother
\makeatletter
\makeatother
\makeatletter
\@ifpackageloaded{caption}{}{\usepackage{caption}}
\AtBeginDocument{%
\ifdefined\contentsname
  \renewcommand*\contentsname{Table of contents}
\else
  \newcommand\contentsname{Table of contents}
\fi
\ifdefined\listfigurename
  \renewcommand*\listfigurename{List of Figures}
\else
  \newcommand\listfigurename{List of Figures}
\fi
\ifdefined\listtablename
  \renewcommand*\listtablename{List of Tables}
\else
  \newcommand\listtablename{List of Tables}
\fi
\ifdefined\figurename
  \renewcommand*\figurename{Figure}
\else
  \newcommand\figurename{Figure}
\fi
\ifdefined\tablename
  \renewcommand*\tablename{Table}
\else
  \newcommand\tablename{Table}
\fi
}
\@ifpackageloaded{float}{}{\usepackage{float}}
\floatstyle{ruled}
\@ifundefined{c@chapter}{\newfloat{codelisting}{h}{lop}}{\newfloat{codelisting}{h}{lop}[chapter]}
\floatname{codelisting}{Listing}
\newcommand*\listoflistings{\listof{codelisting}{List of Listings}}
\makeatother
\makeatletter
\@ifpackageloaded{caption}{}{\usepackage{caption}}
\@ifpackageloaded{subcaption}{}{\usepackage{subcaption}}
\makeatother
\makeatletter
\@ifpackageloaded{tcolorbox}{}{\usepackage[skins,breakable]{tcolorbox}}
\makeatother
\makeatletter
\@ifundefined{shadecolor}{\definecolor{shadecolor}{rgb}{.97, .97, .97}}
\makeatother
\makeatletter
\makeatother
\makeatletter
\makeatother
\ifLuaTeX
\usepackage[bidi=basic]{babel}
\else
\usepackage[bidi=default]{babel}
\fi
\babelprovide[main,import]{english}
% get rid of language-specific shorthands (see #6817):
\let\LanguageShortHands\languageshorthands
\def\languageshorthands#1{}
\ifLuaTeX
  \usepackage{selnolig}  % disable illegal ligatures
\fi
\IfFileExists{bookmark.sty}{\usepackage{bookmark}}{\usepackage{hyperref}}
\IfFileExists{xurl.sty}{\usepackage{xurl}}{} % add URL line breaks if available
\urlstyle{same} % disable monospaced font for URLs
\hypersetup{
  pdftitle={Developmental trajectories of bilingual word recognition},
  pdfauthor={Gonzalo Garcia-Castro; Serene Siow; Nuria Sebastian-Galles; Kim Plunkett},
  pdflang={en},
  pdfkeywords={cognate, word recognition, lexicon, language
acquisition, vocabulary, bilingualism, general additive mixed
models, bayesian},
  colorlinks=true,
  linkcolor={blue},
  filecolor={Maroon},
  citecolor={Blue},
  urlcolor={Blue},
  pdfcreator={LaTeX via pandoc}}

\title{Developmental trajectories of bilingual word recognition}
\author{Gonzalo Garcia-Castro \and Serene Siow \and Nuria
Sebastian-Galles \and Kim Plunkett}
\date{}

\begin{document}
\maketitle
\ifdefined\Shaded\renewenvironment{Shaded}{\begin{tcolorbox}[interior hidden, sharp corners, frame hidden, borderline west={3pt}{0pt}{shadecolor}, breakable, boxrule=0pt, enhanced]}{\end{tcolorbox}}\fi

\hypertarget{methods}{%
\section{Methods}\label{methods}}

All materials, data, and reproducible code can be found at the OSF
(\href{https://osf.io/ckydb/}{https://osf.io/hy984/}) and GitHub
(\url{https://github.com/gongcastro/cognate-priming}) repositories. This
study was conducted according to guidelines laid down in the Declaration
of Helsinki, and was approved by the Drug Research Ethical Committee
(CEIm) of the IMIM Parc de Salut Mar, reference 2020/9080/I. Before
every testing session, caregivers were asked to read and sign an
informed consent form, and were given a token of appreciation at the end
of it.

\hypertarget{participants}{%
\subsection{Participants}\label{participants}}

\begin{Shaded}
\begin{Highlighting}[]
\NormalTok{n\_participants\_total }\OtherTok{\textless{}{-}} \FunctionTok{length}\NormalTok{(}\FunctionTok{unique}\NormalTok{(participants}\SpecialCharTok{$}\NormalTok{id))}

\NormalTok{n\_participants\_sessions }\OtherTok{\textless{}{-}} \FunctionTok{count}\NormalTok{(participants, id, }\AttributeTok{name =} \StringTok{"n\_sessions"}\NormalTok{) }\SpecialCharTok{|\textgreater{}} 
    \FunctionTok{count}\NormalTok{(n\_sessions) }\SpecialCharTok{|\textgreater{}} 
    \FunctionTok{group\_split}\NormalTok{(n\_sessions) }\SpecialCharTok{|\textgreater{}} 
    \FunctionTok{set\_names}\NormalTok{(}\FunctionTok{paste0}\NormalTok{(}\StringTok{"session\_"}\NormalTok{, }\DecValTok{1}\SpecialCharTok{:}\DecValTok{3}\NormalTok{))}

\NormalTok{n\_sessions\_total }\OtherTok{\textless{}{-}} \FunctionTok{count}\NormalTok{(participants)}

\NormalTok{n\_sessions\_age\_group }\OtherTok{\textless{}{-}}\NormalTok{ participants }\SpecialCharTok{|\textgreater{}} 
    \FunctionTok{group\_by}\NormalTok{(age\_group) }\SpecialCharTok{|\textgreater{}} 
    \FunctionTok{summarise}\NormalTok{(}\FunctionTok{across}\NormalTok{(age, }\FunctionTok{lst}\NormalTok{(mean, sd, min, max)),}
              \AttributeTok{n =} \FunctionTok{n}\NormalTok{(),}
              \AttributeTok{.groups =} \StringTok{"drop"}\NormalTok{) }\SpecialCharTok{|\textgreater{}} 
    \FunctionTok{mutate}\NormalTok{(}\FunctionTok{across}\NormalTok{(age\_mean}\SpecialCharTok{:}\NormalTok{age\_max, \textbackslash{}(x) }\FunctionTok{round}\NormalTok{(x, }\DecValTok{2}\NormalTok{))) }\SpecialCharTok{|\textgreater{}} 
    \FunctionTok{group\_split}\NormalTok{(age\_group) }\SpecialCharTok{|\textgreater{}} 
    \FunctionTok{set\_names}\NormalTok{(}\FunctionTok{c}\NormalTok{(}\StringTok{"age\_21"}\NormalTok{, }\StringTok{"age\_25"}\NormalTok{, }\StringTok{"age\_30"}\NormalTok{))}

\NormalTok{n\_sessions\_dominance }\OtherTok{\textless{}{-}} \FunctionTok{count}\NormalTok{(participants, test\_language) }\SpecialCharTok{|\textgreater{}} 
    \FunctionTok{group\_split}\NormalTok{(test\_language) }\SpecialCharTok{|\textgreater{}} 
    \FunctionTok{set\_names}\NormalTok{(}\FunctionTok{c}\NormalTok{(}\StringTok{"catalan"}\NormalTok{, }\StringTok{"spanish"}\NormalTok{))}

\NormalTok{n\_sessions\_dominance\_age\_group }\OtherTok{\textless{}{-}} \FunctionTok{count}\NormalTok{(participants, age\_group, test\_language) }\SpecialCharTok{|\textgreater{}} 
    \FunctionTok{group\_split}\NormalTok{(test\_language) }\SpecialCharTok{|\textgreater{}} 
    \FunctionTok{set\_names}\NormalTok{(}\FunctionTok{c}\NormalTok{(}\StringTok{"catalan"}\NormalTok{, }\StringTok{"spanish"}\NormalTok{)) }\SpecialCharTok{|\textgreater{}} 
    \FunctionTok{map}\NormalTok{(\textbackslash{}(x) }\FunctionTok{group\_split}\NormalTok{(x, age\_group) }\SpecialCharTok{|\textgreater{}} 
            \FunctionTok{set\_names}\NormalTok{(}\FunctionTok{c}\NormalTok{(}\StringTok{"age\_21"}\NormalTok{, }\StringTok{"age\_25"}\NormalTok{, }\StringTok{"age\_30"}\NormalTok{)))}

\NormalTok{n\_sessions\_lp }\OtherTok{\textless{}{-}}\NormalTok{ participants }\SpecialCharTok{|\textgreater{}} 
    \FunctionTok{count}\NormalTok{(lp) }\SpecialCharTok{|\textgreater{}} 
    \FunctionTok{group\_split}\NormalTok{(lp) }\SpecialCharTok{|\textgreater{}} 
    \FunctionTok{set\_names}\NormalTok{(}\FunctionTok{c}\NormalTok{(}\StringTok{"monolingual"}\NormalTok{, }\StringTok{"bilingual"}\NormalTok{))}

\NormalTok{n\_sessions\_lp\_age\_group }\OtherTok{\textless{}{-}}\NormalTok{ participants }\SpecialCharTok{|\textgreater{}} 
    \FunctionTok{count}\NormalTok{(lp, age\_group) }\SpecialCharTok{|\textgreater{}} 
    \FunctionTok{group\_split}\NormalTok{(lp) }\SpecialCharTok{|\textgreater{}} 
    \FunctionTok{set\_names}\NormalTok{(}\FunctionTok{c}\NormalTok{(}\StringTok{"monolingual"}\NormalTok{, }\StringTok{"bilingual"}\NormalTok{)) }\SpecialCharTok{|\textgreater{}} 
    \FunctionTok{map}\NormalTok{(\textbackslash{}(x) }\FunctionTok{group\_split}\NormalTok{(x, age\_group) }\SpecialCharTok{|\textgreater{}} 
            \FunctionTok{set\_names}\NormalTok{(}\FunctionTok{c}\NormalTok{(}\StringTok{"age\_21"}\NormalTok{, }\StringTok{"age\_25"}\NormalTok{, }\StringTok{"age\_30"}\NormalTok{)))}
\end{Highlighting}
\end{Shaded}

We collected data from 180 monolingual and bilingual participants living
in the Metropolitan Area of Barcelona (Spain), who were exposed to at
least Catalan and/or Spanish from birth. Families were recruited from
maternity room in private hospitals in Barcelona, and contacted via
phone when the child's age spanned between our age intervals of
interest. Families were invited to participate at three age points: 21,
25, and 30 months. 94 participants were tested at one age point, 56 at
two age points, and 30 at the three age points. In total, we gathered
data from 296 testing sessions: 94 at 21 months (\emph{Mean} = 20.98,
\emph{SD} = 0.97, \emph{Range} = 20--26.35), 96 at 25 months
(\emph{Mean} = 25.15, \emph{SD} = 0.99, \emph{Range} = 23.35--30.77),
and 106 at 30 months (\emph{Mean} = 29.42, \emph{SD} = 0.99,
\emph{Range} = 19.37--36.07).

\hypertarget{sec-lp}{%
\subsubsection{Language profile}\label{sec-lp}}

We assessed participants' language profile using the Language Exposure
Questionnaire (LEQ, \protect\hyperlink{ref-bosch2001evidence}{Bosch \&
Sebastián-Gallés, 2001}). Before each experimental session, the
experimenter asked the caretakers to estimate the amount of hours per
day they and other people in the infant's social circle have spent
speaking to the infant in any language since birth. The output of this
interview is an estimated degree of exposure (DoE) to each language,
indicated by the proportion of time the infant was reported to have
listened to each language. According to this estimate, we classified
participants as Catalan- or Spanish-dominant if the language with
highest DoE was Catalan or Spanish, respectively, and tested the
participant in the stimuli set that contained words in their native
language. We collected data from 178 Catalan-dominant participants in
Catalan (52 at 21 months, 62 at 25 months, and 64 at 30 months). We
further classified participants as monolinguals if the DoE to their
dominant language exceeded 80\% of the total DoE to Catalan and Spanish,
and as bilinguals otherwise. Participants with DoE to language other
than Catalan or Spanish were excluded from analyses. This divided the
sample into 162 monolinguals ( at 21 months, at 25 months, and at 30
months), and 134 bilinguals ( at 21 months, 2, 2, 44 at 25 months, and
2, 3, 46 at 30 months). Table~\ref{tbl-participants-lp} shows a detailed
description of the linguistic profile of our sample.

\begin{Shaded}
\begin{Highlighting}[]
\NormalTok{participants }\SpecialCharTok{|\textgreater{}} 
    \FunctionTok{filter}\NormalTok{(is\_valid\_participant) }\SpecialCharTok{|\textgreater{}} 
    \FunctionTok{select}\NormalTok{(id, age\_group, age, lp,}
\NormalTok{           doe\_catalan, doe\_spanish, test\_language) }\SpecialCharTok{|\textgreater{}} 
    \FunctionTok{mutate}\NormalTok{(}\AttributeTok{id =} \FunctionTok{paste0}\NormalTok{(id, }\StringTok{" ("}\NormalTok{, age\_group, }\StringTok{")"}\NormalTok{)) }\SpecialCharTok{|\textgreater{}}
    \FunctionTok{add\_count}\NormalTok{(lp, }
              \AttributeTok{name =} \StringTok{"n\_lp"}\NormalTok{) }\SpecialCharTok{|\textgreater{}} 
    \FunctionTok{add\_count}\NormalTok{(age\_group, }
              \AttributeTok{name =} \StringTok{"n\_age\_group"}\NormalTok{) }\SpecialCharTok{|\textgreater{}} 
    \FunctionTok{pivot\_longer}\NormalTok{(}\FunctionTok{starts\_with}\NormalTok{(}\StringTok{"doe\_"}\NormalTok{),}
                 \AttributeTok{names\_to =} \StringTok{"language"}\NormalTok{,}
                 \AttributeTok{values\_to =} \StringTok{"doe"}\NormalTok{) }\SpecialCharTok{|\textgreater{}}
    \FunctionTok{add\_count}\NormalTok{(age\_group,}
\NormalTok{              test\_language,}
              \AttributeTok{name =} \StringTok{"n\_age\_test"}\NormalTok{) }\SpecialCharTok{|\textgreater{}} 
    \FunctionTok{mutate}\NormalTok{(}\AttributeTok{language =} \FunctionTok{str\_to\_sentence}\NormalTok{(}\FunctionTok{str\_remove\_all}\NormalTok{(language, }\StringTok{"doe\_"}\NormalTok{)),}
           \AttributeTok{age\_group =} \FunctionTok{paste0}\NormalTok{(age\_group, }\StringTok{" (N = "}\NormalTok{, n\_age\_group, }\StringTok{")"}\NormalTok{),}
           \AttributeTok{test\_language =} \FunctionTok{paste0}\NormalTok{(}\StringTok{"Tested in "}\NormalTok{, test\_language, }
                               \StringTok{" (N = "}\NormalTok{, n\_age\_test, }\StringTok{")"}\NormalTok{),}
           \AttributeTok{lp =} \FunctionTok{factor}\NormalTok{(lp, }\AttributeTok{levels =} \FunctionTok{rev}\NormalTok{(}\FunctionTok{unique}\NormalTok{(lp))))  }\SpecialCharTok{|\textgreater{}} 
    \FunctionTok{summarise}\NormalTok{(}\FunctionTok{across}\NormalTok{(}\FunctionTok{c}\NormalTok{(doe, age), }\FunctionTok{lst}\NormalTok{(mean, sd)),}
              \AttributeTok{.by =} \FunctionTok{c}\NormalTok{(age\_group, lp, test\_language, language)) }\SpecialCharTok{|\textgreater{}} 
    \FunctionTok{pivot\_wider}\NormalTok{(}\AttributeTok{id\_cols =} \FunctionTok{c}\NormalTok{(age\_group, test\_language),}
                \AttributeTok{names\_from =} \FunctionTok{c}\NormalTok{(language, lp),}
                \AttributeTok{values\_from =} \FunctionTok{c}\NormalTok{(}\FunctionTok{matches}\NormalTok{(}\StringTok{"doe"}\NormalTok{), age\_mean, age\_sd),}
                \AttributeTok{names\_repair =}\NormalTok{ janitor}\SpecialCharTok{::}\NormalTok{make\_clean\_names) }\SpecialCharTok{|\textgreater{}} 
    \FunctionTok{relocate}\NormalTok{(age\_group, test\_language,}
             \FunctionTok{matches}\NormalTok{(}\StringTok{"monolingual"}\NormalTok{),}
             \FunctionTok{matches}\NormalTok{(}\StringTok{"bilingual"}\NormalTok{)) }\SpecialCharTok{|\textgreater{}} 
    \FunctionTok{select}\NormalTok{(}\SpecialCharTok{{-}}\FunctionTok{c}\NormalTok{(age\_mean\_spanish\_monolingual,}
\NormalTok{              age\_mean\_spanish\_bilingual,}
\NormalTok{              age\_sd\_spanish\_monolingual,}
\NormalTok{              age\_sd\_spanish\_bilingual)) }\SpecialCharTok{|\textgreater{}} 
    \FunctionTok{arrange}\NormalTok{(age\_group, test\_language) }\SpecialCharTok{|\textgreater{}} 
    \FunctionTok{gt}\NormalTok{(}\AttributeTok{rowname\_col =} \StringTok{"test\_language"}\NormalTok{, }
       \AttributeTok{groupname\_col =} \StringTok{"age\_group"}\NormalTok{, }
       \AttributeTok{row\_group.sep =} \StringTok{": "}\NormalTok{) }\SpecialCharTok{|\textgreater{}} 
    \FunctionTok{tab\_spanner}\NormalTok{(}\FunctionTok{md}\NormalTok{(}\StringTok{"Monolingual (*N* = 162)"}\NormalTok{), }\FunctionTok{matches}\NormalTok{(}\StringTok{"monolingual"}\NormalTok{)) }\SpecialCharTok{|\textgreater{}} 
    \FunctionTok{tab\_spanner}\NormalTok{(}\FunctionTok{md}\NormalTok{(}\StringTok{"Bilingual (*N* = 133)"}\NormalTok{), }\FunctionTok{matches}\NormalTok{(}\StringTok{"bilingual"}\NormalTok{)) }\SpecialCharTok{|\textgreater{}}
    \FunctionTok{fmt\_number}\NormalTok{(}\FunctionTok{matches}\NormalTok{(}\StringTok{"doe"}\NormalTok{), }\AttributeTok{decimals =} \DecValTok{1}\NormalTok{, }\AttributeTok{scale\_by =} \DecValTok{100}\NormalTok{) }\SpecialCharTok{|\textgreater{}} 
    \FunctionTok{fmt\_number}\NormalTok{(}\FunctionTok{matches}\NormalTok{(}\StringTok{"age"}\NormalTok{), }\AttributeTok{decimals =} \DecValTok{1}\NormalTok{) }\SpecialCharTok{|\textgreater{}} 
    \FunctionTok{cols\_merge\_uncert}\NormalTok{(}\AttributeTok{col\_val =}\NormalTok{ age\_mean\_catalan\_monolingual, }
                      \AttributeTok{col\_uncert =}\NormalTok{ age\_sd\_catalan\_monolingual) }\SpecialCharTok{|\textgreater{}} 
    \FunctionTok{cols\_merge\_uncert}\NormalTok{(}\AttributeTok{col\_val =}\NormalTok{ age\_mean\_catalan\_bilingual, }
                      \AttributeTok{col\_uncert =}\NormalTok{ age\_sd\_catalan\_bilingual) }\SpecialCharTok{|\textgreater{}} 
    \FunctionTok{cols\_merge\_uncert}\NormalTok{(}\AttributeTok{col\_val =}\NormalTok{ doe\_mean\_catalan\_monolingual,}
                      \AttributeTok{col\_uncert =}\NormalTok{ doe\_sd\_catalan\_monolingual) }\SpecialCharTok{|\textgreater{}} 
    \FunctionTok{cols\_merge\_uncert}\NormalTok{(}\AttributeTok{col\_val =}\NormalTok{ doe\_mean\_catalan\_bilingual,}
                      \AttributeTok{col\_uncert =}\NormalTok{ doe\_sd\_catalan\_bilingual) }\SpecialCharTok{|\textgreater{}} 
    \FunctionTok{cols\_merge\_uncert}\NormalTok{(}\AttributeTok{col\_val =}\NormalTok{ doe\_mean\_spanish\_monolingual, }
                      \AttributeTok{col\_uncert =}\NormalTok{ doe\_sd\_spanish\_monolingual) }\SpecialCharTok{|\textgreater{}} 
    \FunctionTok{cols\_merge\_uncert}\NormalTok{(}\AttributeTok{col\_val =}\NormalTok{ doe\_mean\_spanish\_bilingual, }
                      \AttributeTok{col\_uncert =}\NormalTok{ doe\_sd\_spanish\_bilingual) }\SpecialCharTok{|\textgreater{}} 
    \FunctionTok{cols\_label}\NormalTok{(}\AttributeTok{age\_mean\_catalan\_monolingual =} \StringTok{"Age (months)"}\NormalTok{,}
               \AttributeTok{age\_mean\_catalan\_bilingual =} \StringTok{"Age (months)"}\NormalTok{,}
               \AttributeTok{doe\_mean\_catalan\_monolingual =} \StringTok{"Catalan (\%)"}\NormalTok{,}
               \AttributeTok{doe\_mean\_catalan\_bilingual =} \StringTok{"Catalan (\%)"}\NormalTok{,}
               \AttributeTok{doe\_mean\_spanish\_monolingual =} \StringTok{"Spanish (\%)"}\NormalTok{,}
               \AttributeTok{doe\_mean\_spanish\_bilingual =} \StringTok{"Spanish (\%)"}\NormalTok{) }\SpecialCharTok{|\textgreater{}} 
    \FunctionTok{tab\_style}\NormalTok{(}\FunctionTok{cell\_text}\NormalTok{(}\AttributeTok{weight =} \StringTok{"bold"}\NormalTok{),}
              \FunctionTok{list}\NormalTok{(}\FunctionTok{cells\_column\_spanners}\NormalTok{())) }\SpecialCharTok{|\textgreater{}} 
    \FunctionTok{tab\_style}\NormalTok{(}\FunctionTok{cell\_text}\NormalTok{(}\AttributeTok{size =} \StringTok{"medium"}\NormalTok{),}
              \FunctionTok{list}\NormalTok{(}\FunctionTok{cells\_body}\NormalTok{(),}
                 \FunctionTok{cells\_stub}\NormalTok{()))}
\end{Highlighting}
\end{Shaded}

\hypertarget{tbl-participants-lp}{}
\begin{longtable}{l|rrrrrr}
\caption{\label{tbl-participants-lp}Description of language profile of test participants. Data are
summarised for each age group, and for monolinguals and bilinguals
separately. }\tabularnewline

\toprule
\multicolumn{1}{l}{} & \multicolumn{3}{c}{Monolingual (\emph{N} = 162)} & \multicolumn{3}{c}{Bilingual (\emph{N} = 133)} \\ 
\cmidrule(lr){2-4} \cmidrule(lr){5-7}
\multicolumn{1}{l}{} & Catalan (\%) & Spanish (\%) & Age (months) & Catalan (\%) & Spanish (\%) & Age (months) \\ 
\midrule
\multicolumn{7}{l}{21 months (N = 29)} \\ 
\midrule
Tested in Catalan (N = 38) & $98.8$ ± $1.8$ & $1.2$ ± $1.8$ & $20.8$ ± $0.4$ & $61.6$ ± $7.3$ & $38.4$ ± $7.3$ & $21.1$ ± $0.4$ \\ 
Tested in Spanish (N = 20) & $10.1$ ± $6.2$ & $89.9$ ± $6.2$ & $20.8$ ± $0.5$ & $56.0$ ± $17.0$ & $44.0$ ± $17.0$ & $20.4$ ± $0.3$ \\ 
\midrule
\multicolumn{7}{l}{25 months (N = 45)} \\ 
\midrule
Tested in Catalan (N = 68) & $85.5$ ± $25.1$ & $14.5$ ± $25.1$ & $25.1$ ± $0.5$ & $58.9$ ± $12.7$ & $40.3$ ± $13.1$ & $24.7$ ± $0.6$ \\ 
Tested in Spanish (N = 22) & $45.9$ ± $44.3$ & $53.7$ ± $43.9$ & $25.5$ ± $0.4$ & $43.8$ ± $6.2$ & $55.2$ ± $6.2$ & $25.3$ ± $0.6$ \\ 
\midrule
\multicolumn{7}{l}{30 months (N = 39)} \\ 
\midrule
Tested in Catalan (N = 52) & $94.7$ ± $5.9$ & $5.3$ ± $5.8$ & $30.2$ ± $1.1$ & $59.7$ ± $8.8$ & $40.0$ ± $8.7$ & $29.6$ ± $1.3$ \\ 
Tested in Spanish (N = 26) & $9.2$ ± $7.8$ & $89.4$ ± $6.6$ & $29.7$ ± $1.0$ & $42.5$ ± $12.2$ & $55.9$ ± $13.8$ & $29.9$ ± $1.1$ \\ 
\bottomrule
\end{longtable}

\hypertarget{vocabulary-size}{%
\subsection{Vocabulary size}\label{vocabulary-size}}

\begin{Shaded}
\begin{Highlighting}[]
\NormalTok{n\_imputed }\OtherTok{\textless{}{-}} \FunctionTok{table}\NormalTok{(vocabulary}\SpecialCharTok{$}\NormalTok{is\_imputed)[}\DecValTok{2}\NormalTok{]}
\NormalTok{prop\_imputed }\OtherTok{\textless{}{-}}\NormalTok{ scales}\SpecialCharTok{::}\FunctionTok{percent}\NormalTok{(n\_imputed}\SpecialCharTok{/}\FunctionTok{nrow}\NormalTok{(vocabulary))}
\NormalTok{n\_pool }\OtherTok{\textless{}{-}} \FunctionTok{nrow}\NormalTok{(bvq\_data}\SpecialCharTok{$}\NormalTok{vocabulary)}
\end{Highlighting}
\end{Shaded}

We collected vocabulary data using parental responses to the Barcelona
Vocabulary Inventory (BVQ,
\protect\hyperlink{ref-garcia-castro2023bvq}{Garcia-Castro et al.,
2023}), an online vocabulary checklist inspired in several adaptations
of the the Communicative Developmental Inventory (CDI,
\protect\hyperlink{ref-fenson1994variability}{Fenson et al., 1994})
developed to assess the vocabulary size of Catalan-Spanish bilingual
toddlers. Families received a link to the BVQ immediately after each
experimental session, and were given two weeks to fill it.

We calculated several measures of receptive vocabulary size from each
participant's vocabulary: L1 vocabulary size (proportion of words
reported as acquired in the checklist of the dominant language), L2
vocabulary size (proportion of words reported as acquired in the
checklist of the non-dominant language), total vocabulary size
(proportion of the words in both checklists reported as acquired),
conceptual vocabulary (proportion of concepts for which the participant
was reported to have acquired at least one label, in any language), and
translation equivalent vocabulary (proportion of concepts for which the
participant was reported to have acquired at two labels, one in each
language).

141 (48\%) Families failed to provide a complete response to the BVQ
within the two-week time limit, or did not provide a successful response
to the questionnaire. For missing questionnaire responses, we imputed
the vocabulary size of the participant using single imputation, using
the vocabulary size scores of a pool of 600 additional participants for
which a successful response for the questionnaire had been gathered. We
used participants age in months and their language profile
(monolingual/bilingual) as predictors. We used the \texttt{mice} R
package (\protect\hyperlink{ref-van2011mice}{Van Buuren \&
Groothuis-Oudshoorn, 2011}) to perform imputation using the Bayesian
linear regression method.

\begin{Shaded}
\begin{Highlighting}[]
\NormalTok{vocabulary }\SpecialCharTok{|\textgreater{}} 
    \FunctionTok{left\_join}\NormalTok{(}\FunctionTok{select}\NormalTok{(participants, filename, id, age\_group, lp),}
              \AttributeTok{by =} \FunctionTok{join\_by}\NormalTok{(filename)) }\SpecialCharTok{|\textgreater{}} 
    \FunctionTok{summarise}\NormalTok{(}\FunctionTok{across}\NormalTok{(}\FunctionTok{matches}\NormalTok{(}\StringTok{"prop"}\NormalTok{), }
\NormalTok{                     tibble}\SpecialCharTok{::}\FunctionTok{lst}\NormalTok{(mean, sd)),}
              \AttributeTok{.by =} \FunctionTok{c}\NormalTok{(age\_group, lp)) }\SpecialCharTok{|\textgreater{}}
    \FunctionTok{arrange}\NormalTok{(age\_group, lp) }\SpecialCharTok{|\textgreater{}} 
    \FunctionTok{gt}\NormalTok{(}\AttributeTok{groupname\_col =} \StringTok{"age\_group"}\NormalTok{,}
       \AttributeTok{rowname\_col =} \StringTok{"lp"}\NormalTok{) }\SpecialCharTok{|\textgreater{}} 
    \FunctionTok{cols\_hide}\NormalTok{(lp) }\SpecialCharTok{|\textgreater{}} 
    \FunctionTok{tab\_spanner}\NormalTok{(}\StringTok{"Total "}\NormalTok{, }\FunctionTok{matches}\NormalTok{(}\StringTok{"total"}\NormalTok{)) }\SpecialCharTok{|\textgreater{}} 
    \FunctionTok{tab\_spanner}\NormalTok{(}\StringTok{"L1"}\NormalTok{, }\FunctionTok{matches}\NormalTok{(}\StringTok{"l1"}\NormalTok{)) }\SpecialCharTok{|\textgreater{}} 
    \FunctionTok{tab\_spanner}\NormalTok{(}\StringTok{"L2"}\NormalTok{, }\FunctionTok{matches}\NormalTok{(}\StringTok{"l2"}\NormalTok{)) }\SpecialCharTok{|\textgreater{}} 
    \FunctionTok{tab\_spanner}\NormalTok{(}\StringTok{"Conceptual"}\NormalTok{, }\FunctionTok{matches}\NormalTok{(}\StringTok{"concept"}\NormalTok{)) }\SpecialCharTok{|\textgreater{}} 
    \FunctionTok{tab\_spanner}\NormalTok{(}\StringTok{"TE"}\NormalTok{, }\FunctionTok{matches}\NormalTok{(}\StringTok{"te\_"}\NormalTok{)) }\SpecialCharTok{|\textgreater{}} 
    \FunctionTok{tab\_spanner}\NormalTok{(}\StringTok{"Vocabulary size (\%)"}\NormalTok{, }\FunctionTok{matches}\NormalTok{(}\StringTok{"prop\_"}\NormalTok{)) }\SpecialCharTok{|\textgreater{}} 
    \FunctionTok{cols\_merge\_uncert}\NormalTok{(total\_prop\_mean, total\_prop\_sd) }\SpecialCharTok{|\textgreater{}} 
    \FunctionTok{cols\_merge\_uncert}\NormalTok{(l1\_prop\_mean, l1\_prop\_sd) }\SpecialCharTok{|\textgreater{}} 
    \FunctionTok{cols\_merge\_uncert}\NormalTok{(l2\_prop\_mean, l2\_prop\_sd) }\SpecialCharTok{|\textgreater{}} 
    \FunctionTok{cols\_merge\_uncert}\NormalTok{(concept\_prop\_mean, concept\_prop\_sd) }\SpecialCharTok{|\textgreater{}} 
    \FunctionTok{cols\_merge\_uncert}\NormalTok{(te\_prop\_mean, te\_prop\_sd) }\SpecialCharTok{|\textgreater{}} 
    \FunctionTok{fmt\_number}\NormalTok{(is.numeric, }
               \AttributeTok{drop\_trailing\_zeros =} \ConstantTok{FALSE}\NormalTok{) }\SpecialCharTok{|\textgreater{}} 
    \FunctionTok{cols\_label}\NormalTok{(}\AttributeTok{total\_prop\_mean =} \StringTok{"Mean (SD)"}\NormalTok{,}
               \AttributeTok{l1\_prop\_mean =} \StringTok{"Mean (SD)"}\NormalTok{,}
               \AttributeTok{l2\_prop\_mean =} \StringTok{"Mean (SD)"}\NormalTok{,}
               \AttributeTok{concept\_prop\_mean =} \StringTok{"Mean (SD)"}\NormalTok{,}
               \AttributeTok{te\_prop\_mean =} \StringTok{"Mean (SD)"}\NormalTok{) }\SpecialCharTok{|\textgreater{}} 
    \FunctionTok{tab\_stub\_indent}\NormalTok{(}\AttributeTok{rows =} \FunctionTok{everything}\NormalTok{(),}
                    \AttributeTok{indent =} \DecValTok{2}\NormalTok{) }\SpecialCharTok{|\textgreater{}} 
    \FunctionTok{tab\_style}\NormalTok{(}\FunctionTok{cell\_text}\NormalTok{(}\AttributeTok{align =} \StringTok{"left"}\NormalTok{),}
              \FunctionTok{cells\_stub}\NormalTok{())}
\end{Highlighting}
\end{Shaded}

\hypertarget{tbl-vocabulary}{}
\begin{longtable}{l|rrrrr}
\caption{\label{tbl-vocabulary}Vocabulary sizes. Total: proportion of words in both languages marked as
\emph{Understands}. L1: proportion of words in the dominant language
marked as \emph{Understands}. L2: proportion of words in the
non-cominant language marked as \emph{Understands}. Conceptual:
proportion of translation equivalents for which at least one of the
words has ben marked as \emph{Understands}. TE: proportion of
translation equivalents for which \emph{both} words have been marked as
\emph{Understands}. }\tabularnewline

\toprule
\multicolumn{1}{l}{} & \multicolumn{5}{c}{Vocabulary size (\%)} \\ 
\cmidrule(lr){2-6}
\multicolumn{1}{l}{} & Total  & L1 & L2 & Conceptual & TE \\ 
\cmidrule(lr){2-2} \cmidrule(lr){3-3} \cmidrule(lr){4-4} \cmidrule(lr){5-5} \cmidrule(lr){6-6}
\multicolumn{1}{l}{} & Mean (SD) & Mean (SD) & Mean (SD) & Mean (SD) & Mean (SD) \\ 
\midrule
\multicolumn{6}{l}{21 months} \\ 
\midrule
\hspace*{10px} Monolingual & $0.39$ ± $0.18$ & $0.53$ ± $0.18$ & $0.26$ ± $0.22$ & $0.61$ ± $0.18$ & $0.19$ ± $0.20$ \\ 
\hspace*{10px} Bilingual & $0.42$ ± $0.23$ & $0.50$ ± $0.22$ & $0.42$ ± $0.24$ & $0.56$ ± $0.21$ & $0.33$ ± $0.23$ \\ 
\midrule
\multicolumn{6}{l}{25 months} \\ 
\midrule
\hspace*{10px} Monolingual & $0.55$ ± $0.22$ & $0.76$ ± $0.14$ & $0.48$ ± $0.29$ & $0.77$ ± $0.14$ & $0.39$ ± $0.27$ \\ 
\hspace*{10px} Bilingual & $0.64$ ± $0.20$ & $0.71$ ± $0.17$ & $0.57$ ± $0.25$ & $0.75$ ± $0.20$ & $0.51$ ± $0.24$ \\ 
\midrule
\multicolumn{6}{l}{30 months} \\ 
\midrule
\hspace*{10px} Monolingual & $0.67$ ± $0.20$ & $0.81$ ± $0.16$ & $0.53$ ± $0.28$ & $0.83$ ± $0.15$ & $0.43$ ± $0.29$ \\ 
\hspace*{10px} Bilingual & $0.76$ ± $0.18$ & $0.79$ ± $0.14$ & $0.71$ ± $0.23$ & $0.85$ ± $0.10$ & $0.64$ ± $0.21$ \\ 
\bottomrule
\end{longtable}

\hypertarget{stimuli}{%
\subsection{Stimuli}\label{stimuli}}

\begin{Shaded}
\begin{Highlighting}[]
\NormalTok{n\_words }\OtherTok{\textless{}{-}} \FunctionTok{length}\NormalTok{(}\FunctionTok{unique}\NormalTok{(}\FunctionTok{with}\NormalTok{(stimuli, }\FunctionTok{unlist}\NormalTok{(prime, target, distractor))))}
\end{Highlighting}
\end{Shaded}

We used 62 distinct words included in the BVQ to create the stimuli
lists. We created six stimuli lists: three in Catalan, and three in
Spanish. Each list contained 32 trials, each involving a
prime-target-distractor group. Each word played a role as either prime,
\emph{or} as target and distractor across the three lists in their
corresponding language. For instance, the Catalan word \emph{cadira}
appeared as \emph{prime} in the three lists, but never as \emph{target}
or \emph{distractor}; the Catalan word \emph{bici} appeared as
\emph{target} and \emph{distractor} across the three lists, but never as
a prime. Target-distractor pairings were held constant across the three
lists in each language. For instance, in all Catalan lists the word
\emph{bici} was paired with the word \emph{porta}. Target-distractor
pairings were also yoked, so that each member of the same
target-distractor pair appeared once as target and once as distractor in
each list. For instance, the \emph{bici}-\emph{porta} paired appeared
twice in each of the three Catalan lists: once with \emph{bici} as
target and \emph{porta} as distractor, and once with \emph{porta} as
target and \emph{bici} as distractor. This counterbalancing avoided
participants encountering looking at the target word guided solely by
that word having being named in a previous trial. Finally, prime words
appeared only once in each list: each target-distractor pair was
associated with a different prime word in both appearances. In each
list, the same prime word was presented alongside a different
target-distractor pair. For instance, the Catalan prime word
\emph{barret} was presented with the \emph{bici}-\emph{porta}
target-distractor pair in one list, with the \emph{bici}-\emph{porta}
pair in another list, and with \emph{berenar}-\emph{amanida} in the
remaining list. The order of the trials was randomised across
experimental session, so that each time a participant was tested, the
order in which the prime-target-distractor was presented was randomised.
Each participant was randomly assigned to one of the three lists in the
corresponding language (their dominant language, see
Section~\ref{sec-lp}), and always the same list across their
experimental sessions in the case of a recurrent participant.

In 16 of the 32 trials of the same list (henceforth \emph{related}
trials), the prime and the target words were phonologically related,
sharing phonological onset (at least first phoneme). In the other 16
trials (\emph{unrelated} trials), prime and target did not share
phonological onset. 8 of the 16 \emph{related} trials included a cognate
prime (\emph{cognate} trials), and the other 8 included a non-cognate
trials (\emph{non-cognate} trials). A prime word was considered cognate
if its Catalan and Spanish translation shared phonological onset.
Especial attention was paid to avoiding semantic or taxonomic
relationships between prime and target words, and between prime and
distractor words. Target and distractor word pairs were phonologically
unrelated (did not share phonological onset). Some of them shared
semantic features or a taxonomic relationship. This is the case of words
associated with especially salient referents such as animals or food. To
avoid infants guiding their gaze to these objects based on their
saliency, we paired animals and food items together. The position of the
target and distractor pictures (right or left) for each
target-distractor pair was alternated, so that in one list the target
would appear on the left, in another list it would appear on the right,
and so on.

We examined the overall equivalence of the three trial types by
comparing them across three variables relating to the target word:
lexical frequency, word prevalence, animacy Table~\ref{tbl-stimuli}
shows a detailed summary of the stimuli properties, broken down by trial
type and testing language. Lexical frequencies were extracted from the
Catalan and Spanish corpora of the CHILDES database
(\protect\hyperlink{ref-reference}{\textbf{reference?}};
\protect\hyperlink{ref-reference}{\textbf{reference?}}) as counts per
million words, and transformed into Zipf scores for easier
cross-language comparison
(\protect\hyperlink{ref-reference}{\textbf{reference?}};
\protect\hyperlink{ref-reference}{\textbf{reference?}}). We defined word
prevalence as the proportion of same-aged infants who were reported to
understand the word in the BVQ database.

\hypertarget{auditory-stimuli}{%
\subsubsection{Auditory stimuli}\label{auditory-stimuli}}

The auditory stimuli were natural exemplars of the selected target
words, spoken by Catalan-Spanish proficient bilingual female speaker who
was instructed to pronounce each word in a toddler-directed manner.
Recordings were made with an Audio-Tecnica 328 microphone (AT2050) at a
sampling rate of 44100 Hz, in a soundproof room at the \emph{Laboratori
de Recerca en Infancia} at University Pompeu Fabra. We used the Audacity
(\protect\hyperlink{ref-reference}{\textbf{reference?}}) and Praat
(\protect\hyperlink{ref-boersma2001speak}{Boersma \& Van Heuven, 2001})
to record and edit the audio files. The speaker was presented with a
list of words in Catalan. The order of the words was pseudo-randomised,
and each word was produced three times in a row before moving to the
next word in the list. After going through all the words in the list,
the speaker went through the word list again generating three tokens for
each word, now in an inverse order (from bottom of the list to the top).
We then repeated the same procedure for the list of Spanish words. The
resulting audios were manually chunked into individual word-forms. For
each of the six tokens produced for each word, the most adequate was
selected for further processing. The audios were then transformed to
stereo by duplicating them into two channels, denoised, and finally
normalised. The mean duration of the final audios was 1.23 (\emph{SD} =
0.17) and 1.08 (\emph{SD} = 0.14) seconds for the Catalan and Spanish
lists.

To make the pronunciation of the words as familiar as possible to each
infant, we generated additional pronunciation variants for some words in
Catalan and Spanish. Catalan words involving the /\textipa{L}/ phoneme
in their Central Catalan variant (e.g., /\textipa{'Lu.n@}) were also
recorded with such phoneme replaced by /j/ (e.g., /\textipa{'ju.n@}), a
phonological process common in the Metropolitan Area of Barcelona
(\protect\hyperlink{ref-reference}{\textbf{reference?}}). Spanish words
involving the /\textipa{T}/ phoneme were also generated replacing such
phoneme with /\textipa{s}/ to better accommodate Latin variants of
Spanish. Before every experimental session, caregivers were asked to
utter three written words involving the /\textipa{L}/ phoneme (in the
case of participants tested in Catalan) or the /\textipa{T}/ phoneme (in
the case of participants tested in Spanish). Each token contained the
critical phoneme at onset, inter-vocalic position, and coda. The
experimenter assigned the participant to the Catalan or Spanish stimuli
list involving the closest variant to that of caregivers'.

\hypertarget{visual-stimuli}{%
\subsubsection{Visual stimuli}\label{visual-stimuli}}

For each word, we created a picture with a typical referent. To avoid
competition between target and distractor pictures, semantically related
target-distractor pairs were perceptually distinct
(\protect\hyperlink{ref-floccia2020translation}{\textbf{floccia2020translation?}};
\protect\hyperlink{ref-arias-trejo2010}{\textbf{arias-trejo2010?}}).

\begin{Shaded}
\begin{Highlighting}[]
\NormalTok{stimuli }\SpecialCharTok{|\textgreater{}} 
    \FunctionTok{summarise}\NormalTok{(}\FunctionTok{across}\NormalTok{(}\FunctionTok{c}\NormalTok{(}\FunctionTok{matches}\NormalTok{(}\StringTok{"familiarity\_|freq\_|animate\_"}\NormalTok{), duration),}
\NormalTok{                     tibble}\SpecialCharTok{::}\FunctionTok{lst}\NormalTok{(mean, sd, min, max)),}
              \AttributeTok{.by =} \FunctionTok{c}\NormalTok{(test\_language, trial\_type)) }\SpecialCharTok{|\textgreater{}} 
    \FunctionTok{select}\NormalTok{(}\SpecialCharTok{{-}}\FunctionTok{matches}\NormalTok{(}\StringTok{"familiarity\_se"}\NormalTok{), }\SpecialCharTok{{-}}\FunctionTok{matches}\NormalTok{(}\StringTok{"prime"}\NormalTok{),}
           \SpecialCharTok{{-}}\FunctionTok{c}\NormalTok{(is\_animate\_target\_sd,}
\NormalTok{              is\_animate\_target\_min,}
\NormalTok{              is\_animate\_target\_max)) }\SpecialCharTok{|\textgreater{}} 
    \FunctionTok{gt}\NormalTok{(}\AttributeTok{groupname\_col =} \StringTok{"test\_language"}\NormalTok{,}
       \AttributeTok{rowname\_col =} \StringTok{"list"}\NormalTok{) }\SpecialCharTok{|\textgreater{}} 
    \FunctionTok{tab\_spanner}\NormalTok{(}\StringTok{"Prevalence (\%)"}\NormalTok{, }\FunctionTok{matches}\NormalTok{(}\StringTok{"familiarity"}\NormalTok{)) }\SpecialCharTok{|\textgreater{}} 
    \FunctionTok{tab\_spanner}\NormalTok{(}\StringTok{"Frequency (Zipf)"}\NormalTok{, }\FunctionTok{matches}\NormalTok{(}\StringTok{"freq"}\NormalTok{)) }\SpecialCharTok{|\textgreater{}}
    \FunctionTok{tab\_spanner}\NormalTok{(}\StringTok{"Animacy (\%)"}\NormalTok{, }\FunctionTok{matches}\NormalTok{(}\StringTok{"animate"}\NormalTok{)) }\SpecialCharTok{|\textgreater{}} 
    \FunctionTok{tab\_spanner}\NormalTok{(}\StringTok{"Duration (s)"}\NormalTok{, }\FunctionTok{matches}\NormalTok{(}\StringTok{"duration"}\NormalTok{)) }\SpecialCharTok{|\textgreater{}} 
    \FunctionTok{fmt\_number}\NormalTok{(is.numeric) }\SpecialCharTok{|\textgreater{}} 
    \FunctionTok{cols\_merge\_range}\NormalTok{(familiarity\_target\_min, familiarity\_target\_max) }\SpecialCharTok{|\textgreater{}} 
    \FunctionTok{cols\_merge\_range}\NormalTok{(freq\_target\_min, freq\_target\_max) }\SpecialCharTok{|\textgreater{}} 
    \FunctionTok{cols\_merge\_range}\NormalTok{(duration\_min, duration\_max) }\SpecialCharTok{|\textgreater{}}
    \FunctionTok{cols\_merge\_uncert}\NormalTok{(freq\_target\_mean, freq\_target\_sd) }\SpecialCharTok{|\textgreater{}} 
    \FunctionTok{cols\_merge\_uncert}\NormalTok{(familiarity\_target\_mean, familiarity\_target\_sd) }\SpecialCharTok{|\textgreater{}} 
    \FunctionTok{cols\_merge\_uncert}\NormalTok{(duration\_mean, duration\_sd) }\SpecialCharTok{|\textgreater{}} 
    \FunctionTok{cols\_label}\NormalTok{(}\AttributeTok{trial\_type =} \StringTok{""}\NormalTok{,}
               \AttributeTok{familiarity\_target\_mean =} \StringTok{"Mean ± SD"}\NormalTok{,}
               \CommentTok{\# familiarity\_target\_sd = "SD",}
               \AttributeTok{familiarity\_target\_min =} \StringTok{"Range"}\NormalTok{,}
               \AttributeTok{freq\_target\_mean =} \StringTok{"Mean ± SD"}\NormalTok{,}
               \CommentTok{\# freq\_target\_sd = "SD",}
               \AttributeTok{freq\_target\_min =} \StringTok{"Range"}\NormalTok{,}
               \AttributeTok{is\_animate\_target\_mean =} \StringTok{""}\NormalTok{,}
               \AttributeTok{duration\_mean =} \StringTok{"Mean ± SD"}\NormalTok{,}
               \CommentTok{\# duration\_sd = "SD",}
               \AttributeTok{duration\_min =} \StringTok{"Range"}\NormalTok{) }
\end{Highlighting}
\end{Shaded}

\hypertarget{tbl-stimuli}{}
\begin{longtable}{lrrrrrrr}
\caption{\label{tbl-stimuli}Summary of stimuli properties by trial type. }\tabularnewline

\toprule
 & \multicolumn{2}{c}{Prevalence (\%)} & \multicolumn{2}{c}{Frequency (Zipf)} & Animacy (\%) & \multicolumn{2}{c}{Duration (s)} \\ 
\cmidrule(lr){2-3} \cmidrule(lr){4-5} \cmidrule(lr){6-6} \cmidrule(lr){7-8}
 & Mean ± SD & Range & Mean ± SD & Range &  & Mean ± SD & Range \\ 
\midrule
\multicolumn{8}{l}{Catalan} \\ 
\midrule
Cognate & $0.37$ ± $0.18$ & $0.09$–$0.84$ & $3.90$ ± $0.61$ & $3.04$–$5.05$ & $0.21$ & $1.20$ ± $0.17$ & $0.86$–$1.55$ \\ 
Non-cognate & $0.38$ ± $0.16$ & $0.09$–$0.84$ & $3.90$ ± $0.52$ & $3.04$–$5.05$ & $0.25$ & $1.25$ ± $0.17$ & $0.88$–$1.55$ \\ 
Unrelated & $0.36$ ± $0.15$ & $0.09$–$0.84$ & $3.86$ ± $0.58$ & $2.94$–$5.05$ & $0.22$ & $1.23$ ± $0.17$ & $0.86$–$1.55$ \\ 
\midrule
\multicolumn{8}{l}{Spanish} \\ 
\midrule
Cognate & $0.29$ ± $0.13$ & $0.04$–$0.50$ & $4.21$ ± $0.51$ & $2.94$–$5.11$ & $0.10$ & $1.11$ ± $0.14$ & $0.85$–$1.45$ \\ 
Non-cognate & $0.30$ ± $0.14$ & $0.04$–$0.50$ & $4.27$ ± $0.53$ & $2.94$–$5.11$ & $0.15$ & $1.09$ ± $0.14$ & $0.83$–$1.45$ \\ 
Unrelated & $0.28$ ± $0.14$ & $0.04$–$0.50$ & $4.23$ ± $0.46$ & $2.94$–$5.11$ & $0.17$ & $1.07$ ± $0.15$ & $0.83$–$1.54$ \\ 
\bottomrule
\end{longtable}

\hypertarget{procedure}{%
\subsection{Procedure}\label{procedure}}

Testing took place in a sound-proof room. Participants sat on their
caregivers' lap in a dimly lit testing booth while the experimenter
conducted the experiment from outside. Caregivers were instructed to
keep their eyes shut (to avoid recording their gaze, instead of the
participant's), to be still, and to avoid interacting with the
participant verbally or non-verbally. Participants sat at approximately
65 cm from the eye-tracker and a XX-in screen of \(1929\times1080\)
screen resolution. We used a custom Matlab XXXX script using the
PsychToolbox XXX extension
(\protect\hyperlink{ref-brainard}{\textbf{brainard?}}) to present the
stimuli, and the Tobii Analytics SDK 3.0 to interact with the
eye-tracking while the experiment was running. Sampling rate was set at
120 Hz. A 5-point calibration was performed before every experimental
session, in which the picture of a colourful beach ball was presented.
We set a 55\% grey background for the calibration and stimuli
presentation. Auditory stimuli were presented through two loudspeakers
located behind the screen, one to each side. The experimenter monitored
the experimental from outside the room using a centrally located video
camera place above the screen. After a successful calibration the
experimenter triggered the onset of the first trial. Trials were
presented uninterruptedly and without intervention of the experimenter
until the 32 trials were presented, or the experimental session had to
be stopped because of the participant's behaviour.

Each trial started with the presentation of an attention getter for
3,000 milliseconds. Then, the prime picture was presented in silence in
the centre of the screen for 1,500 milliseconds. Fifty milliseconds
after the offset of the prime image, an auditory label was played from
the loudspeakers and, 700 milliseconds after the onset of the auditory
label, the target and distractor pictures were presented side-by-side
during 1,000 milliseconds until the end of the trial. After this, the
attention getter of the next trial was immediately presented. Each
experimental session took approximately 10 minutes.

\hypertarget{data-analysis}{%
\subsection{Data analysis}\label{data-analysis}}

We defined our time window of interest from 300 ms after the onset of
the test phase (target and distractor presentation) until the end of the
test phase (2,000 ms). For each trial, we chunked the time domain into
17 time bins of 100 ms of duration. We then calculated, for each
experimental session, time bin, and condition, participant's proportion
of target and distractor fixations. Finally, we computed the empirical
logit of target fixations, which we introduced in the statistical
analyses as our response variable. Missing eye-tracker samples were
interpolated using the last-observation-carried-forward (see
\protect\hyperlink{ref-zettersten2022peekbank}{Zettersten et al., 2022}
for a similar approach).

We conducted two main analyses. First, we estimated the effect of
phonological priming on participants' target looking, comparing
\emph{related} trials with \emph{unrelated} trials. This analysis
included all trials on the data set. Second, we estimated the effect of
cognateness on phonological priming, comparing \emph{cognate} with
\emph{non-cognate} trials, leaving out \emph{unrelated} trials. In both
analyses, we used General Additive Mixed Models (GAMMs) to model the
probability of target fixations across the time course of the trial
using a normal distribution.

In the first analysis. We included \emph{Relatedness} (\texttt{Related}
vs.~\texttt{Unrelated}, sum-coded as \texttt{-0.5} and \texttt{+0.5}),
\emph{Group} (\texttt{Monolingual} vs.~\texttt{Bilingual}, sum-coded as
\texttt{-0.5} and \texttt{+0.5}), and \emph{Age} (participants'
standardised age in months) as fixed, main effects. We also included
cubic regression splines for the main effect of \emph{Time}, and one for
an adjustment of the previous cubic spline by \emph{Group}
(\protect\hyperlink{ref-wood2017generalized}{Wood, 2017}). For both
splines, we specified \(k = 10\) basis functions or \emph{knots}--half
the number of time bins, for computational convenience. Finally, we
added by-participant random intercepts, and random slopes for the main
effect of \emph{Relatedness} and the main effect of \emph{Age}, both
including repeated measures per participant.

To test the contribution of each of the predictors of
interest--\emph{Relatedness}/\emph{Cognateness}, and \emph{Group}--, we
compared each model (\(\mathcal{M_0}\)) against a simplified model
dropping each of the main effects, \emph{Relatedness}/\emph{Cognateness}
(\(\mathcal{M}_1\)) or \emph{Group} (\(\mathcal{M_2}\)). In both
simplified models, the interaction term was dropped. We used
leave-one-out cross-validation (LOO-CV) as a benchmark of model
performance, using Pareto-smoothed importance sampling (PSIS) to
approximate it. We then examined the posterior predictions of the
best-performing model for interpretation.

\[
\begin{aligned}
\textbf{Likelihood:} \\
y_i &\sim \mathcal{N}(\mu_i, \sigma_i) \\ \\
\textbf{Linear model} \\
\text{logit}(\mu_i) &= (\beta_0 + u _{0_{i}}) + (\beta_1 + u _{1_{i}}) \cdot \text{Relatedness} + \beta_{2} \cdot \text{Group} + \\
&\beta_{3} \cdot (\text{Relatedness} \times \text{Group}) + (\beta_4 + u_{3_{i}}) \cdot \text{Age} + \\
&\sum_{j = 1}^k b_{j_{1}}(\beta_{5_{k}} + u_{4_{i}}) \cdot \text{Time} + \\
&\sum_{j = 1}^k b_{j_{1}} (\beta_{6_{k }} + u_{5_{i}}) \cdot (\text{Time} \times \text{Group}) \\
\text{where:} \\
&k \text{ is the number of knots in the spline (10)} \\
\textbf{Prior:} \\
\beta_{0-6} &\sim \mathcal{N}(0, 1) \\
b_{0-1} &\sim MVN(0, 1) \\
\sigma_i &\sim Exp(4) 
\end{aligned}
\]

\hypertarget{results}{%
\section{Results}\label{results}}

We know present the results under four different trial-level inclusion
criteria In Analysis 1, the child is required to understand the prime
\emph{and} the target words, on top of having to fixate both target and
distractor for at last 50 ms each during the target-distractor phase. In
Analysis 2, the child is required to understand the target word, on top
of having to fixate both target and distractor for at last 50 ms each
during the target-distractor phase. In Analysis 3, the child is not
required to understand any word, but is still required to fixate both
target and distractor for at last 50 ms each during the
target-distractor phase. In Analysis 4, the child is not required to
understand any word, nor to fixate at both target and distractor.

For reference:

\begin{itemize}
\tightlist
\item
  (\protect\hyperlink{ref-floccia2020}{\textbf{floccia2020?}}) requires
  participants to know prime and target, but not to look at target
  \emph{and} distractor (looking at at least one of them is enough).
\end{itemize}

The fact that in some time course analyses participants seem to be
looking predominantly towards one of the objects (despite the target and
distractor pictures having been presented less than 300 ms before) is a
common finding in previous studies using preferential looking procedures
(e.g.,
\protect\hyperlink{ref-floccia2020translation}{\textbf{floccia2020translation?}}).

\begin{Shaded}
\begin{Highlighting}[]
\NormalTok{tibble}\SpecialCharTok{::}\FunctionTok{lst}\NormalTok{(data\_time\_related,}
\NormalTok{            data\_time\_related\_vtarget,}
\NormalTok{            data\_time\_related\_vnone,}
\NormalTok{            data\_time\_related\_noeach,}
\NormalTok{            data\_time\_related\_vnoeach) }\SpecialCharTok{|\textgreater{}} 
    \FunctionTok{bind\_rows}\NormalTok{(}\AttributeTok{.id =} \StringTok{"analysis"}\NormalTok{) }\SpecialCharTok{|\textgreater{}}
    \FunctionTok{mutate}\NormalTok{(}\AttributeTok{analysis =} \FunctionTok{case\_when}\NormalTok{(}
\NormalTok{        analysis}\SpecialCharTok{==}\StringTok{"data\_time\_related"} \SpecialCharTok{\textasciitilde{}} \StringTok{"Analysis 1"}\NormalTok{,}
\NormalTok{        analysis}\SpecialCharTok{==}\StringTok{"data\_time\_related\_vtarget"} \SpecialCharTok{\textasciitilde{}} \StringTok{"Analysis 2"}\NormalTok{,}
\NormalTok{        analysis}\SpecialCharTok{==}\StringTok{"data\_time\_related\_vnone"} \SpecialCharTok{\textasciitilde{}} \StringTok{"Analysis 3"}\NormalTok{,}
\NormalTok{        analysis}\SpecialCharTok{==}\StringTok{"data\_time\_related\_noeach"} \SpecialCharTok{\textasciitilde{}} \StringTok{"Analysis 4"}\NormalTok{,}
\NormalTok{        analysis}\SpecialCharTok{==}\StringTok{"data\_time\_related\_vnoeach"} \SpecialCharTok{\textasciitilde{}} \StringTok{"Analysis 5"}
\NormalTok{    )) }\SpecialCharTok{|\textgreater{}} 
    \CommentTok{\# summarise(.elog\_mean = mean(.elog),}
    \CommentTok{\#         n = n(),}
    \CommentTok{\#         .elog\_std = sd(.elog),}
    \CommentTok{\#         .by = c(condition, timebin, analysis, lp)) |\textgreater{}}
    \CommentTok{\# mutate(.elog\_se = .elog\_std/n) |\textgreater{} }
    \FunctionTok{ggplot}\NormalTok{(}\FunctionTok{aes}\NormalTok{(timebin, .elog,}
               \AttributeTok{colour =}\NormalTok{ condition,}
               \AttributeTok{fill =}\NormalTok{ condition,}
               \AttributeTok{shape =}\NormalTok{ condition)) }\SpecialCharTok{+}
    \FunctionTok{facet\_grid}\NormalTok{(analysis}\SpecialCharTok{\textasciitilde{}}\NormalTok{lp) }\SpecialCharTok{+}
    \FunctionTok{geom\_hline}\NormalTok{(}\AttributeTok{yintercept =} \FloatTok{0.5}\NormalTok{, }
               \AttributeTok{linewidth =} \DecValTok{1}\SpecialCharTok{/}\DecValTok{2}\NormalTok{,}
               \AttributeTok{colour =} \StringTok{"black"}\NormalTok{,}
               \AttributeTok{linetype =} \StringTok{"dotted"}\NormalTok{) }\SpecialCharTok{+}
    \FunctionTok{geom\_smooth}\NormalTok{(}\AttributeTok{method =}\NormalTok{ lm, }
                \AttributeTok{se =} \ConstantTok{FALSE}\NormalTok{, }
                \AttributeTok{formula =}\NormalTok{ y }\SpecialCharTok{\textasciitilde{}}\NormalTok{ splines}\SpecialCharTok{::}\FunctionTok{bs}\NormalTok{(x, }\DecValTok{4}\NormalTok{)) }\SpecialCharTok{+}
    \CommentTok{\# geom\_errorbar(aes(ymin = .elog\_mean {-} .elog\_se*1.96, }
    \CommentTok{\#                 ymax = .elog\_mean + .elog\_se*1.96),}
    \CommentTok{\#             width = 0.25,}
    \CommentTok{\#             linewidth = 3/4) + }
    \CommentTok{\# geom\_line(size = 3/4) + }
    \FunctionTok{stat\_summary}\NormalTok{(}\AttributeTok{fun =}\NormalTok{ mean, }\AttributeTok{geom =} \StringTok{"point"}\NormalTok{) }\SpecialCharTok{+}
    \FunctionTok{stat\_summary}\NormalTok{(}\AttributeTok{fun.data =}\NormalTok{ mean\_se,}
                 \AttributeTok{geom =} \StringTok{"errorbar"}\NormalTok{,}
                 \AttributeTok{width =} \FloatTok{0.25}\NormalTok{) }\SpecialCharTok{+}
    \FunctionTok{labs}\NormalTok{(}\AttributeTok{x =} \StringTok{"Time (ms)"}\NormalTok{,}
         \AttributeTok{y =} \StringTok{"P(Target looking)"}\NormalTok{,}
         \AttributeTok{colour =} \StringTok{"Condition"}\NormalTok{,}
         \AttributeTok{fill =} \StringTok{"Condition"}\NormalTok{,}
         \AttributeTok{linetype =} \StringTok{"Condition"}\NormalTok{,}
         \AttributeTok{shape =} \StringTok{"Condition"}\NormalTok{) }\SpecialCharTok{+}
    \FunctionTok{scale\_linetype\_manual}\NormalTok{(}\AttributeTok{values =} \FunctionTok{rev}\NormalTok{(}\FunctionTok{c}\NormalTok{(}\StringTok{"solid"}\NormalTok{, }\StringTok{"dashed"}\NormalTok{))) }\SpecialCharTok{+}
    \CommentTok{\# scale\_shape\_manual(values = c(1, 2)) +}
    \FunctionTok{scale\_x\_continuous}\NormalTok{(}\AttributeTok{labels =}\NormalTok{ \textbackslash{}(x) }\FunctionTok{format}\NormalTok{((x }\SpecialCharTok{*} \FloatTok{1e2}\NormalTok{)}\SpecialCharTok{+}\DecValTok{300}\NormalTok{, }
                                            \AttributeTok{big.mark =} \StringTok{","}\NormalTok{)) }\SpecialCharTok{+}
    \FunctionTok{theme}\NormalTok{(}\AttributeTok{legend.title =} \FunctionTok{element\_blank}\NormalTok{(),}
          \AttributeTok{axis.title.x =} \FunctionTok{element\_blank}\NormalTok{()) }
\end{Highlighting}
\end{Shaded}

\begin{figure}[H]

{\centering \includegraphics{manuscript_files/figure-pdf/fig-related-summary-1.pdf}

}

\caption{\label{fig-related-summary}Marginal posterior predictions of
the GAMMs. (A) Mean posterior probability of target looking across the
time course of the trial. Black lines and intervals indicate the
psoterior mean and 95\% credible intervals. Points indicate the mean
probability of target looking across participants. (B) Difference in
posterior probability of target looking between \emph{related} and
\emph{unrelated} trials. The yellow rectangle indicates, in both A and
B, the range of time points in which the 95\% credible interval of the
differences excluded zero.}

\end{figure}

\begin{Shaded}
\begin{Highlighting}[]
\NormalTok{tibble}\SpecialCharTok{::}\FunctionTok{lst}\NormalTok{(data\_time\_cognate,}
\NormalTok{            data\_time\_cognate\_vtarget,}
\NormalTok{            data\_time\_cognate\_vnone,}
\NormalTok{            data\_time\_cognate\_noeach,}
\NormalTok{            data\_time\_cognate\_vnoeach) }\SpecialCharTok{|\textgreater{}} 
    \FunctionTok{bind\_rows}\NormalTok{(}\AttributeTok{.id =} \StringTok{"analysis"}\NormalTok{) }\SpecialCharTok{|\textgreater{}}
    \FunctionTok{mutate}\NormalTok{(}\AttributeTok{analysis =} \FunctionTok{case\_when}\NormalTok{(}
\NormalTok{        analysis}\SpecialCharTok{==}\StringTok{"data\_time\_cognate"} \SpecialCharTok{\textasciitilde{}} \StringTok{"Analysis 1"}\NormalTok{,}
\NormalTok{        analysis}\SpecialCharTok{==}\StringTok{"data\_time\_cognate\_vtarget"} \SpecialCharTok{\textasciitilde{}} \StringTok{"Analysis 2"}\NormalTok{,}
\NormalTok{        analysis}\SpecialCharTok{==}\StringTok{"data\_time\_cognate\_vnone"} \SpecialCharTok{\textasciitilde{}} \StringTok{"Analysis 3"}\NormalTok{,}
\NormalTok{        analysis}\SpecialCharTok{==}\StringTok{"data\_time\_cognate\_noeach"} \SpecialCharTok{\textasciitilde{}} \StringTok{"Analysis 4"}\NormalTok{,}
\NormalTok{        analysis}\SpecialCharTok{==}\StringTok{"data\_time\_cognate\_vnoeach"} \SpecialCharTok{\textasciitilde{}} \StringTok{"Analysis 5"}
\NormalTok{    )) }\SpecialCharTok{|\textgreater{}} 
    \CommentTok{\# summarise(.elog\_mean = mean(.elog),}
    \CommentTok{\#         n = n(),}
    \CommentTok{\#         .elog\_std = sd(.elog),}
    \CommentTok{\#         .by = c(condition, timebin, analysis, lp)) |\textgreater{}}
    \CommentTok{\# mutate(.elog\_se = .elog\_std/n) |\textgreater{} }
    \FunctionTok{ggplot}\NormalTok{(}\FunctionTok{aes}\NormalTok{(timebin, .elog,}
               \AttributeTok{colour =}\NormalTok{ condition,}
               \AttributeTok{fill =}\NormalTok{ condition,}
               \AttributeTok{shape =}\NormalTok{ condition)) }\SpecialCharTok{+}
    \FunctionTok{facet\_grid}\NormalTok{(analysis}\SpecialCharTok{\textasciitilde{}}\NormalTok{lp) }\SpecialCharTok{+}
    \FunctionTok{geom\_hline}\NormalTok{(}\AttributeTok{yintercept =} \FloatTok{0.5}\NormalTok{, }
               \AttributeTok{linewidth =} \DecValTok{1}\SpecialCharTok{/}\DecValTok{2}\NormalTok{,}
               \AttributeTok{colour =} \StringTok{"black"}\NormalTok{,}
               \AttributeTok{linetype =} \StringTok{"dotted"}\NormalTok{) }\SpecialCharTok{+}
    \FunctionTok{geom\_smooth}\NormalTok{(}\AttributeTok{method =}\NormalTok{ lm, }
                \AttributeTok{se =} \ConstantTok{FALSE}\NormalTok{, }
                \AttributeTok{formula =}\NormalTok{ y }\SpecialCharTok{\textasciitilde{}}\NormalTok{ splines}\SpecialCharTok{::}\FunctionTok{bs}\NormalTok{(x, }\DecValTok{4}\NormalTok{)) }\SpecialCharTok{+}
    \CommentTok{\# geom\_errorbar(aes(ymin = .elog\_mean {-} .elog\_se*1.96, }
    \CommentTok{\#                 ymax = .elog\_mean + .elog\_se*1.96),}
    \CommentTok{\#             width = 0.25,}
    \CommentTok{\#             linewidth = 3/4) + }
    \CommentTok{\# geom\_line(size = 3/4) + }
    \FunctionTok{stat\_summary}\NormalTok{(}\AttributeTok{fun =}\NormalTok{ mean, }\AttributeTok{geom =} \StringTok{"point"}\NormalTok{) }\SpecialCharTok{+}
    \FunctionTok{stat\_summary}\NormalTok{(}\AttributeTok{fun.data =}\NormalTok{ mean\_se,}
                 \AttributeTok{geom =} \StringTok{"errorbar"}\NormalTok{,}
                 \AttributeTok{width =} \FloatTok{0.25}\NormalTok{) }\SpecialCharTok{+}
    \FunctionTok{labs}\NormalTok{(}\AttributeTok{x =} \StringTok{"Time (ms)"}\NormalTok{,}
         \AttributeTok{y =} \StringTok{"P(Target looking)"}\NormalTok{,}
         \AttributeTok{colour =} \StringTok{"Condition"}\NormalTok{,}
         \AttributeTok{fill =} \StringTok{"Condition"}\NormalTok{,}
         \AttributeTok{linetype =} \StringTok{"Condition"}\NormalTok{,}
         \AttributeTok{shape =} \StringTok{"Condition"}\NormalTok{) }\SpecialCharTok{+}
    \FunctionTok{scale\_linetype\_manual}\NormalTok{(}\AttributeTok{values =} \FunctionTok{rev}\NormalTok{(}\FunctionTok{c}\NormalTok{(}\StringTok{"solid"}\NormalTok{, }\StringTok{"dashed"}\NormalTok{))) }\SpecialCharTok{+}
    \CommentTok{\# scale\_shape\_manual(values = c(1, 2)) +}
    \FunctionTok{scale\_x\_continuous}\NormalTok{(}\AttributeTok{labels =}\NormalTok{ \textbackslash{}(x) }\FunctionTok{format}\NormalTok{((x }\SpecialCharTok{*} \FloatTok{1e2}\NormalTok{)}\SpecialCharTok{+}\DecValTok{300}\NormalTok{, }
                                            \AttributeTok{big.mark =} \StringTok{","}\NormalTok{)) }\SpecialCharTok{+}
    \FunctionTok{theme}\NormalTok{(}\AttributeTok{legend.title =} \FunctionTok{element\_blank}\NormalTok{(),}
          \AttributeTok{axis.title.x =} \FunctionTok{element\_blank}\NormalTok{()) }
\end{Highlighting}
\end{Shaded}

\begin{figure}[H]

{\centering \includegraphics{manuscript_files/figure-pdf/fig-cognate-summary-1.pdf}

}

\caption{\label{fig-cognate-summary}Marginal posterior predictions of
the GAMMs. (A) Mean posterior probability of target looking across the
time course of the trial. Black lines and intervals indicate the
psoterior mean and 95\% credible intervals. Points indicate the mean
probability of target looking across participants. (B) Difference in
posterior probability of target looking between \emph{related} and
\emph{unrelated} trials. The yellow rectangle indicates, in both A and
B, the range of time points in which the 95\% credible interval of the
differences excluded zero.}

\end{figure}

\hypertarget{analysis-1}{%
\subsection{Analysis 1}\label{analysis-1}}

Participants must know \textbf{prime} \emph{and} \textbf{target} words,
and must look at least 10 ms to each target and distractor.

\begin{Shaded}
\begin{Highlighting}[]
\NormalTok{n\_trials }\OtherTok{\textless{}{-}} \FunctionTok{nrow}\NormalTok{(attrition\_trials)}
\NormalTok{n\_total }\OtherTok{\textless{}{-}} \FunctionTok{n\_distinct}\NormalTok{(attrition\_trials}\SpecialCharTok{$}\NormalTok{id)}
\NormalTok{n\_trials\_valid }\OtherTok{\textless{}{-}} \FunctionTok{inner\_join}\NormalTok{(attrition\_participants,}
\NormalTok{                             attrition\_trials,}
                             \AttributeTok{by =} \FunctionTok{join\_by}\NormalTok{(id)) }\SpecialCharTok{|\textgreater{}} 
    \FunctionTok{filter}\NormalTok{(is\_valid\_participant) }\SpecialCharTok{|\textgreater{}} 
    \FunctionTok{pull}\NormalTok{(is\_valid\_trial) }\SpecialCharTok{|\textgreater{}} 
    \FunctionTok{sum}\NormalTok{()}
\NormalTok{n\_participants\_valid }\OtherTok{\textless{}{-}} \FunctionTok{sum}\NormalTok{(attrition\_participants}\SpecialCharTok{$}\NormalTok{is\_valid\_participant)}
\NormalTok{n\_exc\_prime }\OtherTok{\textless{}{-}} \FunctionTok{sum}\NormalTok{(}\SpecialCharTok{!}\NormalTok{attrition\_trials}\SpecialCharTok{$}\NormalTok{is\_valid\_gaze\_prime)}
\NormalTok{n\_exc\_test }\OtherTok{\textless{}{-}} \FunctionTok{sum}\NormalTok{(}\SpecialCharTok{!}\NormalTok{attrition\_trials}\SpecialCharTok{$}\NormalTok{is\_valid\_gaze\_test)}
\NormalTok{n\_exc\_test\_each }\OtherTok{\textless{}{-}} \FunctionTok{sum}\NormalTok{(}\SpecialCharTok{!}\NormalTok{attrition\_trials}\SpecialCharTok{$}\NormalTok{is\_valid\_gaze\_test\_each)}
\NormalTok{n\_exc\_vocab }\OtherTok{\textless{}{-}} \FunctionTok{sum}\NormalTok{(}\SpecialCharTok{!}\NormalTok{attrition\_trials}\SpecialCharTok{$}\NormalTok{is\_valid\_vocab)}
\NormalTok{n\_exc\_cognate }\OtherTok{\textless{}{-}} \FunctionTok{sum}\NormalTok{(}\SpecialCharTok{!}\NormalTok{attrition\_participants}\SpecialCharTok{$}\NormalTok{is\_valid\_cognate)}
\NormalTok{n\_exc\_noncognate }\OtherTok{\textless{}{-}} \FunctionTok{sum}\NormalTok{(}\SpecialCharTok{!}\NormalTok{attrition\_participants}\SpecialCharTok{$}\NormalTok{is\_valid\_noncognate)}
\NormalTok{n\_exc\_unrelated }\OtherTok{\textless{}{-}} \FunctionTok{sum}\NormalTok{(}\SpecialCharTok{!}\NormalTok{attrition\_participants}\SpecialCharTok{$}\NormalTok{is\_valid\_unrelated)}

\NormalTok{n\_longitudinal }\OtherTok{\textless{}{-}}\NormalTok{ attrition\_participants }\SpecialCharTok{|\textgreater{}}
    \FunctionTok{filter}\NormalTok{(is\_valid\_participant) }\SpecialCharTok{|\textgreater{}}
    \FunctionTok{count}\NormalTok{(id, }\AttributeTok{name =} \StringTok{"times"}\NormalTok{) }\SpecialCharTok{|\textgreater{}} 
    \FunctionTok{count}\NormalTok{(times)}
\end{Highlighting}
\end{Shaded}

We gathered data from 9,472 trials from 180 distinct participants. We
excluded trials in which participants failed to provide 50\% valid
eye-tracking samples during the prime phase (\emph{n} = 1,810) or during
the target-distractor phase (\emph{n} = 1,493). We also excluded trials
in which participants did not provide at least 5\% of valid samples to
\emph{both} target and distractor in the test phase (\emph{n} = 2,793).
Finally, we excluded trials in which participants did not understand the
prime \emph{or} the target word, according to a supplementary vocabulary
checklist filled by their caregivers upon experiment completion
(\emph{n} = 5,884). After applying these trial-level inclusion criteria,
we excluded participants who did not provide at least two valid trials
in the \emph{cognate prime} condition (\emph{n} = 173), the
\emph{non-cognate prime} condition (\emph{n} = 168), or the
\emph{unrelated prime} condition (\emph{n} = 157). The resulting dataset
included 3,393 trials from 113 participants. Of those participants, 60
provided data from one experimental session, 19 provided data from two
experimental sessions, and 5 provided data from three experimental
sessions. Table~\ref{tbl-attrition-trials} shows a detailed description
of the trial attrition.

\begin{Shaded}
\begin{Highlighting}[]
\NormalTok{attrition\_trials }\SpecialCharTok{|\textgreater{}} 
    \FunctionTok{filter}\NormalTok{(id }\SpecialCharTok{\%in\%}\NormalTok{ attrition\_participants}\SpecialCharTok{$}\NormalTok{id[attrition\_participants}\SpecialCharTok{$}\NormalTok{is\_valid\_participant]) }\SpecialCharTok{|\textgreater{}} 
    \FunctionTok{left\_join}\NormalTok{(}\FunctionTok{select}\NormalTok{(participants, filename, id, age\_group),}
              \AttributeTok{by =} \FunctionTok{join\_by}\NormalTok{(filename, id, age\_group)) }\SpecialCharTok{|\textgreater{}} 
    \FunctionTok{summarise}\NormalTok{(}\AttributeTok{n\_valid =} \FunctionTok{sum}\NormalTok{(is\_valid\_trial),}
              \AttributeTok{n\_total =} \FunctionTok{n}\NormalTok{(),}
              \AttributeTok{.by =} \FunctionTok{c}\NormalTok{(id, age\_group, trial\_type)) }\SpecialCharTok{|\textgreater{}} 
    \FunctionTok{summarise}\NormalTok{(}\FunctionTok{across}\NormalTok{(n\_valid, }\FunctionTok{lst}\NormalTok{(sum, mean, sd),}
                     \AttributeTok{.names =} \StringTok{"\{.fn\}"}\NormalTok{),}
              \AttributeTok{n\_total =} \FunctionTok{sum}\NormalTok{(n\_total),}
              \AttributeTok{.by =} \FunctionTok{c}\NormalTok{(age\_group, trial\_type)) }\SpecialCharTok{|\textgreater{}} 
    \FunctionTok{mutate}\NormalTok{(}\AttributeTok{n\_excluded =}\NormalTok{ n\_total}\SpecialCharTok{{-}}\NormalTok{sum) }\SpecialCharTok{|\textgreater{}} 
    \FunctionTok{select}\NormalTok{(}\SpecialCharTok{{-}}\FunctionTok{c}\NormalTok{(n\_total)) }\SpecialCharTok{|\textgreater{}} 
    \FunctionTok{pivot\_wider}\NormalTok{(}\AttributeTok{names\_from =}\NormalTok{ trial\_type,}
                \AttributeTok{values\_from =} \FunctionTok{c}\NormalTok{(sum}\SpecialCharTok{:}\NormalTok{sd, n\_excluded),}
                \AttributeTok{names\_repair =}\NormalTok{ janitor}\SpecialCharTok{::}\NormalTok{make\_clean\_names) }\SpecialCharTok{|\textgreater{}} 
    \FunctionTok{rename\_with}\NormalTok{(\textbackslash{}(x) }\FunctionTok{gsub}\NormalTok{(}\StringTok{"non\_cognate"}\NormalTok{, }
                          \StringTok{"noncognate"}\NormalTok{,}
\NormalTok{                          x)) }\SpecialCharTok{|\textgreater{}} 
    \FunctionTok{arrange}\NormalTok{(age\_group) }\SpecialCharTok{|\textgreater{}} 
    \FunctionTok{relocate}\NormalTok{(age\_group,}
             \FunctionTok{matches}\NormalTok{(}\StringTok{"\_cognate"}\NormalTok{),}
             \FunctionTok{matches}\NormalTok{(}\StringTok{"noncognate"}\NormalTok{)) }\SpecialCharTok{|\textgreater{}} 
    \FunctionTok{gt}\NormalTok{(}\AttributeTok{rowname\_col =} \StringTok{"age\_group"}\NormalTok{) }\SpecialCharTok{|\textgreater{}} 
    \FunctionTok{grand\_summary\_rows}\NormalTok{(}\AttributeTok{columns =} \FunctionTok{matches}\NormalTok{(}\StringTok{"mean\_"}\NormalTok{),}
                       \AttributeTok{fns =} \FunctionTok{lst}\NormalTok{(Mean }\SpecialCharTok{\textasciitilde{}} \FunctionTok{mean}\NormalTok{(.)),}
                       \AttributeTok{fmt =} \SpecialCharTok{\textasciitilde{}}\FunctionTok{fmt\_number}\NormalTok{(.)) }\SpecialCharTok{|\textgreater{}}
    \FunctionTok{grand\_summary\_rows}\NormalTok{(}\AttributeTok{columns =} \FunctionTok{matches}\NormalTok{(}\StringTok{"sum\_"}\NormalTok{),}
                       \AttributeTok{fns =} \FunctionTok{lst}\NormalTok{(Sum }\SpecialCharTok{\textasciitilde{}} \FunctionTok{sum}\NormalTok{(.)),}
                       \AttributeTok{fmt =} \SpecialCharTok{\textasciitilde{}}\FunctionTok{fmt\_integer}\NormalTok{(.)) }\SpecialCharTok{|\textgreater{}}
    \FunctionTok{cols\_merge}\NormalTok{(}\FunctionTok{c}\NormalTok{(sum\_cognate, n\_excluded\_cognate), }
               \AttributeTok{pattern =} \StringTok{"\{1\} (\{2\})"}\NormalTok{) }\SpecialCharTok{|\textgreater{}} 
    \FunctionTok{cols\_merge}\NormalTok{(}\FunctionTok{c}\NormalTok{(sum\_noncognate, n\_excluded\_noncognate), }
               \AttributeTok{pattern =} \StringTok{"\{1\} (\{2\})"}\NormalTok{) }\SpecialCharTok{|\textgreater{}} 
    \FunctionTok{cols\_merge}\NormalTok{(}\FunctionTok{c}\NormalTok{(sum\_unrelated, n\_excluded\_unrelated),}
               \AttributeTok{pattern =} \StringTok{"\{1\} (\{2\})"}\NormalTok{) }\SpecialCharTok{|\textgreater{}} 
    \FunctionTok{cols\_merge\_uncert}\NormalTok{(mean\_cognate, sd\_cognate) }\SpecialCharTok{|\textgreater{}} 
    \FunctionTok{cols\_merge\_uncert}\NormalTok{(mean\_noncognate, sd\_noncognate) }\SpecialCharTok{|\textgreater{}} 
    \FunctionTok{cols\_merge\_uncert}\NormalTok{(mean\_unrelated, sd\_unrelated) }\SpecialCharTok{|\textgreater{}} 
    \FunctionTok{tab\_spanner}\NormalTok{(}\StringTok{"Cognate trials"}\NormalTok{, }\FunctionTok{ends\_with}\NormalTok{(}\StringTok{"\_cognate"}\NormalTok{)) }\SpecialCharTok{|\textgreater{}}
    \FunctionTok{tab\_spanner}\NormalTok{(}\StringTok{"Non{-}cognate trials"}\NormalTok{, }\FunctionTok{ends\_with}\NormalTok{(}\StringTok{"noncognate"}\NormalTok{)) }\SpecialCharTok{|\textgreater{}} 
    \FunctionTok{tab\_spanner}\NormalTok{(}\StringTok{"Unrelated trials"}\NormalTok{, }\FunctionTok{ends\_with}\NormalTok{(}\StringTok{"unrelated"}\NormalTok{)) }\SpecialCharTok{|\textgreater{}} 
    \FunctionTok{tab\_spanner}\NormalTok{(}\StringTok{"Related trials"}\NormalTok{, }\FunctionTok{matches}\NormalTok{(}\StringTok{"cognate"}\NormalTok{)) }\SpecialCharTok{|\textgreater{}}
    \FunctionTok{fmt\_number}\NormalTok{(}\FunctionTok{matches}\NormalTok{(}\StringTok{"mean|sd"}\NormalTok{)) }\SpecialCharTok{|\textgreater{}} 
    \FunctionTok{fmt\_integer}\NormalTok{(}\FunctionTok{matches}\NormalTok{(}\StringTok{"sum"}\NormalTok{), }\AttributeTok{sep\_mark =} \StringTok{","}\NormalTok{) }\SpecialCharTok{|\textgreater{}} 
    \FunctionTok{cols\_label}\NormalTok{(}\AttributeTok{sum\_cognate =} \StringTok{"N"}\NormalTok{,}
               \AttributeTok{sum\_noncognate =} \StringTok{"N"}\NormalTok{,}
               \AttributeTok{sum\_unrelated =} \StringTok{"N"}\NormalTok{,}
               \AttributeTok{mean\_cognate =} \StringTok{"Mean"}\NormalTok{,}
               \AttributeTok{mean\_noncognate =} \StringTok{"Mean"}\NormalTok{,}
               \AttributeTok{mean\_unrelated =} \StringTok{"Mean"}\NormalTok{)}
\end{Highlighting}
\end{Shaded}

\hypertarget{tbl-attrition-trials}{}
\begin{longtable}{l|rrrrrr}
\caption{\label{tbl-attrition-trials}Trial attrition rate by condition for included participants. Additional
excluded trials are indicated between parentheses. }\tabularnewline

\toprule
\multicolumn{1}{l}{} & \multicolumn{4}{c}{Related trials} &  &  \\ 
\cmidrule(lr){2-5}
\multicolumn{1}{l}{} & \multicolumn{2}{c}{Cognate trials} & \multicolumn{2}{c}{Non-cognate trials} & \multicolumn{2}{c}{Unrelated trials} \\ 
\cmidrule(lr){2-3} \cmidrule(lr){4-5} \cmidrule(lr){6-7}
\multicolumn{1}{l}{} & N & Mean & N & Mean & N & Mean \\ 
\midrule
21 months & $131$ (245) & $2.79$ ± $2.25$ & $140$ (236) & $2.98$ ± $1.98$ & $237$ (515) & $5.04$ ± $3.61$ \\ 
25 months & $208$ (272) & $3.47$ ± $2.28$ & $203$ (277) & $3.38$ ± $2.37$ & $369$ (591) & $6.15$ ± $4.03$ \\ 
30 months & $218$ (230) & $3.89$ ± $2.89$ & $214$ (234) & $3.82$ ± $2.80$ & $379$ (517) & $6.77$ ± $5.08$ \\ 
\midrule 
\midrule 
Mean & — & $3.38$ & — & $3.39$ & — & $5.99$ \\ 
Sum & $557$ & — & $557$ & — & $985$ & — \\ 
\bottomrule
\end{longtable}

\hypertarget{phonological-priming-related-vs.-unrelated}{%
\subsubsection{Phonological priming: Related
vs.~Unrelated}\label{phonological-priming-related-vs.-unrelated}}

A model including the \emph{Relatedness} \(\times\) \emph{Group}
interaction showed the best of-of-sample predictive performance,
although the model including only \emph{Relatedness} performed
equivalently
(\(\text{ELPD}_{\mathcal{M_0}} - \text{ELPD}_{\mathcal{M_1}}\) = -5.146,
\emph{SE} = 4.634). Both models showed substantially better predictive
performance than the model including only \emph{Group}
(\(\text{ELPD}_{\mathcal{M_0}} - \text{ELPD}_{\mathcal{M_2}}\) =
-55.182, \emph{SE} = 42.373). This indicates that including the
\emph{Relatedness} predictor improved the predictive performance of the
model significantly, that including its interaction with \emph{Group}
slightly increased the performance of the model, and that the main
effect of \emph{Group} by itself barely changed the predictive
performance of the model.

\begin{Shaded}
\begin{Highlighting}[]
\NormalTok{epreds }\OtherTok{\textless{}{-}} \FunctionTok{expand\_grid}\NormalTok{(}\AttributeTok{condition =} \FunctionTok{levels}\NormalTok{(data\_time\_related}\SpecialCharTok{$}\NormalTok{condition),}
                      \AttributeTok{timebin =} \FunctionTok{seq}\NormalTok{(}\DecValTok{0}\NormalTok{, }\DecValTok{17}\NormalTok{, }\AttributeTok{length.out =} \DecValTok{100}\NormalTok{),}
                      \AttributeTok{age =} \FunctionTok{mean}\NormalTok{(data\_time\_related}\SpecialCharTok{$}\NormalTok{age),}
                      \AttributeTok{lp =} \FunctionTok{levels}\NormalTok{(data\_time\_related}\SpecialCharTok{$}\NormalTok{lp),}
                      \AttributeTok{.nsamples =} \DecValTok{1}\NormalTok{) }\SpecialCharTok{|\textgreater{}}
    \FunctionTok{add\_epred\_draws}\NormalTok{(model\_fits\_related[[}\DecValTok{4}\NormalTok{]],}
                    \AttributeTok{ndraws =} \ConstantTok{NULL}\NormalTok{,}
                    \AttributeTok{re\_formula =} \ConstantTok{NA}\NormalTok{, }
                    \AttributeTok{value =} \StringTok{".value"}\NormalTok{) }\SpecialCharTok{|\textgreater{}} 
    \FunctionTok{mutate}\NormalTok{(}\AttributeTok{lp =} \FunctionTok{factor}\NormalTok{(lp, }\AttributeTok{levels =} \FunctionTok{c}\NormalTok{(}\StringTok{"Monolingual"}\NormalTok{, }\StringTok{"Bilingual"}\NormalTok{)))}

\NormalTok{epreds\_diff }\OtherTok{\textless{}{-}}\NormalTok{ epreds }\SpecialCharTok{|\textgreater{}} 
    \FunctionTok{pivot\_wider}\NormalTok{(}\AttributeTok{names\_from =}\NormalTok{ condition,}
                \AttributeTok{values\_from =}\NormalTok{ .value,}
                \AttributeTok{id\_cols =} \FunctionTok{c}\NormalTok{(timebin, age, lp, .draw),}
                \AttributeTok{names\_repair =}\NormalTok{ janitor}\SpecialCharTok{::}\NormalTok{make\_clean\_names) }\SpecialCharTok{|\textgreater{}} 
    \FunctionTok{mutate}\NormalTok{(}\AttributeTok{diff =}\NormalTok{ related }\SpecialCharTok{{-}}\NormalTok{ unrelated) }

\CommentTok{\# diff\_rect \textless{}{-} epreds\_diff |\textgreater{} }
\CommentTok{\#   mean\_qi(diff) |\textgreater{} }
\CommentTok{\#   mutate(is\_cluster = .lower \textgreater{} 0 | .upper \textless{} 0) }
\CommentTok{\# }
\CommentTok{\# clusters \textless{}{-} rle(diff(diff\_rect$is\_cluster))}
\CommentTok{\# diff\_rect$cluster\_id \textless{}{-} c(0, rep(clusters$values, clusters$lengths))}
\CommentTok{\# }
\CommentTok{\# diff\_rect \textless{}{-} diff\_rect |\textgreater{} }
\CommentTok{\#   arrange(lp, timebin)}
\CommentTok{\# }
\CommentTok{\# diff\_rect \textless{}{-} }
\CommentTok{\#   cluster\_number = }
\CommentTok{\#   summarise(xmin = min(timebin),}
\CommentTok{\#             xmax = max(timebin),}
\CommentTok{\#             .by = c(lp, is\_cluster)) |\textgreater{} }
\CommentTok{\#   filter(is\_cluster)}

\CommentTok{\# diff\_obs \textless{}{-} data\_time\_related |\textgreater{}}
\CommentTok{\#   pivot\_wider(names\_from = condition, }
\CommentTok{\#               values\_from = elog,}
\CommentTok{\#               names\_repair = janitor::make\_clean\_names) |\textgreater{} mutate(diff = related {-} unrelated)}

\NormalTok{data\_time\_related }\SpecialCharTok{|\textgreater{}} 
    \FunctionTok{summarise}\NormalTok{(}\AttributeTok{.prop =} \FunctionTok{mean}\NormalTok{(.prop),}
              \AttributeTok{.by =} \FunctionTok{c}\NormalTok{(id, timebin, lp, condition, age)) }\SpecialCharTok{|\textgreater{}} 
    \FunctionTok{ggplot}\NormalTok{(}\FunctionTok{aes}\NormalTok{(timebin, .prop, }
               \AttributeTok{colour =}\NormalTok{ condition,}
               \AttributeTok{fill =}\NormalTok{ condition,}
               \AttributeTok{shape =}\NormalTok{ condition,}
               \AttributeTok{linetype =}\NormalTok{ condition)) }\SpecialCharTok{+}
    \FunctionTok{facet\_wrap}\NormalTok{(}\SpecialCharTok{\textasciitilde{}}\NormalTok{lp) }\SpecialCharTok{+}
    \CommentTok{\# geom\_rect(data = diff\_rect,}
    \CommentTok{\#         aes(xmin = xmin,}
    \CommentTok{\#           xmax = xmax,}
    \CommentTok{\#           ymin = {-}Inf,}
    \CommentTok{\#           ymax = Inf),}
    \CommentTok{\#         colour = NA,}
    \CommentTok{\#         fill = "orange",}
    \CommentTok{\#         alpha = 1/2,}
    \CommentTok{\#         inherit.aes = FALSE) +}
    \CommentTok{\# geom\_line(data = epreds,}
    \CommentTok{\#         aes(y = .epred,}
\CommentTok{\#           group = interaction(condition, .draw)),}
\CommentTok{\#         linetype = "solid",}
\CommentTok{\#         alpha = 0.1,}
\CommentTok{\#         linewidth = 3/4) +}
\FunctionTok{stat\_summary}\NormalTok{(}\AttributeTok{data =}\NormalTok{ epreds,}
             \FunctionTok{aes}\NormalTok{(}\AttributeTok{y =}\NormalTok{ .value),}
             \AttributeTok{fun.data =}\NormalTok{ \textbackslash{}(x) }\FunctionTok{mean\_qi}\NormalTok{(x, }\AttributeTok{.width =} \FloatTok{0.95}\NormalTok{),}
             \AttributeTok{geom =} \StringTok{"ribbon"}\NormalTok{,}
             \AttributeTok{alpha =} \FloatTok{0.5}\NormalTok{,}
             \AttributeTok{linewidth =} \DecValTok{0}\NormalTok{) }\SpecialCharTok{+}
    \FunctionTok{stat\_summary}\NormalTok{(}\AttributeTok{data =}\NormalTok{ epreds,}
                 \FunctionTok{aes}\NormalTok{(}\AttributeTok{y =}\NormalTok{ .value,}
                    \AttributeTok{linetype =}\NormalTok{ condition),}
                 \AttributeTok{fun =} \StringTok{"mean"}\NormalTok{,}
                 \AttributeTok{geom =} \StringTok{"line"}\NormalTok{,}
                 \AttributeTok{colour =} \StringTok{"black"}\NormalTok{,}
                 \AttributeTok{linewidth =} \DecValTok{3}\SpecialCharTok{/}\DecValTok{4}\NormalTok{) }\SpecialCharTok{+}
    \FunctionTok{geom\_hline}\NormalTok{(}\AttributeTok{yintercept =} \DecValTok{1}\SpecialCharTok{/}\DecValTok{2}\NormalTok{, }
               \AttributeTok{linewidth =} \DecValTok{1}\SpecialCharTok{/}\DecValTok{2}\NormalTok{,}
               \AttributeTok{colour =} \StringTok{"black"}\NormalTok{,}
               \AttributeTok{linetype =} \StringTok{"dotted"}\NormalTok{) }\SpecialCharTok{+}
    \FunctionTok{stat\_summary}\NormalTok{(}\AttributeTok{fun =}\NormalTok{ mean,}
                 \AttributeTok{geom =} \StringTok{"point"}\NormalTok{,}
                 \AttributeTok{colour =} \StringTok{"black"}\NormalTok{,}
                 \AttributeTok{size =} \FloatTok{2.5}\NormalTok{,}
                 \AttributeTok{stroke =} \DecValTok{3}\SpecialCharTok{/}\DecValTok{4}\NormalTok{) }\SpecialCharTok{+}
    \FunctionTok{labs}\NormalTok{(}\AttributeTok{x =} \StringTok{"Time (ms)"}\NormalTok{,}
         \AttributeTok{y =} \StringTok{"P(Target looking)"}\NormalTok{,}
         \AttributeTok{colour =} \StringTok{"Condition"}\NormalTok{,}
         \AttributeTok{fill =} \StringTok{"Condition"}\NormalTok{,}
         \AttributeTok{linetype =} \StringTok{"Condition"}\NormalTok{,}
         \AttributeTok{shape =} \StringTok{"Condition"}\NormalTok{) }\SpecialCharTok{+}
    \FunctionTok{theme}\NormalTok{(}\AttributeTok{legend.title =} \FunctionTok{element\_blank}\NormalTok{(),}
          \AttributeTok{axis.title.x =} \FunctionTok{element\_blank}\NormalTok{()) }\SpecialCharTok{+}
    
\NormalTok{    epreds\_diff }\SpecialCharTok{|\textgreater{}} 
    \FunctionTok{ggplot}\NormalTok{(}\FunctionTok{aes}\NormalTok{(timebin, diff)) }\SpecialCharTok{+}
    \FunctionTok{facet\_wrap}\NormalTok{(}\SpecialCharTok{\textasciitilde{}}\NormalTok{lp) }\SpecialCharTok{+}
    \CommentTok{\# geom\_rect(data = diff\_rect,}
    \CommentTok{\#         aes(xmin = xmin,}
    \CommentTok{\#           xmax = xmax,}
    \CommentTok{\#           ymin = {-}Inf,}
    \CommentTok{\#           ymax = Inf),}
    \CommentTok{\#         colour = NA,}
    \CommentTok{\#         fill = "orange",}
    \CommentTok{\#         alpha = 1/2,}
    \CommentTok{\#         inherit.aes = FALSE) +}
    \FunctionTok{stat\_lineribbon}\NormalTok{(}\AttributeTok{.width =} \FloatTok{0.95}\NormalTok{,}
                    \AttributeTok{linewidth =} \DecValTok{0}\NormalTok{,}
                    \AttributeTok{fill =} \StringTok{"grey"}\NormalTok{) }\SpecialCharTok{+}
    \FunctionTok{stat\_summary}\NormalTok{(}\AttributeTok{data =}\NormalTok{ epreds\_diff,}
                 \AttributeTok{fun =} \StringTok{"mean"}\NormalTok{,}
                 \AttributeTok{geom =} \StringTok{"line"}\NormalTok{,}
                 \AttributeTok{colour =} \StringTok{"black"}\NormalTok{,}
                 \AttributeTok{linewidth =} \DecValTok{3}\SpecialCharTok{/}\DecValTok{4}\NormalTok{) }\SpecialCharTok{+}
    \FunctionTok{geom\_hline}\NormalTok{(}\AttributeTok{yintercept =} \DecValTok{0}\NormalTok{, }
               \AttributeTok{linewidth =} \DecValTok{1}\SpecialCharTok{/}\DecValTok{2}\NormalTok{,}
               \AttributeTok{colour =} \StringTok{"black"}\NormalTok{,}
               \AttributeTok{linetype =} \StringTok{"dotted"}\NormalTok{) }\SpecialCharTok{+}
    \CommentTok{\# geom\_point(data = diff\_obs) +}
    \FunctionTok{labs}\NormalTok{(}\AttributeTok{x =} \StringTok{"Time (ms)"}\NormalTok{,}
         \AttributeTok{y =} \StringTok{"P(Target looking)"}\NormalTok{,}
         \AttributeTok{fill =} \StringTok{"CrI"}\NormalTok{) }\SpecialCharTok{+}
    \FunctionTok{theme}\NormalTok{(}\AttributeTok{strip.text =} \FunctionTok{element\_blank}\NormalTok{(),}
          \AttributeTok{legend.position =} \StringTok{"none"}\NormalTok{) }\SpecialCharTok{+}
    
    \FunctionTok{plot\_layout}\NormalTok{(}\AttributeTok{ncol =} \DecValTok{1}\NormalTok{) }\SpecialCharTok{\&}
    \FunctionTok{plot\_annotation}\NormalTok{(}\AttributeTok{tag\_levels =} \StringTok{"A"}\NormalTok{) }\SpecialCharTok{+}
    \FunctionTok{scale\_linetype\_manual}\NormalTok{(}\AttributeTok{values =} \FunctionTok{rev}\NormalTok{(}\FunctionTok{c}\NormalTok{(}\StringTok{"solid"}\NormalTok{, }\StringTok{"dashed"}\NormalTok{))) }\SpecialCharTok{\&}
    \FunctionTok{scale\_shape\_manual}\NormalTok{(}\AttributeTok{values =} \FunctionTok{c}\NormalTok{(}\DecValTok{1}\NormalTok{, }\DecValTok{2}\NormalTok{)) }\SpecialCharTok{\&}
    \FunctionTok{scale\_x\_continuous}\NormalTok{(}\AttributeTok{labels =}\NormalTok{ \textbackslash{}(x) }\FunctionTok{format}\NormalTok{((x }\SpecialCharTok{*} \FloatTok{1e2}\NormalTok{)}\SpecialCharTok{+}\DecValTok{300}\NormalTok{, }
                                            \AttributeTok{big.mark =} \StringTok{","}\NormalTok{)) }\SpecialCharTok{\&}
    \FunctionTok{theme}\NormalTok{(}\AttributeTok{panel.grid =} \FunctionTok{element\_blank}\NormalTok{(),}
          \AttributeTok{legend.position =} \StringTok{"top"}\NormalTok{) }
\end{Highlighting}
\end{Shaded}

\begin{figure}[H]

{\centering \includegraphics{manuscript_files/figure-pdf/fig-related-1.pdf}

}

\caption{\label{fig-related}Marginal posterior predictions of the GAMMs.
(A) Mean posterior probability of target looking across the time course
of the trial. Black lines and intervals indicate the psoterior mean and
95\% credible intervals. Points indicate the mean probability of target
looking across participants. (B) Difference in posterior probability of
target looking between \emph{related} and \emph{unrelated} trials. The
yellow rectangle indicates, in both A and B, the range of time points in
which the 95\% credible interval of the differences excluded zero.}

\end{figure}

\hypertarget{cognate-priming-cognate-vs.-non-cognate}{%
\subsubsection{Cognate priming: Cognate
vs.~Non-cognate}\label{cognate-priming-cognate-vs.-non-cognate}}

A model including the \emph{Cognateness} \(\times\) \emph{Group}
interaction showed the best of-of-sample predictive performance,
although the model including only \emph{Cognateness} performed
equivalently
(\(\text{ELPD}_{\mathcal{M_0}} - \text{ELPD}_{\mathcal{M_1}}\) =
-25.189, \emph{SE} = 5.023). Both models showed substantially better
predictive performance than the model including only \emph{Group}
(\(\text{ELPD}_{\mathcal{M_0}} - \text{ELPD}_{\mathcal{M_1}}\) =
-60.098, \emph{SE} = 41.472). This indicates that including the
\emph{Cognateness} predictor improved the predictive performance of the
model significantly, that including its interaction with \emph{Group}
slightly increased the performance of the model, and that the main
effect of \emph{Group} by itself barely changed the predictive
performance of the model.

\begin{Shaded}
\begin{Highlighting}[]
\NormalTok{epreds }\OtherTok{\textless{}{-}} \FunctionTok{expand\_grid}\NormalTok{(}\AttributeTok{condition =} \FunctionTok{levels}\NormalTok{(data\_time\_cognate}\SpecialCharTok{$}\NormalTok{condition),}
                      \AttributeTok{timebin =} \FunctionTok{seq}\NormalTok{(}\DecValTok{0}\NormalTok{, }\DecValTok{17}\NormalTok{, }\AttributeTok{length.out =} \DecValTok{100}\NormalTok{),}
                      \AttributeTok{age =} \FunctionTok{mean}\NormalTok{(data\_time\_cognate}\SpecialCharTok{$}\NormalTok{age),}
                      \AttributeTok{lp =} \FunctionTok{levels}\NormalTok{(data\_time\_cognate}\SpecialCharTok{$}\NormalTok{lp),}
                      \AttributeTok{.nsamples =} \DecValTok{1}\NormalTok{) }\SpecialCharTok{|\textgreater{}}
    \FunctionTok{add\_epred\_draws}\NormalTok{(model\_fits\_cognate[[}\DecValTok{4}\NormalTok{]],}
                    \AttributeTok{ndraws =} \ConstantTok{NULL}\NormalTok{,}
                    \AttributeTok{re\_formula =} \ConstantTok{NA}\NormalTok{) }\SpecialCharTok{|\textgreater{}} 
    \FunctionTok{mutate}\NormalTok{(}\AttributeTok{lp =} \FunctionTok{factor}\NormalTok{(lp, }\AttributeTok{levels =} \FunctionTok{c}\NormalTok{(}\StringTok{"Monolingual"}\NormalTok{, }\StringTok{"Bilingual"}\NormalTok{)))}

\NormalTok{epreds\_diff }\OtherTok{\textless{}{-}}\NormalTok{ epreds }\SpecialCharTok{|\textgreater{}} 
    \FunctionTok{pivot\_wider}\NormalTok{(}\AttributeTok{names\_from =}\NormalTok{ condition,}
                \AttributeTok{values\_from =}\NormalTok{ .epred,}
                \AttributeTok{id\_cols =} \FunctionTok{c}\NormalTok{(timebin, age, lp, .draw),}
                \AttributeTok{names\_repair =}\NormalTok{ janitor}\SpecialCharTok{::}\NormalTok{make\_clean\_names) }\SpecialCharTok{|\textgreater{}} 
    \FunctionTok{mutate}\NormalTok{(}\AttributeTok{diff =}\NormalTok{ cognate }\SpecialCharTok{{-}}\NormalTok{ non\_cognate) }
\CommentTok{\# }
\CommentTok{\# diff\_rect \textless{}{-} epreds\_diff |\textgreater{} }
\CommentTok{\#   mean\_qi(diff) |\textgreater{} }
\CommentTok{\#   filter(.lower \textgreater{} 0 | .upper \textless{} 0) |\textgreater{} }
\CommentTok{\#   summarise(xmin = min(timebin),}
\CommentTok{\#             xmax = max(timebin),}
\CommentTok{\#             .by = lp)}

\NormalTok{data\_time\_cognate }\SpecialCharTok{|\textgreater{}} 
    \FunctionTok{summarise}\NormalTok{(}\AttributeTok{.prop =} \FunctionTok{mean}\NormalTok{(.prop),}
              \AttributeTok{.by =} \FunctionTok{c}\NormalTok{(id, timebin, lp, condition, age)) }\SpecialCharTok{|\textgreater{}} 
    \FunctionTok{ggplot}\NormalTok{(}\FunctionTok{aes}\NormalTok{(timebin, .prop, }
               \AttributeTok{colour =}\NormalTok{ condition,}
               \AttributeTok{fill =}\NormalTok{ condition,}
               \AttributeTok{shape =}\NormalTok{ condition)) }\SpecialCharTok{+}
    \FunctionTok{facet\_wrap}\NormalTok{(}\SpecialCharTok{\textasciitilde{}}\NormalTok{lp) }\SpecialCharTok{+}
    \CommentTok{\# geom\_rect(data = diff\_rect,}
    \CommentTok{\#         aes(xmin = xmin,}
    \CommentTok{\#           xmax = xmax,}
    \CommentTok{\#           ymin = {-}1.5,}
    \CommentTok{\#           ymax = 1.5),}
    \CommentTok{\#         colour = NA,}
    \CommentTok{\#         fill = "orange",}
    \CommentTok{\#         alpha = 1/2,}
    \CommentTok{\#         inherit.aes = FALSE) +}
    \CommentTok{\# geom\_line(data = epreds,}
    \CommentTok{\#         aes(y = .epred,}
\CommentTok{\#           group = interaction(condition, .draw)),}
\CommentTok{\#         linetype = "solid",}
\CommentTok{\#         alpha = 0.1,}
\CommentTok{\#         linewidth = 3/4) +}
\FunctionTok{stat\_summary}\NormalTok{(}\AttributeTok{data =}\NormalTok{ epreds,}
             \FunctionTok{aes}\NormalTok{(}\AttributeTok{y =}\NormalTok{ .epred),}
             \AttributeTok{fun.data =}\NormalTok{ \textbackslash{}(x) }\FunctionTok{mean\_qi}\NormalTok{(x, }\AttributeTok{.width =} \FloatTok{0.95}\NormalTok{),}
             \AttributeTok{geom =} \StringTok{"ribbon"}\NormalTok{,}
             \AttributeTok{alpha =} \FloatTok{0.5}\NormalTok{,}
             \AttributeTok{linewidth =} \DecValTok{0}\NormalTok{) }\SpecialCharTok{+}
    \FunctionTok{stat\_summary}\NormalTok{(}\AttributeTok{data =}\NormalTok{ epreds,}
                 \FunctionTok{aes}\NormalTok{(}\AttributeTok{y =}\NormalTok{ .epred,}
                    \AttributeTok{linetype =}\NormalTok{ condition),}
                 \AttributeTok{fun =} \StringTok{"mean"}\NormalTok{,}
                 \AttributeTok{geom =} \StringTok{"line"}\NormalTok{,}
                 \AttributeTok{colour =} \StringTok{"black"}\NormalTok{,}
                 \AttributeTok{linewidth =} \DecValTok{3}\SpecialCharTok{/}\DecValTok{4}\NormalTok{) }\SpecialCharTok{+}
    \FunctionTok{geom\_hline}\NormalTok{(}\AttributeTok{yintercept =} \FloatTok{0.5}\NormalTok{, }
               \AttributeTok{linewidth =} \DecValTok{1}\SpecialCharTok{/}\DecValTok{2}\NormalTok{,}
               \AttributeTok{colour =} \StringTok{"black"}\NormalTok{,}
               \AttributeTok{linetype =} \StringTok{"dotted"}\NormalTok{) }\SpecialCharTok{+}
    \FunctionTok{stat\_summary}\NormalTok{(}\AttributeTok{fun =}\NormalTok{ mean,}
                 \AttributeTok{geom =} \StringTok{"point"}\NormalTok{,}
                 \AttributeTok{colour =} \StringTok{"black"}\NormalTok{,}
                 \AttributeTok{size =} \FloatTok{2.5}\NormalTok{,}
                 \AttributeTok{stroke =} \DecValTok{3}\SpecialCharTok{/}\DecValTok{4}\NormalTok{) }\SpecialCharTok{+}
    \FunctionTok{labs}\NormalTok{(}\AttributeTok{x =} \StringTok{"Time (ms)"}\NormalTok{,}
         \AttributeTok{y =} \StringTok{"P(Target looking)"}\NormalTok{,}
         \AttributeTok{colour =} \StringTok{"Prime type"}\NormalTok{,}
         \AttributeTok{fill =} \StringTok{"Prime type"}\NormalTok{,}
         \AttributeTok{linetype =} \StringTok{"Prime type"}\NormalTok{,}
         \AttributeTok{shape =} \StringTok{"Prime type"}\NormalTok{) }\SpecialCharTok{+}
    \FunctionTok{theme}\NormalTok{(}\AttributeTok{legend.title =} \FunctionTok{element\_blank}\NormalTok{(),}
          \AttributeTok{axis.title.x =} \FunctionTok{element\_blank}\NormalTok{()) }\SpecialCharTok{+}
    
\NormalTok{    epreds\_diff }\SpecialCharTok{|\textgreater{}} 
    \FunctionTok{ggplot}\NormalTok{(}\FunctionTok{aes}\NormalTok{(timebin, diff)) }\SpecialCharTok{+}
    \FunctionTok{facet\_wrap}\NormalTok{(}\SpecialCharTok{\textasciitilde{}}\NormalTok{lp) }\SpecialCharTok{+}
    \CommentTok{\# geom\_rect(data = diff\_rect,}
    \CommentTok{\#         aes(xmin = xmin,}
    \CommentTok{\#           xmax = xmax,}
    \CommentTok{\#           ymin = {-}3/4,}
    \CommentTok{\#           ymax = 3/4),}
    \CommentTok{\#         colour = NA,}
    \CommentTok{\#         fill = "orange",}
    \CommentTok{\#         alpha = 1/2,}
    \CommentTok{\#         inherit.aes = FALSE) +}
    \FunctionTok{stat\_lineribbon}\NormalTok{(}\AttributeTok{.width =} \FloatTok{0.95}\NormalTok{,}
                    \AttributeTok{linewidth =} \DecValTok{0}\NormalTok{,}
                    \AttributeTok{fill =} \StringTok{"grey"}\NormalTok{) }\SpecialCharTok{+}
    \FunctionTok{stat\_summary}\NormalTok{(}\AttributeTok{data =}\NormalTok{ epreds\_diff,}
                 \AttributeTok{fun =} \StringTok{"mean"}\NormalTok{,}
                 \AttributeTok{geom =} \StringTok{"line"}\NormalTok{,}
                 \AttributeTok{colour =} \StringTok{"black"}\NormalTok{,}
                 \AttributeTok{linewidth =} \DecValTok{3}\SpecialCharTok{/}\DecValTok{4}\NormalTok{) }\SpecialCharTok{+}
    \FunctionTok{geom\_hline}\NormalTok{(}\AttributeTok{yintercept =} \DecValTok{0}\NormalTok{, }
               \AttributeTok{linewidth =} \DecValTok{1}\SpecialCharTok{/}\DecValTok{2}\NormalTok{,}
               \AttributeTok{colour =} \StringTok{"black"}\NormalTok{,}
               \AttributeTok{linetype =} \StringTok{"dotted"}\NormalTok{) }\SpecialCharTok{+}
    \FunctionTok{labs}\NormalTok{(}\AttributeTok{x =} \StringTok{"Time (ms)"}\NormalTok{,}
         \AttributeTok{y =} \StringTok{"P(Target looking)"}\NormalTok{,}
         \AttributeTok{fill =} \StringTok{"CrI"}\NormalTok{) }\SpecialCharTok{+}
    \FunctionTok{theme}\NormalTok{(}\AttributeTok{strip.text =} \FunctionTok{element\_blank}\NormalTok{(),}
          \AttributeTok{legend.position =} \StringTok{"none"}\NormalTok{) }\SpecialCharTok{+}
    
    \FunctionTok{plot\_layout}\NormalTok{(}\AttributeTok{ncol =} \DecValTok{1}\NormalTok{) }\SpecialCharTok{\&}
    \FunctionTok{plot\_annotation}\NormalTok{(}\AttributeTok{tag\_levels =} \StringTok{"A"}\NormalTok{) }\SpecialCharTok{+}
    \FunctionTok{scale\_linetype\_manual}\NormalTok{(}\AttributeTok{values =} \FunctionTok{rev}\NormalTok{(}\FunctionTok{c}\NormalTok{(}\StringTok{"solid"}\NormalTok{, }\StringTok{"dashed"}\NormalTok{))) }\SpecialCharTok{\&}
    \FunctionTok{scale\_shape\_manual}\NormalTok{(}\AttributeTok{values =} \FunctionTok{c}\NormalTok{(}\DecValTok{1}\NormalTok{, }\DecValTok{2}\NormalTok{)) }\SpecialCharTok{\&}
    \FunctionTok{scale\_x\_continuous}\NormalTok{(}\AttributeTok{labels =}\NormalTok{ \textbackslash{}(x) }\FunctionTok{format}\NormalTok{((x }\SpecialCharTok{*} \FloatTok{1e2}\NormalTok{)}\SpecialCharTok{+}\DecValTok{300}\NormalTok{, }
                                            \AttributeTok{big.mark =} \StringTok{","}\NormalTok{)) }\SpecialCharTok{\&}
    \FunctionTok{theme}\NormalTok{(}\AttributeTok{panel.grid =} \FunctionTok{element\_blank}\NormalTok{(),}
          \AttributeTok{legend.position =} \StringTok{"top"}\NormalTok{) }
\end{Highlighting}
\end{Shaded}

\begin{figure}[H]

{\centering \includegraphics{manuscript_files/figure-pdf/fig-cognate-1.pdf}

}

\caption{\label{fig-cognate}Marginal posterior predictions of the GAMMs.
(A) Mean posterior probability of target looking across the time course
of the trial. Black lines and intervals indicate the psoterior mean and
95\% credible intervals. Points indicate the mean probability of target
looking across participants. (B) Difference in posterior probability of
target looking between \emph{cognate} and \emph{non-cognate} trials. The
yellow rectangle indicates, in both A and B, the range of time points in
which the 95\% credible interval of the differences excluded zero.}

\end{figure}

\hypertarget{analysis-2}{%
\subsection{Analysis 2}\label{analysis-2}}

Participants must know the target \textbf{target} word (no need to know
the prime word), and must look at least 10 ms to each target and
distractor.

\begin{Shaded}
\begin{Highlighting}[]
\NormalTok{n\_trials }\OtherTok{\textless{}{-}} \FunctionTok{nrow}\NormalTok{(attrition\_trials\_vtarget)}
\NormalTok{n\_trials\_valid }\OtherTok{\textless{}{-}} \FunctionTok{inner\_join}\NormalTok{(attrition\_participants\_vtarget,}
\NormalTok{                             attrition\_trials\_vtarget) }\SpecialCharTok{|\textgreater{}} 
    \FunctionTok{filter}\NormalTok{(is\_valid\_participant) }\SpecialCharTok{|\textgreater{}} 
    \FunctionTok{pull}\NormalTok{(is\_valid\_trial) }\SpecialCharTok{|\textgreater{}} 
    \FunctionTok{sum}\NormalTok{()}
\NormalTok{n\_participants\_valid }\OtherTok{\textless{}{-}} \FunctionTok{sum}\NormalTok{(attrition\_participants\_vtarget}\SpecialCharTok{$}\NormalTok{is\_valid\_participant)}
\NormalTok{n\_exc\_prime }\OtherTok{\textless{}{-}} \FunctionTok{sum}\NormalTok{(}\SpecialCharTok{!}\NormalTok{attrition\_trials\_vtarget}\SpecialCharTok{$}\NormalTok{is\_valid\_gaze\_prime)}
\NormalTok{n\_exc\_test }\OtherTok{\textless{}{-}} \FunctionTok{sum}\NormalTok{(}\SpecialCharTok{!}\NormalTok{attrition\_trials\_vtarget}\SpecialCharTok{$}\NormalTok{is\_valid\_gaze\_test)}
\NormalTok{n\_exc\_test\_each }\OtherTok{\textless{}{-}} \FunctionTok{sum}\NormalTok{(}\SpecialCharTok{!}\NormalTok{attrition\_trials\_vtarget}\SpecialCharTok{$}\NormalTok{is\_valid\_gaze\_test\_each)}
\NormalTok{n\_exc\_vocab }\OtherTok{\textless{}{-}} \FunctionTok{sum}\NormalTok{(}\SpecialCharTok{!}\NormalTok{attrition\_trials\_vtarget}\SpecialCharTok{$}\NormalTok{is\_valid\_vocab)}
\NormalTok{n\_exc\_cognate }\OtherTok{\textless{}{-}} \FunctionTok{sum}\NormalTok{(}\SpecialCharTok{!}\NormalTok{attrition\_participants\_vtarget}\SpecialCharTok{$}\NormalTok{is\_valid\_cognate)}
\NormalTok{n\_exc\_noncognate }\OtherTok{\textless{}{-}} \FunctionTok{sum}\NormalTok{(}\SpecialCharTok{!}\NormalTok{attrition\_participants\_vtarget}\SpecialCharTok{$}\NormalTok{is\_valid\_noncognate)}
\NormalTok{n\_exc\_unrelated }\OtherTok{\textless{}{-}} \FunctionTok{sum}\NormalTok{(}\SpecialCharTok{!}\NormalTok{attrition\_participants\_vtarget}\SpecialCharTok{$}\NormalTok{is\_valid\_unrelated)}

\NormalTok{n\_longitudinal }\OtherTok{\textless{}{-}}\NormalTok{ attrition\_participants\_vtarget }\SpecialCharTok{|\textgreater{}}
    \FunctionTok{filter}\NormalTok{(is\_valid\_participant) }\SpecialCharTok{|\textgreater{}}
    \FunctionTok{count}\NormalTok{(id, }\AttributeTok{name =} \StringTok{"times"}\NormalTok{) }\SpecialCharTok{|\textgreater{}} 
    \FunctionTok{count}\NormalTok{(times)}
\end{Highlighting}
\end{Shaded}

We gathered data from 9,472 trials from 180 distinct participants. We
excluded trials in which participants failed to provide 50\% valid
eye-tracking samples during the prime phase (\emph{n} = 1,810) or during
the target-distractor phase (\emph{n} = 1,493). We also excluded trials
in which participants did not provide at least 5\% of valid samples to
\emph{both} target and distractor in the test phase (\emph{n} = 2,793).
Finally, we excluded trials in which participants did not understand the
target word, according to a supplementary vocabulary checklist filled by
their caregivers upon experiment completion (\emph{n} = 5,539). After
applying these trial-level inclusion criteria, we excluded participants
who did not provide at least two valid trials in the \emph{cognate
prime} condition (\emph{n} = 165), the \emph{non-cognate prime}
condition (\emph{n} = 165), or the \emph{unrelated prime} condition
(\emph{n} = 151). The resulting dataset included 2,219 trials from 121
participants. Of those participants, 62 provided data from one
experimental session, 22 provided data from two experimental sessions,
and 5 provided data from three experimental sessions.
Table~\ref{tbl-attrition-trials} shows a detailed description of the
trial attrition.

\begin{Shaded}
\begin{Highlighting}[]
\NormalTok{attrition\_trials\_vtarget }\SpecialCharTok{|\textgreater{}} 
    \FunctionTok{filter}\NormalTok{(id }\SpecialCharTok{\%in\%}\NormalTok{ attrition\_participants\_vtarget}\SpecialCharTok{$}\NormalTok{id[attrition\_participants\_vtarget}\SpecialCharTok{$}\NormalTok{is\_valid\_participant]) }\SpecialCharTok{|\textgreater{}} 
    \FunctionTok{left\_join}\NormalTok{(}\FunctionTok{select}\NormalTok{(participants, filename, id, age\_group),}
              \AttributeTok{by =} \FunctionTok{join\_by}\NormalTok{(filename, id, age\_group)) }\SpecialCharTok{|\textgreater{}} 
    \FunctionTok{summarise}\NormalTok{(}\AttributeTok{n\_valid =} \FunctionTok{sum}\NormalTok{(is\_valid\_trial),}
              \AttributeTok{n\_total =} \FunctionTok{n}\NormalTok{(),}
              \AttributeTok{.by =} \FunctionTok{c}\NormalTok{(id, age\_group, trial\_type)) }\SpecialCharTok{|\textgreater{}} 
    \FunctionTok{summarise}\NormalTok{(}\FunctionTok{across}\NormalTok{(n\_valid, }\FunctionTok{lst}\NormalTok{(sum, mean, sd),}
                     \AttributeTok{.names =} \StringTok{"\{.fn\}"}\NormalTok{),}
              \AttributeTok{n\_total =} \FunctionTok{sum}\NormalTok{(n\_total),}
              \AttributeTok{.by =} \FunctionTok{c}\NormalTok{(age\_group, trial\_type)) }\SpecialCharTok{|\textgreater{}} 
    \FunctionTok{mutate}\NormalTok{(}\AttributeTok{n\_excluded =}\NormalTok{ n\_total}\SpecialCharTok{{-}}\NormalTok{sum) }\SpecialCharTok{|\textgreater{}} 
    \FunctionTok{select}\NormalTok{(}\SpecialCharTok{{-}}\FunctionTok{c}\NormalTok{(n\_total)) }\SpecialCharTok{|\textgreater{}} 
    \FunctionTok{pivot\_wider}\NormalTok{(}\AttributeTok{names\_from =}\NormalTok{ trial\_type,}
                \AttributeTok{values\_from =} \FunctionTok{c}\NormalTok{(sum}\SpecialCharTok{:}\NormalTok{sd, n\_excluded),}
                \AttributeTok{names\_repair =}\NormalTok{ janitor}\SpecialCharTok{::}\NormalTok{make\_clean\_names) }\SpecialCharTok{|\textgreater{}} 
    \FunctionTok{rename\_with}\NormalTok{(\textbackslash{}(x) }\FunctionTok{gsub}\NormalTok{(}\StringTok{"non\_cognate"}\NormalTok{, }
                          \StringTok{"noncognate"}\NormalTok{,}
\NormalTok{                          x)) }\SpecialCharTok{|\textgreater{}} 
    \FunctionTok{arrange}\NormalTok{(age\_group) }\SpecialCharTok{|\textgreater{}} 
    \FunctionTok{relocate}\NormalTok{(age\_group,}
             \FunctionTok{matches}\NormalTok{(}\StringTok{"\_cognate"}\NormalTok{),}
             \FunctionTok{matches}\NormalTok{(}\StringTok{"noncognate"}\NormalTok{)) }\SpecialCharTok{|\textgreater{}} 
    \FunctionTok{gt}\NormalTok{(}\AttributeTok{rowname\_col =} \StringTok{"age\_group"}\NormalTok{) }\SpecialCharTok{|\textgreater{}} 
    \FunctionTok{grand\_summary\_rows}\NormalTok{(}\AttributeTok{columns =} \FunctionTok{matches}\NormalTok{(}\StringTok{"mean\_"}\NormalTok{),}
                       \AttributeTok{fns =} \FunctionTok{lst}\NormalTok{(Mean }\SpecialCharTok{\textasciitilde{}} \FunctionTok{mean}\NormalTok{(.)),}
                       \AttributeTok{fmt =} \SpecialCharTok{\textasciitilde{}}\FunctionTok{fmt\_number}\NormalTok{(.)) }\SpecialCharTok{|\textgreater{}}
    \FunctionTok{grand\_summary\_rows}\NormalTok{(}\AttributeTok{columns =} \FunctionTok{matches}\NormalTok{(}\StringTok{"sum\_"}\NormalTok{),}
                       \AttributeTok{fns =} \FunctionTok{lst}\NormalTok{(Sum }\SpecialCharTok{\textasciitilde{}} \FunctionTok{sum}\NormalTok{(.)),}
                       \AttributeTok{fmt =} \SpecialCharTok{\textasciitilde{}}\FunctionTok{fmt\_integer}\NormalTok{(.)) }\SpecialCharTok{|\textgreater{}}
    \FunctionTok{cols\_merge}\NormalTok{(}\FunctionTok{c}\NormalTok{(sum\_cognate, n\_excluded\_cognate), }
               \AttributeTok{pattern =} \StringTok{"\{1\} (\{2\})"}\NormalTok{) }\SpecialCharTok{|\textgreater{}} 
    \FunctionTok{cols\_merge}\NormalTok{(}\FunctionTok{c}\NormalTok{(sum\_noncognate, n\_excluded\_noncognate), }
               \AttributeTok{pattern =} \StringTok{"\{1\} (\{2\})"}\NormalTok{) }\SpecialCharTok{|\textgreater{}} 
    \FunctionTok{cols\_merge}\NormalTok{(}\FunctionTok{c}\NormalTok{(sum\_unrelated, n\_excluded\_unrelated),}
               \AttributeTok{pattern =} \StringTok{"\{1\} (\{2\})"}\NormalTok{) }\SpecialCharTok{|\textgreater{}} 
    \FunctionTok{cols\_merge\_uncert}\NormalTok{(mean\_cognate, sd\_cognate) }\SpecialCharTok{|\textgreater{}} 
    \FunctionTok{cols\_merge\_uncert}\NormalTok{(mean\_noncognate, sd\_noncognate) }\SpecialCharTok{|\textgreater{}} 
    \FunctionTok{cols\_merge\_uncert}\NormalTok{(mean\_unrelated, sd\_unrelated) }\SpecialCharTok{|\textgreater{}} 
    \FunctionTok{tab\_spanner}\NormalTok{(}\StringTok{"Cognate trials"}\NormalTok{, }\FunctionTok{ends\_with}\NormalTok{(}\StringTok{"\_cognate"}\NormalTok{)) }\SpecialCharTok{|\textgreater{}}
    \FunctionTok{tab\_spanner}\NormalTok{(}\StringTok{"Non{-}cognate trials"}\NormalTok{, }\FunctionTok{ends\_with}\NormalTok{(}\StringTok{"noncognate"}\NormalTok{)) }\SpecialCharTok{|\textgreater{}} 
    \FunctionTok{tab\_spanner}\NormalTok{(}\StringTok{"Unrelated trials"}\NormalTok{, }\FunctionTok{ends\_with}\NormalTok{(}\StringTok{"unrelated"}\NormalTok{)) }\SpecialCharTok{|\textgreater{}} 
    \FunctionTok{tab\_spanner}\NormalTok{(}\StringTok{"Related trials"}\NormalTok{, }\FunctionTok{matches}\NormalTok{(}\StringTok{"cognate"}\NormalTok{)) }\SpecialCharTok{|\textgreater{}}
    \FunctionTok{fmt\_number}\NormalTok{(}\FunctionTok{matches}\NormalTok{(}\StringTok{"mean|sd"}\NormalTok{)) }\SpecialCharTok{|\textgreater{}} 
    \FunctionTok{fmt\_integer}\NormalTok{(}\FunctionTok{matches}\NormalTok{(}\StringTok{"sum"}\NormalTok{), }\AttributeTok{sep\_mark =} \StringTok{","}\NormalTok{) }\SpecialCharTok{|\textgreater{}} 
    \FunctionTok{cols\_label}\NormalTok{(}\AttributeTok{sum\_cognate =} \StringTok{"N"}\NormalTok{,}
               \AttributeTok{sum\_noncognate =} \StringTok{"N"}\NormalTok{,}
               \AttributeTok{sum\_unrelated =} \StringTok{"N"}\NormalTok{,}
               \AttributeTok{mean\_cognate =} \StringTok{"Mean"}\NormalTok{,}
               \AttributeTok{mean\_noncognate =} \StringTok{"Mean"}\NormalTok{,}
               \AttributeTok{mean\_unrelated =} \StringTok{"Mean"}\NormalTok{)}
\end{Highlighting}
\end{Shaded}

\hypertarget{tbl-attrition-trials-vtarget}{}
\begin{longtable}{l|rrrrrr}
\caption{\label{tbl-attrition-trials-vtarget}Trial attrition rate by condition for included participants. Additional
excluded trials are indicated between parentheses. }\tabularnewline

\toprule
\multicolumn{1}{l}{} & \multicolumn{4}{c}{Related trials} &  &  \\ 
\cmidrule(lr){2-5}
\multicolumn{1}{l}{} & \multicolumn{2}{c}{Cognate trials} & \multicolumn{2}{c}{Non-cognate trials} & \multicolumn{2}{c}{Unrelated trials} \\ 
\cmidrule(lr){2-3} \cmidrule(lr){4-5} \cmidrule(lr){6-7}
\multicolumn{1}{l}{} & N & Mean & N & Mean & N & Mean \\ 
\midrule
21 months & $152$ (256) & $2.98$ ± $2.10$ & $158$ (250) & $3.10$ ± $1.94$ & $317$ (499) & $6.22$ ± $3.80$ \\ 
25 months & $220$ (284) & $3.49$ ± $2.39$ & $214$ (290) & $3.40$ ± $2.46$ & $416$ (592) & $6.60$ ± $4.51$ \\ 
30 months & $226$ (254) & $3.77$ ± $2.96$ & $219$ (261) & $3.65$ ± $2.85$ & $409$ (551) & $6.82$ ± $5.31$ \\ 
\midrule 
\midrule 
Mean & — & $3.41$ & — & $3.38$ & — & $6.55$ \\ 
Sum & $598$ & — & $591$ & — & $1,142$ & — \\ 
\bottomrule
\end{longtable}

\hypertarget{phonological-priming-related-vs.-unrelated-1}{%
\subsubsection{Phonological priming: Related
vs.~Unrelated}\label{phonological-priming-related-vs.-unrelated-1}}

A model including the \emph{Relatedness} \(\times\) \emph{Group}
interaction showed the best of-of-sample predictive performance,
although the model including only \emph{Relatedness} performed
equivalently
(\(\text{ELPD}_{\mathcal{M_0}} - \text{ELPD}_{\mathcal{M_1}}\) =
-14.134, \emph{SE} = 5.25). Both models showed substantially better
predictive performance than the model including only \emph{Group}
(\(\text{ELPD}_{\mathcal{M_0}} - \text{ELPD}_{\mathcal{M_2}}\) =
-38.505, \emph{SE} = 38.46). This indicates that including the
\emph{Relatedness} predictor improved the predictive performance of the
model significantly, that including its interaction with \emph{Group}
slightly increased the performance of the model, and that the main
effect of \emph{Group} by itself barely changed the predictive
performance of the model.

\begin{Shaded}
\begin{Highlighting}[]
\NormalTok{epreds }\OtherTok{\textless{}{-}} \FunctionTok{expand\_grid}\NormalTok{(}\AttributeTok{condition =} \FunctionTok{levels}\NormalTok{(data\_time\_related\_vtarget}\SpecialCharTok{$}\NormalTok{condition),}
                      \AttributeTok{timebin =} \FunctionTok{seq}\NormalTok{(}\DecValTok{0}\NormalTok{, }\DecValTok{17}\NormalTok{, }\AttributeTok{length.out =} \DecValTok{100}\NormalTok{),}
                      \AttributeTok{age =} \FunctionTok{mean}\NormalTok{(data\_time\_related\_vtarget}\SpecialCharTok{$}\NormalTok{age),}
                      \AttributeTok{lp =} \FunctionTok{levels}\NormalTok{(data\_time\_related\_vtarget}\SpecialCharTok{$}\NormalTok{lp),}
                      \AttributeTok{.nsamples =} \DecValTok{1}\NormalTok{) }\SpecialCharTok{|\textgreater{}}
    \FunctionTok{add\_epred\_draws}\NormalTok{(model\_fits\_related\_vtarget[[}\DecValTok{4}\NormalTok{]],}
                    \AttributeTok{ndraws =} \ConstantTok{NULL}\NormalTok{,}
                    \AttributeTok{re\_formula =} \ConstantTok{NA}\NormalTok{, }
                    \AttributeTok{value =} \StringTok{".value"}\NormalTok{) }\SpecialCharTok{|\textgreater{}} 
    \FunctionTok{mutate}\NormalTok{(}\AttributeTok{lp =} \FunctionTok{factor}\NormalTok{(lp, }\AttributeTok{levels =} \FunctionTok{c}\NormalTok{(}\StringTok{"Monolingual"}\NormalTok{, }\StringTok{"Bilingual"}\NormalTok{)))}

\NormalTok{epreds\_diff }\OtherTok{\textless{}{-}}\NormalTok{ epreds }\SpecialCharTok{|\textgreater{}} 
    \FunctionTok{pivot\_wider}\NormalTok{(}\AttributeTok{names\_from =}\NormalTok{ condition,}
                \AttributeTok{values\_from =}\NormalTok{ .value,}
                \AttributeTok{id\_cols =} \FunctionTok{c}\NormalTok{(timebin, age, lp, .draw),}
                \AttributeTok{names\_repair =}\NormalTok{ janitor}\SpecialCharTok{::}\NormalTok{make\_clean\_names) }\SpecialCharTok{|\textgreater{}} 
    \FunctionTok{mutate}\NormalTok{(}\AttributeTok{diff =}\NormalTok{ related }\SpecialCharTok{{-}}\NormalTok{ unrelated) }

\CommentTok{\# diff\_rect \textless{}{-} epreds\_diff |\textgreater{} }
\CommentTok{\#   mean\_qi(diff) |\textgreater{} }
\CommentTok{\#   mutate(is\_cluster = .lower \textgreater{} 0 | .upper \textless{} 0) }
\CommentTok{\# }
\CommentTok{\# clusters \textless{}{-} rle(diff(diff\_rect$is\_cluster))}
\CommentTok{\# diff\_rect$cluster\_id \textless{}{-} c(0, rep(clusters$values, clusters$lengths))}
\CommentTok{\# }
\CommentTok{\# diff\_rect \textless{}{-} diff\_rect |\textgreater{} }
\CommentTok{\#   arrange(lp, timebin)}
\CommentTok{\# }
\CommentTok{\# diff\_rect \textless{}{-} }
\CommentTok{\#   cluster\_number = }
\CommentTok{\#   summarise(xmin = min(timebin),}
\CommentTok{\#             xmax = max(timebin),}
\CommentTok{\#             .by = c(lp, is\_cluster)) |\textgreater{} }
\CommentTok{\#   filter(is\_cluster)}

\CommentTok{\# diff\_obs \textless{}{-} data\_time\_related |\textgreater{}}
\CommentTok{\#   pivot\_wider(names\_from = condition, }
\CommentTok{\#               values\_from = elog,}
\CommentTok{\#               names\_repair = janitor::make\_clean\_names) |\textgreater{} mutate(diff = related {-} unrelated)}

\NormalTok{data\_time\_related\_vtarget }\SpecialCharTok{|\textgreater{}} 
    \FunctionTok{summarise}\NormalTok{(}\AttributeTok{.prop =} \FunctionTok{mean}\NormalTok{(.prop),}
              \AttributeTok{.by =} \FunctionTok{c}\NormalTok{(id, timebin, lp, condition, age)) }\SpecialCharTok{|\textgreater{}} 
    \FunctionTok{ggplot}\NormalTok{(}\FunctionTok{aes}\NormalTok{(timebin, .prop, }
               \AttributeTok{colour =}\NormalTok{ condition,}
               \AttributeTok{fill =}\NormalTok{ condition,}
               \AttributeTok{shape =}\NormalTok{ condition,}
               \AttributeTok{linetype =}\NormalTok{ condition)) }\SpecialCharTok{+}
    \FunctionTok{facet\_wrap}\NormalTok{(}\SpecialCharTok{\textasciitilde{}}\NormalTok{lp) }\SpecialCharTok{+}
    \CommentTok{\# geom\_rect(data = diff\_rect,}
    \CommentTok{\#         aes(xmin = xmin,}
    \CommentTok{\#           xmax = xmax,}
    \CommentTok{\#           ymin = {-}Inf,}
    \CommentTok{\#           ymax = Inf),}
    \CommentTok{\#         colour = NA,}
    \CommentTok{\#         fill = "orange",}
    \CommentTok{\#         alpha = 1/2,}
    \CommentTok{\#         inherit.aes = FALSE) +}
    \CommentTok{\# geom\_line(data = epreds,}
    \CommentTok{\#         aes(y = .epred,}
\CommentTok{\#           group = interaction(condition, .draw)),}
\CommentTok{\#         linetype = "solid",}
\CommentTok{\#         alpha = 0.1,}
\CommentTok{\#         linewidth = 3/4) +}
\FunctionTok{stat\_summary}\NormalTok{(}\AttributeTok{data =}\NormalTok{ epreds,}
             \FunctionTok{aes}\NormalTok{(}\AttributeTok{y =}\NormalTok{ .value),}
             \AttributeTok{fun.data =}\NormalTok{ \textbackslash{}(x) }\FunctionTok{mean\_qi}\NormalTok{(x, }\AttributeTok{.width =} \FloatTok{0.95}\NormalTok{),}
             \AttributeTok{geom =} \StringTok{"ribbon"}\NormalTok{,}
             \AttributeTok{alpha =} \FloatTok{0.5}\NormalTok{,}
             \AttributeTok{linewidth =} \DecValTok{0}\NormalTok{) }\SpecialCharTok{+}
    \FunctionTok{stat\_summary}\NormalTok{(}\AttributeTok{data =}\NormalTok{ epreds,}
                 \FunctionTok{aes}\NormalTok{(}\AttributeTok{y =}\NormalTok{ .value,}
                    \AttributeTok{linetype =}\NormalTok{ condition),}
                 \AttributeTok{fun =} \StringTok{"mean"}\NormalTok{,}
                 \AttributeTok{geom =} \StringTok{"line"}\NormalTok{,}
                 \AttributeTok{colour =} \StringTok{"black"}\NormalTok{,}
                 \AttributeTok{linewidth =} \DecValTok{3}\SpecialCharTok{/}\DecValTok{4}\NormalTok{) }\SpecialCharTok{+}
    \FunctionTok{geom\_hline}\NormalTok{(}\AttributeTok{yintercept =} \DecValTok{1}\SpecialCharTok{/}\DecValTok{2}\NormalTok{, }
               \AttributeTok{linewidth =} \DecValTok{1}\SpecialCharTok{/}\DecValTok{2}\NormalTok{,}
               \AttributeTok{colour =} \StringTok{"black"}\NormalTok{,}
               \AttributeTok{linetype =} \StringTok{"dotted"}\NormalTok{) }\SpecialCharTok{+}
    \FunctionTok{stat\_summary}\NormalTok{(}\AttributeTok{fun =}\NormalTok{ mean,}
                 \AttributeTok{geom =} \StringTok{"point"}\NormalTok{,}
                 \AttributeTok{colour =} \StringTok{"black"}\NormalTok{,}
                 \AttributeTok{size =} \FloatTok{2.5}\NormalTok{,}
                 \AttributeTok{stroke =} \DecValTok{3}\SpecialCharTok{/}\DecValTok{4}\NormalTok{) }\SpecialCharTok{+}
    \FunctionTok{labs}\NormalTok{(}\AttributeTok{x =} \StringTok{"Time (ms)"}\NormalTok{,}
         \AttributeTok{y =} \StringTok{"P(Target looking)"}\NormalTok{,}
         \AttributeTok{colour =} \StringTok{"Condition"}\NormalTok{,}
         \AttributeTok{fill =} \StringTok{"Condition"}\NormalTok{,}
         \AttributeTok{linetype =} \StringTok{"Condition"}\NormalTok{,}
         \AttributeTok{shape =} \StringTok{"Condition"}\NormalTok{) }\SpecialCharTok{+}
    \FunctionTok{theme}\NormalTok{(}\AttributeTok{legend.title =} \FunctionTok{element\_blank}\NormalTok{(),}
          \AttributeTok{axis.title.x =} \FunctionTok{element\_blank}\NormalTok{()) }\SpecialCharTok{+}
    
\NormalTok{    epreds\_diff }\SpecialCharTok{|\textgreater{}} 
    \FunctionTok{ggplot}\NormalTok{(}\FunctionTok{aes}\NormalTok{(timebin, diff)) }\SpecialCharTok{+}
    \FunctionTok{facet\_wrap}\NormalTok{(}\SpecialCharTok{\textasciitilde{}}\NormalTok{lp) }\SpecialCharTok{+}
    \CommentTok{\# geom\_rect(data = diff\_rect,}
    \CommentTok{\#         aes(xmin = xmin,}
    \CommentTok{\#           xmax = xmax,}
    \CommentTok{\#           ymin = {-}Inf,}
    \CommentTok{\#           ymax = Inf),}
    \CommentTok{\#         colour = NA,}
    \CommentTok{\#         fill = "orange",}
    \CommentTok{\#         alpha = 1/2,}
    \CommentTok{\#         inherit.aes = FALSE) +}
    \FunctionTok{stat\_lineribbon}\NormalTok{(}\AttributeTok{.width =} \FloatTok{0.95}\NormalTok{,}
                    \AttributeTok{linewidth =} \DecValTok{0}\NormalTok{,}
                    \AttributeTok{fill =} \StringTok{"grey"}\NormalTok{) }\SpecialCharTok{+}
    \FunctionTok{stat\_summary}\NormalTok{(}\AttributeTok{data =}\NormalTok{ epreds\_diff,}
                 \AttributeTok{fun =} \StringTok{"mean"}\NormalTok{,}
                 \AttributeTok{geom =} \StringTok{"line"}\NormalTok{,}
                 \AttributeTok{colour =} \StringTok{"black"}\NormalTok{,}
                 \AttributeTok{linewidth =} \DecValTok{3}\SpecialCharTok{/}\DecValTok{4}\NormalTok{) }\SpecialCharTok{+}
    \FunctionTok{geom\_hline}\NormalTok{(}\AttributeTok{yintercept =} \DecValTok{0}\NormalTok{, }
               \AttributeTok{linewidth =} \DecValTok{1}\SpecialCharTok{/}\DecValTok{2}\NormalTok{,}
               \AttributeTok{colour =} \StringTok{"black"}\NormalTok{,}
               \AttributeTok{linetype =} \StringTok{"dotted"}\NormalTok{) }\SpecialCharTok{+}
    \CommentTok{\# geom\_point(data = diff\_obs) +}
    \FunctionTok{labs}\NormalTok{(}\AttributeTok{x =} \StringTok{"Time (ms)"}\NormalTok{,}
         \AttributeTok{y =} \StringTok{"P(Target looking)"}\NormalTok{,}
         \AttributeTok{fill =} \StringTok{"CrI"}\NormalTok{) }\SpecialCharTok{+}
    \FunctionTok{theme}\NormalTok{(}\AttributeTok{strip.text =} \FunctionTok{element\_blank}\NormalTok{(),}
          \AttributeTok{legend.position =} \StringTok{"none"}\NormalTok{) }\SpecialCharTok{+}
    
    \FunctionTok{plot\_layout}\NormalTok{(}\AttributeTok{ncol =} \DecValTok{1}\NormalTok{) }\SpecialCharTok{\&}
    \FunctionTok{plot\_annotation}\NormalTok{(}\AttributeTok{tag\_levels =} \StringTok{"A"}\NormalTok{) }\SpecialCharTok{+}
    \FunctionTok{scale\_linetype\_manual}\NormalTok{(}\AttributeTok{values =} \FunctionTok{rev}\NormalTok{(}\FunctionTok{c}\NormalTok{(}\StringTok{"solid"}\NormalTok{, }\StringTok{"dashed"}\NormalTok{))) }\SpecialCharTok{\&}
    \FunctionTok{scale\_shape\_manual}\NormalTok{(}\AttributeTok{values =} \FunctionTok{c}\NormalTok{(}\DecValTok{1}\NormalTok{, }\DecValTok{2}\NormalTok{)) }\SpecialCharTok{\&}
    \FunctionTok{scale\_x\_continuous}\NormalTok{(}\AttributeTok{labels =}\NormalTok{ \textbackslash{}(x) }\FunctionTok{format}\NormalTok{((x }\SpecialCharTok{*} \FloatTok{1e2}\NormalTok{)}\SpecialCharTok{+}\DecValTok{300}\NormalTok{, }
                                            \AttributeTok{big.mark =} \StringTok{","}\NormalTok{)) }\SpecialCharTok{\&}
    \FunctionTok{theme}\NormalTok{(}\AttributeTok{panel.grid =} \FunctionTok{element\_blank}\NormalTok{(),}
          \AttributeTok{legend.position =} \StringTok{"top"}\NormalTok{) }
\end{Highlighting}
\end{Shaded}

\begin{figure}[H]

{\centering \includegraphics{manuscript_files/figure-pdf/fig-related-vtarget-1.pdf}

}

\caption{\label{fig-related-vtarget}Marginal posterior predictions of
the GAMMs. (A) Mean posterior probability of target looking across the
time course of the trial. Black lines and intervals indicate the
psoterior mean and 95\% credible intervals. Points indicate the mean
probability of target looking across participants. (B) Difference in
posterior probability of target looking between \emph{related} and
\emph{unrelated} trials. The yellow rectangle indicates, in both A and
B, the range of time points in which the 95\% credible interval of the
differences excluded zero.}

\end{figure}

\hypertarget{cognate-priming-cognate-vs.-non-cognate-1}{%
\subsubsection{Cognate priming: Cognate
vs.~Non-cognate}\label{cognate-priming-cognate-vs.-non-cognate-1}}

A model including the \emph{Cognateness} \(\times\) \emph{Group}
interaction showed the best of-of-sample predictive performance,
although the model including only \emph{Cognateness} performed
equivalently
(\(\text{ELPD}_{\mathcal{M_0}} - \text{ELPD}_{\mathcal{M_1}}\) = -3.684,
\emph{SE} = 4.241). Both models showed substantially better predictive
performance than the model including only \emph{Group}
(\(\text{ELPD}_{\mathcal{M_0}} - \text{ELPD}_{\mathcal{M_1}}\) =
-50.588, \emph{SE} = 43.666). This indicates that including the
\emph{Cognateness} predictor improved the predictive performance of the
model significantly, that including its interaction with \emph{Group}
slightly increased the performance of the model, and that the main
effect of \emph{Group} by itself barely changed the predictive
performance of the model.

\begin{Shaded}
\begin{Highlighting}[]
\NormalTok{epreds }\OtherTok{\textless{}{-}} \FunctionTok{expand\_grid}\NormalTok{(}\AttributeTok{condition =} \FunctionTok{levels}\NormalTok{(data\_time\_cognate\_vtarget}\SpecialCharTok{$}\NormalTok{condition),}
                      \AttributeTok{timebin =} \FunctionTok{seq}\NormalTok{(}\DecValTok{0}\NormalTok{, }\DecValTok{17}\NormalTok{, }\AttributeTok{length.out =} \DecValTok{100}\NormalTok{),}
                      \AttributeTok{age =} \FunctionTok{mean}\NormalTok{(data\_time\_cognate\_vtarget}\SpecialCharTok{$}\NormalTok{age),}
                      \AttributeTok{lp =} \FunctionTok{levels}\NormalTok{(data\_time\_cognate\_vtarget}\SpecialCharTok{$}\NormalTok{lp),}
                      \AttributeTok{.nsamples =} \DecValTok{1}\NormalTok{) }\SpecialCharTok{|\textgreater{}}
    \FunctionTok{add\_epred\_draws}\NormalTok{(model\_fits\_cognate\_vtarget[[}\DecValTok{4}\NormalTok{]],}
                    \AttributeTok{ndraws =} \ConstantTok{NULL}\NormalTok{,}
                    \AttributeTok{re\_formula =} \ConstantTok{NA}\NormalTok{) }\SpecialCharTok{|\textgreater{}} 
    \FunctionTok{mutate}\NormalTok{(}\AttributeTok{lp =} \FunctionTok{factor}\NormalTok{(lp, }\AttributeTok{levels =} \FunctionTok{c}\NormalTok{(}\StringTok{"Monolingual"}\NormalTok{, }\StringTok{"Bilingual"}\NormalTok{)))}

\NormalTok{epreds\_diff }\OtherTok{\textless{}{-}}\NormalTok{ epreds }\SpecialCharTok{|\textgreater{}} 
    \FunctionTok{pivot\_wider}\NormalTok{(}\AttributeTok{names\_from =}\NormalTok{ condition,}
                \AttributeTok{values\_from =}\NormalTok{ .epred,}
                \AttributeTok{id\_cols =} \FunctionTok{c}\NormalTok{(timebin, age, lp, .draw),}
                \AttributeTok{names\_repair =}\NormalTok{ janitor}\SpecialCharTok{::}\NormalTok{make\_clean\_names) }\SpecialCharTok{|\textgreater{}} 
    \FunctionTok{mutate}\NormalTok{(}\AttributeTok{diff =}\NormalTok{ cognate }\SpecialCharTok{{-}}\NormalTok{ non\_cognate) }
\CommentTok{\# }
\CommentTok{\# diff\_rect \textless{}{-} epreds\_diff |\textgreater{} }
\CommentTok{\#   mean\_qi(diff) |\textgreater{} }
\CommentTok{\#   filter(.lower \textgreater{} 0 | .upper \textless{} 0) |\textgreater{} }
\CommentTok{\#   summarise(xmin = min(timebin),}
\CommentTok{\#             xmax = max(timebin),}
\CommentTok{\#             .by = lp)}

\NormalTok{data\_time\_cognate\_vtarget }\SpecialCharTok{|\textgreater{}} 
    \FunctionTok{summarise}\NormalTok{(}\AttributeTok{.prop =} \FunctionTok{mean}\NormalTok{(.prop),}
              \AttributeTok{.by =} \FunctionTok{c}\NormalTok{(id, timebin, lp, condition, age)) }\SpecialCharTok{|\textgreater{}} 
    \FunctionTok{ggplot}\NormalTok{(}\FunctionTok{aes}\NormalTok{(timebin, .prop, }
               \AttributeTok{colour =}\NormalTok{ condition,}
               \AttributeTok{fill =}\NormalTok{ condition,}
               \AttributeTok{shape =}\NormalTok{ condition)) }\SpecialCharTok{+}
    \FunctionTok{facet\_wrap}\NormalTok{(}\SpecialCharTok{\textasciitilde{}}\NormalTok{lp) }\SpecialCharTok{+}
    \CommentTok{\# geom\_rect(data = diff\_rect,}
    \CommentTok{\#         aes(xmin = xmin,}
    \CommentTok{\#           xmax = xmax,}
    \CommentTok{\#           ymin = {-}1.5,}
    \CommentTok{\#           ymax = 1.5),}
    \CommentTok{\#         colour = NA,}
    \CommentTok{\#         fill = "orange",}
    \CommentTok{\#         alpha = 1/2,}
    \CommentTok{\#         inherit.aes = FALSE) +}
    \CommentTok{\# geom\_line(data = epreds,}
    \CommentTok{\#         aes(y = .epred,}
\CommentTok{\#           group = interaction(condition, .draw)),}
\CommentTok{\#         linetype = "solid",}
\CommentTok{\#         alpha = 0.1,}
\CommentTok{\#         linewidth = 3/4) +}
\FunctionTok{stat\_summary}\NormalTok{(}\AttributeTok{data =}\NormalTok{ epreds,}
             \FunctionTok{aes}\NormalTok{(}\AttributeTok{y =}\NormalTok{ .epred),}
             \AttributeTok{fun.data =}\NormalTok{ \textbackslash{}(x) }\FunctionTok{mean\_qi}\NormalTok{(x, }\AttributeTok{.width =} \FloatTok{0.95}\NormalTok{),}
             \AttributeTok{geom =} \StringTok{"ribbon"}\NormalTok{,}
             \AttributeTok{alpha =} \FloatTok{0.5}\NormalTok{,}
             \AttributeTok{linewidth =} \DecValTok{0}\NormalTok{) }\SpecialCharTok{+}
    \FunctionTok{stat\_summary}\NormalTok{(}\AttributeTok{data =}\NormalTok{ epreds,}
                 \FunctionTok{aes}\NormalTok{(}\AttributeTok{y =}\NormalTok{ .epred,}
                    \AttributeTok{linetype =}\NormalTok{ condition),}
                 \AttributeTok{fun =} \StringTok{"mean"}\NormalTok{,}
                 \AttributeTok{geom =} \StringTok{"line"}\NormalTok{,}
                 \AttributeTok{colour =} \StringTok{"black"}\NormalTok{,}
                 \AttributeTok{linewidth =} \DecValTok{3}\SpecialCharTok{/}\DecValTok{4}\NormalTok{) }\SpecialCharTok{+}
    \FunctionTok{geom\_hline}\NormalTok{(}\AttributeTok{yintercept =} \FloatTok{0.5}\NormalTok{, }
               \AttributeTok{linewidth =} \DecValTok{1}\SpecialCharTok{/}\DecValTok{2}\NormalTok{,}
               \AttributeTok{colour =} \StringTok{"black"}\NormalTok{,}
               \AttributeTok{linetype =} \StringTok{"dotted"}\NormalTok{) }\SpecialCharTok{+}
    \FunctionTok{stat\_summary}\NormalTok{(}\AttributeTok{fun =}\NormalTok{ mean,}
                 \AttributeTok{geom =} \StringTok{"point"}\NormalTok{,}
                 \AttributeTok{colour =} \StringTok{"black"}\NormalTok{,}
                 \AttributeTok{size =} \FloatTok{2.5}\NormalTok{,}
                 \AttributeTok{stroke =} \DecValTok{3}\SpecialCharTok{/}\DecValTok{4}\NormalTok{) }\SpecialCharTok{+}
    \FunctionTok{labs}\NormalTok{(}\AttributeTok{x =} \StringTok{"Time (ms)"}\NormalTok{,}
         \AttributeTok{y =} \StringTok{"P(Target looking)"}\NormalTok{,}
         \AttributeTok{colour =} \StringTok{"Prime type"}\NormalTok{,}
         \AttributeTok{fill =} \StringTok{"Prime type"}\NormalTok{,}
         \AttributeTok{linetype =} \StringTok{"Prime type"}\NormalTok{,}
         \AttributeTok{shape =} \StringTok{"Prime type"}\NormalTok{) }\SpecialCharTok{+}
    \FunctionTok{theme}\NormalTok{(}\AttributeTok{legend.title =} \FunctionTok{element\_blank}\NormalTok{(),}
          \AttributeTok{axis.title.x =} \FunctionTok{element\_blank}\NormalTok{()) }\SpecialCharTok{+}
    
\NormalTok{    epreds\_diff }\SpecialCharTok{|\textgreater{}} 
    \FunctionTok{ggplot}\NormalTok{(}\FunctionTok{aes}\NormalTok{(timebin, diff)) }\SpecialCharTok{+}
    \FunctionTok{facet\_wrap}\NormalTok{(}\SpecialCharTok{\textasciitilde{}}\NormalTok{lp) }\SpecialCharTok{+}
    \CommentTok{\# geom\_rect(data = diff\_rect,}
    \CommentTok{\#         aes(xmin = xmin,}
    \CommentTok{\#           xmax = xmax,}
    \CommentTok{\#           ymin = {-}3/4,}
    \CommentTok{\#           ymax = 3/4),}
    \CommentTok{\#         colour = NA,}
    \CommentTok{\#         fill = "orange",}
    \CommentTok{\#         alpha = 1/2,}
    \CommentTok{\#         inherit.aes = FALSE) +}
    \FunctionTok{stat\_lineribbon}\NormalTok{(}\AttributeTok{.width =} \FloatTok{0.95}\NormalTok{,}
                    \AttributeTok{linewidth =} \DecValTok{0}\NormalTok{,}
                    \AttributeTok{fill =} \StringTok{"grey"}\NormalTok{) }\SpecialCharTok{+}
    \FunctionTok{stat\_summary}\NormalTok{(}\AttributeTok{data =}\NormalTok{ epreds\_diff,}
                 \AttributeTok{fun =} \StringTok{"mean"}\NormalTok{,}
                 \AttributeTok{geom =} \StringTok{"line"}\NormalTok{,}
                 \AttributeTok{colour =} \StringTok{"black"}\NormalTok{,}
                 \AttributeTok{linewidth =} \DecValTok{3}\SpecialCharTok{/}\DecValTok{4}\NormalTok{) }\SpecialCharTok{+}
    \FunctionTok{geom\_hline}\NormalTok{(}\AttributeTok{yintercept =} \DecValTok{0}\NormalTok{, }
               \AttributeTok{linewidth =} \DecValTok{1}\SpecialCharTok{/}\DecValTok{2}\NormalTok{,}
               \AttributeTok{colour =} \StringTok{"black"}\NormalTok{,}
               \AttributeTok{linetype =} \StringTok{"dotted"}\NormalTok{) }\SpecialCharTok{+}
    \FunctionTok{labs}\NormalTok{(}\AttributeTok{x =} \StringTok{"Time (ms)"}\NormalTok{,}
         \AttributeTok{y =} \StringTok{"P(Target looking)"}\NormalTok{,}
         \AttributeTok{fill =} \StringTok{"CrI"}\NormalTok{) }\SpecialCharTok{+}
    \FunctionTok{theme}\NormalTok{(}\AttributeTok{strip.text =} \FunctionTok{element\_blank}\NormalTok{(),}
          \AttributeTok{legend.position =} \StringTok{"none"}\NormalTok{) }\SpecialCharTok{+}
    
    \FunctionTok{plot\_layout}\NormalTok{(}\AttributeTok{ncol =} \DecValTok{1}\NormalTok{) }\SpecialCharTok{\&}
    \FunctionTok{plot\_annotation}\NormalTok{(}\AttributeTok{tag\_levels =} \StringTok{"A"}\NormalTok{) }\SpecialCharTok{+}
    \FunctionTok{scale\_linetype\_manual}\NormalTok{(}\AttributeTok{values =} \FunctionTok{rev}\NormalTok{(}\FunctionTok{c}\NormalTok{(}\StringTok{"solid"}\NormalTok{, }\StringTok{"dashed"}\NormalTok{))) }\SpecialCharTok{\&}
    \FunctionTok{scale\_shape\_manual}\NormalTok{(}\AttributeTok{values =} \FunctionTok{c}\NormalTok{(}\DecValTok{1}\NormalTok{, }\DecValTok{2}\NormalTok{)) }\SpecialCharTok{\&}
    \FunctionTok{scale\_x\_continuous}\NormalTok{(}\AttributeTok{labels =}\NormalTok{ \textbackslash{}(x) }\FunctionTok{format}\NormalTok{((x }\SpecialCharTok{*} \FloatTok{1e2}\NormalTok{)}\SpecialCharTok{+}\DecValTok{300}\NormalTok{, }
                                            \AttributeTok{big.mark =} \StringTok{","}\NormalTok{)) }\SpecialCharTok{\&}
    \FunctionTok{theme}\NormalTok{(}\AttributeTok{panel.grid =} \FunctionTok{element\_blank}\NormalTok{(),}
          \AttributeTok{legend.position =} \StringTok{"top"}\NormalTok{) }
\end{Highlighting}
\end{Shaded}

\begin{figure}[H]

{\centering \includegraphics{manuscript_files/figure-pdf/fig-cognate-vtarget-1.pdf}

}

\caption{\label{fig-cognate-vtarget}Marginal posterior predictions of
the GAMMs. (A) Mean posterior probability of target looking across the
time course of the trial. Black lines and intervals indicate the
psoterior mean and 95\% credible intervals. Points indicate the mean
probability of target looking across participants. (B) Difference in
posterior probability of target looking between \emph{cognate} and
\emph{non-cognate} trials. The yellow rectangle indicates, in both A and
B, the range of time points in which the 95\% credible interval of the
differences excluded zero.}

\end{figure}

\hypertarget{analysis-3}{%
\subsection{Analysis 3}\label{analysis-3}}

Participants do not need to know the prime or target words, but must
look at least 10 ms to each target and distractor.

\begin{Shaded}
\begin{Highlighting}[]
\NormalTok{n\_trials }\OtherTok{\textless{}{-}} \FunctionTok{nrow}\NormalTok{(attrition\_trials\_vnone)}
\NormalTok{n\_trials\_valid }\OtherTok{\textless{}{-}} \FunctionTok{inner\_join}\NormalTok{(attrition\_participants\_vnone,}
\NormalTok{                             attrition\_trials\_vnone) }\SpecialCharTok{|\textgreater{}} 
    \FunctionTok{filter}\NormalTok{(is\_valid\_participant) }\SpecialCharTok{|\textgreater{}} 
    \FunctionTok{pull}\NormalTok{(is\_valid\_trial) }\SpecialCharTok{|\textgreater{}} 
    \FunctionTok{sum}\NormalTok{()}
\NormalTok{n\_participants\_valid }\OtherTok{\textless{}{-}} \FunctionTok{sum}\NormalTok{(attrition\_participants\_vnone}\SpecialCharTok{$}\NormalTok{is\_valid\_participant)}
\NormalTok{n\_exc\_prime }\OtherTok{\textless{}{-}} \FunctionTok{sum}\NormalTok{(}\SpecialCharTok{!}\NormalTok{attrition\_trials\_vnone}\SpecialCharTok{$}\NormalTok{is\_valid\_gaze\_prime)}
\NormalTok{n\_exc\_test }\OtherTok{\textless{}{-}} \FunctionTok{sum}\NormalTok{(}\SpecialCharTok{!}\NormalTok{attrition\_trials\_vnone}\SpecialCharTok{$}\NormalTok{is\_valid\_gaze\_test)}
\NormalTok{n\_exc\_test\_each }\OtherTok{\textless{}{-}} \FunctionTok{sum}\NormalTok{(}\SpecialCharTok{!}\NormalTok{attrition\_trials\_vnone}\SpecialCharTok{$}\NormalTok{is\_valid\_gaze\_test\_each)}
\NormalTok{n\_exc\_vocab }\OtherTok{\textless{}{-}} \FunctionTok{sum}\NormalTok{(}\SpecialCharTok{!}\NormalTok{attrition\_trials\_vnone}\SpecialCharTok{$}\NormalTok{is\_valid\_vocab)}
\NormalTok{n\_exc\_cognate }\OtherTok{\textless{}{-}} \FunctionTok{sum}\NormalTok{(}\SpecialCharTok{!}\NormalTok{attrition\_participants\_vnone}\SpecialCharTok{$}\NormalTok{is\_valid\_cognate)}
\NormalTok{n\_exc\_noncognate }\OtherTok{\textless{}{-}} \FunctionTok{sum}\NormalTok{(}\SpecialCharTok{!}\NormalTok{attrition\_participants\_vnone}\SpecialCharTok{$}\NormalTok{is\_valid\_noncognate)}
\NormalTok{n\_exc\_unrelated }\OtherTok{\textless{}{-}} \FunctionTok{sum}\NormalTok{(}\SpecialCharTok{!}\NormalTok{attrition\_participants\_vnone}\SpecialCharTok{$}\NormalTok{is\_valid\_unrelated)}

\NormalTok{n\_longitudinal }\OtherTok{\textless{}{-}}\NormalTok{ attrition\_participants\_vnone }\SpecialCharTok{|\textgreater{}}
    \FunctionTok{filter}\NormalTok{(is\_valid\_participant) }\SpecialCharTok{|\textgreater{}}
    \FunctionTok{count}\NormalTok{(id, }\AttributeTok{name =} \StringTok{"times"}\NormalTok{) }\SpecialCharTok{|\textgreater{}} 
    \FunctionTok{count}\NormalTok{(times)}
\end{Highlighting}
\end{Shaded}

We gathered data from 9,472 trials from 180 distinct participants. We
excluded trials in which participants failed to provide 50\% valid
eye-tracking samples during the prime phase (\emph{n} = 1,810) or during
the target-distractor phase (\emph{n} = 1,493). We also excluded trials
in which participants did not provide at least 5\% of valid samples to
\emph{both} target and distractor in the test phase (\emph{n} = 2,793).
After applying these trial-level inclusion criteria, we excluded
participants who did not provide at least two valid trials in the
\emph{cognate prime} condition (\emph{n} = 22), the \emph{non-cognate
prime} condition (\emph{n} = 23), or the \emph{unrelated prime}
condition (\emph{n} = 12). The resulting dataset included 5,671 trials
from 264 participants. Of those participants, 92 provided data from one
experimental session, 53 provided data from two experimental sessions,
and 22 provided data from three experimental sessions.
Table~\ref{tbl-attrition-trials} shows a detailed description of the
trial attrition.

\begin{Shaded}
\begin{Highlighting}[]
\NormalTok{attrition\_trials\_vnone }\SpecialCharTok{|\textgreater{}} 
    \FunctionTok{filter}\NormalTok{(id }\SpecialCharTok{\%in\%}\NormalTok{ attrition\_participants\_vnone}\SpecialCharTok{$}\NormalTok{id[attrition\_participants\_vnone}\SpecialCharTok{$}\NormalTok{is\_valid\_participant]) }\SpecialCharTok{|\textgreater{}} 
    \FunctionTok{left\_join}\NormalTok{(}\FunctionTok{select}\NormalTok{(participants, filename, id, age\_group),}
              \AttributeTok{by =} \FunctionTok{join\_by}\NormalTok{(filename, id, age\_group)) }\SpecialCharTok{|\textgreater{}} 
    \FunctionTok{summarise}\NormalTok{(}\AttributeTok{n\_valid =} \FunctionTok{sum}\NormalTok{(is\_valid\_trial),}
              \AttributeTok{n\_total =} \FunctionTok{n}\NormalTok{(),}
              \AttributeTok{.by =} \FunctionTok{c}\NormalTok{(id, age\_group, trial\_type)) }\SpecialCharTok{|\textgreater{}} 
    \FunctionTok{summarise}\NormalTok{(}\FunctionTok{across}\NormalTok{(n\_valid, }\FunctionTok{lst}\NormalTok{(sum, mean, sd),}
                     \AttributeTok{.names =} \StringTok{"\{.fn\}"}\NormalTok{),}
              \AttributeTok{n\_total =} \FunctionTok{sum}\NormalTok{(n\_total),}
              \AttributeTok{.by =} \FunctionTok{c}\NormalTok{(age\_group, trial\_type)) }\SpecialCharTok{|\textgreater{}} 
    \FunctionTok{mutate}\NormalTok{(}\AttributeTok{n\_excluded =}\NormalTok{ n\_total}\SpecialCharTok{{-}}\NormalTok{sum) }\SpecialCharTok{|\textgreater{}} 
    \FunctionTok{select}\NormalTok{(}\SpecialCharTok{{-}}\FunctionTok{c}\NormalTok{(n\_total)) }\SpecialCharTok{|\textgreater{}} 
    \FunctionTok{pivot\_wider}\NormalTok{(}\AttributeTok{names\_from =}\NormalTok{ trial\_type,}
                \AttributeTok{values\_from =} \FunctionTok{c}\NormalTok{(sum}\SpecialCharTok{:}\NormalTok{sd, n\_excluded),}
                \AttributeTok{names\_repair =}\NormalTok{ janitor}\SpecialCharTok{::}\NormalTok{make\_clean\_names) }\SpecialCharTok{|\textgreater{}} 
    \FunctionTok{rename\_with}\NormalTok{(\textbackslash{}(x) }\FunctionTok{gsub}\NormalTok{(}\StringTok{"non\_cognate"}\NormalTok{, }
                          \StringTok{"noncognate"}\NormalTok{,}
\NormalTok{                          x)) }\SpecialCharTok{|\textgreater{}} 
    \FunctionTok{arrange}\NormalTok{(age\_group) }\SpecialCharTok{|\textgreater{}} 
    \FunctionTok{relocate}\NormalTok{(age\_group,}
             \FunctionTok{matches}\NormalTok{(}\StringTok{"\_cognate"}\NormalTok{),}
             \FunctionTok{matches}\NormalTok{(}\StringTok{"noncognate"}\NormalTok{)) }\SpecialCharTok{|\textgreater{}} 
    \FunctionTok{gt}\NormalTok{(}\AttributeTok{rowname\_col =} \StringTok{"age\_group"}\NormalTok{) }\SpecialCharTok{|\textgreater{}} 
    \FunctionTok{grand\_summary\_rows}\NormalTok{(}\AttributeTok{columns =} \FunctionTok{matches}\NormalTok{(}\StringTok{"mean\_"}\NormalTok{),}
                       \AttributeTok{fns =} \FunctionTok{lst}\NormalTok{(Mean }\SpecialCharTok{\textasciitilde{}} \FunctionTok{mean}\NormalTok{(.)),}
                       \AttributeTok{fmt =} \SpecialCharTok{\textasciitilde{}}\FunctionTok{fmt\_number}\NormalTok{(.)) }\SpecialCharTok{|\textgreater{}}
    \FunctionTok{grand\_summary\_rows}\NormalTok{(}\AttributeTok{columns =} \FunctionTok{matches}\NormalTok{(}\StringTok{"sum\_"}\NormalTok{),}
                       \AttributeTok{fns =} \FunctionTok{lst}\NormalTok{(Sum }\SpecialCharTok{\textasciitilde{}} \FunctionTok{sum}\NormalTok{(.)),}
                       \AttributeTok{fmt =} \SpecialCharTok{\textasciitilde{}}\FunctionTok{fmt\_integer}\NormalTok{(.)) }\SpecialCharTok{|\textgreater{}}
    \FunctionTok{cols\_merge}\NormalTok{(}\FunctionTok{c}\NormalTok{(sum\_cognate, n\_excluded\_cognate), }
               \AttributeTok{pattern =} \StringTok{"\{1\} (\{2\})"}\NormalTok{) }\SpecialCharTok{|\textgreater{}} 
    \FunctionTok{cols\_merge}\NormalTok{(}\FunctionTok{c}\NormalTok{(sum\_noncognate, n\_excluded\_noncognate), }
               \AttributeTok{pattern =} \StringTok{"\{1\} (\{2\})"}\NormalTok{) }\SpecialCharTok{|\textgreater{}} 
    \FunctionTok{cols\_merge}\NormalTok{(}\FunctionTok{c}\NormalTok{(sum\_unrelated, n\_excluded\_unrelated),}
               \AttributeTok{pattern =} \StringTok{"\{1\} (\{2\})"}\NormalTok{) }\SpecialCharTok{|\textgreater{}} 
    \FunctionTok{cols\_merge\_uncert}\NormalTok{(mean\_cognate, sd\_cognate) }\SpecialCharTok{|\textgreater{}} 
    \FunctionTok{cols\_merge\_uncert}\NormalTok{(mean\_noncognate, sd\_noncognate) }\SpecialCharTok{|\textgreater{}} 
    \FunctionTok{cols\_merge\_uncert}\NormalTok{(mean\_unrelated, sd\_unrelated) }\SpecialCharTok{|\textgreater{}} 
    \FunctionTok{tab\_spanner}\NormalTok{(}\StringTok{"Cognate trials"}\NormalTok{, }\FunctionTok{ends\_with}\NormalTok{(}\StringTok{"\_cognate"}\NormalTok{)) }\SpecialCharTok{|\textgreater{}}
    \FunctionTok{tab\_spanner}\NormalTok{(}\StringTok{"Non{-}cognate trials"}\NormalTok{, }\FunctionTok{ends\_with}\NormalTok{(}\StringTok{"noncognate"}\NormalTok{)) }\SpecialCharTok{|\textgreater{}} 
    \FunctionTok{tab\_spanner}\NormalTok{(}\StringTok{"Unrelated trials"}\NormalTok{, }\FunctionTok{ends\_with}\NormalTok{(}\StringTok{"unrelated"}\NormalTok{)) }\SpecialCharTok{|\textgreater{}} 
    \FunctionTok{tab\_spanner}\NormalTok{(}\StringTok{"Related trials"}\NormalTok{, }\FunctionTok{matches}\NormalTok{(}\StringTok{"cognate"}\NormalTok{)) }\SpecialCharTok{|\textgreater{}}
    \FunctionTok{fmt\_number}\NormalTok{(}\FunctionTok{matches}\NormalTok{(}\StringTok{"mean|sd"}\NormalTok{)) }\SpecialCharTok{|\textgreater{}} 
    \FunctionTok{fmt\_integer}\NormalTok{(}\FunctionTok{matches}\NormalTok{(}\StringTok{"sum"}\NormalTok{), }\AttributeTok{sep\_mark =} \StringTok{","}\NormalTok{) }\SpecialCharTok{|\textgreater{}} 
    \FunctionTok{cols\_label}\NormalTok{(}\AttributeTok{sum\_cognate =} \StringTok{"N"}\NormalTok{,}
               \AttributeTok{sum\_noncognate =} \StringTok{"N"}\NormalTok{,}
               \AttributeTok{sum\_unrelated =} \StringTok{"N"}\NormalTok{,}
               \AttributeTok{mean\_cognate =} \StringTok{"Mean"}\NormalTok{,}
               \AttributeTok{mean\_noncognate =} \StringTok{"Mean"}\NormalTok{,}
               \AttributeTok{mean\_unrelated =} \StringTok{"Mean"}\NormalTok{)}
\end{Highlighting}
\end{Shaded}

\hypertarget{tbl-attrition-trials-vnone}{}
\begin{longtable}{l|rrrrrr}
\caption{\label{tbl-attrition-trials-vnone}Trial attrition rate by condition for included participants. Additional
excluded trials are indicated between parentheses. }\tabularnewline

\toprule
\multicolumn{1}{l}{} & \multicolumn{4}{c}{Related trials} &  &  \\ 
\cmidrule(lr){2-5}
\multicolumn{1}{l}{} & \multicolumn{2}{c}{Cognate trials} & \multicolumn{2}{c}{Non-cognate trials} & \multicolumn{2}{c}{Unrelated trials} \\ 
\cmidrule(lr){2-3} \cmidrule(lr){4-5} \cmidrule(lr){6-7}
\multicolumn{1}{l}{} & N & Mean & N & Mean & N & Mean \\ 
\midrule
21 months & $428$ (300) & $4.70$ ± $2.00$ & $425$ (303) & $4.67$ ± $1.87$ & $854$ (602) & $9.38$ ± $3.30$ \\ 
25 months & $481$ (247) & $5.29$ ± $1.81$ & $469$ (259) & $5.15$ ± $1.78$ & $952$ (504) & $10.46$ ± $3.36$ \\ 
30 months & $556$ (236) & $5.62$ ± $1.91$ & $536$ (256) & $5.41$ ± $2.11$ & $1,068$ (516) & $10.79$ ± $3.54$ \\ 
\midrule 
\midrule 
Mean & — & $5.20$ & — & $5.08$ & — & $10.21$ \\ 
Sum & $1,465$ & — & $1,430$ & — & $2,874$ & — \\ 
\bottomrule
\end{longtable}

\hypertarget{phonological-priming-related-vs.-unrelated-2}{%
\subsubsection{Phonological priming: Related
vs.~Unrelated}\label{phonological-priming-related-vs.-unrelated-2}}

A model including the \emph{Relatedness} \(\times\) \emph{Group}
interaction showed the best of-of-sample predictive performance,
although the model including only \emph{Relatedness} performed
equivalently
(\(\text{ELPD}_{\mathcal{M_0}} - \text{ELPD}_{\mathcal{M_1}}\) = -6.303,
\emph{SE} = 7.273). Both models showed substantially better predictive
performance than the model including only \emph{Group}
(\(\text{ELPD}_{\mathcal{M_0}} - \text{ELPD}_{\mathcal{M_2}}\) =
-135.716, \emph{SE} = 52.934). This indicates that including the
\emph{Relatedness} predictor improved the predictive performance of the
model significantly, that including its interaction with \emph{Group}
slightly increased the performance of the model, and that the main
effect of \emph{Group} by itself barely changed the predictive
performance of the model.

\begin{Shaded}
\begin{Highlighting}[]
\NormalTok{epreds }\OtherTok{\textless{}{-}} \FunctionTok{expand\_grid}\NormalTok{(}\AttributeTok{condition =} \FunctionTok{levels}\NormalTok{(data\_time\_related\_vnone}\SpecialCharTok{$}\NormalTok{condition),}
                      \AttributeTok{timebin =} \FunctionTok{seq}\NormalTok{(}\DecValTok{0}\NormalTok{, }\DecValTok{17}\NormalTok{, }\AttributeTok{length.out =} \DecValTok{100}\NormalTok{),}
                      \AttributeTok{age =} \FunctionTok{mean}\NormalTok{(data\_time\_related\_vnone}\SpecialCharTok{$}\NormalTok{age),}
                      \AttributeTok{lp =} \FunctionTok{levels}\NormalTok{(data\_time\_related\_vnone}\SpecialCharTok{$}\NormalTok{lp),}
                      \AttributeTok{.nsamples =} \DecValTok{1}\NormalTok{) }\SpecialCharTok{|\textgreater{}}
    \FunctionTok{add\_epred\_draws}\NormalTok{(model\_fits\_related\_vnone[[}\DecValTok{4}\NormalTok{]],}
                    \AttributeTok{ndraws =} \ConstantTok{NULL}\NormalTok{,}
                    \AttributeTok{re\_formula =} \ConstantTok{NA}\NormalTok{, }
                    \AttributeTok{value =} \StringTok{".value"}\NormalTok{) }\SpecialCharTok{|\textgreater{}} 
    \FunctionTok{mutate}\NormalTok{(}\AttributeTok{lp =} \FunctionTok{factor}\NormalTok{(lp, }\AttributeTok{levels =} \FunctionTok{c}\NormalTok{(}\StringTok{"Monolingual"}\NormalTok{, }\StringTok{"Bilingual"}\NormalTok{)))}

\NormalTok{epreds\_diff }\OtherTok{\textless{}{-}}\NormalTok{ epreds }\SpecialCharTok{|\textgreater{}} 
    \FunctionTok{pivot\_wider}\NormalTok{(}\AttributeTok{names\_from =}\NormalTok{ condition,}
                \AttributeTok{values\_from =}\NormalTok{ .value,}
                \AttributeTok{id\_cols =} \FunctionTok{c}\NormalTok{(timebin, age, lp, .draw),}
                \AttributeTok{names\_repair =}\NormalTok{ janitor}\SpecialCharTok{::}\NormalTok{make\_clean\_names) }\SpecialCharTok{|\textgreater{}} 
    \FunctionTok{mutate}\NormalTok{(}\AttributeTok{diff =}\NormalTok{ related }\SpecialCharTok{{-}}\NormalTok{ unrelated) }

\CommentTok{\# diff\_rect \textless{}{-} epreds\_diff |\textgreater{} }
\CommentTok{\#   mean\_qi(diff) |\textgreater{} }
\CommentTok{\#   mutate(is\_cluster = .lower \textgreater{} 0 | .upper \textless{} 0) }
\CommentTok{\# }
\CommentTok{\# clusters \textless{}{-} rle(diff(diff\_rect$is\_cluster))}
\CommentTok{\# diff\_rect$cluster\_id \textless{}{-} c(0, rep(clusters$values, clusters$lengths))}
\CommentTok{\# }
\CommentTok{\# diff\_rect \textless{}{-} diff\_rect |\textgreater{} }
\CommentTok{\#   arrange(lp, timebin)}
\CommentTok{\# }
\CommentTok{\# diff\_rect \textless{}{-} }
\CommentTok{\#   cluster\_number = }
\CommentTok{\#   summarise(xmin = min(timebin),}
\CommentTok{\#             xmax = max(timebin),}
\CommentTok{\#             .by = c(lp, is\_cluster)) |\textgreater{} }
\CommentTok{\#   filter(is\_cluster)}

\CommentTok{\# diff\_obs \textless{}{-} data\_time\_related |\textgreater{}}
\CommentTok{\#   pivot\_wider(names\_from = condition, }
\CommentTok{\#               values\_from = elog,}
\CommentTok{\#               names\_repair = janitor::make\_clean\_names) |\textgreater{} mutate(diff = related {-} unrelated)}

\NormalTok{data\_time\_related\_vnone }\SpecialCharTok{|\textgreater{}} 
    \FunctionTok{summarise}\NormalTok{(}\AttributeTok{.prop =} \FunctionTok{mean}\NormalTok{(.prop),}
              \AttributeTok{.by =} \FunctionTok{c}\NormalTok{(id, timebin, lp, condition, age)) }\SpecialCharTok{|\textgreater{}} 
    \FunctionTok{ggplot}\NormalTok{(}\FunctionTok{aes}\NormalTok{(timebin, .prop, }
               \AttributeTok{colour =}\NormalTok{ condition,}
               \AttributeTok{fill =}\NormalTok{ condition,}
               \AttributeTok{shape =}\NormalTok{ condition,}
               \AttributeTok{linetype =}\NormalTok{ condition)) }\SpecialCharTok{+}
    \FunctionTok{facet\_wrap}\NormalTok{(}\SpecialCharTok{\textasciitilde{}}\NormalTok{lp) }\SpecialCharTok{+}
    \CommentTok{\# geom\_rect(data = diff\_rect,}
    \CommentTok{\#         aes(xmin = xmin,}
    \CommentTok{\#           xmax = xmax,}
    \CommentTok{\#           ymin = {-}Inf,}
    \CommentTok{\#           ymax = Inf),}
    \CommentTok{\#         colour = NA,}
    \CommentTok{\#         fill = "orange",}
    \CommentTok{\#         alpha = 1/2,}
    \CommentTok{\#         inherit.aes = FALSE) +}
    \CommentTok{\# geom\_line(data = epreds,}
    \CommentTok{\#         aes(y = .epred,}
\CommentTok{\#           group = interaction(condition, .draw)),}
\CommentTok{\#         linetype = "solid",}
\CommentTok{\#         alpha = 0.1,}
\CommentTok{\#         linewidth = 3/4) +}
\FunctionTok{stat\_summary}\NormalTok{(}\AttributeTok{data =}\NormalTok{ epreds,}
             \FunctionTok{aes}\NormalTok{(}\AttributeTok{y =}\NormalTok{ .value),}
             \AttributeTok{fun.data =}\NormalTok{ \textbackslash{}(x) }\FunctionTok{mean\_qi}\NormalTok{(x, }\AttributeTok{.width =} \FloatTok{0.95}\NormalTok{),}
             \AttributeTok{geom =} \StringTok{"ribbon"}\NormalTok{,}
             \AttributeTok{alpha =} \FloatTok{0.5}\NormalTok{,}
             \AttributeTok{linewidth =} \DecValTok{0}\NormalTok{) }\SpecialCharTok{+}
    \FunctionTok{stat\_summary}\NormalTok{(}\AttributeTok{data =}\NormalTok{ epreds,}
                 \FunctionTok{aes}\NormalTok{(}\AttributeTok{y =}\NormalTok{ .value,}
                    \AttributeTok{linetype =}\NormalTok{ condition),}
                 \AttributeTok{fun =} \StringTok{"mean"}\NormalTok{,}
                 \AttributeTok{geom =} \StringTok{"line"}\NormalTok{,}
                 \AttributeTok{colour =} \StringTok{"black"}\NormalTok{,}
                 \AttributeTok{linewidth =} \DecValTok{3}\SpecialCharTok{/}\DecValTok{4}\NormalTok{) }\SpecialCharTok{+}
    \FunctionTok{geom\_hline}\NormalTok{(}\AttributeTok{yintercept =} \DecValTok{1}\SpecialCharTok{/}\DecValTok{2}\NormalTok{, }
               \AttributeTok{linewidth =} \DecValTok{1}\SpecialCharTok{/}\DecValTok{2}\NormalTok{,}
               \AttributeTok{colour =} \StringTok{"black"}\NormalTok{,}
               \AttributeTok{linetype =} \StringTok{"dotted"}\NormalTok{) }\SpecialCharTok{+}
    \FunctionTok{stat\_summary}\NormalTok{(}\AttributeTok{fun =}\NormalTok{ mean,}
                 \AttributeTok{geom =} \StringTok{"point"}\NormalTok{,}
                 \AttributeTok{colour =} \StringTok{"black"}\NormalTok{,}
                 \AttributeTok{size =} \FloatTok{2.5}\NormalTok{,}
                 \AttributeTok{stroke =} \DecValTok{3}\SpecialCharTok{/}\DecValTok{4}\NormalTok{) }\SpecialCharTok{+}
    \FunctionTok{labs}\NormalTok{(}\AttributeTok{x =} \StringTok{"Time (ms)"}\NormalTok{,}
         \AttributeTok{y =} \StringTok{"P(Target looking)"}\NormalTok{,}
         \AttributeTok{colour =} \StringTok{"Condition"}\NormalTok{,}
         \AttributeTok{fill =} \StringTok{"Condition"}\NormalTok{,}
         \AttributeTok{linetype =} \StringTok{"Condition"}\NormalTok{,}
         \AttributeTok{shape =} \StringTok{"Condition"}\NormalTok{) }\SpecialCharTok{+}
    \FunctionTok{theme}\NormalTok{(}\AttributeTok{legend.title =} \FunctionTok{element\_blank}\NormalTok{(),}
          \AttributeTok{axis.title.x =} \FunctionTok{element\_blank}\NormalTok{()) }\SpecialCharTok{+}
    
\NormalTok{    epreds\_diff }\SpecialCharTok{|\textgreater{}} 
    \FunctionTok{ggplot}\NormalTok{(}\FunctionTok{aes}\NormalTok{(timebin, diff)) }\SpecialCharTok{+}
    \FunctionTok{facet\_wrap}\NormalTok{(}\SpecialCharTok{\textasciitilde{}}\NormalTok{lp) }\SpecialCharTok{+}
    \CommentTok{\# geom\_rect(data = diff\_rect,}
    \CommentTok{\#         aes(xmin = xmin,}
    \CommentTok{\#           xmax = xmax,}
    \CommentTok{\#           ymin = {-}Inf,}
    \CommentTok{\#           ymax = Inf),}
    \CommentTok{\#         colour = NA,}
    \CommentTok{\#         fill = "orange",}
    \CommentTok{\#         alpha = 1/2,}
    \CommentTok{\#         inherit.aes = FALSE) +}
    \FunctionTok{stat\_lineribbon}\NormalTok{(}\AttributeTok{.width =} \FloatTok{0.95}\NormalTok{,}
                    \AttributeTok{linewidth =} \DecValTok{0}\NormalTok{,}
                    \AttributeTok{fill =} \StringTok{"grey"}\NormalTok{) }\SpecialCharTok{+}
    \FunctionTok{stat\_summary}\NormalTok{(}\AttributeTok{data =}\NormalTok{ epreds\_diff,}
                 \AttributeTok{fun =} \StringTok{"mean"}\NormalTok{,}
                 \AttributeTok{geom =} \StringTok{"line"}\NormalTok{,}
                 \AttributeTok{colour =} \StringTok{"black"}\NormalTok{,}
                 \AttributeTok{linewidth =} \DecValTok{3}\SpecialCharTok{/}\DecValTok{4}\NormalTok{) }\SpecialCharTok{+}
    \FunctionTok{geom\_hline}\NormalTok{(}\AttributeTok{yintercept =} \DecValTok{0}\NormalTok{, }
               \AttributeTok{linewidth =} \DecValTok{1}\SpecialCharTok{/}\DecValTok{2}\NormalTok{,}
               \AttributeTok{colour =} \StringTok{"black"}\NormalTok{,}
               \AttributeTok{linetype =} \StringTok{"dotted"}\NormalTok{) }\SpecialCharTok{+}
    \CommentTok{\# geom\_point(data = diff\_obs) +}
    \FunctionTok{labs}\NormalTok{(}\AttributeTok{x =} \StringTok{"Time (ms)"}\NormalTok{,}
         \AttributeTok{y =} \StringTok{"P(Target looking)"}\NormalTok{,}
         \AttributeTok{fill =} \StringTok{"CrI"}\NormalTok{) }\SpecialCharTok{+}
    \FunctionTok{theme}\NormalTok{(}\AttributeTok{strip.text =} \FunctionTok{element\_blank}\NormalTok{(),}
          \AttributeTok{legend.position =} \StringTok{"none"}\NormalTok{) }\SpecialCharTok{+}
    
    \FunctionTok{plot\_layout}\NormalTok{(}\AttributeTok{ncol =} \DecValTok{1}\NormalTok{) }\SpecialCharTok{\&}
    \FunctionTok{plot\_annotation}\NormalTok{(}\AttributeTok{tag\_levels =} \StringTok{"A"}\NormalTok{) }\SpecialCharTok{+}
    \FunctionTok{scale\_linetype\_manual}\NormalTok{(}\AttributeTok{values =} \FunctionTok{rev}\NormalTok{(}\FunctionTok{c}\NormalTok{(}\StringTok{"solid"}\NormalTok{, }\StringTok{"dashed"}\NormalTok{))) }\SpecialCharTok{\&}
    \FunctionTok{scale\_shape\_manual}\NormalTok{(}\AttributeTok{values =} \FunctionTok{c}\NormalTok{(}\DecValTok{1}\NormalTok{, }\DecValTok{2}\NormalTok{)) }\SpecialCharTok{\&}
    \FunctionTok{scale\_x\_continuous}\NormalTok{(}\AttributeTok{labels =}\NormalTok{ \textbackslash{}(x) }\FunctionTok{format}\NormalTok{((x }\SpecialCharTok{*} \FloatTok{1e2}\NormalTok{)}\SpecialCharTok{+}\DecValTok{300}\NormalTok{, }
                                            \AttributeTok{big.mark =} \StringTok{","}\NormalTok{)) }\SpecialCharTok{\&}
    \FunctionTok{theme}\NormalTok{(}\AttributeTok{panel.grid =} \FunctionTok{element\_blank}\NormalTok{(),}
          \AttributeTok{legend.position =} \StringTok{"top"}\NormalTok{) }
\end{Highlighting}
\end{Shaded}

\begin{figure}[H]

{\centering \includegraphics{manuscript_files/figure-pdf/fig-related-vnone-1.pdf}

}

\caption{\label{fig-related-vnone}Marginal posterior predictions of the
GAMMs. (A) Mean posterior probability of target looking across the time
course of the trial. Black lines and intervals indicate the psoterior
mean and 95\% credible intervals. Points indicate the mean probability
of target looking across participants. (B) Difference in posterior
probability of target looking between \emph{related} and
\emph{unrelated} trials. The yellow rectangle indicates, in both A and
B, the range of time points in which the 95\% credible interval of the
differences excluded zero.}

\end{figure}

\hypertarget{cognate-priming-cognate-vs.-non-cognate-2}{%
\subsubsection{Cognate priming: Cognate
vs.~Non-cognate}\label{cognate-priming-cognate-vs.-non-cognate-2}}

A model including the \emph{Cognateness} \(\times\) \emph{Group}
interaction showed the best of-of-sample predictive performance,
although the model including only \emph{Cognateness} performed
equivalently
(\(\text{ELPD}_{\mathcal{M_0}} - \text{ELPD}_{\mathcal{M_1}}\) = -1.476,
\emph{SE} = 5.759). Both models showed substantially better predictive
performance than the model including only \emph{Group}
(\(\text{ELPD}_{\mathcal{M_0}} - \text{ELPD}_{\mathcal{M_1}}\) =
-131.841, \emph{SE} = 49.45). This indicates that including the
\emph{Cognateness} predictor improved the predictive performance of the
model significantly, that including its interaction with \emph{Group}
slightly increased the performance of the model, and that the main
effect of \emph{Group} by itself barely changed the predictive
performance of the model.

\begin{Shaded}
\begin{Highlighting}[]
\NormalTok{epreds }\OtherTok{\textless{}{-}} \FunctionTok{expand\_grid}\NormalTok{(}\AttributeTok{condition =} \FunctionTok{levels}\NormalTok{(data\_time\_cognate\_vnone}\SpecialCharTok{$}\NormalTok{condition),}
                      \AttributeTok{timebin =} \FunctionTok{seq}\NormalTok{(}\DecValTok{0}\NormalTok{, }\DecValTok{17}\NormalTok{, }\AttributeTok{length.out =} \DecValTok{100}\NormalTok{),}
                      \AttributeTok{age =} \FunctionTok{mean}\NormalTok{(data\_time\_cognate\_vnone}\SpecialCharTok{$}\NormalTok{age),}
                      \AttributeTok{lp =} \FunctionTok{levels}\NormalTok{(data\_time\_cognate\_vnone}\SpecialCharTok{$}\NormalTok{lp),}
                      \AttributeTok{.nsamples =} \DecValTok{1}\NormalTok{) }\SpecialCharTok{|\textgreater{}}
    \FunctionTok{add\_epred\_draws}\NormalTok{(model\_fits\_cognate\_vnone[[}\DecValTok{4}\NormalTok{]],}
                    \AttributeTok{ndraws =} \ConstantTok{NULL}\NormalTok{,}
                    \AttributeTok{re\_formula =} \ConstantTok{NA}\NormalTok{) }\SpecialCharTok{|\textgreater{}} 
    \FunctionTok{mutate}\NormalTok{(}\AttributeTok{lp =} \FunctionTok{factor}\NormalTok{(lp, }\AttributeTok{levels =} \FunctionTok{c}\NormalTok{(}\StringTok{"Monolingual"}\NormalTok{, }\StringTok{"Bilingual"}\NormalTok{)))}

\NormalTok{epreds\_diff }\OtherTok{\textless{}{-}}\NormalTok{ epreds }\SpecialCharTok{|\textgreater{}} 
    \FunctionTok{pivot\_wider}\NormalTok{(}\AttributeTok{names\_from =}\NormalTok{ condition,}
                \AttributeTok{values\_from =}\NormalTok{ .epred,}
                \AttributeTok{id\_cols =} \FunctionTok{c}\NormalTok{(timebin, age, lp, .draw),}
                \AttributeTok{names\_repair =}\NormalTok{ janitor}\SpecialCharTok{::}\NormalTok{make\_clean\_names) }\SpecialCharTok{|\textgreater{}} 
    \FunctionTok{mutate}\NormalTok{(}\AttributeTok{diff =}\NormalTok{ cognate }\SpecialCharTok{{-}}\NormalTok{ non\_cognate) }
\CommentTok{\# }
\CommentTok{\# diff\_rect \textless{}{-} epreds\_diff |\textgreater{} }
\CommentTok{\#   mean\_qi(diff) |\textgreater{} }
\CommentTok{\#   filter(.lower \textgreater{} 0 | .upper \textless{} 0) |\textgreater{} }
\CommentTok{\#   summarise(xmin = min(timebin),}
\CommentTok{\#             xmax = max(timebin),}
\CommentTok{\#             .by = lp)}

\NormalTok{data\_time\_cognate\_vnone }\SpecialCharTok{|\textgreater{}} 
    \FunctionTok{summarise}\NormalTok{(}\AttributeTok{.prop =} \FunctionTok{mean}\NormalTok{(.prop),}
              \AttributeTok{.by =} \FunctionTok{c}\NormalTok{(id, timebin, lp, condition, age)) }\SpecialCharTok{|\textgreater{}} 
    \FunctionTok{ggplot}\NormalTok{(}\FunctionTok{aes}\NormalTok{(timebin, .prop, }
               \AttributeTok{colour =}\NormalTok{ condition,}
               \AttributeTok{fill =}\NormalTok{ condition,}
               \AttributeTok{shape =}\NormalTok{ condition)) }\SpecialCharTok{+}
    \FunctionTok{facet\_wrap}\NormalTok{(}\SpecialCharTok{\textasciitilde{}}\NormalTok{lp) }\SpecialCharTok{+}
    \CommentTok{\# geom\_rect(data = diff\_rect,}
    \CommentTok{\#         aes(xmin = xmin,}
    \CommentTok{\#           xmax = xmax,}
    \CommentTok{\#           ymin = {-}1.5,}
    \CommentTok{\#           ymax = 1.5),}
    \CommentTok{\#         colour = NA,}
    \CommentTok{\#         fill = "orange",}
    \CommentTok{\#         alpha = 1/2,}
    \CommentTok{\#         inherit.aes = FALSE) +}
    \CommentTok{\# geom\_line(data = epreds,}
    \CommentTok{\#         aes(y = .epred,}
\CommentTok{\#           group = interaction(condition, .draw)),}
\CommentTok{\#         linetype = "solid",}
\CommentTok{\#         alpha = 0.1,}
\CommentTok{\#         linewidth = 3/4) +}
\FunctionTok{stat\_summary}\NormalTok{(}\AttributeTok{data =}\NormalTok{ epreds,}
             \FunctionTok{aes}\NormalTok{(}\AttributeTok{y =}\NormalTok{ .epred),}
             \AttributeTok{fun.data =}\NormalTok{ \textbackslash{}(x) }\FunctionTok{mean\_qi}\NormalTok{(x, }\AttributeTok{.width =} \FloatTok{0.95}\NormalTok{),}
             \AttributeTok{geom =} \StringTok{"ribbon"}\NormalTok{,}
             \AttributeTok{alpha =} \FloatTok{0.5}\NormalTok{,}
             \AttributeTok{linewidth =} \DecValTok{0}\NormalTok{) }\SpecialCharTok{+}
    \FunctionTok{stat\_summary}\NormalTok{(}\AttributeTok{data =}\NormalTok{ epreds,}
                 \FunctionTok{aes}\NormalTok{(}\AttributeTok{y =}\NormalTok{ .epred,}
                    \AttributeTok{linetype =}\NormalTok{ condition),}
                 \AttributeTok{fun =} \StringTok{"mean"}\NormalTok{,}
                 \AttributeTok{geom =} \StringTok{"line"}\NormalTok{,}
                 \AttributeTok{colour =} \StringTok{"black"}\NormalTok{,}
                 \AttributeTok{linewidth =} \DecValTok{3}\SpecialCharTok{/}\DecValTok{4}\NormalTok{) }\SpecialCharTok{+}
    \FunctionTok{geom\_hline}\NormalTok{(}\AttributeTok{yintercept =} \FloatTok{0.5}\NormalTok{, }
               \AttributeTok{linewidth =} \DecValTok{1}\SpecialCharTok{/}\DecValTok{2}\NormalTok{,}
               \AttributeTok{colour =} \StringTok{"black"}\NormalTok{,}
               \AttributeTok{linetype =} \StringTok{"dotted"}\NormalTok{) }\SpecialCharTok{+}
    \FunctionTok{stat\_summary}\NormalTok{(}\AttributeTok{fun =}\NormalTok{ mean,}
                 \AttributeTok{geom =} \StringTok{"point"}\NormalTok{,}
                 \AttributeTok{colour =} \StringTok{"black"}\NormalTok{,}
                 \AttributeTok{size =} \FloatTok{2.5}\NormalTok{,}
                 \AttributeTok{stroke =} \DecValTok{3}\SpecialCharTok{/}\DecValTok{4}\NormalTok{) }\SpecialCharTok{+}
    \FunctionTok{labs}\NormalTok{(}\AttributeTok{x =} \StringTok{"Time (ms)"}\NormalTok{,}
         \AttributeTok{y =} \StringTok{"P(Target looking)"}\NormalTok{,}
         \AttributeTok{colour =} \StringTok{"Prime type"}\NormalTok{,}
         \AttributeTok{fill =} \StringTok{"Prime type"}\NormalTok{,}
         \AttributeTok{linetype =} \StringTok{"Prime type"}\NormalTok{,}
         \AttributeTok{shape =} \StringTok{"Prime type"}\NormalTok{) }\SpecialCharTok{+}
    \FunctionTok{theme}\NormalTok{(}\AttributeTok{legend.title =} \FunctionTok{element\_blank}\NormalTok{(),}
          \AttributeTok{axis.title.x =} \FunctionTok{element\_blank}\NormalTok{()) }\SpecialCharTok{+}
    
\NormalTok{    epreds\_diff }\SpecialCharTok{|\textgreater{}} 
    \FunctionTok{ggplot}\NormalTok{(}\FunctionTok{aes}\NormalTok{(timebin, diff)) }\SpecialCharTok{+}
    \FunctionTok{facet\_wrap}\NormalTok{(}\SpecialCharTok{\textasciitilde{}}\NormalTok{lp) }\SpecialCharTok{+}
    \CommentTok{\# geom\_rect(data = diff\_rect,}
    \CommentTok{\#         aes(xmin = xmin,}
    \CommentTok{\#           xmax = xmax,}
    \CommentTok{\#           ymin = {-}3/4,}
    \CommentTok{\#           ymax = 3/4),}
    \CommentTok{\#         colour = NA,}
    \CommentTok{\#         fill = "orange",}
    \CommentTok{\#         alpha = 1/2,}
    \CommentTok{\#         inherit.aes = FALSE) +}
    \FunctionTok{stat\_lineribbon}\NormalTok{(}\AttributeTok{.width =} \FloatTok{0.95}\NormalTok{,}
                    \AttributeTok{linewidth =} \DecValTok{0}\NormalTok{,}
                    \AttributeTok{fill =} \StringTok{"grey"}\NormalTok{) }\SpecialCharTok{+}
    \FunctionTok{stat\_summary}\NormalTok{(}\AttributeTok{data =}\NormalTok{ epreds\_diff,}
                 \AttributeTok{fun =} \StringTok{"mean"}\NormalTok{,}
                 \AttributeTok{geom =} \StringTok{"line"}\NormalTok{,}
                 \AttributeTok{colour =} \StringTok{"black"}\NormalTok{,}
                 \AttributeTok{linewidth =} \DecValTok{3}\SpecialCharTok{/}\DecValTok{4}\NormalTok{) }\SpecialCharTok{+}
    \FunctionTok{geom\_hline}\NormalTok{(}\AttributeTok{yintercept =} \DecValTok{0}\NormalTok{, }
               \AttributeTok{linewidth =} \DecValTok{1}\SpecialCharTok{/}\DecValTok{2}\NormalTok{,}
               \AttributeTok{colour =} \StringTok{"black"}\NormalTok{,}
               \AttributeTok{linetype =} \StringTok{"dotted"}\NormalTok{) }\SpecialCharTok{+}
    \FunctionTok{labs}\NormalTok{(}\AttributeTok{x =} \StringTok{"Time (ms)"}\NormalTok{,}
         \AttributeTok{y =} \StringTok{"P(Target looking)"}\NormalTok{,}
         \AttributeTok{fill =} \StringTok{"CrI"}\NormalTok{) }\SpecialCharTok{+}
    \FunctionTok{theme}\NormalTok{(}\AttributeTok{strip.text =} \FunctionTok{element\_blank}\NormalTok{(),}
          \AttributeTok{legend.position =} \StringTok{"none"}\NormalTok{) }\SpecialCharTok{+}
    
    \FunctionTok{plot\_layout}\NormalTok{(}\AttributeTok{ncol =} \DecValTok{1}\NormalTok{) }\SpecialCharTok{\&}
    \FunctionTok{plot\_annotation}\NormalTok{(}\AttributeTok{tag\_levels =} \StringTok{"A"}\NormalTok{) }\SpecialCharTok{+}
    \FunctionTok{scale\_linetype\_manual}\NormalTok{(}\AttributeTok{values =} \FunctionTok{rev}\NormalTok{(}\FunctionTok{c}\NormalTok{(}\StringTok{"solid"}\NormalTok{, }\StringTok{"dashed"}\NormalTok{))) }\SpecialCharTok{\&}
    \FunctionTok{scale\_shape\_manual}\NormalTok{(}\AttributeTok{values =} \FunctionTok{c}\NormalTok{(}\DecValTok{1}\NormalTok{, }\DecValTok{2}\NormalTok{)) }\SpecialCharTok{\&}
    \FunctionTok{scale\_x\_continuous}\NormalTok{(}\AttributeTok{labels =}\NormalTok{ \textbackslash{}(x) }\FunctionTok{format}\NormalTok{((x }\SpecialCharTok{*} \FloatTok{1e2}\NormalTok{)}\SpecialCharTok{+}\DecValTok{300}\NormalTok{, }
                                            \AttributeTok{big.mark =} \StringTok{","}\NormalTok{)) }\SpecialCharTok{\&}
    \FunctionTok{theme}\NormalTok{(}\AttributeTok{panel.grid =} \FunctionTok{element\_blank}\NormalTok{(),}
          \AttributeTok{legend.position =} \StringTok{"top"}\NormalTok{) }
\end{Highlighting}
\end{Shaded}

\begin{figure}[H]

{\centering \includegraphics{manuscript_files/figure-pdf/fig-cognate-vnone-1.pdf}

}

\caption{\label{fig-cognate-vnone}Marginal posterior predictions of the
GAMMs. (A) Mean posterior probability of target looking across the time
course of the trial. Black lines and intervals indicate the psoterior
mean and 95\% credible intervals. Points indicate the mean probability
of target looking across participants. (B) Difference in posterior
probability of target looking between \emph{cognate} and
\emph{non-cognate} trials. The yellow rectangle indicates, in both A and
B, the range of time points in which the 95\% credible interval of the
differences excluded zero.}

\end{figure}

\hypertarget{analysis-4}{%
\subsection{Analysis 4}\label{analysis-4}}

Participants do not need to know the prime or target words, and do not
need to look to both pictures.

\begin{Shaded}
\begin{Highlighting}[]
\NormalTok{n\_trials }\OtherTok{\textless{}{-}} \FunctionTok{nrow}\NormalTok{(attrition\_trials\_noeach)}
\NormalTok{n\_trials\_valid }\OtherTok{\textless{}{-}} \FunctionTok{inner\_join}\NormalTok{(attrition\_participants\_noeach,}
\NormalTok{                             attrition\_trials\_noeach) }\SpecialCharTok{|\textgreater{}} 
    \FunctionTok{filter}\NormalTok{(is\_valid\_participant) }\SpecialCharTok{|\textgreater{}} 
    \FunctionTok{pull}\NormalTok{(is\_valid\_trial) }\SpecialCharTok{|\textgreater{}} 
    \FunctionTok{sum}\NormalTok{()}
\NormalTok{n\_participants\_valid }\OtherTok{\textless{}{-}} \FunctionTok{sum}\NormalTok{(attrition\_participants\_noeach}\SpecialCharTok{$}\NormalTok{is\_valid\_participant)}
\NormalTok{n\_exc\_prime }\OtherTok{\textless{}{-}} \FunctionTok{sum}\NormalTok{(}\SpecialCharTok{!}\NormalTok{attrition\_trials\_noeach}\SpecialCharTok{$}\NormalTok{is\_valid\_gaze\_prime)}
\NormalTok{n\_exc\_test }\OtherTok{\textless{}{-}} \FunctionTok{sum}\NormalTok{(}\SpecialCharTok{!}\NormalTok{attrition\_trials\_noeach}\SpecialCharTok{$}\NormalTok{is\_valid\_gaze\_test)}
\NormalTok{n\_exc\_test\_each }\OtherTok{\textless{}{-}} \FunctionTok{sum}\NormalTok{(}\SpecialCharTok{!}\NormalTok{attrition\_trials\_noeach}\SpecialCharTok{$}\NormalTok{is\_valid\_gaze\_test\_each)}
\NormalTok{n\_exc\_vocab }\OtherTok{\textless{}{-}} \FunctionTok{sum}\NormalTok{(}\SpecialCharTok{!}\NormalTok{attrition\_trials\_noeach}\SpecialCharTok{$}\NormalTok{is\_valid\_vocab)}
\NormalTok{n\_exc\_cognate }\OtherTok{\textless{}{-}} \FunctionTok{sum}\NormalTok{(}\SpecialCharTok{!}\NormalTok{attrition\_participants\_noeach}\SpecialCharTok{$}\NormalTok{is\_valid\_cognate)}
\NormalTok{n\_exc\_noncognate }\OtherTok{\textless{}{-}} \FunctionTok{sum}\NormalTok{(}\SpecialCharTok{!}\NormalTok{attrition\_participants\_noeach}\SpecialCharTok{$}\NormalTok{is\_valid\_noncognate)}
\NormalTok{n\_exc\_unrelated }\OtherTok{\textless{}{-}} \FunctionTok{sum}\NormalTok{(}\SpecialCharTok{!}\NormalTok{attrition\_participants\_noeach}\SpecialCharTok{$}\NormalTok{is\_valid\_unrelated)}

\NormalTok{n\_longitudinal }\OtherTok{\textless{}{-}}\NormalTok{ attrition\_participants\_noeach }\SpecialCharTok{|\textgreater{}}
    \FunctionTok{filter}\NormalTok{(is\_valid\_participant) }\SpecialCharTok{|\textgreater{}}
    \FunctionTok{count}\NormalTok{(id, }\AttributeTok{name =} \StringTok{"times"}\NormalTok{) }\SpecialCharTok{|\textgreater{}} 
    \FunctionTok{count}\NormalTok{(times)}
\end{Highlighting}
\end{Shaded}

We gathered data from 9,472 trials from 180 distinct participants. We
excluded trials in which participants failed to provide 50\% valid
eye-tracking samples during the prime phase (\emph{n} = 1,810) or during
the target-distractor phase (\emph{n} = 1,493). After applying these
trial-level inclusion criteria, we excluded participants who did not
provide at least two valid trials in the \emph{cognate prime} condition
(\emph{n} = 15), the \emph{non-cognate prime} condition (\emph{n} = 17),
or the \emph{unrelated prime} condition (\emph{n} = 7). The resulting
dataset included 7,078 trials from 277 participants. Of those
participants, 93 provided data from one experimental session, 53
provided data from two experimental sessions, and 26 provided data from
three experimental sessions. Table~\ref{tbl-attrition-trials} shows a
detailed description of the trial attrition.

\begin{Shaded}
\begin{Highlighting}[]
\NormalTok{attrition\_trials\_noeach }\SpecialCharTok{|\textgreater{}} 
    \FunctionTok{filter}\NormalTok{(id }\SpecialCharTok{\%in\%}\NormalTok{ attrition\_participants\_noeach}\SpecialCharTok{$}\NormalTok{id[attrition\_participants\_noeach}\SpecialCharTok{$}\NormalTok{is\_valid\_participant]) }\SpecialCharTok{|\textgreater{}} 
    \FunctionTok{left\_join}\NormalTok{(}\FunctionTok{select}\NormalTok{(participants, filename, id, age\_group),}
              \AttributeTok{by =} \FunctionTok{join\_by}\NormalTok{(filename, id, age\_group)) }\SpecialCharTok{|\textgreater{}} 
    \FunctionTok{summarise}\NormalTok{(}\AttributeTok{n\_valid =} \FunctionTok{sum}\NormalTok{(is\_valid\_trial),}
              \AttributeTok{n\_total =} \FunctionTok{n}\NormalTok{(),}
              \AttributeTok{.by =} \FunctionTok{c}\NormalTok{(id, age\_group, trial\_type)) }\SpecialCharTok{|\textgreater{}} 
    \FunctionTok{summarise}\NormalTok{(}\FunctionTok{across}\NormalTok{(n\_valid, }\FunctionTok{lst}\NormalTok{(sum, mean, sd),}
                     \AttributeTok{.names =} \StringTok{"\{.fn\}"}\NormalTok{),}
              \AttributeTok{n\_total =} \FunctionTok{sum}\NormalTok{(n\_total),}
              \AttributeTok{.by =} \FunctionTok{c}\NormalTok{(age\_group, trial\_type)) }\SpecialCharTok{|\textgreater{}} 
    \FunctionTok{mutate}\NormalTok{(}\AttributeTok{n\_excluded =}\NormalTok{ n\_total}\SpecialCharTok{{-}}\NormalTok{sum) }\SpecialCharTok{|\textgreater{}} 
    \FunctionTok{select}\NormalTok{(}\SpecialCharTok{{-}}\FunctionTok{c}\NormalTok{(n\_total)) }\SpecialCharTok{|\textgreater{}} 
    \FunctionTok{pivot\_wider}\NormalTok{(}\AttributeTok{names\_from =}\NormalTok{ trial\_type,}
                \AttributeTok{values\_from =} \FunctionTok{c}\NormalTok{(sum}\SpecialCharTok{:}\NormalTok{sd, n\_excluded),}
                \AttributeTok{names\_repair =}\NormalTok{ janitor}\SpecialCharTok{::}\NormalTok{make\_clean\_names) }\SpecialCharTok{|\textgreater{}} 
    \FunctionTok{rename\_with}\NormalTok{(\textbackslash{}(x) }\FunctionTok{gsub}\NormalTok{(}\StringTok{"non\_cognate"}\NormalTok{, }
                          \StringTok{"noncognate"}\NormalTok{,}
\NormalTok{                          x)) }\SpecialCharTok{|\textgreater{}} 
    \FunctionTok{arrange}\NormalTok{(age\_group) }\SpecialCharTok{|\textgreater{}} 
    \FunctionTok{relocate}\NormalTok{(age\_group,}
             \FunctionTok{matches}\NormalTok{(}\StringTok{"\_cognate"}\NormalTok{),}
             \FunctionTok{matches}\NormalTok{(}\StringTok{"noncognate"}\NormalTok{)) }\SpecialCharTok{|\textgreater{}} 
    \FunctionTok{gt}\NormalTok{(}\AttributeTok{rowname\_col =} \StringTok{"age\_group"}\NormalTok{) }\SpecialCharTok{|\textgreater{}} 
    \FunctionTok{grand\_summary\_rows}\NormalTok{(}\AttributeTok{columns =} \FunctionTok{matches}\NormalTok{(}\StringTok{"mean\_"}\NormalTok{),}
                       \AttributeTok{fns =} \FunctionTok{lst}\NormalTok{(Mean }\SpecialCharTok{\textasciitilde{}} \FunctionTok{mean}\NormalTok{(.)),}
                       \AttributeTok{fmt =} \SpecialCharTok{\textasciitilde{}}\FunctionTok{fmt\_number}\NormalTok{(.)) }\SpecialCharTok{|\textgreater{}}
    \FunctionTok{grand\_summary\_rows}\NormalTok{(}\AttributeTok{columns =} \FunctionTok{matches}\NormalTok{(}\StringTok{"sum\_"}\NormalTok{),}
                       \AttributeTok{fns =} \FunctionTok{lst}\NormalTok{(Sum }\SpecialCharTok{\textasciitilde{}} \FunctionTok{sum}\NormalTok{(.)),}
                       \AttributeTok{fmt =} \SpecialCharTok{\textasciitilde{}}\FunctionTok{fmt\_integer}\NormalTok{(.)) }\SpecialCharTok{|\textgreater{}}
    \FunctionTok{cols\_merge}\NormalTok{(}\FunctionTok{c}\NormalTok{(sum\_cognate, n\_excluded\_cognate), }
               \AttributeTok{pattern =} \StringTok{"\{1\} (\{2\})"}\NormalTok{) }\SpecialCharTok{|\textgreater{}} 
    \FunctionTok{cols\_merge}\NormalTok{(}\FunctionTok{c}\NormalTok{(sum\_noncognate, n\_excluded\_noncognate), }
               \AttributeTok{pattern =} \StringTok{"\{1\} (\{2\})"}\NormalTok{) }\SpecialCharTok{|\textgreater{}} 
    \FunctionTok{cols\_merge}\NormalTok{(}\FunctionTok{c}\NormalTok{(sum\_unrelated, n\_excluded\_unrelated),}
               \AttributeTok{pattern =} \StringTok{"\{1\} (\{2\})"}\NormalTok{) }\SpecialCharTok{|\textgreater{}} 
    \FunctionTok{cols\_merge\_uncert}\NormalTok{(mean\_cognate, sd\_cognate) }\SpecialCharTok{|\textgreater{}} 
    \FunctionTok{cols\_merge\_uncert}\NormalTok{(mean\_noncognate, sd\_noncognate) }\SpecialCharTok{|\textgreater{}} 
    \FunctionTok{cols\_merge\_uncert}\NormalTok{(mean\_unrelated, sd\_unrelated) }\SpecialCharTok{|\textgreater{}} 
    \FunctionTok{tab\_spanner}\NormalTok{(}\StringTok{"Cognate trials"}\NormalTok{, }\FunctionTok{ends\_with}\NormalTok{(}\StringTok{"\_cognate"}\NormalTok{)) }\SpecialCharTok{|\textgreater{}}
    \FunctionTok{tab\_spanner}\NormalTok{(}\StringTok{"Non{-}cognate trials"}\NormalTok{, }\FunctionTok{ends\_with}\NormalTok{(}\StringTok{"noncognate"}\NormalTok{)) }\SpecialCharTok{|\textgreater{}} 
    \FunctionTok{tab\_spanner}\NormalTok{(}\StringTok{"Unrelated trials"}\NormalTok{, }\FunctionTok{ends\_with}\NormalTok{(}\StringTok{"unrelated"}\NormalTok{)) }\SpecialCharTok{|\textgreater{}} 
    \FunctionTok{tab\_spanner}\NormalTok{(}\StringTok{"Related trials"}\NormalTok{, }\FunctionTok{matches}\NormalTok{(}\StringTok{"cognate"}\NormalTok{)) }\SpecialCharTok{|\textgreater{}}
    \FunctionTok{fmt\_number}\NormalTok{(}\FunctionTok{matches}\NormalTok{(}\StringTok{"mean|sd"}\NormalTok{)) }\SpecialCharTok{|\textgreater{}} 
    \FunctionTok{fmt\_integer}\NormalTok{(}\FunctionTok{matches}\NormalTok{(}\StringTok{"sum"}\NormalTok{), }\AttributeTok{sep\_mark =} \StringTok{","}\NormalTok{) }\SpecialCharTok{|\textgreater{}} 
    \FunctionTok{cols\_label}\NormalTok{(}\AttributeTok{sum\_cognate =} \StringTok{"N"}\NormalTok{,}
               \AttributeTok{sum\_noncognate =} \StringTok{"N"}\NormalTok{,}
               \AttributeTok{sum\_unrelated =} \StringTok{"N"}\NormalTok{,}
               \AttributeTok{mean\_cognate =} \StringTok{"Mean"}\NormalTok{,}
               \AttributeTok{mean\_noncognate =} \StringTok{"Mean"}\NormalTok{,}
               \AttributeTok{mean\_unrelated =} \StringTok{"Mean"}\NormalTok{)}
\end{Highlighting}
\end{Shaded}

\hypertarget{tbl-attrition-trials-noeach}{}
\begin{longtable}{l|rrrrrr}
\caption{\label{tbl-attrition-trials-noeach}Trial attrition rate by condition for included participants. Additional
excluded trials are indicated between parentheses. }\tabularnewline

\toprule
\multicolumn{1}{l}{} & \multicolumn{4}{c}{Related trials} &  &  \\ 
\cmidrule(lr){2-5}
\multicolumn{1}{l}{} & \multicolumn{2}{c}{Cognate trials} & \multicolumn{2}{c}{Non-cognate trials} & \multicolumn{2}{c}{Unrelated trials} \\ 
\cmidrule(lr){2-3} \cmidrule(lr){4-5} \cmidrule(lr){6-7}
\multicolumn{1}{l}{} & N & Mean & N & Mean & N & Mean \\ 
\midrule
21 months & $541$ (195) & $5.88$ ± $1.94$ & $561$ (175) & $6.10$ ± $1.82$ & $1,089$ (383) & $11.84$ ± $3.46$ \\ 
25 months & $579$ (173) & $6.16$ ± $1.94$ & $586$ (166) & $6.23$ ± $1.85$ & $1,169$ (335) & $12.44$ ± $3.43$ \\ 
30 months & $649$ (167) & $6.36$ ± $2.01$ & $644$ (172) & $6.31$ ± $2.16$ & $1,300$ (332) & $12.75$ ± $3.65$ \\ 
\midrule 
\midrule 
Mean & — & $6.13$ & — & $6.22$ & — & $12.34$ \\ 
Sum & $1,769$ & — & $1,791$ & — & $3,558$ & — \\ 
\bottomrule
\end{longtable}

\hypertarget{phonological-priming-related-vs.-unrelated-3}{%
\subsubsection{Phonological priming: Related
vs.~Unrelated}\label{phonological-priming-related-vs.-unrelated-3}}

A model including the \emph{Relatedness} \(\times\) \emph{Group}
interaction showed the best of-of-sample predictive performance,
although the model including only \emph{Relatedness} performed
equivalently
(\(\text{ELPD}_{\mathcal{M_0}} - \text{ELPD}_{\mathcal{M_1}}\) =
-27.368, \emph{SE} = 5.299). Both models showed substantially better
predictive performance than the model including only \emph{Group}
(\(\text{ELPD}_{\mathcal{M_0}} - \text{ELPD}_{\mathcal{M_2}}\) =
-93.323, \emph{SE} = 42.373). This indicates that including the
\emph{Relatedness} predictor improved the predictive performance of the
model significantly, that including its interaction with \emph{Group}
slightly increased the performance of the model, and that the main
effect of \emph{Group} by itself barely changed the predictive
performance of the model.

\begin{Shaded}
\begin{Highlighting}[]
\NormalTok{epreds }\OtherTok{\textless{}{-}} \FunctionTok{expand\_grid}\NormalTok{(}\AttributeTok{condition =} \FunctionTok{levels}\NormalTok{(data\_time\_related\_noeach}\SpecialCharTok{$}\NormalTok{condition),}
                      \AttributeTok{timebin =} \FunctionTok{seq}\NormalTok{(}\DecValTok{0}\NormalTok{, }\DecValTok{17}\NormalTok{, }\AttributeTok{length.out =} \DecValTok{100}\NormalTok{),}
                      \AttributeTok{age =} \FunctionTok{mean}\NormalTok{(data\_time\_related\_noeach}\SpecialCharTok{$}\NormalTok{age),}
                      \AttributeTok{lp =} \FunctionTok{levels}\NormalTok{(data\_time\_related\_noeach}\SpecialCharTok{$}\NormalTok{lp),}
                      \AttributeTok{.nsamples =} \DecValTok{1}\NormalTok{) }\SpecialCharTok{|\textgreater{}}
    \FunctionTok{add\_epred\_draws}\NormalTok{(model\_fits\_related\_noeach[[}\DecValTok{4}\NormalTok{]],}
                    \AttributeTok{ndraws =} \ConstantTok{NULL}\NormalTok{,}
                    \AttributeTok{re\_formula =} \ConstantTok{NA}\NormalTok{, }
                    \AttributeTok{value =} \StringTok{".value"}\NormalTok{) }\SpecialCharTok{|\textgreater{}} 
    \FunctionTok{mutate}\NormalTok{(}\AttributeTok{lp =} \FunctionTok{factor}\NormalTok{(lp, }\AttributeTok{levels =} \FunctionTok{c}\NormalTok{(}\StringTok{"Monolingual"}\NormalTok{, }\StringTok{"Bilingual"}\NormalTok{)))}

\NormalTok{epreds\_diff }\OtherTok{\textless{}{-}}\NormalTok{ epreds }\SpecialCharTok{|\textgreater{}} 
    \FunctionTok{pivot\_wider}\NormalTok{(}\AttributeTok{names\_from =}\NormalTok{ condition,}
                \AttributeTok{values\_from =}\NormalTok{ .value,}
                \AttributeTok{id\_cols =} \FunctionTok{c}\NormalTok{(timebin, age, lp, .draw),}
                \AttributeTok{names\_repair =}\NormalTok{ janitor}\SpecialCharTok{::}\NormalTok{make\_clean\_names) }\SpecialCharTok{|\textgreater{}} 
    \FunctionTok{mutate}\NormalTok{(}\AttributeTok{diff =}\NormalTok{ related }\SpecialCharTok{{-}}\NormalTok{ unrelated) }

\CommentTok{\# diff\_rect \textless{}{-} epreds\_diff |\textgreater{} }
\CommentTok{\#   mean\_qi(diff) |\textgreater{} }
\CommentTok{\#   mutate(is\_cluster = .lower \textgreater{} 0 | .upper \textless{} 0) }
\CommentTok{\# }
\CommentTok{\# clusters \textless{}{-} rle(diff(diff\_rect$is\_cluster))}
\CommentTok{\# diff\_rect$cluster\_id \textless{}{-} c(0, rep(clusters$values, clusters$lengths))}
\CommentTok{\# }
\CommentTok{\# diff\_rect \textless{}{-} diff\_rect |\textgreater{} }
\CommentTok{\#   arrange(lp, timebin)}
\CommentTok{\# }
\CommentTok{\# diff\_rect \textless{}{-} }
\CommentTok{\#   cluster\_number = }
\CommentTok{\#   summarise(xmin = min(timebin),}
\CommentTok{\#             xmax = max(timebin),}
\CommentTok{\#             .by = c(lp, is\_cluster)) |\textgreater{} }
\CommentTok{\#   filter(is\_cluster)}

\CommentTok{\# diff\_obs \textless{}{-} data\_time\_related |\textgreater{}}
\CommentTok{\#   pivot\_wider(names\_from = condition, }
\CommentTok{\#               values\_from = elog,}
\CommentTok{\#               names\_repair = janitor::make\_clean\_names) |\textgreater{} mutate(diff = related {-} unrelated)}

\NormalTok{data\_time\_related\_noeach }\SpecialCharTok{|\textgreater{}} 
    \FunctionTok{summarise}\NormalTok{(}\AttributeTok{.prop =} \FunctionTok{mean}\NormalTok{(.prop),}
              \AttributeTok{.by =} \FunctionTok{c}\NormalTok{(id, timebin, lp, condition, age)) }\SpecialCharTok{|\textgreater{}} 
    \FunctionTok{ggplot}\NormalTok{(}\FunctionTok{aes}\NormalTok{(timebin, .prop, }
               \AttributeTok{colour =}\NormalTok{ condition,}
               \AttributeTok{fill =}\NormalTok{ condition,}
               \AttributeTok{shape =}\NormalTok{ condition,}
               \AttributeTok{linetype =}\NormalTok{ condition)) }\SpecialCharTok{+}
    \FunctionTok{facet\_wrap}\NormalTok{(}\SpecialCharTok{\textasciitilde{}}\NormalTok{lp) }\SpecialCharTok{+}
    \CommentTok{\# geom\_rect(data = diff\_rect,}
    \CommentTok{\#         aes(xmin = xmin,}
    \CommentTok{\#           xmax = xmax,}
    \CommentTok{\#           ymin = {-}Inf,}
    \CommentTok{\#           ymax = Inf),}
    \CommentTok{\#         colour = NA,}
    \CommentTok{\#         fill = "orange",}
    \CommentTok{\#         alpha = 1/2,}
    \CommentTok{\#         inherit.aes = FALSE) +}
    \CommentTok{\# geom\_line(data = epreds,}
    \CommentTok{\#         aes(y = .epred,}
\CommentTok{\#           group = interaction(condition, .draw)),}
\CommentTok{\#         linetype = "solid",}
\CommentTok{\#         alpha = 0.1,}
\CommentTok{\#         linewidth = 3/4) +}
\FunctionTok{stat\_summary}\NormalTok{(}\AttributeTok{data =}\NormalTok{ epreds,}
             \FunctionTok{aes}\NormalTok{(}\AttributeTok{y =}\NormalTok{ .value),}
             \AttributeTok{fun.data =}\NormalTok{ \textbackslash{}(x) }\FunctionTok{mean\_qi}\NormalTok{(x, }\AttributeTok{.width =} \FloatTok{0.95}\NormalTok{),}
             \AttributeTok{geom =} \StringTok{"ribbon"}\NormalTok{,}
             \AttributeTok{alpha =} \FloatTok{0.5}\NormalTok{,}
             \AttributeTok{linewidth =} \DecValTok{0}\NormalTok{) }\SpecialCharTok{+}
    \FunctionTok{stat\_summary}\NormalTok{(}\AttributeTok{data =}\NormalTok{ epreds,}
                 \FunctionTok{aes}\NormalTok{(}\AttributeTok{y =}\NormalTok{ .value,}
                    \AttributeTok{linetype =}\NormalTok{ condition),}
                 \AttributeTok{fun =} \StringTok{"mean"}\NormalTok{,}
                 \AttributeTok{geom =} \StringTok{"line"}\NormalTok{,}
                 \AttributeTok{colour =} \StringTok{"black"}\NormalTok{,}
                 \AttributeTok{linewidth =} \DecValTok{3}\SpecialCharTok{/}\DecValTok{4}\NormalTok{) }\SpecialCharTok{+}
    \FunctionTok{geom\_hline}\NormalTok{(}\AttributeTok{yintercept =} \DecValTok{1}\SpecialCharTok{/}\DecValTok{2}\NormalTok{, }
               \AttributeTok{linewidth =} \DecValTok{1}\SpecialCharTok{/}\DecValTok{2}\NormalTok{,}
               \AttributeTok{colour =} \StringTok{"black"}\NormalTok{,}
               \AttributeTok{linetype =} \StringTok{"dotted"}\NormalTok{) }\SpecialCharTok{+}
    \FunctionTok{stat\_summary}\NormalTok{(}\AttributeTok{fun =}\NormalTok{ mean,}
                 \AttributeTok{geom =} \StringTok{"point"}\NormalTok{,}
                 \AttributeTok{colour =} \StringTok{"black"}\NormalTok{,}
                 \AttributeTok{size =} \FloatTok{2.5}\NormalTok{,}
                 \AttributeTok{stroke =} \DecValTok{3}\SpecialCharTok{/}\DecValTok{4}\NormalTok{) }\SpecialCharTok{+}
    \FunctionTok{labs}\NormalTok{(}\AttributeTok{x =} \StringTok{"Time (ms)"}\NormalTok{,}
         \AttributeTok{y =} \StringTok{"P(Target looking)"}\NormalTok{,}
         \AttributeTok{colour =} \StringTok{"Condition"}\NormalTok{,}
         \AttributeTok{fill =} \StringTok{"Condition"}\NormalTok{,}
         \AttributeTok{linetype =} \StringTok{"Condition"}\NormalTok{,}
         \AttributeTok{shape =} \StringTok{"Condition"}\NormalTok{) }\SpecialCharTok{+}
    \FunctionTok{theme}\NormalTok{(}\AttributeTok{legend.title =} \FunctionTok{element\_blank}\NormalTok{(),}
          \AttributeTok{axis.title.x =} \FunctionTok{element\_blank}\NormalTok{()) }\SpecialCharTok{+}
    
\NormalTok{    epreds\_diff }\SpecialCharTok{|\textgreater{}} 
    \FunctionTok{ggplot}\NormalTok{(}\FunctionTok{aes}\NormalTok{(timebin, diff)) }\SpecialCharTok{+}
    \FunctionTok{facet\_wrap}\NormalTok{(}\SpecialCharTok{\textasciitilde{}}\NormalTok{lp) }\SpecialCharTok{+}
    \CommentTok{\# geom\_rect(data = diff\_rect,}
    \CommentTok{\#         aes(xmin = xmin,}
    \CommentTok{\#           xmax = xmax,}
    \CommentTok{\#           ymin = {-}Inf,}
    \CommentTok{\#           ymax = Inf),}
    \CommentTok{\#         colour = NA,}
    \CommentTok{\#         fill = "orange",}
    \CommentTok{\#         alpha = 1/2,}
    \CommentTok{\#         inherit.aes = FALSE) +}
    \FunctionTok{stat\_lineribbon}\NormalTok{(}\AttributeTok{.width =} \FloatTok{0.95}\NormalTok{,}
                    \AttributeTok{linewidth =} \DecValTok{0}\NormalTok{,}
                    \AttributeTok{fill =} \StringTok{"grey"}\NormalTok{) }\SpecialCharTok{+}
    \FunctionTok{stat\_summary}\NormalTok{(}\AttributeTok{data =}\NormalTok{ epreds\_diff,}
                 \AttributeTok{fun =} \StringTok{"mean"}\NormalTok{,}
                 \AttributeTok{geom =} \StringTok{"line"}\NormalTok{,}
                 \AttributeTok{colour =} \StringTok{"black"}\NormalTok{,}
                 \AttributeTok{linewidth =} \DecValTok{3}\SpecialCharTok{/}\DecValTok{4}\NormalTok{) }\SpecialCharTok{+}
    \FunctionTok{geom\_hline}\NormalTok{(}\AttributeTok{yintercept =} \DecValTok{0}\NormalTok{, }
               \AttributeTok{linewidth =} \DecValTok{1}\SpecialCharTok{/}\DecValTok{2}\NormalTok{,}
               \AttributeTok{colour =} \StringTok{"black"}\NormalTok{,}
               \AttributeTok{linetype =} \StringTok{"dotted"}\NormalTok{) }\SpecialCharTok{+}
    \CommentTok{\# geom\_point(data = diff\_obs) +}
    \FunctionTok{labs}\NormalTok{(}\AttributeTok{x =} \StringTok{"Time (ms)"}\NormalTok{,}
         \AttributeTok{y =} \StringTok{"P(Target looking)"}\NormalTok{,}
         \AttributeTok{fill =} \StringTok{"CrI"}\NormalTok{) }\SpecialCharTok{+}
    \FunctionTok{theme}\NormalTok{(}\AttributeTok{strip.text =} \FunctionTok{element\_blank}\NormalTok{(),}
          \AttributeTok{legend.position =} \StringTok{"none"}\NormalTok{) }\SpecialCharTok{+}
    
    \FunctionTok{plot\_layout}\NormalTok{(}\AttributeTok{ncol =} \DecValTok{1}\NormalTok{) }\SpecialCharTok{\&}
    \FunctionTok{plot\_annotation}\NormalTok{(}\AttributeTok{tag\_levels =} \StringTok{"A"}\NormalTok{) }\SpecialCharTok{+}
    \FunctionTok{scale\_linetype\_manual}\NormalTok{(}\AttributeTok{values =} \FunctionTok{rev}\NormalTok{(}\FunctionTok{c}\NormalTok{(}\StringTok{"solid"}\NormalTok{, }\StringTok{"dashed"}\NormalTok{))) }\SpecialCharTok{\&}
    \FunctionTok{scale\_shape\_manual}\NormalTok{(}\AttributeTok{values =} \FunctionTok{c}\NormalTok{(}\DecValTok{1}\NormalTok{, }\DecValTok{2}\NormalTok{)) }\SpecialCharTok{\&}
    \FunctionTok{scale\_x\_continuous}\NormalTok{(}\AttributeTok{labels =}\NormalTok{ \textbackslash{}(x) }\FunctionTok{format}\NormalTok{((x }\SpecialCharTok{*} \FloatTok{1e2}\NormalTok{)}\SpecialCharTok{+}\DecValTok{300}\NormalTok{, }
                                            \AttributeTok{big.mark =} \StringTok{","}\NormalTok{)) }\SpecialCharTok{\&}
    \FunctionTok{theme}\NormalTok{(}\AttributeTok{panel.grid =} \FunctionTok{element\_blank}\NormalTok{(),}
          \AttributeTok{legend.position =} \StringTok{"top"}\NormalTok{) }
\end{Highlighting}
\end{Shaded}

\begin{figure}[H]

{\centering \includegraphics{manuscript_files/figure-pdf/fig-related-noeach-1.pdf}

}

\caption{\label{fig-related-noeach}Marginal posterior predictions of the
GAMMs. (A) Mean posterior probability of target looking across the time
course of the trial. Black lines and intervals indicate the psoterior
mean and 95\% credible intervals. Points indicate the mean probability
of target looking across participants. (B) Difference in posterior
probability of target looking between \emph{related} and
\emph{unrelated} trials. The yellow rectangle indicates, in both A and
B, the range of time points in which the 95\% credible interval of the
differences excluded zero.}

\end{figure}

\hypertarget{cognate-priming-cognate-vs.-non-cognate-3}{%
\subsubsection{Cognate priming: Cognate
vs.~Non-cognate}\label{cognate-priming-cognate-vs.-non-cognate-3}}

A model including the \emph{Cognateness} \(\times\) \emph{Group}
interaction showed the best of-of-sample predictive performance,
although the model including only \emph{Cognateness} performed
equivalently
(\(\text{ELPD}_{\mathcal{M_0}} - \text{ELPD}_{\mathcal{M_1}}\) =
-25.189, \emph{SE} = 5.023). Both models showed substantially better
predictive performance than the model including only \emph{Group}
(\(\text{ELPD}_{\mathcal{M_0}} - \text{ELPD}_{\mathcal{M_1}}\) =
-60.098, \emph{SE} = 41.472). This indicates that including the
\emph{Cognateness} predictor improved the predictive performance of the
model significantly, that including its interaction with \emph{Group}
slightly increased the performance of the model, and that the main
effect of \emph{Group} by itself barely changed the predictive
performance of the model.

\begin{Shaded}
\begin{Highlighting}[]
\NormalTok{epreds }\OtherTok{\textless{}{-}} \FunctionTok{expand\_grid}\NormalTok{(}\AttributeTok{condition =} \FunctionTok{levels}\NormalTok{(data\_time\_cognate\_noeach}\SpecialCharTok{$}\NormalTok{condition),}
                      \AttributeTok{timebin =} \FunctionTok{seq}\NormalTok{(}\DecValTok{0}\NormalTok{, }\DecValTok{17}\NormalTok{, }\AttributeTok{length.out =} \DecValTok{100}\NormalTok{),}
                      \AttributeTok{age =} \FunctionTok{mean}\NormalTok{(data\_time\_cognate\_noeach}\SpecialCharTok{$}\NormalTok{age),}
                      \AttributeTok{lp =} \FunctionTok{levels}\NormalTok{(data\_time\_cognate\_noeach}\SpecialCharTok{$}\NormalTok{lp),}
                      \AttributeTok{.nsamples =} \DecValTok{1}\NormalTok{) }\SpecialCharTok{|\textgreater{}}
    \FunctionTok{add\_epred\_draws}\NormalTok{(model\_fits\_cognate\_noeach[[}\DecValTok{4}\NormalTok{]],}
                    \AttributeTok{ndraws =} \ConstantTok{NULL}\NormalTok{,}
                    \AttributeTok{re\_formula =} \ConstantTok{NA}\NormalTok{) }\SpecialCharTok{|\textgreater{}} 
    \FunctionTok{mutate}\NormalTok{(}\AttributeTok{lp =} \FunctionTok{factor}\NormalTok{(lp, }\AttributeTok{levels =} \FunctionTok{c}\NormalTok{(}\StringTok{"Monolingual"}\NormalTok{, }\StringTok{"Bilingual"}\NormalTok{)))}

\NormalTok{epreds\_diff }\OtherTok{\textless{}{-}}\NormalTok{ epreds }\SpecialCharTok{|\textgreater{}} 
    \FunctionTok{pivot\_wider}\NormalTok{(}\AttributeTok{names\_from =}\NormalTok{ condition,}
                \AttributeTok{values\_from =}\NormalTok{ .epred,}
                \AttributeTok{id\_cols =} \FunctionTok{c}\NormalTok{(timebin, age, lp, .draw),}
                \AttributeTok{names\_repair =}\NormalTok{ janitor}\SpecialCharTok{::}\NormalTok{make\_clean\_names) }\SpecialCharTok{|\textgreater{}} 
    \FunctionTok{mutate}\NormalTok{(}\AttributeTok{diff =}\NormalTok{ cognate }\SpecialCharTok{{-}}\NormalTok{ non\_cognate) }
\CommentTok{\# }
\CommentTok{\# diff\_rect \textless{}{-} epreds\_diff |\textgreater{} }
\CommentTok{\#   mean\_qi(diff) |\textgreater{} }
\CommentTok{\#   filter(.lower \textgreater{} 0 | .upper \textless{} 0) |\textgreater{} }
\CommentTok{\#   summarise(xmin = min(timebin),}
\CommentTok{\#             xmax = max(timebin),}
\CommentTok{\#             .by = lp)}

\NormalTok{data\_time\_cognate\_noeach }\SpecialCharTok{|\textgreater{}} 
    \FunctionTok{summarise}\NormalTok{(}\AttributeTok{.prop =} \FunctionTok{mean}\NormalTok{(.prop),}
              \AttributeTok{.by =} \FunctionTok{c}\NormalTok{(id, timebin, lp, condition, age)) }\SpecialCharTok{|\textgreater{}} 
    \FunctionTok{ggplot}\NormalTok{(}\FunctionTok{aes}\NormalTok{(timebin, .prop, }
               \AttributeTok{colour =}\NormalTok{ condition,}
               \AttributeTok{fill =}\NormalTok{ condition,}
               \AttributeTok{shape =}\NormalTok{ condition)) }\SpecialCharTok{+}
    \FunctionTok{facet\_wrap}\NormalTok{(}\SpecialCharTok{\textasciitilde{}}\NormalTok{lp) }\SpecialCharTok{+}
    \CommentTok{\# geom\_rect(data = diff\_rect,}
    \CommentTok{\#         aes(xmin = xmin,}
    \CommentTok{\#           xmax = xmax,}
    \CommentTok{\#           ymin = {-}1.5,}
    \CommentTok{\#           ymax = 1.5),}
    \CommentTok{\#         colour = NA,}
    \CommentTok{\#         fill = "orange",}
    \CommentTok{\#         alpha = 1/2,}
    \CommentTok{\#         inherit.aes = FALSE) +}
    \CommentTok{\# geom\_line(data = epreds,}
    \CommentTok{\#         aes(y = .epred,}
\CommentTok{\#           group = interaction(condition, .draw)),}
\CommentTok{\#         linetype = "solid",}
\CommentTok{\#         alpha = 0.1,}
\CommentTok{\#         linewidth = 3/4) +}
\FunctionTok{stat\_summary}\NormalTok{(}\AttributeTok{data =}\NormalTok{ epreds,}
             \FunctionTok{aes}\NormalTok{(}\AttributeTok{y =}\NormalTok{ .epred),}
             \AttributeTok{fun.data =}\NormalTok{ \textbackslash{}(x) }\FunctionTok{mean\_qi}\NormalTok{(x, }\AttributeTok{.width =} \FloatTok{0.95}\NormalTok{),}
             \AttributeTok{geom =} \StringTok{"ribbon"}\NormalTok{,}
             \AttributeTok{alpha =} \FloatTok{0.5}\NormalTok{,}
             \AttributeTok{linewidth =} \DecValTok{0}\NormalTok{) }\SpecialCharTok{+}
    \FunctionTok{stat\_summary}\NormalTok{(}\AttributeTok{data =}\NormalTok{ epreds,}
                 \FunctionTok{aes}\NormalTok{(}\AttributeTok{y =}\NormalTok{ .epred,}
                    \AttributeTok{linetype =}\NormalTok{ condition),}
                 \AttributeTok{fun =} \StringTok{"mean"}\NormalTok{,}
                 \AttributeTok{geom =} \StringTok{"line"}\NormalTok{,}
                 \AttributeTok{colour =} \StringTok{"black"}\NormalTok{,}
                 \AttributeTok{linewidth =} \DecValTok{3}\SpecialCharTok{/}\DecValTok{4}\NormalTok{) }\SpecialCharTok{+}
    \FunctionTok{geom\_hline}\NormalTok{(}\AttributeTok{yintercept =} \FloatTok{0.5}\NormalTok{, }
               \AttributeTok{linewidth =} \DecValTok{1}\SpecialCharTok{/}\DecValTok{2}\NormalTok{,}
               \AttributeTok{colour =} \StringTok{"black"}\NormalTok{,}
               \AttributeTok{linetype =} \StringTok{"dotted"}\NormalTok{) }\SpecialCharTok{+}
    \FunctionTok{stat\_summary}\NormalTok{(}\AttributeTok{fun =}\NormalTok{ mean,}
                 \AttributeTok{geom =} \StringTok{"point"}\NormalTok{,}
                 \AttributeTok{colour =} \StringTok{"black"}\NormalTok{,}
                 \AttributeTok{size =} \FloatTok{2.5}\NormalTok{,}
                 \AttributeTok{stroke =} \DecValTok{3}\SpecialCharTok{/}\DecValTok{4}\NormalTok{) }\SpecialCharTok{+}
    \FunctionTok{labs}\NormalTok{(}\AttributeTok{x =} \StringTok{"Time (ms)"}\NormalTok{,}
         \AttributeTok{y =} \StringTok{"P(Target looking)"}\NormalTok{,}
         \AttributeTok{colour =} \StringTok{"Prime type"}\NormalTok{,}
         \AttributeTok{fill =} \StringTok{"Prime type"}\NormalTok{,}
         \AttributeTok{linetype =} \StringTok{"Prime type"}\NormalTok{,}
         \AttributeTok{shape =} \StringTok{"Prime type"}\NormalTok{) }\SpecialCharTok{+}
    \FunctionTok{theme}\NormalTok{(}\AttributeTok{legend.title =} \FunctionTok{element\_blank}\NormalTok{(),}
          \AttributeTok{axis.title.x =} \FunctionTok{element\_blank}\NormalTok{()) }\SpecialCharTok{+}
    
\NormalTok{    epreds\_diff }\SpecialCharTok{|\textgreater{}} 
    \FunctionTok{ggplot}\NormalTok{(}\FunctionTok{aes}\NormalTok{(timebin, diff)) }\SpecialCharTok{+}
    \FunctionTok{facet\_wrap}\NormalTok{(}\SpecialCharTok{\textasciitilde{}}\NormalTok{lp) }\SpecialCharTok{+}
    \CommentTok{\# geom\_rect(data = diff\_rect,}
    \CommentTok{\#         aes(xmin = xmin,}
    \CommentTok{\#           xmax = xmax,}
    \CommentTok{\#           ymin = {-}3/4,}
    \CommentTok{\#           ymax = 3/4),}
    \CommentTok{\#         colour = NA,}
    \CommentTok{\#         fill = "orange",}
    \CommentTok{\#         alpha = 1/2,}
    \CommentTok{\#         inherit.aes = FALSE) +}
    \FunctionTok{stat\_lineribbon}\NormalTok{(}\AttributeTok{.width =} \FloatTok{0.95}\NormalTok{,}
                    \AttributeTok{linewidth =} \DecValTok{0}\NormalTok{,}
                    \AttributeTok{fill =} \StringTok{"grey"}\NormalTok{) }\SpecialCharTok{+}
    \FunctionTok{stat\_summary}\NormalTok{(}\AttributeTok{data =}\NormalTok{ epreds\_diff,}
                 \AttributeTok{fun =} \StringTok{"mean"}\NormalTok{,}
                 \AttributeTok{geom =} \StringTok{"line"}\NormalTok{,}
                 \AttributeTok{colour =} \StringTok{"black"}\NormalTok{,}
                 \AttributeTok{linewidth =} \DecValTok{3}\SpecialCharTok{/}\DecValTok{4}\NormalTok{) }\SpecialCharTok{+}
    \FunctionTok{geom\_hline}\NormalTok{(}\AttributeTok{yintercept =} \DecValTok{0}\NormalTok{, }
               \AttributeTok{linewidth =} \DecValTok{1}\SpecialCharTok{/}\DecValTok{2}\NormalTok{,}
               \AttributeTok{colour =} \StringTok{"black"}\NormalTok{,}
               \AttributeTok{linetype =} \StringTok{"dotted"}\NormalTok{) }\SpecialCharTok{+}
    \FunctionTok{labs}\NormalTok{(}\AttributeTok{x =} \StringTok{"Time (ms)"}\NormalTok{,}
         \AttributeTok{y =} \StringTok{"P(Target looking)"}\NormalTok{,}
         \AttributeTok{fill =} \StringTok{"CrI"}\NormalTok{) }\SpecialCharTok{+}
    \FunctionTok{theme}\NormalTok{(}\AttributeTok{strip.text =} \FunctionTok{element\_blank}\NormalTok{(),}
          \AttributeTok{legend.position =} \StringTok{"none"}\NormalTok{) }\SpecialCharTok{+}
    
    \FunctionTok{plot\_layout}\NormalTok{(}\AttributeTok{ncol =} \DecValTok{1}\NormalTok{) }\SpecialCharTok{\&}
    \FunctionTok{plot\_annotation}\NormalTok{(}\AttributeTok{tag\_levels =} \StringTok{"A"}\NormalTok{) }\SpecialCharTok{+}
    \FunctionTok{scale\_linetype\_manual}\NormalTok{(}\AttributeTok{values =} \FunctionTok{rev}\NormalTok{(}\FunctionTok{c}\NormalTok{(}\StringTok{"solid"}\NormalTok{, }\StringTok{"dashed"}\NormalTok{))) }\SpecialCharTok{\&}
    \FunctionTok{scale\_shape\_manual}\NormalTok{(}\AttributeTok{values =} \FunctionTok{c}\NormalTok{(}\DecValTok{1}\NormalTok{, }\DecValTok{2}\NormalTok{)) }\SpecialCharTok{\&}
    \FunctionTok{scale\_x\_continuous}\NormalTok{(}\AttributeTok{labels =}\NormalTok{ \textbackslash{}(x) }\FunctionTok{format}\NormalTok{((x }\SpecialCharTok{*} \FloatTok{1e2}\NormalTok{)}\SpecialCharTok{+}\DecValTok{300}\NormalTok{, }
                                            \AttributeTok{big.mark =} \StringTok{","}\NormalTok{)) }\SpecialCharTok{\&}
    \FunctionTok{theme}\NormalTok{(}\AttributeTok{panel.grid =} \FunctionTok{element\_blank}\NormalTok{(),}
          \AttributeTok{legend.position =} \StringTok{"top"}\NormalTok{) }
\end{Highlighting}
\end{Shaded}

\begin{figure}[H]

{\centering \includegraphics{manuscript_files/figure-pdf/fig-cognate-noeach-1.pdf}

}

\caption{\label{fig-cognate-noeach}Marginal posterior predictions of the
GAMMs. (A) Mean posterior probability of target looking across the time
course of the trial. Black lines and intervals indicate the psoterior
mean and 95\% credible intervals. Points indicate the mean probability
of target looking across participants. (B) Difference in posterior
probability of target looking between \emph{cognate} and
\emph{non-cognate} trials. The yellow rectangle indicates, in both A and
B, the range of time points in which the 95\% credible interval of the
differences excluded zero.}

\end{figure}

\hypertarget{discussion}{%
\section{Discussion}\label{discussion}}

\hypertarget{appendix}{%
\section{Appendix}\label{appendix}}

\hypertarget{appendix-a-imputing-voabulary-size-scores}{%
\subsection{Appendix A: imputing voabulary size
scores}\label{appendix-a-imputing-voabulary-size-scores}}

\begin{Shaded}
\begin{Highlighting}[]
\NormalTok{vocabulary\_tmp }\OtherTok{\textless{}{-}}\NormalTok{ vocabulary }\SpecialCharTok{|\textgreater{}} 
    \FunctionTok{left\_join}\NormalTok{(}\FunctionTok{select}\NormalTok{(participants, }\AttributeTok{id =}\NormalTok{ id\_db, age, age\_group, filename),}
              \AttributeTok{by =} \FunctionTok{join\_by}\NormalTok{(filename)) }\SpecialCharTok{|\textgreater{}} 
    \FunctionTok{relocate}\NormalTok{(id, age) }\SpecialCharTok{|\textgreater{}} 
    \FunctionTok{select}\NormalTok{(}\SpecialCharTok{{-}}\NormalTok{filename) }\SpecialCharTok{|\textgreater{}} 
    \FunctionTok{filter}\NormalTok{(is\_imputed) }\SpecialCharTok{|\textgreater{}} 
    \FunctionTok{mutate}\NormalTok{(}\AttributeTok{id =} \FunctionTok{as.character}\NormalTok{(id))}

\NormalTok{bvq\_data}\SpecialCharTok{$}\NormalTok{vocabulary }\SpecialCharTok{|\textgreater{}} 
    \FunctionTok{inner\_join}\NormalTok{(}\FunctionTok{distinct}\NormalTok{(bvq\_data}\SpecialCharTok{$}\NormalTok{logs, id, age\_group, age),}
               \AttributeTok{by =} \FunctionTok{join\_by}\NormalTok{(id, age\_group)) }\SpecialCharTok{|\textgreater{}} 
    \FunctionTok{mutate}\NormalTok{(}\AttributeTok{is\_imputed =} \ConstantTok{FALSE}\NormalTok{) }\SpecialCharTok{|\textgreater{}} 
    \FunctionTok{bind\_rows}\NormalTok{(vocabulary\_tmp) }\SpecialCharTok{|\textgreater{}} 
    \FunctionTok{pivot\_longer}\NormalTok{(}\FunctionTok{ends\_with}\NormalTok{(}\StringTok{"\_prop"}\NormalTok{),}
                 \AttributeTok{names\_to =} \StringTok{"measure"}\NormalTok{,}
                 \AttributeTok{values\_to =} \StringTok{"prop"}\NormalTok{) }\SpecialCharTok{|\textgreater{}} 
    \FunctionTok{drop\_na}\NormalTok{(prop) }\SpecialCharTok{|\textgreater{}} 
    \FunctionTok{mutate}\NormalTok{(}\AttributeTok{is\_imputed =} \FunctionTok{ifelse}\NormalTok{(is\_imputed, }\StringTok{"Imputed"}\NormalTok{, }\StringTok{"Observed"}\NormalTok{),}
           \AttributeTok{measure =} \FunctionTok{factor}\NormalTok{(measure,}
                         \AttributeTok{levels =} \FunctionTok{c}\NormalTok{(}\StringTok{"total\_prop"}\NormalTok{,}
                                   \StringTok{"l1\_prop"}\NormalTok{,}
                                   \StringTok{"l2\_prop"}\NormalTok{,}
                                   \StringTok{"concept\_prop"}\NormalTok{,}
                                   \StringTok{"te\_prop"}\NormalTok{),}
                         \AttributeTok{labels =} \FunctionTok{c}\NormalTok{(}\StringTok{"Total"}\NormalTok{,}
                                   \StringTok{"L1"}\NormalTok{,}
                                   \StringTok{"L2"}\NormalTok{,}
                                   \StringTok{"Conceptual"}\NormalTok{,}
                                   \StringTok{"TE"}\NormalTok{))) }\SpecialCharTok{|\textgreater{}} 
    \FunctionTok{ggplot}\NormalTok{(}\FunctionTok{aes}\NormalTok{(age, prop, }
               \AttributeTok{colour =}\NormalTok{ is\_imputed,}
               \AttributeTok{fill =}\NormalTok{ is\_imputed)) }\SpecialCharTok{+}
    \FunctionTok{facet\_wrap}\NormalTok{(}\SpecialCharTok{\textasciitilde{}}\NormalTok{measure) }\SpecialCharTok{+}
    \FunctionTok{geom\_point}\NormalTok{(}\AttributeTok{alpha =} \DecValTok{1}\SpecialCharTok{/}\DecValTok{4}\NormalTok{, }\AttributeTok{size =} \DecValTok{1}\NormalTok{) }\SpecialCharTok{+}
    \FunctionTok{geom\_smooth}\NormalTok{(}\AttributeTok{method =} \StringTok{"glm"}\NormalTok{, }
                \AttributeTok{formula =} \StringTok{"y \textasciitilde{} x"}\NormalTok{,}
                \AttributeTok{method.args =} \FunctionTok{list}\NormalTok{(}\AttributeTok{family =} \StringTok{"binomial"}\NormalTok{), }
                \CommentTok{\# se = FALSE,}
                \AttributeTok{size =} \DecValTok{1}\NormalTok{) }\SpecialCharTok{+}
    \FunctionTok{labs}\NormalTok{(}\AttributeTok{x =} \StringTok{"Age (months)"}\NormalTok{,}
         \AttributeTok{y =} \StringTok{"Vocabulary size"}\NormalTok{,}
         \AttributeTok{colour =} \StringTok{"Imputed"}\NormalTok{) }\SpecialCharTok{+}
    \FunctionTok{theme}\NormalTok{(}\AttributeTok{legend.position =} \FunctionTok{c}\NormalTok{(}\DecValTok{1}\NormalTok{, }\DecValTok{0}\NormalTok{),}
          \AttributeTok{legend.justification =} \FunctionTok{c}\NormalTok{(}\DecValTok{1}\NormalTok{, }\DecValTok{0}\NormalTok{),}
          \AttributeTok{legend.title =} \FunctionTok{element\_blank}\NormalTok{())}
\end{Highlighting}
\end{Shaded}

\begin{figure}[H]

{\centering \includegraphics{manuscript_files/figure-pdf/fig-vocabulary-imputation-1.pdf}

}

\caption{\label{fig-vocabulary-imputation}\textbf{?(caption)}}

\end{figure}

\hypertarget{appendix-b-distribution-of-prime-and-target-looking-times}{%
\subsection{Appendix B: distribution of prime and target looking
times}\label{appendix-b-distribution-of-prime-and-target-looking-times}}

\begin{Shaded}
\begin{Highlighting}[]
\NormalTok{looking\_times |\textgreater{} }
\NormalTok{    mutate(prime\_time = cut(prime\_time, }
\NormalTok{                            seq(0, 1.5, 0.1),}
\NormalTok{                            labels = FALSE,}
\NormalTok{                            include.lowest = TRUE)) |\textgreater{} }
\NormalTok{    add\_count(lp, age\_group, name = "n\_total") |\textgreater{} }
\NormalTok{    count(lp, age\_group, prime\_time, n\_total) |\textgreater{} }
\NormalTok{    mutate(n = n / n\_total) |\textgreater{} }
\NormalTok{    ggplot(aes(prime\_time, n)) +}
\NormalTok{    facet\_grid(lp\textasciitilde{}age\_group) +}
\NormalTok{    annotate(geom = "rect",}
\NormalTok{             fill = "orange",}
\NormalTok{             ymin = {-}Inf,}
\NormalTok{             ymax = Inf,}
\NormalTok{             xmin = 0,}
\NormalTok{             xmax = 0.75*10,}
\NormalTok{             alpha = 1/3,}
\NormalTok{             colour = NA) +}
\NormalTok{    \# annotate(label = "Excluded trials",}
\NormalTok{    \#        geom = "text",}
\NormalTok{    \#         x = (0.75/2)*10,}
\NormalTok{    \#         y = 1)}
\NormalTok{    geom\_col(fill = "black") +}
\NormalTok{    labs(x = "Prime looking time (s)",}
\NormalTok{         y = "Proportion of trials") +}
\NormalTok{    scale\_x\_continuous(labels = \textbackslash{}(x) x/10) +}
\NormalTok{    theme(panel.grid.major.y = element\_line(linetype = "dotted",}
\NormalTok{                                            colour = "grey"))}
\end{Highlighting}
\end{Shaded}

\begin{Shaded}
\begin{Highlighting}[]
\NormalTok{looking\_times |\textgreater{} }
\NormalTok{    mutate(target\_time = cut(target\_time, }
\NormalTok{                            seq(0, 2, 0.1),}
\NormalTok{                            labels = FALSE,}
\NormalTok{                            include.lowest = TRUE)) |\textgreater{} }
\NormalTok{    add\_count(lp, age\_group, name = "n\_total") |\textgreater{} }
\NormalTok{    count(lp, age\_group, target\_time, n\_total) |\textgreater{} }
\NormalTok{    mutate(n = n / n\_total) |\textgreater{} }
\NormalTok{    ggplot(aes(target\_time, n)) +}
\NormalTok{    facet\_grid(lp\textasciitilde{}age\_group) +}
\NormalTok{    \# annotate(label = "Excluded trials",}
\NormalTok{    \#        geom = "text",}
\NormalTok{    \#         x = (0.75/2)*10,}
\NormalTok{    \#         y = 1)}
\NormalTok{    geom\_col(fill = "black") +}
\NormalTok{    labs(x = "Targets looking time (s)",}
\NormalTok{         y = "Proportion of trials") +}
\NormalTok{    scale\_x\_continuous(labels = \textbackslash{}(x) x/10) +}
\NormalTok{    theme(panel.grid.major.y = element\_line(linetype = "dotted",}
\NormalTok{                                            colour = "grey"))}
\end{Highlighting}
\end{Shaded}

\hypertarget{appendix-c-prime-and-test-looking-times}{%
\subsection{Appendix C: prime and test looking
times}\label{appendix-c-prime-and-test-looking-times}}

\begin{Shaded}
\begin{Highlighting}[]
\NormalTok{looking\_times }\SpecialCharTok{|\textgreater{}}
    \FunctionTok{left\_join}\NormalTok{(}\FunctionTok{select}\NormalTok{(attrition\_trials, }
\NormalTok{                     filename, trial, is\_valid\_trial),}
              \AttributeTok{by =} \FunctionTok{join\_by}\NormalTok{(filename, trial)) }\SpecialCharTok{|\textgreater{}}
    \FunctionTok{filter}\NormalTok{(is\_valid\_trial) }\SpecialCharTok{|\textgreater{}} 
    \FunctionTok{ggplot}\NormalTok{(}\FunctionTok{aes}\NormalTok{(duration,}
\NormalTok{               prime\_time,}
               \AttributeTok{colour =}\NormalTok{ trial\_type, }
               \AttributeTok{fill =}\NormalTok{ trial\_type,}
               \AttributeTok{shape =}\NormalTok{ trial\_type)) }\SpecialCharTok{+}
    \FunctionTok{facet\_grid}\NormalTok{(lp}\SpecialCharTok{\textasciitilde{}}\NormalTok{age\_group) }\SpecialCharTok{+}
    \FunctionTok{geom\_point}\NormalTok{(}\AttributeTok{alpha =} \FloatTok{0.5}\NormalTok{, }
               \AttributeTok{size =} \FloatTok{0.5}\NormalTok{) }\SpecialCharTok{+}
    \FunctionTok{geom\_smooth}\NormalTok{(}\AttributeTok{method =} \StringTok{"lm"}\NormalTok{) }\SpecialCharTok{+}
    \FunctionTok{labs}\NormalTok{(}\AttributeTok{x =} \StringTok{"Audio duration (s)"}\NormalTok{,}
         \AttributeTok{y =} \StringTok{"Looking time (1.0{-}2.0 seconds)"}\NormalTok{,}
         \AttributeTok{colour =} \StringTok{"Prime type"}\NormalTok{,}
         \AttributeTok{fill =} \StringTok{"Prime type"}\NormalTok{,}
         \AttributeTok{shape =} \StringTok{"Prime type"}\NormalTok{)}
\end{Highlighting}
\end{Shaded}

\begin{figure}[H]

{\centering \includegraphics{manuscript_files/figure-pdf/fig-looking-times-1.pdf}

}

\caption{\label{fig-looking-times}Looking time (s) to the prime AOI
against audio duration. In longer audios, participants are expected to
look longer to the target picture.}

\end{figure}

\begin{Shaded}
\begin{Highlighting}[]
\NormalTok{looking\_times }\SpecialCharTok{|\textgreater{}}
    \FunctionTok{left\_join}\NormalTok{(}\FunctionTok{select}\NormalTok{(attrition\_trials, }
\NormalTok{                     filename, trial, is\_valid\_trial),}
              \AttributeTok{by =} \FunctionTok{join\_by}\NormalTok{(filename, trial)) }\SpecialCharTok{|\textgreater{}}
    \FunctionTok{filter}\NormalTok{(is\_valid\_trial) }\SpecialCharTok{|\textgreater{}} 
    \FunctionTok{ggplot}\NormalTok{(}\FunctionTok{aes}\NormalTok{(duration,}
\NormalTok{               target\_time,}
               \AttributeTok{colour =}\NormalTok{ trial\_type, }
               \AttributeTok{fill =}\NormalTok{ trial\_type,}
               \AttributeTok{shape =}\NormalTok{ trial\_type)) }\SpecialCharTok{+}
    \FunctionTok{facet\_grid}\NormalTok{(lp}\SpecialCharTok{\textasciitilde{}}\NormalTok{age\_group) }\SpecialCharTok{+}
    \FunctionTok{geom\_point}\NormalTok{(}\AttributeTok{alpha =} \FloatTok{0.5}\NormalTok{,}
               \AttributeTok{size =} \FloatTok{0.5}\NormalTok{) }\SpecialCharTok{+}
    \FunctionTok{geom\_smooth}\NormalTok{(}\AttributeTok{method =} \StringTok{"lm"}\NormalTok{,}
                \AttributeTok{formula =} \StringTok{"y \textasciitilde{} x"}\NormalTok{) }\SpecialCharTok{+}
    \FunctionTok{labs}\NormalTok{(}\AttributeTok{x =} \StringTok{"Audio duration (s)"}\NormalTok{,}
         \AttributeTok{y =} \StringTok{"Looking time (1.0{-}2.0 seconds)"}\NormalTok{,}
         \AttributeTok{colour =} \StringTok{"Prime type"}\NormalTok{,}
         \AttributeTok{fill =} \StringTok{"Prime type"}\NormalTok{,}
         \AttributeTok{shape =} \StringTok{"Prime type"}\NormalTok{)}
\end{Highlighting}
\end{Shaded}

\begin{figure}[H]

{\centering \includegraphics{manuscript_files/figure-pdf/fig-target-times-1.pdf}

}

\caption{\label{fig-target-times}Looking time (s) to the prime AOI
against audio duration. In longer audios, participants are expected to
look longer to the target picture.}

\end{figure}

\hypertarget{refs}{}
\begin{CSLReferences}{1}{0}
\leavevmode\vadjust pre{\hypertarget{ref-boersma2001speak}{}}%
Boersma, P., \& Van Heuven, V. (2001). Speak and unSpeak with PRAAT.
\emph{Glot International}, \emph{5}(9/10), 341--347.

\leavevmode\vadjust pre{\hypertarget{ref-bosch2001evidence}{}}%
Bosch, L., \& Sebastián-Gallés, N. (2001). Evidence of early language
discrimination abilities in infants from bilingual environments.
\emph{Infancy}, \emph{2}(1), 29--49.

\leavevmode\vadjust pre{\hypertarget{ref-fenson1994variability}{}}%
Fenson, L., Dale, P. S., Reznick, J. S., Bates, E., Thal, D. J.,
Pethick, S. J., Tomasello, M., Mervis, C. B., \& Stiles, J. (1994).
Variability in early communicative development. \emph{Monographs of the
Society for Research in Child Development}, i--185.

\leavevmode\vadjust pre{\hypertarget{ref-garcia-castro2023bvq}{}}%
Garcia-Castro, G., Ávila-Varela, D. S., \& Sebastian-Galles, N. (2023).
\emph{Bvq: Barcelona vocabulary questionnaire database and helper
functions}. \url{https://gongcastro.github.io/bvq}

\leavevmode\vadjust pre{\hypertarget{ref-van2011mice}{}}%
Van Buuren, S., \& Groothuis-Oudshoorn, K. (2011). Mice: Multivariate
imputation by chained equations in r. \emph{Journal of Statistical
Software}, \emph{45}, 1--67.

\leavevmode\vadjust pre{\hypertarget{ref-wood2017generalized}{}}%
Wood, S. N. (2017). \emph{Generalized additive models: An introduction
with r}. CRC press.

\leavevmode\vadjust pre{\hypertarget{ref-zettersten2022peekbank}{}}%
Zettersten, M., Yurovsky, D., Xu, T. L., Uner, S., Tsui, A. S. M.,
Schneider, R. M., Saleh, A. N., Meylan, S. C., Marchman, V. A.,
Mankewitz, J., et al. (2022). Peekbank: An open, large-scale repository
for developmental eye-tracking data of children's word recognition.
\emph{Behavior Research Methods}, 1--16.

\leavevmode\vadjust pre{\hypertarget{ref-boersma2001speak}{}}%
Boersma, P., \& Van Heuven, V. (2001). Speak and unSpeak with PRAAT.
\emph{Glot International}, \emph{5}(9/10), 341--347.

\leavevmode\vadjust pre{\hypertarget{ref-bosch2001evidence}{}}%
Bosch, L., \& Sebastián-Gallés, N. (2001). Evidence of early language
discrimination abilities in infants from bilingual environments.
\emph{Infancy}, \emph{2}(1), 29--49.

\leavevmode\vadjust pre{\hypertarget{ref-fenson1994variability}{}}%
Fenson, L., Dale, P. S., Reznick, J. S., Bates, E., Thal, D. J.,
Pethick, S. J., Tomasello, M., Mervis, C. B., \& Stiles, J. (1994).
Variability in early communicative development. \emph{Monographs of the
Society for Research in Child Development}, i--185.

\leavevmode\vadjust pre{\hypertarget{ref-garcia-castro2023bvq}{}}%
Garcia-Castro, G., Ávila-Varela, D. S., \& Sebastian-Galles, N. (2023).
\emph{Bvq: Barcelona vocabulary questionnaire database and helper
functions}. \url{https://gongcastro.github.io/bvq}

\leavevmode\vadjust pre{\hypertarget{ref-van2011mice}{}}%
Van Buuren, S., \& Groothuis-Oudshoorn, K. (2011). Mice: Multivariate
imputation by chained equations in r. \emph{Journal of Statistical
Software}, \emph{45}, 1--67.

\leavevmode\vadjust pre{\hypertarget{ref-wood2017generalized}{}}%
Wood, S. N. (2017). \emph{Generalized additive models: An introduction
with r}. CRC press.

\leavevmode\vadjust pre{\hypertarget{ref-zettersten2022peekbank}{}}%
Zettersten, M., Yurovsky, D., Xu, T. L., Uner, S., Tsui, A. S. M.,
Schneider, R. M., Saleh, A. N., Meylan, S. C., Marchman, V. A.,
Mankewitz, J., et al. (2022). Peekbank: An open, large-scale repository
for developmental eye-tracking data of children's word recognition.
\emph{Behavior Research Methods}, 1--16.

\end{CSLReferences}



\end{document}
